%%%%%%%%%%%%%%%%%%%%%%%%%%%%%%%%%%%%%%%%%%%%%%%%%%%%%%%%%%%%%%%%%%%%%%%%%%%
%
%			Mathematics 132 Course Notes
%
%			 Department of Mathematics,
%   			  University of Melbourne
%
%		Stephen Simmons			Lee White
%
% 8 Feb-96 SS: Updated with corrections from semester 2, 1995
%
%%%%%%%%%%%%%%%%%%% Copyright (C) 1995-96 Stephen Simmons %%%%%%%%%%%%%%%%%

%%%%%%%%%%%%%%%%%%%%%%%%%%%%%%%%%%%%%%%%%%%%%%%%%%%%%%%%%%%%%%%%%%%%%%%%%%%%%
\chapter{Single Particle Dynamics}
\label{spd chp}

A \name{particle} is a moving geometric point at which matter is
concentrated.  \name{Newton's Laws of Motion} describe how particles behave:
\begin{description}
\item[First Law] A particle moves at constant velocity $\dot{\vect{r}}$ 
relative to an inertial frame of reference unless acted upon by a force.

\item[Second Law] If a particle is acted upon by a force $\vect{F}$, it will
accelerate relative to an inertial frame of reference such that
$$\vect{F}=m\ddot{\vect{r}}$$
The constant of proportionality $m$ is the \name{inertial mass} of the
particle.

\item[Third Law] When two particles act on one another, the two forces
acting on the particles are equal in magnitude and opposite in direction and
along the line of their centres
$$\vect{F}_{12}=-\vect{F}_{21}$$
\end{description}

Particles are also affected by \name{Newton's postulate of gravitation},
which says that any two particles in the universe attract each other with a
gravitational force of magnitude
$$\frac{Gm_1m_2}{r^2}$$
along the line of their centres, where $r$ is the distance between the
particles, $m_1$ and $m_2$ are their inertial masses and $G$ is the
\name{gravitational constant}
$$G=6.67\times 10^{-11}\rm\,m^3kg^{-1}s^{-2}$$

The gravitational force between a uniform sphere of mass $M$ and a particle
of mass $m$ is
$$\vect{F}_p=-\frac{GmM}{r^2}\rvect$$
This is derived in Fowles, pp. 134--136.

The near gravitational field of the Earth can be found using
$$\vect{F}_p=-\frac{GmM}{r^2}\rvect$$
where $M$ is the mass of the Earth and $r$ is the distance of the particle
from the centre of the Earth.  Write $r=R+h$ where $R$ is the radius of the
Earth and $h$ is the height of the particle above the surface of the Earth. 
Then
$$\vect{F}_p=-\frac{GmM}{(R+h)^2}\rvect$$
If the particle is close to the Earth's surface,
$$\vect{F}_p=-\frac{GM}{R^2}m\rvect + O\left(\frac{h}{R}\right)$$
Since $h\ll R$, the $h/R$ terms can be neglected so that
$$\vect{F}_p=-\frac{GM}{R^2}m\rvect=-gm\rvect$$
where $g$ is gravitational acceleration at the Earth's surface
$$g=\frac{GM}{R^2}=9.81\rm\,ms^{-2}$$
This simplification is valid as long as $h\ll R$ and it assumes that the Earth
is composed of uniform spherical shells.

%%%%%%%%%%%%%%%%%%%%%%%%%%%%%%%%%%%%%%%%%%%%%%%%%%%%%%%%%%%%%%%%%%%%%%%%%%%%%
\section{Rectilinear Motion}

If the particle moves in a straight line, the force must always be parallel
to the direction of its motion.  Suppose that the force is constant
$$m\ddot{\vect{r}}=\vect{F}$$
Choose the $x$ axis to lie in the direction of $\vect{F}$.  Then
$\vect{F}=F\ivect$ and
\begin{eqnarray*}
m\ddot{x}&=&F\\
m\ddot{y}&=&0\\
m\ddot{z}&=&0
\end{eqnarray*}
Integrating gives the particle's velocities 
\begin{eqnarray*}
\dot{x}(t)&=&\frac{F}{m}t+\dot{x}(0)\\
\dot{y}(t)&=&\dot{y}(0)=0\\
\dot{z}(t)&=&\dot{z}(0)=0
\end{eqnarray*}
because there is no motion in the $y$ and $z$ directions.

Integrating again gives the particle's position
\begin{eqnarray*}
x(t)&=&\frac{1}{2}\frac{F}{m}t^2+\dot{x}(0)t+x(0)\\
y(t)&=&y(0)\\
z(t)&=&z(0)
\end{eqnarray*}


Using this, note that
\begin{eqnarray*}
2\frac{F}{m}\left[x(t)-x(0)\right]
&=&\left(\frac{F}{m}t\right)^2+2\frac{F}{m}t\dot{x}(0) \\
&=&\left(\dot{x}(t)-\dot{x}(0)\right)^2
	+2\dot{x}(0)\left(\dot{x}(t)-\dot{x}(0)\right)\\
&=&\dot{x}^2(t)-\dot{x}^2(0)
\end{eqnarray*}
This can be written as
$$F\left[x(t)-x(0)\right]=\frac{1}{2}m\dot{x}^2(t)-\frac{1}{2}m\dot{x}^2(0)$$
which shows that the work done is equal to the change in kinetic energy.

Alternatively, this can be obtained using
$$\ddot{x}=\der{\dot{x}}{t}=\der{\dot{x}}{x}\der{x}{t}
=\dot{x}\der{\dot{x}}{x}=\frac{1}{2}\dbd{x}\left(\dot{x}^2\right)$$
so that
$$\frac{1}{2}\dbd{x}\left(\dot{x}^2\right)=\frac{F}{m}$$
Integrating both sides with respect to $x$ shows that
$$\dot{x}^2(t)-\dot{x}^2(0)=\frac{2F}{m}\left(x(t)-x(0)\right)$$

In the \name{general rectilinear problem}, only one coordinate is important.
If the $z$ axis lies along the line of motion,
$$\vect{r}=z\kvect$$
and
$$\vect{F}=F\kvect$$
The equation of motion becomes a scalar differential equation in $z$
$$m\ddot{z}=F(z,\dot{z},t)$$

%============================================================================
\begin{example}
\label{spd ex:dropped}

\problem A particle is dropped from rest at a height $z_0$ close to the
Earth's surface, with air resistance proportional to the particle's speed.
Find the particle's height $z(t)$ and its terminal velocity 
$\dot{z}_{\rm ter}$.

\solution
Air resistance is proportional to speed so the force on the particle is
$$\vect{F}=-mg\kvect-\lambda\dot{z}\kvect$$
Therefore the equation of motion is a second order linear \ODE with 
constant coefficients
$$m\ddot{z}=-mg-\lambda\dot{z}$$
with $\dot{z}=0$ and $z=z_0$ when $t=0$.

In standard form, this is
$$\ddot{z}+\frac{\lambda}{m}\dot{z}=-g$$
Try a homogeneous solution of the form $z=e^{\alpha t}$.  This gives
$$\ddot{z}+\frac{\lambda}{m}\dot{z}
=\left[\alpha^2+\frac{\lambda}{m}\alpha\right]e^{\alpha t}=0$$
so that
$$z=A+Be^{-\frac{\lambda}{m}t}+z_p$$
where the particular solution $z_p$ is yet to be determined.

The homogeneous solution with $\alpha=0$ has the same form as the
inhomogeneous term in the differential equation, thus a constant is not an
acceptable particular solution.  Instead try
$$z_p(t)=ct$$
Substituting into the differential equation shows that
$$c=-\frac{mg}{\lambda}$$
so that 
$$z_p(t)=-\frac{mg}{\lambda}t$$
giving the general solution
$$z=A+Be^{-\frac{\lambda}{m}t}-\frac{mg}{\lambda}t$$

When $t=0$, $z_0=A+B$ and $\dot{z}=0=-\lambda B/m-mg/\lambda$,
so the constants are
$$B=-\left(\frac{m}{\lambda}\right)^2g$$
and
$$A=z_0-B=z_0+\left(\frac{m}{\lambda}\right)^2g$$
giving the solution
$$z(t)=z_0+\left(\frac{m}{\lambda}\right)^2g\left[
1-e^{-\frac{\lambda}{m}t}-\frac{\lambda}{m}t\right]$$

The speed of the particle is
$$\dot{z}(t)=-\frac{mg}{\lambda}\left[1-e^{-\frac{\lambda}{m}t}\right]$$
which reaches a terminal velocity of 
$$\dot{z}_{\rm ter}=-\frac{mg}{\lambda}$$
as $t\to\infty$.

\parbreak Alternatively the terminal velocity can be obtained directly 
from the \ODE by noting
that as the terminal velocity is approached, $\dot{z}$ becomes constant so
that $\ddot{z}$ tends to zero.  Therefore the differential equation
$$m\ddot{z}=-mg-\lambda\dot{z}$$
becomes
$$0=-mg-\lambda\dot{z}_{\rm ter}$$
so that the terminal velocity is
$$\dot{z}_{\rm ter}=-\frac{mg}{\lambda}$$

To see what happens in the absence of air resistance, take the limit as
$\lambda\to 0$.  Now 
\begin{eqnarray*}
\lim_{\lambda\to 0} z(t)
&=&z_0+\lim_{\lambda\to 0}\left(\frac{m}{\lambda}\right)^2g\left(
1-e^{-\frac{\lambda}{m}t}-\frac{\lambda}{m}t\right) \\
&=&z_0+g\lim_{\alpha\to 0}\frac{1}{\alpha^2}
	\left(1-e^{-\alpha t}-\alpha t\right) \\
&=&z_0+g\lim_{\alpha\to 0}\frac{1}{\alpha^2}
	\left[(1-\alpha t)-\left(1-\alpha t +\frac{\alpha^2t^2}{2}-\cdots
\right)\right] \\
&=&z_0-\frac{gt^2}{2}
\end{eqnarray*}
which is the free-fall limit.
\end{example}
%============================================================================

%============================================================================
\begin{example}
\problem Find the height reached by a particle launched vertically upwards 
with speed $v$ from the Earth's surface with air resistance proportional 
to the square of the particle's speed.

\solution The equation of motion is 
$$m\ddot{z}=-mg-\lambda\dot{z}^2$$
subject to $z=0$ and $\dot{z}=v$ at $t=0$.

The differential equation is
$$\ddot{z}=-g-\frac{\lambda}{m}\dot{z}^2$$
But $\ddot{z}=\dbd{z}\left(\frac{1}{2}\dot{z}^2\right)$ so put 
$w=\dot{z}^2/2$ in the equation of motion to give
$$\der{w}{z}=-g-\frac{2\lambda}{m}w$$
or
$$\der{w}{z}+\frac{2\lambda}{m}w=-g$$
$w(z)$ is the sum of the homogeneous solution and a particular solution.  
For the homogeneous solution, trying $w=e^{\alpha z}$ shows that
$\alpha=-2\lambda/m$.  For the particular solution, try 
$$w_p=A$$
Substituting into the \ODE shows that
$$\frac{2\lambda}{m}A=-g$$
so that the general solution is
$$w(z)=Ce^{-\frac{2\lambda}{m}z}-\frac{mg}{2\lambda}$$

At $z=0$, $w=v^2/2$ so
$$\frac{1}{2}v^2=C-\frac{mg}{2\lambda}$$
Therefore
$$w(z)=\left(\frac{v^2}{2}+\frac{mg}{2\lambda}\right)
\,e^{-\frac{2\lambda}{m}z}-\frac{mg}{2\lambda}$$

At the greatest height reached, the velocity is zero, hence
$$w(z_{\rm max})=0$$
Therefore
$$0=\left(\frac{v^2}{2}+\frac{mg}{2\lambda}\right)
\,e^{-\frac{2\lambda}{m}z_{\rm max}}-\frac{mg}{2\lambda}$$
so that
$$z_{\rm max}=\frac{m}{2\lambda}\ln\left[1+\frac{\lambda v^2}{mg}\right]$$

To find the greatest height reached when there is no air resistance,
take the limit as $\lambda\to 0$.  Since, for small $x$,
$$\frac{1}{1+x}=1-x+x^2-x^3+\cdots$$
integrating term-by-term gives
$$\ln(1+x)=\int^x \frac{dx}{1+x}=x-\frac{x^2}{x}+\frac{x^3}{3}-\cdots$$
Therefore
$$z_{\rm max}=\frac{m}{2\lambda}\left[
\frac{\lambda v^2}{mg}-\frac{1}{2}\left(\frac{\lambda v^2}{mg}\right)^2
+\frac{1}{3}\left(\frac{\lambda v^2}{mg}\right)^3-\cdots\right]$$
This can be written in the form
$$z_{\rm max}=\frac{v^2}{2g}+O(\lambda)$$
where $O(\lambda)$ indicates terms of order $\lambda$ and higher.  In the
limit as $\lambda\to 0$, the $O(\lambda)$ terms tend to zero, leaving
$$z_{\rm max}=\frac{v^2}{2g}$$
\end{example}
%============================================================================

%%%%%%%%%%%%%%%%%%%%%%%%%%%%%%%%%%%%%%%%%%%%%%%%%%%%%%%%%%%%%%%%%%%%%%%%%%%%%
\section{Motion in a Plane}

For this analysis of projectile motion, ignore the Earth's rotation and
assume that the height of the projectile is always much smaller than the
radius of the Earth.

%----------------------------------------------------------------------------
\begin{figure}\centering
\caption{Motion of a projectile with air resistance $\vect{F}_{\rm res}$. 
The projectile's initial velocity is $v$ at an angle $\alpha$ to the
horizontal.}
\label{spd fig:pm I}

\psset{unit=5cm}
\begin{pspicture}(-0.25,-0.35)(1.65,1.2)
% First the curve, so the axes show through
\psplot[linecolor=gray,linewidth=2pt,plotstyle=curve]{0}{1.2}{2 x mul 
1.2 x sub mul}
% Now the axes and their labels
\psline{<->}(0,0.8)(0,0)(1.4,0)
\uput[dl](0,0){$O$}
\uput[r](1.4,0){$x$}
\uput[u](0,0.8){$z$}
% The velocity at the origin
\psline{->}(0,0)(0.154,0.369)
\uput[u](0.154,0.369){$v$}
\psarc{->}(0,0){0.2}{0}{67}
\uput[r](0.17,0.07){$\alpha$}
% The vector r(t)
\pcline{->}(0,0)(0.9,0.54)
\Bput{$\vect{r}(t)$}
% The weight
\psline{->}(0.9,0.54)(0.9,0.29)
\uput[d](0.9,0.29){$mg$}
% The air resistance
\rput[b]{-50.2}(0.9,0.54){
	\psline[linecolor=black,linestyle=dashed]{-}(0,0)(0.25,0) 
	\rput[b]{*0}(0,0.05){$P$}
	\psline[linecolor=black]{->}(0,0)(-0.25,0) 
	\rput[br]{*0}(-0.26,0){$\vect{F}_{\rm res}$}
}
\end{pspicture}
\end{figure}
%----------------------------------------------------------------------------

From figure \ref{spd fig:pm I}, the equation of motion is
$$m\ddot{\vect{r}}=-mg\kvect+\vect{F}_{\rm res}$$
where the air resistance has magnitude $\phi(v)$ and is directed in the
opposite direction to the projectile's motion
$$\vect{F}_{\rm res}=-\phi(v) \hat{\dot{\vect{r}}}
=-\frac{\phi(v)}{v}\dot{\vect{r}}$$
where the projectile's speed is
$$v=\left|\dot{\vect{r}}\right|=\sqrt{\dot{x}^2+\dot{z}^2}$$

The equation of motion can be written as separate equations for the $x$ and
$z$ directions
$$\ddot{x}=-\frac{\phi(v)}{mv}\dot{x}$$
and
$$\ddot{z}=-g-\frac{\phi(v)}{mv}\dot{z}$$
The solution of these depend on the form of the air resistance function
$\phi(v)$.


%============================================================================
\begin{example}
When $\phi(v)=0$ and there is no air resistance, the equations of motion
become
$$\ddot{x}=0\qquad\mbox{and}\qquad\ddot{z}=-g$$
subject to the initial conditions $x(0)=z(0)=0$, $\dot{x}(0)=v\cos\alpha$ 
and $\dot{z}(0)=v\sin\alpha$.

The solution is 
$$x(t)=vt\cos\alpha$$
$$z(t)=vt\sin\alpha-\frac{1}{2}gt^2$$
Eliminating $t$ from these two equations gives the equation of the
projectile's trajectory
$$z=x\tan\alpha-\frac{gx^2}{2v^2\cos^2\alpha}$$
which is a parabola.  Factorising $z$ as
$$z=\frac{x}{\cos\alpha}\left(\sin\alpha-\frac{gx}{2v^2\cos\alpha}\right)$$
shows that $z=0$ when $x=0$ and when $x=2v^2\sin\alpha\cos\alpha/g$
so that the horizontal range is
$$x_{\rm max}=\frac{v^2}{g}\sin 2\alpha$$
This range is maximised when $\sin 2\alpha=1$, which is an elevation of 
$\alpha=\pi/4$.  

The maximum height occurs when $\dot{z}(t)=0$, which occurs when
$$t=\frac{v\sin\alpha}{g}$$
Therefore the maximum height as a function of the projectile's initial
elevation is
$$z_{\rm max}(\alpha)=\frac{v^2\sin^2\alpha}{2g}$$
which occurs when $x$ is equal to half the range.  Note that
$$z_{\rm max}\left(\frac{\pi}{4}\right)=\frac{v^2}{4g}$$
which is equal to one-quarter of the maximum range.
\end{example}
%============================================================================

%============================================================================
\begin{exercise}
Now try the problem again assuming that the ground is at an angle $\beta$
so that when the projectile lands, $x$ and $z$ must satisfy
$$z=x\tan\beta$$
Combine this with the equation for the projectile's motion to find the $x$
and $z$ coordinates of the point when the projectile lands.
\end{exercise}
%============================================================================


%============================================================================
\begin{example}
\label{spd ex:prop sp}

When $\phi(v)=\lambda v$ and the air resistance is proportional to the
projectile's speed, the equations of motion become
$$\ddot{x}=-\frac{\lambda}{m}\dot{x}$$
and
$$\ddot{z}=-g-\frac{\lambda}{m}\dot{z}$$
subject to the initial conditions $x(0)=z(0)=0$, $\dot{x}(0)=v\cos\alpha$ 
and $\dot{z}(0)=v\sin\alpha$.

This has already been solved for $z$ with different boundary conditions
in example \ref{spd ex:dropped}, so the solution is
$$z(t)=A+Be^{-\frac{\lambda}{m}t}-\frac{m}{\lambda}gt$$
The boundary conditions are
$$z(0)=0=A+B$$
and
$$\dot{z}(0)=v\sin\alpha=-\frac{\lambda}{m}B-\frac{mg}{\lambda}$$
Therefore
$$A=-B=\left(\frac{m}{\lambda}\right)^2g+\frac{m}{\lambda}v\sin\alpha$$
so the $z$ solution is
$$z(t)=\left[\left(\frac{m}{\lambda}\right)^2g+\frac{m}{\lambda}v\sin\alpha
\right]\left(1-e^{-\frac{\lambda}{m}t}\right)-\frac{m}{\lambda}gt$$

The general solution for $x$ is
$$x(t)=C+De^{-\frac{\lambda}{m}t}$$
The boundary conditions are
$$x(0)=0=C+D$$ 
and
$$\dot{x}(0)=v\cos\alpha=-\frac{\lambda}{m}D$$
Therefore
$$C=-D=\frac{m}{\lambda}v\cos\alpha$$
so the $x$ solution is
$$x(t)=\frac{m}{\lambda}v\cos\alpha\left(1-e^{-\frac{\lambda}{m}t}\right)$$

Eliminating $t$ from the equations for $x(t)$ and $z(t)$ gives
$$z=\left(\frac{v\sin\alpha+mg/\lambda}{v\cos\alpha}\right)x
+\left(\frac{m}{\lambda}\right)^2g
\ln\left(1-\frac{\lambda x}{mv\cos\alpha}\right)$$
which is plotted in figure \ref{spd fig:pm II}.
\end{example}
%============================================================================

%----------------------------------------------------------------------------
\begin{figure}\centering
\caption{For projectile motion with air resistance, three points $x_1$,
$x_2$ and $x_3$ can be defined.  $x_1$ is the horizontal position of the
projectile's greatest elevation, $x_2$ the point at which the projectile
hits the ground, and $x_3$ the position of the vertical asymptote if the
projectile were to continue falling through the ground.}
\label{spd fig:pm II}

\psset{unit=5cm}
\begin{pspicture}(-0.25,-0.5)(1.65,1.2)
% First the curve, so the axes show through
\psplot[linecolor=gray,linewidth=2pt,plotstyle=curve]{0}{1.175}{1 x 1.2 div
sub ln 0.4 mul x add 1.4 mul} 
% Now the axes and their labels
\psline{->}(0,0)(1.4,0)
\psline{->}(0,-0.5)(0,0.8)
\uput[l](0,0){$O$}
\uput[r](1.4,0){$x$}
\uput[u](0,0.8){$z$}
% The velocity at the origin
\psline{->}(0,0)(0.29,0.27)
\uput[u](0.29,0.27){$v$}
\psarc{->}(0,0){0.2}{0}{43}
\uput[r](0.17,0.07){$\alpha$}
% The intercepts				
\psline[linecolor=black,linestyle=dashed]{-}(0.8,0)(0.8,0.5) 
\uput[d](0.8,0){$x_1$}
\psline[linecolor=black,linestyle=dashed]{-}(1.2,-0.5)(1.2,0.5) 
\uput[ur](1.2,0){$x_3$}
\uput[dl](1.1,0){$x_2$}
\end{pspicture}
\end{figure}
%----------------------------------------------------------------------------

%============================================================================
\begin{exercise}
Find expressions for the points $x_1$ and $x_3$ in figure \ref{spd fig:pm
II}.
\end{exercise}
%============================================================================

%============================================================================
\begin{example}
The general expression for air resistance is $\phi(v)=\lambda v^m$ for some
power $m$.  The equation of motion is
$$m\ddot{\vect{r}}=-mg\kvect-\phi(v)\Tvect$$
The velocity of the projectile is just the speed in the tangential direction
$\dot{\vect{r}}=\dot{s}\Tvect$ so
$$\left|\dot{\vect{r}}\right|=v=\dot{s}$$
From
$$\ddot{\vect{r}}=\ddot{s}\Tvect+\dot{s}\dot{\psi}\Nvect$$
the equation of motion in the $\Tvect$ direction is
\begin{eqnarray*}
\ddot{s}&=&-g\kvect\cdot\Tvect-\frac{\phi(\dot{s})}{m} \\
&=&-g\sin\psi-\frac{\phi(\dot{s})}{m}
\end{eqnarray*}
and in the $\Nvect$ direction,
\begin{eqnarray*}
\dot{s}\dot{\psi}&=&-g\kvect\cdot\Nvect\\
&=&-g\cos\psi
\end{eqnarray*}
The remainder of the solution depends on the form of $\phi(\dot{s})$.
\end{example}
%============================================================================

%----------------------------------------------------------------------------
\begin{figure}\centering
\caption{Motion of the missile for the Death Star problem.}
\label{spd fig:ds}

\psset{unit=5cm}
\begin{pspicture}(-0.25,-0.65)(1.45,0.9)
% First the curve, so the axes show through
\psplot[linecolor=gray,linewidth=2pt,plotstyle=curve]{0}{1.175}{1 x 1.2 div
sub ln 0.4 mul x add 1.4 mul} 
% Now the axes and their labels
\psline{->}(0,0)(0.4,0)
\psline{->}(0,0)(0,0.4)
\uput[r](0.4,0){$x$}
\uput[u](0,0.4){$z$}
% The velocity at the origin
\psline{->}(0,0)(0.29,0.27)
\uput[u](0.29,0.27){$v$}
\psarc{->}(0,0){0.2}{0}{43}
\uput[r](0.17,0.07){$\alpha$}
% The Death Star
\psline[linecolor=black,linewidth=2pt]{-}(0,-0.3)(1,-0.3)(1,-0.6) 
\psline[linecolor=black,linewidth=2pt]{-}(1.2,-0.6)(1.2,-0.3)(1.4,-0.3)
% The D, H and d labels with arrows
\pcline{<->}(0,-0.55)(1.2,-0.55)
\Bput{$D$}
\pcline{<->}(1,-0.35)(1.2,-0.35)
\Aput{$d$}
\pcline{<->}(0,-0.3)(0,0)
\Aput{$H$}
% The weight
\psline{->}(0.95,0.45)(0.95,0.2)
\uput[d](0.95,0.2){$mg$}
% The air resistance
\rput[br]{-40.3}(0.95,0.45){
	\psline[linecolor=black]{->}(0,0)(-0.25,0) 
	\rput[br]{*0}(-0.26,0){$\vect{F}_{\rm res}=-\lambda\dot{\vect{r}}$}
}
\end{pspicture}
\end{figure}
%----------------------------------------------------------------------------

%============================================================================
\begin{example}[The Death Star]

\problem From the diagram in figure \ref{spd fig:ds}, you have to fire a 
missile	that falls down a narrow chute of width $d$.  The missile is
launched at an initial angle $\alpha$ with speed $v$ from a height $H$ and
a distance $D$ from the far edge of the chute.  Given that air resistance is
proportional to speed, calculate the maximum value of $D$.  Also calculate
the minimum height $H$ from which the missile can be released at this
maximum distance $D$ and still fall down the chute.

\solution
With air resistance proportional to speed, the equation of motion is
$$m\ddot{\vect{r}}=-mg\kvect-\lambda \dot{\vect{r}}$$
This is identical to the situation considered in example \ref{spd ex:prop sp}
with solution
$$z=\left(\frac{v\sin\alpha+mg/\lambda}{v\cos\alpha}\right)x
+\left(\frac{m}{\lambda}\right)^2g
\ln\left(1-\frac{\lambda x}{mv\cos\alpha}\right)$$

The maximum horizontal range is the furthest $x$ distance reached as
$z\to-\infty$.  This occurs when
$$1-\frac{\lambda x}{mv\cos\alpha}=0$$
or when
$$x=\frac{mv}{\lambda}\cos\alpha$$

The maximum range is maximised when $\alpha=0$, so this is achieved when the
missile is fired horizontally, and 
$$D=x_{\rm max}=\frac{mv}{\lambda}$$
Putting $\alpha=0$, the trajectory becomes
$$z=\left(\frac{mg}{\lambda v}\right)x
+\left(\frac{m}{\lambda}\right)^2g\ln\left(1-\frac{\lambda x}{mv}\right)$$

The trajectory with minimum height just grazes the inner lip of the chute
and the asymptote is the chute's back wall.  Therefore the trajectory
passes through
$$z=-H\qquad\mbox{at}\qquad x=D-d$$
Substituting this into the trajectory with $\alpha=0$ gives
$$-H=\left(\frac{mg}{\lambda v}\right)(D-d)
+\left(\frac{m}{\lambda}\right)^2g\ln\left(1-\frac{\lambda (D-d)}{mv}\right)$$
which gives
$$H=\frac{gD^2}{v^2}\left[-\ln\left(\frac{d}{D}\right)+\frac{d}{D}-1\right]$$
\end{example}
%============================================================================

%----------------------------------------------------------------------------
\begin{figure}\centering
\caption{This shows a projectile $P$ launched from the equator $E$ of an airless
planet with speed $v$ at an angle $\alpha$ to the vertical towards the
north.  $\vect{F}_{\rm gr}$ is the gravitational attraction.}
\label{spd fig:ap}

\psset{unit=5cm}
\begin{pspicture}(-0.6,-0.6)(1.2,0.9)
% Here is the Earth with its axes
\pscircle(0,0){0.5}
\psline{->}(-0.6,0)(1,0)
\psline{->}(0,-0.6)(0,0.6)
\uput[dr](0.5,0){$E$}
\uput[u](0,0.6){$N$}
% The parabola describes the projectile's trajectory
\parametricplot[linecolor=gray,linewidth=2pt,plotstyle=curve,arrows=->]%
{-0.3}{0.2}{1 t t mul 0.18 div sub t 0.3 add}
% The velocity at the origin
\SpecialCoor
\psline{->}(0.5,0)(0.8,0.1)
\uput[r](0.8,0.1){$v$}
\psarc{->}(0.5,0){0.2}{0}{(0.3,0.1)}
\uput[ur](0.7,0){$\alpha$}
% The point P
\uput[r](1,0.3){$P$}
\psarc{->}(0,0){0.2}{0}{(1,0.3)}
\uput[ur](0.2,0){$\theta$}
\pcline{->}(0,0)(1,0.3)
\Aput{$\vect{r}(t)$}
\pcline{->}(1,0.3)(0.8,0.24)
\Bput{$\vect{F}_{gr}$}
% Radius of the Earth
\pcline{->}(0,0)(0.5;300)\Aput{$R$}
\end{pspicture}
\end{figure}
%----------------------------------------------------------------------------

%============================================================================
\begin{example}
\problem
A projectile is fired from the equator $E$ of an airless planet of radius
$R$ northwards with initial speed $v$ and at an angle $\alpha$ to the
vertical.  The acceleration due to gravity at the surface is $g$.
This is illustrated in figure \ref{spd fig:ap}. 

\begin{description}
\item[(a)] Show that the projectile's motion satisfies
$$\ddot{r}-r\dot{\theta}^2=-\frac{gR^2}{r^2}$$
and
$$r\ddot{\theta}+2\dot{r}\dot{\theta}=0$$

\item[(b)] Hence show that 
$$\dot{\theta}=\frac{Rv\sin\alpha}{r^2}$$

\item[(c)] Also show that the maximum height of the projectile is given by
$$\frac{r}{R}=\frac{1}{2(\lambda-1)}\left(
\sqrt{\lambda^2-4(\lambda-1)\sin^2\alpha}+\lambda\right)$$
where $\lambda=2gR/v^2$ provided $\lambda>1$.
\end{description}

\solution
With $m$ the projectile's mass and $M$ the planet's mass, the force due to 
gravity is
$$\vect{F}=-\frac{GmM}{r^2}\rvect$$
At the surface, when $r=R$, the force due to gravity is
$$\vect{F}=-mg\rvect=-\frac{GmM}{R^2}\rvect$$
so that
$$\vect{F}=-\frac{mgR^2}{r^2}\rvect$$

There is no air resistance, so from Newton's second law,
$$m\ddot{\vect{r}}=\vect{F}$$
Choosing a polar coordinate system,
$$\ddot{\vect{r}}=(\ddot{r}-r\dot{\theta}^2)\rvect
+(r\ddot{\theta}+2\dot{r}\dot{\theta})\tvect=-\frac{gR^2}{r^2}\rvect$$
Equating radial and polar components yields
$$\ddot{r}-r\dot{\theta}^2=-\frac{gR^2}{r^2}$$
and
$$r\ddot{\theta}+2\dot{r}\dot{\theta}=0$$
as required.


From the initial conditions, $r(0)=R$ and $\theta(0)=0$.  Initially,
$$\dot{\vect{r}}(0)=v\cos\alpha\rvect+v\sin\alpha\tvect$$
Using
$$\dot{\vect{r}}=\dot{r}\rvect+r\dot{\theta}\tvect$$
shows that
$$\dot{r}(0)=v\cos\alpha$$
and
$$R\dot{\theta}(0)=v\sin\alpha$$

To show that
$$\dot{\theta}=\frac{Rv\sin\alpha}{r^2}$$
multiply
$$r\ddot{\theta}+2\dot{r}\dot{\theta}=0$$
by $r$ and use product rule to show that
$$\dbd{t}(r^2\dot{\theta})=0$$
Integrating shows that
$$r^2\dot{\theta}=C$$
for some constant $C$.  At $t=0$, $r=R$ and
$\dot{\theta}=v\sin\alpha/R$.  Therefore
$$r^2\dot{\theta}=R^2\frac{v}{R}\sin\alpha$$
so that
$$\dot{\theta}=\frac{Rv\sin\alpha}{r^2}$$

To show that the maximum height of the projectile is given by
$$\frac{r}{R}=\frac{1}{2(\lambda-1)}\left(
\sqrt{\lambda^2-4(\lambda-1)\sin^2\alpha}+\lambda\right)$$
where $\lambda=2gR/v^2$ provided $\lambda>1$, we know that
$$\dot{\theta}=\frac{Rv\sin\alpha}{r^2}$$
and
$$\ddot{r}-r\dot{\theta}^2=-\frac{gR^2}{r^2}$$
Substituting for $\dot{\theta}$ shows that
$$\ddot{r}=\frac{v^2R^2\sin^2\alpha}{r^3}-\frac{gR^2}{r^2}$$
Using the usual identity,
$$\ddot{r}=\dbd{r}\left(\frac{1}{2}\dot{r}^2\right)=
\frac{v^2R^2\sin^2\alpha}{r^3}-\frac{gR^2}{r^2}$$
Integrating with respect to $r$ gives
$$\frac{1}{2}\dot{r}^2=
-\frac{1}{2}\frac{v^2R^2\sin^2\alpha}{r^2}+\frac{gR^2}{r}+C$$
At time $t=0$, $r=R$ and $\dot{r}=v\cos\alpha$ so
$$\frac{1}{2}v^2\cos^2\alpha=
-\frac{1}{2}\frac{v^2R^2\sin^2\alpha}{R^2}+\frac{gR^2}{R}+C$$
which determines the constant
$$C=\frac{1}{2}v^2-gR$$
Therefore, a first integral of the radial equation is
$$\frac{1}{2}\dot{r}^2=\frac{1}{2}v^2-gR
-\frac{1}{2}\frac{v^2R^2\sin^2\alpha}{r^2}+\frac{gR^2}{r}$$
The maximum height occurs when $\dot{r}=0$, thus
$$0=\frac{1}{2}v^2-gR
-\frac{1}{2}\frac{v^2R^2\sin^2\alpha}{r^2}+\frac{gR^2}{r}$$
Divide by $v^2/2$ and rearrange to give the quadratic
$$\left(\frac{r}{R}\right)^2(\lambda-1)-\lambda\frac{r}{R}+\sin^2\alpha=0$$
Therefore
$$\frac{r}{R}=\frac{1}{2(\lambda-1)}\left(
\lambda\pm\sqrt{\lambda^2-4(\lambda-1)\sin^2\alpha}\right)$$
To determine which of the two solutions is appropriate, note that as
$v\to0$, $r/R$ must tend to $1$.  As $v\to0$, $\lambda\to\infty$,
showing that the positive solution is needed.  This gives the maximum 
height as
$$\frac{r}{R}=\frac{1}{2(\lambda-1)}\left(
\sqrt{\lambda^2-4(\lambda-1)\sin^2\alpha}+\lambda\right)$$
as required.

Note that if $\lambda<1$, then $\sqrt{\lambda^2-4(\lambda-1)\sin^2\alpha}
>\lambda$ so the negative solution must be chosen to ensure that $r/R>0$.
In this case,
$$\frac{r}{R}=\frac{1}{2(1-\lambda)}\left(
\sqrt{\lambda^2-4(\lambda-1)\sin^2\alpha}-\lambda\right)<1$$
so no maximum height is reached.  When $\lambda=1$, the speed of the projectile
is exactly equal to the escape velocity of the planet.
\end{example}
%============================================================================


%%%%%%%%%%%%%%%%%%%%%%%%%%%%%%%%%%%%%%%%%%%%%%%%%%%%%%%%%%%%%%%%%%%%%%%%%%%%%
\section{Momentum and Torque}

\name{Linear momentum} $\vect{p}$ is defined as
$$\vect{p}=m\dot{\vect{r}}$$
From Newton's second law, in an inertial reference frame,
$$\vect{F}=m\ddot{\vect{r}}=\dbd{t}(m\dot{\vect{r}})$$
so that
$$\der{\vect{p}}{t}=\vect{F}$$
When $\vect{F}=0$, $\vect{p}$ is constant, so linear momentum is conserved
when no net force acts on the particle.

The \name{angular momentum} $\vect{L}$ of a particle about the point $O$
is defined as 
$$\vect{L}=\vect{r}\times\vect{p}$$
where $\vect{r}$ is the particle's position relative to $O$ and $\vect{p}$
is the particle's linear momentum.  Note that
$$\der{\vect{L}}{t}=\dbd{t}(\vect{r}\times(m\dot{\vect{r}}))
=\dot{\vect{r}}\times(m\dot{\vect{r}})+\vect{r}\times(m\ddot{\vect{r}})$$
from product rule.  The first cross-product is zero because $\dot{\vect{r}}$
and $m\dot{\vect{r}}$ are parallel.  Writing $\vect{F}=m\ddot{\vect{r}}$
shows that
$$\der{\vect{L}}{t}=\vect{r}\times \vect{F}$$
provided the reference frame is inertial.

The \name{torque} of a force $\vect{F}$ about a point $O$ is
$\vect{r}\times\vect{F}$ where $\vect{r}$ is the position vector from $O$ to
the point of application of the force.  

Therefore, torque is equal to the rate of change of angular momentum.  Angular
momentum is conserved when no torque acts.

%%%%%%%%%%%%%%%%%%%%%%%%%%%%%%%%%%%%%%%%%%%%%%%%%%%%%%%%%%%%%%%%%%%%%%%%%%%%%
\section{Central Forces}

A force $\vect{F}$ which always acts along the position vector $\vect{r}$ is
called a \name{central force}.
$$\vect{F}=f(r)\rvect$$

%============================================================================
\begin{theorem}
The trajectory of a particle acted on by a central force lies in a plane.
\end{theorem}

\begin{proof}
At any instant $t$, the two vectors $\vect{r}(t)$ and $\dot{\vect{r}}(t)$
define a plane.  The acceleration is
$$\ddot{\vect{r}}=\frac{1}{m}\vect{F}=\frac{1}{m}f(r)\rvect$$
which lies in the plane.  Thus the particle's subsequent motion is confined
to that plane.
\end{proof}
%============================================================================

Therefore central forces imply motion in a plane.

%============================================================================
\begin{theorem}
The angular momentum of a particle in a central force field is constant.
\end{theorem}

\begin{proof}
The rate of change is angular momentum is
$$\der{\vect{L}}{t}=\vect{r}\times \vect{F}=\vect{r}\times f(r)\rvect=0$$
because $\vect{F}$ is parallel to $\vect{r}$.  Therefore $\vect{L}$ is
constant.
\end{proof}
%============================================================================

The physical significance of constant angular momentum is related to the
area swept out by the particle's position vector $\vect{r}(t)$.  Suppose
that $\vect{r}(t)$ sweeps out an area $A(t)$ as it moves in the plane.  In a
short time $\delta t$, the area swept out is
$$\delta A\approx\frac{1}{2}\left|\vect{r}(t+\delta t)\times\vect{r}(t)\right|$$
Now $\vect{r}(t+\delta t)=\vect{r}(t)+\delta\vect{r}$ so
$$\vect{r}(t+\delta t)\times\vect{r}(t)=\delta\vect{r}\times\vect{r}(t)$$
Therefore
$$\der{A}{t}=\lim_{\delta t\to 0}\frac{\delta A}{\delta t}
=\lim_{\delta t\to 0}\frac{1}{2}\left|\vect{r}\times\frac{\delta\vect{r}}
{\delta t}\right|=\frac{1}{2}\left|\vect{r}\times\dot{\vect{r}}\right|$$
Writing
$$\frac{1}{2}\left|\vect{r}\times\dot{\vect{r}}\right|
=\frac{1}{2m}\left|\vect{r}\times(m\dot{\vect{r}})\right|$$
shows that
$$\der{A}{t}=\frac{1}{2m}\left|\vect{L}\right|$$
Since angular momentum is constant in a central force field, we have
$$\der{A}{t}=\rm constant$$
This is Kepler's second rule of planetary motion (1609) which was explained
by Newton as a property of central forces in his {\em Principia} of 1687.

In polar coordinates, 
$$\dot{\vect{r}}=\dot{r}\rvect+r\dot{\theta}\tvect$$
so that
$$\vect{L}=\vect{r}\times (m\dot{\vect{r}})=m(r\rvect)\times
(\dot{r}\rvect+r\dot{\theta}\tvect)$$
Therefore 
$$\vect{L}=mr^2\dot{\theta}(\rvect\times\tvect)$$
Now $\rvect\times\tvect$ is the unit vector normal to the plane in which the
particle moves, so the magnitude of the angular momentum is
$$\left|\vect{L}\right|=mr^2\left|\dot{\theta}\right|$$
Substituting this back into the equation for $\der{A}{t}$ shows that
$$\der{A}{t}=\frac{1}{2}r^2\left|\dot{\theta}\right|$$

%%%%%%%%%%%%%%%%%%%%%%%%%%%%%%%%%%%%%%%%%%%%%%%%%%%%%%%%%%%%%%%%%%%%%%%%%%%%%
\section{Orbits in a Central Force Field}

The equation of motion is
$$m\ddot{\vect{r}}=f(r)\rvect$$
where the acceleration can be expressed as
$$\ddot{\vect{r}}=\left(\ddot{r}-r\dot{\theta}^2\right)\rvect
+\left(r\ddot{\theta}+2\dot{r}\dot{\theta}\right)\tvect$$
in polar coordinates.

The radial component is
$$\ddot{r}-r\dot{\theta}^2=\frac{f(r)}{m}$$
and the transverse component is
$$r\ddot{\theta}+2\dot{r}\dot{\theta}=0$$
This can be solved for $\dot{\theta}$ by writing it as
$$\dbd{t}\dot{\theta}+2\frac{\dot{r}}{r}\dot{\theta}=0$$
The integrating factor is $r^2$ which shows that
$$\dbd{t}\left(r^2\dot{\theta}\right)=0$$
so that $r^2\dot{\theta}$ is a constant.  Let this constant be $h$ so that
$$r^2\dot{\theta}=h$$
Note that $\frac{1}{2}\left|h\right|=\der{A}{t}=\frac{1}{2m}\left|\vect{L}
\right|$.

To obtain the equation of the orbit in the form $r(\theta)$, use
$$\dot{r}=\der{r}{\theta}\dot{\theta}=\frac{h}{r^2}\der{r}{\theta}$$
and
$$\ddot{r}=\frac{h}{r^2}\dbd{\theta}
\left(\frac{h}{r^2}\der{r}{\theta}\right)$$
The radial equation of motion can then be written as
$$\frac{h^2}{r^2}\dbd{\theta}\left(\frac{1}{r^2}\der{r}{\theta}\right)
-r\left(\frac{h}{r^2}\right)^2=\frac{f(r)}{m}$$
Rearranging this gives
$$\dbd{\theta}\left(\frac{1}{r^2}\der{r}{\theta}\right)
-\frac{1}{r}=\frac{r^2f(r)}{mh^2}$$
which can be solved to give $r(\theta)$.

%%%%%%%%%%%%%%%%%%%%%%%%%%%%%%%%%%%%%%%%%%%%%%%%%%%%%%%%%%%%%%%%%%%%%%%%%%%%%
\subsection{Motion in a Gravitational Field}

Let the central force be
$$f(r)=-\frac{GMm}{r^2}$$
where $M$ is the mass of the Sun at the origin and $m$ is the mass of the
body.  We will asssume that $M\gg m$ so that the Sun may be assumed to be
fixed at the origin.

The equation for the orbit becomes
$$\dbd{\theta}\left(\frac{1}{r^2}\der{r}{\theta}\right)
-\frac{1}{r}=-\frac{GM}{h^2}$$
Since 
$$\frac{1}{r^2}\der{r}{\theta}=-\dbd{\theta}\left(\frac{1}{r}\right)$$
this can be written
$$\ndbd{\theta}{2}\left(\frac{1}{r}\right)+\frac{1}{r}=\frac{GM}{h^2}$$
The general solution is
$$\frac{1}{r}=B\cos\theta+C\sin\theta+\frac{GM}{h^2}=A\cos(\theta-\theta_0)
+\frac{GM}{h^2}$$
where $A$ and $\theta_0$ are arbitrary constants giving the orbit's
amplitude and phase angle.  In terms of $r$, this is
$$r=\frac{1}{A\cos(\theta-\theta_0)+GM/h^2}$$
$A$ and $\theta_0$ are determined by the initial conditions.  Since
$\theta_0$ determines only the orientation of the orbit in the plane, we can
set it to zero when discussing the orbit's shape.

\typeout{There is no page 74, but my written notes don't have a break.}

Define the orbit's \name{eccentricity} to be
$$e=\frac{Ah^2}{GM}$$
and $r_0$ to be the \name{perihelion distance}
$$r_0=\frac{h^2/GM}{1+e}$$
which is the smallest distance between the body and the Sun during the
orbit.  Then the orbit equation can be written
$$r=r_0\frac{1+e}{1+e\cos\theta}$$
so that $r=r_0$ when $\theta=0$ and $r>r_0$ for all other $\theta$ during
each orbit.

The shape of the orbit depends on the eccentricity $e$.  If $e=0$, the orbit
is a circle, and for $0<e<1$, the orbit is an ellipse.  Both of these are
closed, periodic orbits.  When $e=1$, the orbit is a parabola and for $e>1$,
the orbit is a hyperbola.  Both of these are open, single encounter orbits.

%%%%%%%%%%%%%%%%%%%%%%%%%%%%%%%%%%%%%%%%%%%%%%%%%%%%%%%%%%%%%%%%%%%%%%%%%%%%%
\subsection{Cartesian Equation of an Orbit}

The orbit equation can be written
$$r(1+e\cos\theta)=r_0(1+e)$$
Writing $L=r_0(1+e)$ and putting $x=r\cos\theta$, we have
$$r=L-ex$$
Squaring both sides so that $r^2=x^2+y^2$,
$$x^2+y^2=(L-ex)^2$$
thus
$$(1-e^2)x^2+2Lex+y^2=L^2$$
Completing the square gives
$$\frac{(x-x_0)^2}{L^2/(1-e^2)^2}+\frac{y^2}{L^2/(1-e^2)}=1$$
where
$$x_0=-\frac{Le}{1-e^2}$$
When $e=0$, this simplifies to the equation of a circle
$$x^2+y^2=L^2$$
When $0<e<1$, this simplifies to the equation of an ellipse
$$\frac{(x-x_0)^2}{a^2}+\frac{y^2}{b^2}=1$$
where $a=L/(1-e^2)$ and $b=L/\sqrt{1-e^2}$.  

When  $e=1$, the equation becomes that of a parabola
$$y^2=L^2-2Lx$$
and when $e>1$, it becomes the equation of a hyperbola
$$\frac{(x-x_0)^2}{a^2}-\frac{y^2}{b^2}=1$$
where $a=L/(e^2-1)$ and $b=L/\sqrt{e^2-1}$.  

%%%%%%%%%%%%%%%%%%%%%%%%%%%%%%%%%%%%%%%%%%%%%%%%%%%%%%%%%%%%%%%%%%%%%%%%%%%%%
\subsection{Kepler's Laws}

Centuries of increasingly accurate astronomical observations culminated in
Kepler's laws of planetary motion (1609):

\begin{enumerate}
\item Each planet moves in an ellipse with the Sun as a focus.
\item The radius vector sweeps out equal areas in equal times 
($\der{A}{t}$ is constant).
\item The square of the period of revolution about the Sun is proportional
to the cube of the major axis of the orbit.
\end{enumerate}

Law 2 is a consequence of gravity being a central force.  Law 1 is a
consequence of gravity being an inverse square attractive force.

%%%%%%%%%%%%%%%%%%%%%%%%%%%%%%%%%%%%%%%%%%%%%%%%%%%%%%%%%%%%%%%%%%%%%%%%%%%%%
\subsection{Elliptical Planetary Orbits}

The equations for a planet's orbit are
$$r=r_0\frac{1+e}{1+e\cos\theta}$$
or
$$\frac{(x-x_0)^2}{a^2}+\frac{y^2}{b^2}=1$$
where $a=L/(1-e^2)$ and $b=L/\sqrt{1-e^2}$ (so $a>b$) and
$x_0=-Le/(1-e^2)$. 

%----------------------------------------------------------------------------
\begin{figure}\centering
\caption{Planets move in an ellipse around the Sun.  Their closest approach
is at perihelion and their furthest distance is at aphelion.}
\label{spd fig:pm}

\psset{unit=5cm}
\begin{pspicture}(-0.8,-0.5)(0.5,0.6)
% First the curve, so the axes show through
\psellipse[linecolor=gray,linewidth=2pt,plotstyle=curve](-0.2,0)(0.5,0.3)
% Now the axes and their labels
\psline{->}(-0.8,0)(0.4,0)
\psline{->}(0,-0.35)(0,0.4)
\uput[r](0.4,0){$x$}
\uput[u](0,0.4){$y$}
% Dashed lines
\psline[linecolor=black,linestyle=dashed]{-}(-0.7,-0.5)(-0.7,0) 
\psline[linecolor=black,linestyle=dashed]{-}(-0.2,-0.5)(-0.2,0.4) 
\psline[linecolor=black,linestyle=dashed]{-}(0.3,-0.5)(0.3,0) 
% Arrows and labels
\pcline{<->}(-0.7,-0.4)(0,-0.4)\Aput{$r_1$}
\pcline{<->}(0,-0.4)(0.3,-0.4)\Aput{$r_0$}
\pcline{<->}(-0.2,-0.5)(0.3,-0.5)\Bput{$a$}
\pcline{<->}(-0.2,0)(-0.2,0.3)\Aput{$b$}
\uput[dr](-0.2,0){$x_0$}
\pcline{*-*}(0,0)(0.2,0.18)\Aput{$r$}
\psarc{->}(0,0){0.14}{0}{42}
\uput[r](0.13,0.05){$\theta$}
% Words for aphelion and perihelion
\pnode(0.3,0){PER}
\rput[bl](0.3,0.3){\rnode{PERLAB}{perihelion}}
\ncline[nodesep=3pt]{->}{PERLAB}{PER}
\pnode(-0.7,0){AP}
\rput[br](-0.6,0.3){\rnode{APLAB}{aphelion}}
\ncline[nodesep=3pt]{->}{APLAB}{AP}
\end{pspicture}
\end{figure}
%----------------------------------------------------------------------------

This elliptic orbit is shown in figure \ref{spd fig:pm}.  The \name{aphelion
distance}, $r_1$, is related to the \name{perihelion distance}, $r_0$, 
according to
$$r_1=r_0\frac{1+e}{1-e}$$
The area of the ellipse is 
$$A=\pi ab=\frac{\pi L^2}{(1-e^2)^{3/2}}$$
The rate at which the ellipse's area is swept out is
$$\der{A}{t}=\frac{h}{2}=\frac{1}{2}r^2\dot{\theta}$$
The period $T$ of the orbit is thus
$$T=\frac{A}{\der{A}{t}}=\frac{2\pi L^2}{h(1-e^2)^{3/2}}$$
so that $T^2$ is
$$T^2=\frac{4\pi^2 L^4}{h^2(1-e^2)^3}=\frac{4\pi^2a^3L}{h^2}$$
where $a=L/(1-e^2)$.  But $L=h^2/GM$ so 
$$\frac{a^3}{T^2}=\frac{GM}{4\pi^2}$$
which is the same for all planets.  This is the explanation of Kepler's
third law given by Newton.

%%%%%%%%%%%%%%%%%%%%%%%%%%%%%%%%%%%%%%%%%%%%%%%%%%%%%%%%%%%%%%%%%%%%%%%%%%%%%
\subsection{Orbit in Terms of Perihelion Parameters}

From the definition of $r_0$,
$$e=\frac{h^2}{GMr_0}-1$$
At perihelion, $r=r_0$ and $\dot{r}=0$.  The speed at perihelion $v_0$ is
$$v_0=r_0\dot{\theta}_{\rm per}$$
But $h=r^2\dot{\theta}$ so at perihelion,
$$h=r_0^2\dot{\theta}_{\rm per}=r_0v_0$$
Therefore the eccentricity can be written
$$e=\frac{r_0v_0^2}{GM}-1$$
Define $v_c$ as the perihelion speed necessary to give a circular orbit
$$v_c=\sqrt{\frac{GM}{r_0}}$$
so that another way of writing the eccentricity is
$$e=\left(\frac{v_0}{v_c}\right)^2-1$$
Then the orbit equation can be written as
$$r=\frac{r_0\left(v_0/v_c\right)^2}
{1+\left[\left(v_0/v_c\right)^2-1\right]\cos\theta}$$
At aphelion, $r=r_1$ and $\theta=\pi$ so that
$$r_1=\frac{r_0\left(v_0/v_c\right)^2}
{2-\left(v_0/v_c\right)^2}$$

%============================================================================
\begin{example}
\problem
Find the speed of a satellite executing a circular orbit around the Earth.

\solution
Assuming that the Earth's gravitational field dominates over the Sun's field,
the satellite is moving in a central force field with
$$f(r)=-\frac{GM_em}{r^2}$$
where $M_e$ is the mass of the Earth.

The critical speed for a circular orbit at $r=r_0$ is
$$v_c=\sqrt{\frac{GM_e}{r_0}}$$
Note that $g=GM_e/R_e^2$ where $R_e$ is the Earth's radius so this
critical velocity can also be written
$$v_c=\sqrt{\frac{gR_e^2}{r_0}}$$
For a satellite close to the Earth, $r_0\approx R_e$ so
$$v_c\approx\sqrt{gR_e}=7920\ \rm ms^{-1}$$
\end{example}
%============================================================================

%============================================================================
\begin{example}
\problem
Calculate the escape velocity at perihelion from the solar system.

\solution
When $e=1$, the orbit is open and so the orbiting body may escape.  Since
$e=\left(v_0/v_c\right)^2-1$, $e=1$ when $v_0=\sqrt{2}v_c$. 
Therefore the escape velocity is
$$v_c=\sqrt{\frac{2gR_e^2}{r_0}}$$
\end{example}
%============================================================================

%============================================================================
\begin{example}
\problem
A satellite in a circular orbit $r_0$ fires its rocket to increase its speed
suddenly by a factor of $\alpha$.   Calculate the new apogee distance.

\solution
The old speed was
$$v_c=\sqrt{\frac{GM_e}{r_0}}$$
The new speed 
$$v_0=(1+\alpha)v_c$$
is the perihelion speed since the satellite will immediately move to
increase $r$.  The new orbit is
$$r=\frac{r_0\left(v_0/v_c\right)^2}
{1+\left[\left(v_0/v_c\right)^2-1\right]\cos\theta}
=\frac{r_0(1+\alpha)^2}{1+\left[(1+\alpha)^2-1\right]\cos\theta}$$
The apogee distance is
$$r=\frac{r_0(1+\alpha)^2}{2-(1+\alpha)^2}$$

Note that when $\alpha=0.15$, $r_1=1.95 r_0$.  When
$\alpha=\sqrt{2}-1\approx 0.4$, $r_1=\infty$ and the satellite escapes.
\end{example}
%============================================================================

%%%%%%%%%%%%%%%%%%%%%%%%%%%%%%%%%%%%%%%%%%%%%%%%%%%%%%%%%%%%%%%%%%%%%%%%%%%%%
\subsection{Inverse Cubic Attraction}


In general, the equation of orbital motion is
$$\ndbd{\theta}{2}\left(\frac{1}{r}\right)+\frac{1}{r}=-\frac{r^2f(r)}{mh^2}$$
When the attraction follows an inverse cubic law, 
$$f(r)=-\frac{k}{r^3}$$
the orbital motion with inverse cubic attraction is found by solving
$$\ndbd{\theta}{2}\left(\frac{1}{r}\right)+\frac{1}{r}
=\frac{k}{mh^2}\frac{1}{r}$$
which may be written as
$$\ndbd{\theta}{2}\left(\frac{1}{r}\right)
-\left(\frac{k}{mh^2}-1\right)\left(\frac{1}{r}\right)=0$$
When $k/mh^2>1$, the solution is
$$\frac{1}{r}=Ae^{-\alpha\theta}+Be^{\alpha\theta}$$
where 
$$\alpha=\sqrt{\frac{k}{mh^2}-1}$$
and $A$ and $B$ are determined by the initial conditions.
If $B>0$, $1/r\to Be^{\alpha\theta}\to\infty$ and the satellite
follows a path that spirals into the planet.
If $B<0$, $1/r$ will vanish (which means that $r\to\infty$) at some
finite value of $\theta$.
If $B=0$, $1/r=Ae^{-\alpha\theta}$ so $r\to\infty$ and the satellite
escapes from the planet.

When $k/mh^2=1$, the solution is
$$\frac{1}{r}=A+B\theta$$
If $B>0$, $1/r\to\infty$ as $\theta$ increases, so the satellite is
captured in a decaying spiralling orbit.
If $B<0$, $1/r\to 0$ as $\theta$ tends to some finite
$\theta_{\infty}$ and the satellite escapes.  If $B=0$, $1/r=A$ and
the satellite is in a closed circular orbit.

When $k/mh^2<1$, the solution is
$$\frac{1}{r}=A\cos\alpha\theta+B\sin\alpha\theta
=C\cos\alpha(\theta-\theta_0)$$
where $\alpha=\sqrt{1-k/mh^2}<1$, $C=\sqrt{A^2+B^2}$ and
$\tan\alpha\theta_0=B/A$.  Since $1/r\to0$ as
$\theta\to\theta_0+\pi/2\alpha$, the satellite escapes.

%%%%%%%%%%%%%%%%%%%%%%%%%%%%%%%%%%%%%%%%%%%%%%%%%%%%%%%%%%%%%%%%%%%%%%%%%%%%%
\section{Constrained Motion}

%%%%%%%%%%%%%%%%%%%%%%%%%%%%%%%%%%%%%%%%%%%%%%%%%%%%%%%%%%%%%%%%%%%%%%%%%%%%%
\subsection{Frictionless Constrained Motion}

%----------------------------------------------------------------------------
\begin{figure}\centering
\caption{This shows a bead moving on a smooth wire.  There is no friction, 
so the only forces on the bead are gravity $mg$ and the reaction $R$.}
\label{spd fig:bsw}

\psset{unit=5cm}
\begin{pspicture}(-0.1,-0.1)(1.3,0.8)
% First the curve, so the axes show through
\psplot[linecolor=gray,linewidth=2pt,plotstyle=curve,arrows=->]{0.3}{0.45}%
{1 1 x x mul sub sqrt sub}
\uput[ul](0.45,0.11){$s$}
\psplot[linecolor=gray,linewidth=2pt,plotstyle=curve]{0.25}{0.9}%
{1 1 x x mul sub sqrt sub}
% Now the axes and their labels
\psline{->}(0,0)(1.2,0)
\psline{->}(0,0)(0,0.6)
\uput[r](1.2,0){$x$}
\uput[u](0,0.6){$y$}
\psline[linecolor=black,linewidth=1pt,linestyle=dashed]{-}(0.414,0)(0.707,0.293)
\psarc{->}(0.414,0){0.2}{0}{45}
\uput[r](0.6,0.05){$\psi$}
\rput[B]{45}(0.707,0.293){
	\qdisk(0,0){3pt}
	\psline[linecolor=black]{->}(0,0)(0.2,0) 
	\uput[d](0.2,0){\rput[t]{*0}{$\Tvect$}}
	\psline[linecolor=black]{->}(0,0)(0,0.2) 
	\uput[u](0,0.2){\rput[b]{*0}{$\Nvect$}}
	\psline[linecolor=black]{->}(0,0)(0,0.15) 
	\uput[l](0,0.15){\rput[r]{*0}{$R$}}
}
% The weight
\pcline{->}(0.707,0.293)(0.707,0.1)
\Aput{$mg$}
\end{pspicture}
\end{figure}
%----------------------------------------------------------------------------

Suppose a bead of mass $m$ slides smoothly on a wire whose shape is given by
the function $\vect{r}(s)$.  From section \ref{vf sec:2D ic}, the bead's 
velocity and acceleration in intrinsic coordinates are
$$\dot{\vect{r}}=\dot{s}\Tvect$$
$$\ddot{\vect{r}}=\ddot{s}\Tvect+\dot{s}\dot{\psi}\Nvect$$
The intrinsic coordinates are illustrated in figure \ref{spd fig:bsw}.

For a smooth wire, the only force the wire can exert on the bead is a 
\name{reaction} that is \name{normal}, $R\Nvect$ ($R$ can be either
positive or negative).

The only other force is gravity, giving the total force
$$\vect{F}=-mg\jvect+R\Nvect$$
From Newton's second law, the equation of motion is
$$m\ddot{\vect{r}}=-mg\jvect+R\Nvect$$
The tangential component is
$$m\ddot{s}=-mg\Tvect\cdot\jvect$$
and the normal component is
$$m\dot{s}\dot{\psi}=-mg\Nvect\cdot\jvect+R$$
Since $\Tvect\cdot\jvect=\sin\psi$ and $\Nvect\cdot\jvect=\cos\psi$,
$$\ddot{s}=-g\sin\psi$$
and
$$\dot{s}\dot{\psi}=-g\cos\psi+\frac{R}{m}$$

Suppose the curve can be specified in terms of $\psi=\psi(s)$.  Then
$$\dot{\psi}=\der{\psi}{s}\dot{s}$$
where $\der{\psi}{s}$ is a known function of $s$.  Substituting this into
the normal equation of motion gives
$$R=m\left[\dot{s}^2\der{\psi}{s}+g\cos\psi\right]$$
Therefore, once $\dot{s}$ is known as a function of $s$, $R(s)$ is known at
any point $s$ on the wire.  The tangential component can then be written
$$\ddot{s}=\frac{1}{2}\dbd{s}\left(\dot{s}^2\right)=-g\sin\psi$$
Since $\psi(s)$ is specified, integrate the tangential equation with respect
to $s$ to obtain
$$\dot{s}^2=-2g\int_0^s\sin\psi(s)\,ds +\left.\dot{s}^2\right|_{s=0}$$
Once $\dot{s}(s)$ is known, we can integrate again to obtain $s(t)$.

Note that it may be more convenient to solve with $\psi$ rather than with
$s$ as the variable.  For example, the tangential equation can be written
$$\der{\dot{s}^2}{\psi}=\der{(\dot{s}^2)}{s}\der{s}{\psi}=-2g\sin\psi
\der{s}{\psi}$$
where $\der{s}{\psi}$ is a known function of $\psi$.

%----------------------------------------------------------------------------
\begin{figure}\centering
\caption{This shows a bead moving on a straight wire at an angle $\alpha$ to
the horizontal.  The distance $s$ is measured from the origin.}
\label{spd fig:bstw}

\psset{unit=5cm}
\begin{pspicture}(-0.25,-0.25)(1,1.1)
% First the curve, so the axes show through
\psline[linecolor=gray,linewidth=2pt]{-}(-0.1,-0.1)(0.8,0.8)
% Now the axes and their labels
\psline{->}(-0.1,0)(0.9,0)
\psline{->}(0,-0.1)(0,0.9)
\uput[r](0.9,0){$x$}
\uput[u](0,0.9){$y$}
\psarc{->}(0,0){0.2}{0}{45}
\uput[r](0.2,0.1){$\alpha$}
\rput[B]{45}(0.6,0.6){
	\qdisk(0,0){3pt}
	\pcline[linecolor=black]{->}(0,0)(0,0.2) 
	\uput[u](0,0.2){\rput[b]{*0}{$R$}}
}
% The weight
\psline{->}(0.6,0.6)(0.6,0.4)
\uput[d](0.6,0.4){$mg$}
% The label for s
\psline[linecolor=gray,linewidth=2pt]{->}(0.2,0.2)(0.4,0.4)
\uput[ul](0.4,0.4){$s$}
\end{pspicture}
\end{figure}
%----------------------------------------------------------------------------

%============================================================================
\begin{example}
\problem
Consider a wire at an angle $\alpha$ to the horizontal with the bead on the
wire released from rest at a height $H$, as in figure \ref{spd fig:bstw}.  
Calculate the equation of motion of the bead.

\solution
In this case, $\psi=\alpha$ and
$$\der{\dot{s}^2}{s}=-2g\sin\alpha$$
Integrate to show that
$$\dot{s}^2=-2gs\sin\alpha+C$$
The bead is released from a height $H$ so when $t=0$
$$s=\frac{H}{\sin\alpha}$$
and
$$\dot{s}=0$$
Therefore $C=2gH$ and
$$\dot{s}^2=2g\left(H-s\sin\alpha\right)$$
and the normal equation is
$$R=m\left(\dot{s}^2\der{\psi}{s}+g\cos\psi\right)=mg\cos\alpha$$
because $\psi$ is independent of $s$.

Taking the square root of the expression for $\dot{s}^2$ gives
$$\dot{s}=-\sqrt{2g(H-s\sin\alpha)}$$
where the negative root has been selected because the motion is in the
direction of $s$ decreasing.  This is a separable equation
$$\frac{ds}{\sqrt{H-s\sin\alpha}}=-\sqrt{2g}\,dt$$
Integrating gives
$$\frac{-2\sqrt{H-s\sin\alpha}}{\sin\alpha}=-\sqrt{2g}t+C$$
At $t=0$, $s=H/\sin\alpha$ so $C=0$ and
$$H-s\sin\alpha=\frac{gt^2}{2}\sin^2\alpha$$
which can be rearranged to give
$$s=\frac{H}{\sin\alpha}-\frac{1}{2}gt^2\sin\alpha$$
\end{example}
%============================================================================

%----------------------------------------------------------------------------
\begin{figure}\centering
\caption{For the particle sliding down the sphere in problem
\protect\ref{spd exam:psds}, place the top of the sphere at the origin.}
\label{spd fig:psds}

\psset{unit=5cm}
\begin{pspicture}(-0.25,-1.1)(1.2,0.35)
% First the curve, so the axes show through
\psplot[linecolor=gray,linewidth=2pt,plotstyle=curve]{0}{1}{1 x x mul sub 
sqrt 1 sub}
\psplot[linecolor=gray,linewidth=2pt,plotstyle=curve,arrows=->]{0}{0.25}{1 x x mul sub 
sqrt 1 sub}
% Now the axes and their labels
\psline{->}(-0.1,0)(1,0)
\psline{->}(0,-1)(0,0.1)
\uput[r](1,0){$x$}
\uput[u](0,0.1){$y$}
% Psi
\psline[linestyle=dashed]{-}(0.414,0)(0.707,-0.293)
\psarc{->}(0.414,0){0.08}{0}{315}
\uput[u](0.414,0.08){$\psi$}
% Phi
\pcline[linestyle=dashed]{-}(0,-1)(0.707,-0.293)
\Aput{$a$}
\psarcn{->}(0,-1){0.2}{90}{45}
\uput[u](0.07,-0.8){$\phi$}
% Forces at the particle
\rput[B]{-45}(0.707,-0.293){
	\qdisk(0,0){3pt}
	\psline[linecolor=black]{->}(0,0)(0,0.2) 
	\rput[b]{*0}(0,0.21){$R$}
	\psline[linecolor=black]{->}(0,0)(0.2,0) 
	\rput[l]{*0}(0.21,0){$\Tvect$}
}
% The weight
\psline{->}(0.707,-0.293)(0.707,-0.45)
\uput[d](0.707,-0.45){$mg$}
% The label for s
\uput[d](0.25,-0.05){$s$}
\end{pspicture}
\end{figure}
%----------------------------------------------------------------------------

%============================================================================
\begin{example}
\label{spd exam:psds}

\problem 
A particle slides down the surface of a sphere of radius $a$.  Initially,
the particle is at the top of the sphere and is moving so slowly that it is
nearly at rest.  When does the particle leave the sphere?

\solution
This is illustrated in figure \ref{spd fig:psds}, from which it can be 
seen that
$$\psi=2\pi-\phi$$
The tangential equation of motion is 
$$\dbd{s}(\dot{s}^2)=-2g\sin\psi=2g\sin\phi$$
The normal equation of motion is
$$\dot{s}\dot{\psi}=-g\cos\psi+\frac{R}{m}$$
which can be written in terms of $\phi$ as
$$\dot{s}\dot{\phi}=g\cos\phi-\frac{R}{m}$$

The equation of the surface is
$$s=a\phi$$
where $s$ is measured from the top.  This shows that
$$\dot{\phi}=\frac{\dot{s}}{a}$$
and 
$$R=m\left(g\cos\phi-\frac{\dot{s}^2}{a}\right)$$
The tangential equation becomes
$$\dbd{s}(\dot{s}^2)=2g\sin\phi=2g\sin\frac{s}{a}$$
Integrating gives
$$\dot{s}^2=-2ag\cos\frac{s}{a}+C$$
But when $t=0$, $s=0$ and $\dot{s}=0$ so that $C=2ag$.  Therefore
$$\dot{s}^2=2ag\left(1-\cos\frac{s}{a}\right)=2ag\left(1-\cos\phi\right)$$
Substituting this into the equation for the reaction $R$ shows that
$$R=m\left(g\cos\phi-\frac{2ag(1-\cos\phi)}{a}\right)=mg(3\cos\phi-2)$$

From this, the reaction is zero when 
$$R=mg(3\cos\phi-2)=0$$
which occurs when
$$\cos\phi=\frac{2}{3}$$
which is point at which the particle leaves the sphere.

Note that in problems like this, there is nothing to sustain negative
reactions, unlike the bead on the wire problem.
\end{example}
%============================================================================

%----------------------------------------------------------------------------
\begin{figure}\centering
\caption{This diagram shows a bead on a parabolic wire with equation
$y=-x^2/2L$.  It has been drawn upside down so that gravity points upwards.}
\label{spd fig:bopw}

\psset{unit=5cm}
\begin{pspicture}(-0.1,-0.1)(1.3,1.3)
% Plot the curve twice, one with an arrow so the direction of s is shown
\psplot[linecolor=gray,linewidth=2pt,plotstyle=curve,arrows=->]{0}{0.4}{x x mul}
\psplot[linecolor=gray,linewidth=2pt,plotstyle=curve]{0}{1}{x x mul}
\uput[ul](0.4,0.16){$s$}
% Now the axes and their labels
\psline{->}(0,0)(1.1,0)
\psline{->}(0,0)(0,1.1)
\uput[r](1.1,0){$x$}
\uput[u](0,1.1){$y$}
% The vectors at the point (x,y)
\psline[linecolor=black,linewidth=1pt,linestyle=dashed]{-}(0.3536,0)(0.707,0.5)
\psarc{->}(0.3536,0){0.2}{0}{54.7}
\uput[r](0.53,0.05){$\psi$}
\rput[B]{54.7}(0.707,0.5){
	\qdisk(0,0){3pt}
	\psline[linecolor=black]{->}(0,0)(0.2,0) 
	\uput[d](0.2,0){\rput[t]{*0}{$\Tvect$}}
	\psline[linecolor=black]{->}(0,0)(0,0.2) 
	\uput[u](0,0.2){\rput[b]{*0}{$\Nvect$}}
	\psline[linecolor=black]{->}(0,0)(0,0.15) 
	\uput[l](0,0.15){\rput[r]{*0}{$R$}}
}
% The weight
\psline{->}(0.707,0.5)(0.707,0.7)
\uput[u](0.707,0.7){$mg$}
\uput[r](0.9,0.81){$y=\ds\frac{x^2}{2L}$}
\end{pspicture}
\end{figure}
%----------------------------------------------------------------------------

%============================================================================
\begin{example}
\label{spd ex:bopw}
\problem
Find the motion of a bead under gravity starting with speed $u$ at the
origin, threaded on a frictionless wire that has been bent into a
parabolic shape
$$y=-\frac{x^2}{2L}$$

\solution
This is illustrated upside down (so gravity points upwards) in figure
\ref{spd fig:bopw}.  From this, the tangential equation of motion is
$$m\ddot{s}=mg\sin\psi$$
and the normal equation is
$$m\dot{s}\der{\psi}{s}=mg\cos\psi+R$$

To find the equation of the  wire in intrinsic coordinates, use
$$\der{s}{x}=\sqrt{1+\left(\der{y}{x}\right)^2}$$
where the positive square root has been chosen as $s$ increases with $x$.

Since $y=x^2/2L$ in the inverted diagram, 
$$\der{y}{x}=\frac{x}{L}=\tan\psi$$
Therefore
$$\der{x}{\psi}=\frac{L}{\cos^2\psi}$$
and
$$\der{s}{\psi}=\der{s}{x}\der{x}{\psi}=\sqrt{1+tan^2\psi}
\frac{L}{\cos^2\psi}=\frac{L}{\cos^3\psi}$$
The tangential equation of motion can be written as
$$\der{\dot{s}^2}{\psi}=\der{\dot{s}^2}{s}\der{s}{\psi}=
2\ddot{s}\der{s}{\psi}=2Lg\frac{\sin\psi}{\cos^3\psi}$$
Integrating gives
$$\dot{s}^2=2gL\int^{\psi}_0\frac{sin\psi}{\cos^3\psi}\,d\psi
+\left.\dot{s}^2\right|_{\psi=0}
=\left.\frac{gL}{\cos^2\psi}\right|_0^{\psi}+u^2$$
Therefore
$$\dot{s}^2=gL\left[\frac{1}{\cos^2\psi}-1\right]+u^2$$
Now 
$$\frac{1}{\cos^2\psi}-1=\tan^2\psi=\frac{x^2}{L^2}$$
so
$$\dot{s}^2=\frac{g}{L}x^2+u^2$$

From the normal equation of motion, 
\begin{eqnarray*}
R&=&m\dot{s}^2\der{\psi}{s}-mg\cos\psi \\
&=&m\left[gL\left(\frac{1}{\cos^2\psi}-1\right)+u^2\right]
\frac{\cos^3\psi}{L}-mg\cos\psi \\
&=&m(u^2-gL)\cos^3\psi
\end{eqnarray*}
Therefore the reaction is
$$R=\frac{m(u^2-gL)}{\left(1+x^2/L^2\right)^{3/2}}$$
Note that the reaction is positive for $u>\sqrt{gL}$ and negative for
$u<\sqrt{gL}$.
\end{example}
%============================================================================

%%%%%%%%%%%%%%%%%%%%%%%%%%%%%%%%%%%%%%%%%%%%%%%%%%%%%%%%%%%%%%%%%%%%%%%%%%%%%
\subsection{Friction}

%----------------------------------------------------------------------------
\begin{figure}\centering
\caption{The three forces on a stationary mass on a rough surface are the
reaction $R$, an applied force $f_{\rm app}$ and friction $F$.  The 
direction of the frictional force $F$ opposes the motion that
would occur if friction were absent.}
\label{spd fig:fr I}

\psset{unit=5cm}
\begin{pspicture}(-0.3,-0.1)(1.3,0.9)
% First the ground
\psline[linecolor=black,linewidth=2pt]{-}(0,0.03)(1,0.03)
% The box
\pspolygon[fillstyle=solid,linewidth=2pt,fillcolor=lightgray]%
(0.3,0.05)(0.7,0.05)(0.7,0.45)(0.3,0.45)
% The reaction
\psline{->}(0.5,0.07)(0.5,0.65)\uput[u](0.5,0.65){$R$}
% Friction
\psline{->}(0.3,0.05)(-0.1,0.05)\uput[l](-0.1,0.05){$F$}
% Applied force
\psline{<-}(0.3,0.35)(-0.1,0.35)\uput[l](-0.1,0.35){$f_{\rm app}$}
\end{pspicture}
\end{figure}
%----------------------------------------------------------------------------

\name{Friction} is a force that acts to oppose the relative motion of rough
surfaces.  As shown in figure \ref{spd fig:fr I}, $f_{\rm app}$ is the force
applied to the body, $R$ is the normal reaction force of the surface on the
body and $F$ is the frictional force on the body acting along the surface in
a direction which opposes the motion that would occur if friction were
absent.

\name{Static friction} occurs provided there is no motion, so that
$$F=f_{\rm app}$$
As $f_{\rm app}$ is increased, $F$ increases.  Experiments show that there
is a critical $f_{\rm app}$ at which the body starts to move, which is
proportional to the reaction $R$.  The friction force at this point can be
written
$$F=\mu_sR$$
where $\mu_s$ is the coefficient of limiting static friction.

%----------------------------------------------------------------------------
\begin{figure}\centering
\caption{The three forces on a stationary mass on a rough plane inclined at
an angle $\alpha$ are the reaction $R$, friction $F$ and weight $mg$.}
\label{spd fig:moi}

\psset{unit=5cm}
\begin{pspicture}(-0.3,-0.1)(1.3,0.8)
% The box
\rput[B]{30}(0.45,0.3773){
	\pspolygon[fillstyle=solid,fillcolor=lightgray]%
	(-0.15,-0.1)(0.15,-0.1)(0.15,0.1)(-0.15,0.1)
	\qdisk(0,0){3pt}
	\psline[linecolor=black]{->}(0,0)(0.3,0) 
	\uput[r](0.3,0){\rput[l]{*0}{$F$}}
	\psline[linecolor=black]{->}(0,0)(0,0.3) 
	\uput[u](0,0.3){\rput[b]{*0}{$R$}}
}
% First the ground
\SpecialCoor
\psline[linecolor=black,linewidth=2pt]{-}(1.1;30)(0;30)(1;0)
% The angle alpha
\psarc{->}(0,0){0.2}{0}{30}
\uput[r](0.2,0.05){$\alpha$}
% The weight vector
\psline{->}(0.45,0.3773)(0.45,0.1773)
\uput[d](0.45,0.1773){$mg$}
\end{pspicture}
\end{figure}
%----------------------------------------------------------------------------

%============================================================================
\begin{example}
For a stationary mass on an inclined plane, such as in figure 
\ref{spd fig:moi}, the forces must add to zero.  The tangential force
component is
$$F-mg\sin\alpha=0$$
and the normal force component is
$$R-mg\cos\alpha=0$$
At a critical angle $\alpha_c$ when the mass is just about to slip, we know
that
$$F=\mu_s R$$
Therefore, eliminating $R$ from the force equations at this point,
$$\mu_s=\tan\alpha_c$$

So if $\alpha<\alpha_c$, $F<\mu_sR$ and there is no motion.  If $\alpha\geq
\alpha_c$, the mass slides down the incline.
\end{example}
%============================================================================

Once motion has started, the frictional force still acts to oppose the
motion.  Experimentally, it is observed that
$$F=\mu_k R$$
for low speeds, where $\mu_k$ is the coefficient of \name{kinetic friction}.

Note that $\mu_k\leq\mu_s$.  Friction in problems involving rolling is
slightly different; this is dealt with later.

The kinetic friction acts to oppose the motion, so it can be described by
$$\vect{F}=-\mu_k\left|R\right|\hat{\dot{\vect{r}}}$$
where $\hat{\dot{\vect{r}}}$ is the unit vector in the direction of motion.


%----------------------------------------------------------------------------
\begin{figure}\centering
\caption{This shows one quarter of a rotating floor-polisher, which is
rotating with angular speed $\omega$ and moving with velocity $v$ in the 
$y$ direction.  The elemental unit of area $dA$ extends from $r$ to $r+dr$ 
and from $\theta$ to $\theta+d\theta$.}
\label{spd fig:fp}

\psset{unit=5cm}
\begin{pspicture}(-0.1,-0.1)(1.3,1.3)
\SpecialCoor
% The circular brush
\psarc[linecolor=gray,linewidth=2pt,plotstyle=curve]{-}(0,0){1}{-5}{95}
\uput[dr](1,0){$a$}
% Now the axes and their labels
\psline{->}(-0.1,0)(1.1,0)
\psline{->}(0,-0.1)(0,1.1)
\uput[r](1.1,0){$x$}
\uput[u](0,1.1){$y$}
% Dashed lines giving the elemental area
\psarc[linecolor=black,linewidth=1pt,linestyle=dashed]{-}(0,0){0.45}{0}{90}
\psarc[linecolor=black,linewidth=1pt,linestyle=dashed]{-}(0,0){0.55}{0}{90}
\psline[linecolor=black,linewidth=1pt,linestyle=dashed]{-}(0,0)(0.55;40)
\psline[linecolor=black,linewidth=1pt,linestyle=dashed]{-}(0,0)(0.55;50)
\uput[dr](1,0){$a$}
\uput[dr](0.55,0){$r+dr$}
\uput[dl](0.45,0){$\vphantom{d}r$}
\psarc{->}(0,0){0.2}{0}{40}
\uput[r](0.18,0.05){$\theta$}
\psarc{<-}(0,0){0.2}{50}{70}
\uput[ur](0.2;45){$d\theta$}
% The elemental area
\psarc[linecolor=black,linewidth=2pt]{-}(0,0){0.45}{40}{50}
\psarc[linecolor=black,linewidth=2pt]{-}(0,0){0.55}{40}{50}
\psline[linecolor=black,linewidth=2pt]{-}(0.45;40)(0.55;40)
\psline[linecolor=black,linewidth=2pt]{-}(0.45;50)(0.55;50)
% The vectors from the elemental area
\psarc{->}(0,0){0.5}{45}{65}
\psline{->}(0.3536,0.3536)(0.3536,0.55)
\uput[ul](0.5;65){$\omega r$}
\uput[u](0.3536,0.55){$v$}
\end{pspicture}
\end{figure}
%----------------------------------------------------------------------------

%============================================================================
\begin{example}
\problem
A floor polisher of weight $w$ has a circular brush of radius $a$ which is
rotating with angular speed $\omega$ and slides over a floor with speed $v$. 
Calculate the frictional force, where $\mu$ is the coefficient of sliding
friction.  

\solution
Consider the element of the brush between $r$ to $r+dr$ and $\theta$ to
$\theta+d\theta$, as shown in figure \ref{spd fig:fp}.  The area of this 
element is 
$$dA=r\,dr\,d\theta$$
Assume that the weight is distributed uniformly over all area elements, so
that the reaction force on this element is
$$R_e=\frac{w}{\pi a^2}r\,dr\,d\theta$$
The velocity of the element is
$$\dot{\vect{r}}=v\jvect+\omega r (-\sin\theta\ivect+\cos\theta\jvect)$$
and the unit vector in this direction is
$$\hat{\dot{\vect{r}}}=\frac{v\jvect+\omega r 
(-\sin\theta\ivect+\cos\theta\jvect)}
{\sqrt{v^2+2\omega rv\cos\theta+\omega^2r^2}}$$
The frictional force on the element is
$$-\frac{\mu w r\,dr\,d\theta}{\pi a^2}\hat{\dot{\vect{r}}}$$
The frictional force on the whole annulus from $r$ to $r+dr$ is found by
integrating the frictional force on the element for $\theta$ from $0$ to
$2\pi$
$$-\int_0^{2\pi}\hat{\dot{\vect{r}}}\,d\theta\,\frac{\mu wr\,dr}{\pi a^2}$$

This is difficult to solve exactly, so suppose that $\omega r\gg v$.  Then
$$\frac{1}{\sqrt{v^2+2\omega rv\cos\theta+\omega^2r^2}}\approx
\frac{1}{\omega r}\left[1-\frac{v\cos\theta}{\omega r}\right]$$
So the frictional force on the annulus is
$$-\jvect\frac{\mu wr\,dr}{a^2}\frac{v}{\omega r}$$
and the frictional force on the whole brush is
$$-\jvect\frac{\mu wv}{a\omega}$$

If the brush is not rotating, $\omega=0$, and the frictional force must be
$$\vect{F}_0=-\jvect\mu w$$
Therefore, when the brush is rotating, the friction can be written
$$\left|\vect{F}\right|=\left|\vect{F}_0\right|\frac{v}{aw}$$
which tends to zero as $\omega\to\infty$ and the polisher becomes
essentially frictionless as the rotational speed is increased.
\end{example}
%============================================================================

%%%%%%%%%%%%%%%%%%%%%%%%%%%%%%%%%%%%%%%%%%%%%%%%%%%%%%%%%%%%%%%%%%%%%%%%%%%%%
\subsection{Constrained Motion with Friction}

The equation of motion in intrinsic coordinates with friction is
$$m(\ddot{s}\Tvect+\dot{s}\dot{\psi}\Nvect)=m\vect{g}+R\Nvect-F\Tvect$$
where the friction term is
$$F=\mu\left|R\right|\frac{\dot{s}}{\left|\dot{s}\right|}$$

%----------------------------------------------------------------------------
\begin{figure}\centering
\caption{This shows a bead on a rough parabolic wire, so friction $F$
opposes the direction of motion $s$.  As in figure \protect\ref{spd
fig:bopw}, the diagram has been drawn upside down so that gravity points 
upwards.}
\label{spd fig:bopwf}

\psset{unit=5cm}
\begin{pspicture}(-0.1,-0.1)(1.3,1.3)
% Plot the curve twice, one with an arrow so the direction of s is shown
\psplot[linecolor=gray,linewidth=2pt,plotstyle=curve,arrows=->]{0}{0.4}{x x mul}
\psplot[linecolor=gray,linewidth=2pt,plotstyle=curve]{0}{1}{x x mul}
\uput[ul](0.4,0.16){$s$}
% Now the axes and their labels
\psline{->}(0,0)(1.1,0)
\psline{->}(0,0)(0,1.1)
\uput[r](1.1,0){$x$}
\uput[u](0,1.1){$y$}
% The vectors at the point (x,y)
\psline[linecolor=black,linewidth=1pt,linestyle=dashed]{-}(0.3536,0)(0.707,0.5)
\psarc{->}(0.3536,0){0.2}{0}{54.7}
\uput[r](0.53,0.1){$\psi$}
\rput[B]{54.7}(0.707,0.5){
	\qdisk(0,0){3pt}
	\psline[linecolor=black]{->}(0,0)(0.2,0) 
	\uput[d](0.2,0){\rput[t]{*0}{$\Tvect$}}
	\psline[linecolor=black]{->}(0,0)(-0.2,0) 
	\uput[d](-0.2,0){\rput[t]{*0}{$F$}}
	\psline[linecolor=black]{->}(0,0)(0,0.2) 
	\uput[u](0,0.2){\rput[b]{*0}{$\Nvect$}}
	\psline[linecolor=black]{->}(0,0)(0,0.15) 
	\uput[l](0,0.15){\rput[r]{*0}{$R$}}
}
% The weight
\psline{->}(0.707,0.5)(0.707,0.7)
\uput[u](0.707,0.7){$mg$}
\uput[r](0.9,0.81){$y=\ds\frac{x^2}{2L}$}
\end{pspicture}
\end{figure}
%----------------------------------------------------------------------------

%============================================================================
\begin{example}
\problem
Consider the bead on the parabolic wire from example \ref{spd ex:bopw} where
the wire is no longer smooth, so friction must be included.  This is
illustrated in figure \ref{spd fig:bopwf}.  Calculate the bead's equation of
motion.

\solution
Recall from example \ref{spd ex:bopw} that
$$\der{s}{\psi}=\frac{L}{\cos^3\psi}$$
and 
$$1+\left(\frac{x}{L}\right)^2=\frac{1}{\cos^2\psi}$$

Then tangential equation of motion is
$$m\ddot{s}=mg\sin\psi-F$$
and the normal equation of motion is
$$m\dot{s}\dot{\psi}=mg\cos\psi+R$$

Assume that the initial speed is sufficiently small that $R<0$ at the start. 
We can then write
$$F=-\mu R=-\mu(m\dot{s}\dot{\psi}-mg\cos\psi)$$
Therefore
$$\frac{1}{2}\dbd{s}(\dot{s}^2)=g\sin\psi+\mu(\dot{s}\dot{\psi}-g\cos\psi)$$
so that
$$\dbd{s}(\dot{s}^2)-2\mu\der{\psi}{s}(\dot{s}^2)=2g(\sin\psi-\mu\cos\psi)$$
This is more conveniently expressed with respect to $\psi$ rather than $s$. 
So multiply by $\der{s}{\psi}$ to give
$$\dbd{\psi}(\dot{s}^2)-2\mu(\dot{s}^2)=2g\der{s}{\psi}
(\sin\psi-\mu\cos\psi)$$
The integrating factor is
$$I(\psi)=e^{-2\mu\psi}$$ 
so that
$$\dbd{\psi}\left(e^{-2\mu\psi}\dot{s}^2\right)=
\frac{2gLe^{-2\mu\psi}}{\cos^3\psi}(\sin\psi-\mu\cos\psi)$$
which gives
$$\dot{s}^2e^{-2\mu\psi}=u^2+2gL\int_0^{\psi}
\frac{e^{-2\mu\psi}}{\cos^3\psi}(\sin\psi-\mu\cos\psi)\,d\psi$$
Therefore $\dot{s}^2$ is 
$$\dot{s}^2=u^2e^{2\mu\psi}+2gLe^{2\mu\psi}\int_0^{\psi}
\frac{e^{-2\mu\psi}}{\cos^3\psi}(\sin\psi-\mu\cos\psi)\,d\psi$$
which is substantially harder to solve than when there was no friction.
\end{example}
%============================================================================

%----------------------------------------------------------------------------
\begin{figure}\centering
\caption{A light bead on a rough wire which is rotating in a horizontal
plane.}
\label{spd fig:lbrrw}

\psset{unit=5cm}
\begin{pspicture}(-0.25,-0.25)(1,1.1)
% First the curve, so the axes show through
\psline[linecolor=gray,linewidth=2pt]{-}(-0.1,-0.1)(0.8,0.8)
% Now the axes and their labels
\psline{->}(-0.1,0)(0.9,0)
\psline{->}(0,-0.1)(0,0.9)
\uput[r](0.9,0){$x$}
\uput[u](0,0.9){$y$}
\psarc{->}(0,0){0.2}{0}{45}
\uput[r](0.2,0.1){$\theta$}
\rput[B]{45}(0.6,0.6){
	\qdisk(0,0){3pt}
	\psline[linecolor=black]{->}(0,0)(0.2,0) 
	\uput[d](0.2,0){\rput[t]{*0}{$\rvect$}}
	\psline[linecolor=black]{->}(0,0)(-0.2,0) 
	\uput[d](-0.2,0){\rput[t]{*0}{$F$}}
	\psline[linecolor=black]{->}(0,0)(0,0.2) 
	\uput[u](0,0.2){\rput[b]{*0}{$\tvect$}}
	\psline[linecolor=black]{->}(0,0)(0,0.15) 
	\uput[l](0,0.15){\rput[r]{*0}{$R$}}
}
\psarc{->}(0,0){1.2}{35}{55}
\uput[r](0.9,0.9){$\dot{\theta}=\omega$}
\end{pspicture}
\end{figure}
%----------------------------------------------------------------------------

%============================================================================
\begin{example}
\problem
Consider a light bead on a rough wire that is rotating horizontally, as
shown in figure \ref{spd fig:lbrrw}.  Since the bead is light, its weight 
can be ignored.  Calculate the bead's position on the wire, $r(t)$.

\solution
The equations of motion in polar coordinates are
$$m(\ddot{r}-r\dot{\theta}^2)=-F$$
and
$$m(r\ddot{\theta}+2\dot{r}\dot{\theta})=R$$
Since the wire is rotating with angular speed $\omega$ and the bead is
constained to the wire, 
$$\dot{\theta}=\omega$$
and
$$\ddot{\theta}=0$$
Thus
$$m(\ddot{r}-r\omega^2)=-F$$
and
$$m(2\dot{r}\omega)=R$$

To find the critical rotation rate for the bead to move, at the point of
incipient motion
$$F=\mu_sR$$ 
with $\dot{r}=\ddot{r}=0$.  Then from the equations of motion, initially
$R=0$ and therefore $F=0$ initially.  

For any $\omega$, the initial acceleration is
$$\ddot{r}=r\omega^2$$
Friction only starts to act once $\dot{r}$, and hence $R$, becomes non-zero.
Therefore
$$\omega_{\rm crit}=0$$
although this has to be modified if the weight of the bead is included.


When $\omega\neq 0$ and the bead is sliding, the transverse equation shows 
that
$$F=\mu_kR=2m\omega\mu_k\dot{r}$$
The radial equation is
$$\ddot{r}-r\omega^2=-\frac{F}{m}=-2\omega\mu_k\dot{r}$$
which is
$$\ddot{r}+2\omega\mu_k\dot{r}-\omega^2r=0$$
Try $r=e^{\alpha t}$ in the differential equation to get
$$\alpha^2+2\omega\mu_k\alpha-\omega^2=0$$
which can be expressed as
$$\left(\frac{\alpha}{\omega}\right)^2+2\mu_k\left(\frac{\alpha}{\omega}
\right)-1=0$$
Thus
$$\alpha=\omega\left(\mu_k\pm\sqrt{\mu_k^2+1}\right)$$
This gives the two solutions
$$\alpha_1=-\omega\left(\mu_k+\sqrt{\mu_k^2+1}\right)<0$$
and
$$\alpha_2=\omega\left(\sqrt{\mu_k^2+1}-\mu_k\right)>0$$
so 
$$r(t)=Ae^{\alpha_1t}+Be^{\alpha_2t}$$
At $t=0$, $r=r_0$ and $\dot{r}=0$ so that
$$r_0=A+B\qquad\mbox{and}\qquad 0=\alpha_1A+\alpha_2B$$
Therefore
$$A=-\frac{\alpha_2}{\alpha_1}\left(\frac{r_0}{1-\alpha_2/\alpha_1}
\right)>0$$
and
$$B=\frac{r_0}{1-\alpha_2/\alpha_1}>0$$
This gives the solution
$$r(t)=\frac{r_0}{1-\alpha_2/\alpha_1}\left(
-\frac{\alpha_2}{\alpha_1}e^{\alpha_1t}+e^{\alpha_2t}
\right)$$
\end{example}
%============================================================================

