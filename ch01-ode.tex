%%%%%%%%%%%%%%%%%%%%%%%%%%%%%%%%%%%%%%%%%%%%%%%%%%%%%%%%%%%%%%%%%%%%%%%%%%%
%
%			Mathematics 132 Course Notes
%
%			 Department of Mathematics,
%   			  University of Melbourne
%
%		Stephen Simmons			Lee White
%
% 8 Feb-96 SS: Updated with corrections from semester 2, 1995
%
%%%%%%%%%%%%%%%%%%% Copyright (C) 1995-96 Stephen Simmons %%%%%%%%%%%%%%%%%

\chapter{Ordinary Differential Equations}
\label{ode chp}

An equation of the form
$$f\left(x,y,\der{y}{x},\dder{y}{x},\ldots,\nder{y}{x}{n}\right)=0$$
involving $y(x)$ and its derivatives is an \name{ordinary differential
equation} (\ODE).  The \name{order} of an \ODE is the order of the
highest derivative that appears.  An \ODE is \name{linear} if it can be
written in the form
$$a_n(x)\nder{y}{x}{n}+a_{n-1}(x)\nder{y}{x}{n-1}+\cdots+a_1(x)\der{y}{x}
+a_0(x)y=F(x)$$
A \name{solution} of an \ODE is a function $y(x)$ which when substituted
into the \ODE renders it an identity.  An \name{integral} of an \ODE is
an implicit relationship connecting $y$ and $x$ which when substituted
into the \ODE  renders it an identity (this is sometimes called an
\name{implicit solution}).

%============================================================================
\begin{example}
An integral or implicit solution of the \ODE
$$\der{y}{x}=-\frac{x}{y}$$
is
$$y^2+x^2=C$$
\end{example}
%============================================================================

%%%%%%%%%%%%%%%%%%%%%%%%%%%%%%%%%%%%%%%%%%%%%%%%%%%%%%%%%%%%%%%%%%%%%%%%%%%%%
\section{First Order O.D.E.s}
\label{ode sec:foode}

The general \name{first order \ODE} is written
$$\der{y}{x}=f(x,y)$$
Note that $F(x,y,\der{y}{x})=0$ may not be able to be inverted to obtain
$\der{y}{x}$ in terms of $x$ or $y$, or there may be a multiplicity of
solutions, for example, as occurs in
$$\left(\der{y}{x}\right)^2-g(x,y)=0$$

The \name{initial value problem} is to find the solution of an \ODE
subject to the \name{initial condition}
$$y(x_0)=y_0$$

%%%%%%%%%%%%%%%%%%%%%%%%%%%%%%%%%%%%%%%%%%%%%%%%%%%%%%%%%%%%%%%%%%%%%%%%%%%%%
\subsection{Separable Equations}
\label{ode sec:sep}

An \ODE of the form
$$\der{y}{x}=\frac{g(x)}{h(y)}$$
is called \name{separable}.  Separable \ODEs are solved by considering the 
indefinite integral
$$G(x)=\int^x g(x)\,dx$$
From the fundamental theorem of calculus,
$$\der{G}{x}=g(x)$$
Similarly, the function 
$$H(x)=\int^x h(x)\,dx$$
obeys
$$\der{H}{x}=h(x)$$
Thus, using chain rule,
$$\dbd{x}H\left(y(x)\right)=\der{H}{y}\,\der{y}{x}=h(y)\,\der{y}{x}=g(x)$$
Using $g(x)=\der{G}{x}$ shows that
$$\dbd{x}H\left(y(x)\right)=\dbd{x} G(x)$$
Integrating both sides with respect to $x$ shows that
$$H\left(y(x)\right)=G(x)+C$$
Therefore the general solution is
$$\int^y h(y)\,dy =\int^x g(x)\,dx +C$$
This is in the form of an integral or implicit solution of the \ODE.

%============================================================================
\begin{example}
For these examples, $h(y)=1$ so that the separable equation has the form
$$\der{y}{x}=g(x)$$

\begin{eqnarray*}
\der{y}{x}=x^n		&\mif&	y(x)=\frac{x^{n+1}}{n+1}+C	\\
\der{y}{x}=\sin ax	&\mif&	y(x)=-\frac{1}{a}\cos ax+C	\\
\der{y}{x}=B\,e^{ax}	&\mif&	y(x)=\frac{B}{a}\,e^{ax}+C	\\
\der{y}{x}=\tan ax	&\mif&	y(x)=-\frac{1}{a}\ln\left|\cos ax\right|+C
\end{eqnarray*}
These are all examples of \name{explicit solutions} because the solution
$y(x)$ is given as a function of $x$.
\end{example}
%============================================================================

Note that $C$, the \name{constant of integration}, is determined by the
problem's initial conditions.

%============================================================================
\begin{example}
To solve
$$\der{y}{x}=\frac{y}{1+x}$$
note that $h(y)=1/y$ and $g(x)=1/(1+x)$.  Then
$$\frac{dy}{y}=\frac{dx}{1+x}$$
Integrating both sides gives
$$\int^y \frac{dy}{y}=\int^x\frac{dx}{1+x}+C$$
which means that
$$\ln\left|y\right|=\ln\left|1+x\right|+C$$
Then by writing $C=\ln\left|A\right|$ and removing the logs, the solution is
$$y=A(1+x)$$
\end{example}
%============================================================================

%============================================================================
\begin{example}
To solve
$$\der{y}{x}=-\frac{x}{y}$$
rewrite the \ODE as
$$y\,dy=-x\,dx$$
Integrating both sides gives the solution
$$y^2=-x^2+C$$
which is the equation for a circle of radius $\sqrt{C}$
$$x^2+y^2=C$$
\end{example}
%============================================================================

%============================================================================
\begin{example}
To solve
$$\der{y}{x}=y^2-a^2$$
rewrite it as
$$\frac{dy}{y^2-a^2}=dx$$
Integrating both sides gives
$$\int^y\frac{dy}{y^2-a^2}=x+C$$
Using partial fractions, the integrand is equal to
$$\frac{1}{y^2-a^2}=\frac{1}{2a}\left(\frac{1}{y-a}-\frac{1}{y+a}\right)$$
Therefore
\begin{eqnarray*}
\int^y\frac{dy}{y^2-a^2}
&=&\frac{1}{2a}\int^y \frac{1}{y-a}-\frac{1}{y+a}\,dy	\\
&=&\frac{1}{2a}\ln\left|\frac{y-a}{y+a}\right|		\\
&=&x+C
\end{eqnarray*}
Now writing $\left|A\right|=e^{2aC}$ gives the solution
$$\frac{y-a}{y+a}=A\,e^{2ax}$$
which as a function $y(x)$ is
$$y=a\frac{1+A\,e^{2ax}}{1-A\,e^{2ax}}$$
\end{example}
%============================================================================


%============================================================================
\begin{example}
The \name{Doomsday Model} is
$$\der{P}{t}=kP$$
with initial condition $P(0)=P_0$.  The general solution is
\begin{eqnarray*}
\int^P\frac{dP}{P}&=&k\int^t dt +C	\\
\thus\ln P&=&kt+C
\end{eqnarray*}
At $t=0$, the initial condition can be used to determine $C$
$$\ln P_0=0+C$$
so that
$$\ln\frac{P}{P_0}=kt$$
or
$$P(t)=P_0\,e^{kt}$$
\end{example}
%============================================================================

%============================================================================
\begin{example}
The \name{logistic equation} is
$$\der{P}{t}=kP(1-bP)$$
with initial condition $P(0)=P_0$.  The solution is found using partial
fractions
\begin{eqnarray*}
\int^P\frac{dP}{P(1-bP)}&=&k\int^t dt +C	\\
\thus \ln \frac{P}{1-bP}&=&kt+C
\end{eqnarray*}
At $t=0$, the initial condition can be used to determine $C$, giving the
solution
$$\frac{P(t)}{1-bP(t)}=\frac{P_0}{1-bP_0}\,e^{kt}$$
which can easily be rearranged to give an explicit solution for $P(t)$ in
terms of $t$.
\end{example}
%============================================================================

%============================================================================
\begin{exercise}
Exercise 2.2 of Zill, pp. 44--46, has more examples of separable \ODEs.
\end{exercise}
%============================================================================

%%%%%%%%%%%%%%%%%%%%%%%%%%%%%%%%%%%%%%%%%%%%%%%%%%%%%%%%%%%%%%%%%%%%%%%%%%%%%
\subsection{Homogeneous Equations}

If a function $f(x,y)$ satisfies
$$f(tx,ty)=t^n\,f(x,y)$$
for all $x$ and $y$ and for some constant real number $n$, then $f(x,y)$ is
said to be a \name{homogeneous} function of degree $n$.

%============================================================================
\begin{example}
Here are some examples of functions that are homogeneous or inhomogeneous:
\begin{eqnarray*}
ax^2+bxy+cy^2 		&&  \mbox{homogenous ($n=2$)}	\\
\ds\frac{ax+by}{cx+dy}	&&  \mbox{homogenous ($n=0$)}	\\
\cos(ax^2+by^2)		&&  \mbox{inhomogenous}	\\
\exp\left(\ds\frac{ax^2+by^2}{cx^2+dy^2}\right)	
			&&  \mbox{homogenous ($n=0$)}	
\end{eqnarray*}
\end{example}
%============================================================================

An \ODE of the form
$$N(x,y)\,\der{y}{x}+M(x,y)=0$$
where $M(x,y)$ and $N(x,y)$ are homogeneous functions of the same degree $n$
is called a \name{homogeneous} \ODE.  Homogeneous \ODEs can be turned into 
separable \ODEs with the substitution
$$u=\frac{y}{x}$$
When using this substitution, note what happens to $M(x,y)$, $N(x,y)$ and
$\der{y}{x}$:
$$M(x,y)=M(x,xu)=x^n\,M(1,u)$$
$$N(x,y)=N(x,xu)=x^n\,N(1,u)$$
$$\ds\der{y}{x}=\ds\der{(xu)}{x}=x\ds\der{u}{x}+u$$
Now, substituting these into the \ODE, we have
$$0=x^n\,M(1,u)+x^n\,N(1,u)\left(x\der{u}{x}+u\right)$$
which can be rearranged to give
$$\left(\frac{N(1,u)}{M(1,u)+uN(1,u)}\right)\der{u}{x}=-\frac{1}{x}$$
which is a separable \ODE in $u$ and $x$.

Note that sometimes it may be simpler to use the alternative substitution
$$v=\frac{x}{y}$$
which gives a different separable equation.

Never try to remember these final equations; learn the principle and
rederive each time starting with the substitution $u=y/x$ or $v=x/y$.

%============================================================================
\begin{example}
To solve
$$\der{y}{x}=\frac{x+3y}{3x+y}=\frac{1+3y/x}{3+y/x}$$
put $u=y/x$ so that $y=xu$ and
$$x\,\der{u}{x}+u=\der{y}{x}$$
Therefore
$$x\,\der{u}{x}+u=\frac{1+3u}{3+u}$$
so that
$$x\,\der{u}{x}=\frac{1+3u}{3+u}-u=\frac{1-u^2}{3+u}$$
Separating the terms involving $x$ and $u$,
$$\int^u\frac{3+u}{1-u^2}\,du=\int^x\frac{dx}{x}+C$$
Using partial fractions,
\begin{eqnarray*}
\int^u \frac{1}{1+u}+\frac{2}{1-u}\,du&=&\ln\left|x\right|+C\\
\thus \ln\left|\frac{1+u}{(1-u)^2}\right|&=&\ln\left|x\right|+C
\end{eqnarray*}
Writing the constant of integration as $C=\ln\left|A\right|$ gives the solution
$$\frac{1+u}{(1-u)^2}=Ax$$
\end{example}
%============================================================================

%============================================================================
\begin{exercise}
Exercise 2.3 of Zill, pp. 52--53, has more examples of homogeneous \ODEs.
\end{exercise}
%============================================================================

%%%%%%%%%%%%%%%%%%%%%%%%%%%%%%%%%%%%%%%%%%%%%%%%%%%%%%%%%%%%%%%%%%%%%%%%%%%%%
\subsection{Linear First Order O.D.E.s}

An \ODE of the form
$$\der{y}{x}+p(x)\,y=f(x)$$
is a \name{linear} first order \ODE.  When $f(x)=0$, the \ODE is
\name{homogeneous}.  When $f(x)\neq 0$, the \ODE is \name{inhomogeneous}.

One \name{trivial solution} of an homogeneous linear \ODE is always
$y=0$.  This usually need not be considered since the initial conditions will
usually be non-zero.

To solve a linear first order \ODE, use the \name{integrating factor},
$I(x)$.  $I(x)$ is defined to be any solution of 
$$\der{I}{x}=p(x)\,I$$
This is a separable equation which is solved using
$$\int^I\frac{dI}{I}=\int^x p(x)\,dx+C$$
so that
$$\ln I=\int^x p(x)\,dx+C$$
Set $C=0$ so that the integrating factor is
$$I(x)=\exp\left(\int^x p(x)\,dx\right)$$
Note that $y=1/I(x)$ is the solution of the homogeneous equation
$$\der{y}{x}+p(x)\,y=0$$
To prove this, substitute $y=1/I(x)$ so that
$$\der{y}{x}=-\frac{1}{I^2}\,\der{I}{x}$$
Now substituting this into the homogeneous equation shows that
$$\der{y}{x}+py=-\frac{1}{I^2}\left(\der{I}{x}-pI\right)=0$$
where the term in parentheses is zero from the definition of the integrating
factor.

Now, to complete the solution of the \ODE
$$\der{y}{x}+p(x)\,y=f(x)$$
multiply both sides by $I(x)$ to give
$$I(x)\,\der{y}{x}+p(x)I(x)\,y=I(x)f(x)$$
But from the definition of the integrating factor, $p(x)I(x)=\der{I}{x}$, 
so that
$$I(x)\,\der{y}{x}+\der{I}{x}\,y=I(x)f(x)$$
Using product rule, the left hand side is seen to be 
$\dbd{x}\left(Iy\right)$, hence
$$\dbd{x}\left[I(x)\,y\right]=I(x)f(x)$$
Integrating to remove the derivative gives the solution
$$I(x)\,y=\int^x I(x)f(x)\,dx+C$$
which is
$$y=C\frac{1}{I(x)}+\frac{1}{I(x)}\int^x I(x)f(x)\,dx$$
The first part of this solution is $C$ times the solution of the homogeneous
\ODE, and the second part is a \name{particular solution} which exists
because $f(x)\neq 0$.


%============================================================================
\begin{example}
To solve the \ODE
$$x^2\,\der{y}{x}+xy=1$$
divide by $x^2$ to turn it into standard form
$$\der{y}{x}+\frac{1}{x}\,y=\frac{1}{x^2}$$

The integrating factor is 
$$I(x)=x$$
found by solving
$$\der{I}{x}=\frac{1}{x}\,I$$
and setting the constant in
$$\int^x \frac{dI}{I}=\int^x\frac{dx}{x}+C$$
to zero.

Now solve the \ODE by multiplying it by the integrating factor
$$x\,\der{y}{x}+y=\frac{1}{x}$$
Using product rule, the left hand side is
$$x\,\der{y}{x}+y=\dbd{x}\left(xy\right)$$
so that 
$$xy=\ln\left|x\right|+C$$
Therefore the solution is
$$y(x)=\frac{\ln\left|x\right|}{x}+\frac{C}{x}$$
\end{example}
%============================================================================

%============================================================================
\begin{example}
To solve the \ODE
$$\left(1-\cos x\right)\der{y}{x}+2y\sin x-\tan x=0$$
put it in standard form, so that
$$\der{y}{x}+\left(\frac{2\sin x}{1-\cos x}\right)y=\frac{\tan x}{1-\cos x}$$
To find the integrating factor, solve $\der{I}{x}=p(x) I$.
\begin{eqnarray*}
\int^I\frac{dI}{I}&=&\int^x\frac{2\sin x}{1-\cos{x}}\,dx	\\
\thus \ln I&=&2\ln(1-\cos x)
\end{eqnarray*}
where the constant of integration has been set to zero.  Therefore
$$I(x)=\left(1-\cos x\right)^2$$

Now, solve the \ODE by multiplying by the integrating factor.  This gives
$$\dbd{x}(Iy)=(1-\cos x)^2\frac{\tan x}{1-\cos x}=(1-\cos x)\tan x$$
Integrate with respect to $x$
\begin{eqnarray*}
Iy&=&\int^x(1-\cos x)\tan x\,dx+C \\
&=&\int^x (\tan x -\sin x)\,dx+C \\
&=&-\int^x \frac{d(\cos x)}{\cos x}+\cos x+C \\
&=&-\ln\left|\cos x\right|+\cos x+C
\end{eqnarray*}
Dividing both sides by $I(x)=(1-\cos x)^2$ gives the solution
$$y(x)=\frac{C}{(1-\cos x)^2}
+\frac{\cos x-\ln\left|\cos x\right|}{(1-\cos x)^2}$$
which is of the form of $C$ times the homogeneous solution plus the
particular solution due to $f(x)$.
\end{example}
%============================================================================

%============================================================================
\begin{example}
Solving
$$\der{y}{x}=\frac{y}{y^3+x}$$
is a little harder because the \ODE is neither linear, homogeneous nor 
separable.  But by inverting it and rearranging, it becomes a linear \ODE
in $y$
$$\der{x}{y}=\frac{y^3+x}{y}\mif\der{x}{y}-\frac{1}{y}\,x=y^2$$
The solution gives $x$ as a function of $y$.  The integrating factor $I(y)$ 
satisfies
$$\der{I}{y}=-\frac{1}{y}I$$
so that
$$I(y)=\frac{1}{y}$$
Multiply both sides of the \ODE by $I(y)$ and use product rule on the
left hand side to give
$$\dbd{y}\left(\frac{x}{y}\right)=y$$
Integrate with respect to $y$ 
$$\frac{x}{y}=\int^y y\,dy+C$$
to show that the solution is
$$x=Cy+\frac{y^3}{2}$$
Once again, this takes the form of $C$ times the homogeneous solution 
plus the particular solution due to $f(x)$.
\end{example}
%============================================================================

%============================================================================
\begin{exercise}
Exercise 2.5 of Zill, pp. 69--71, has more examples of linear \ODEs.
\end{exercise}
%============================================================================

%%%%%%%%%%%%%%%%%%%%%%%%%%%%%%%%%%%%%%%%%%%%%%%%%%%%%%%%%%%%%%%%%%%%%%%%%%%%%
\section{Applications of First Order O.D.E.s}


%----------------------------------------------------------------------------
\begin{figure}\centering
\caption{The curve $y(x)$ is traced out by the weight $W$ dragged by a tractor
$T$ in the tractrix problem of example \protect\ref{ode exam:tractrix}.}
\label{ode fig:tractrix}

\psset{xunit=5cm,yunit=5cm}
\begin{pspicture}(-0.25,-0.25)(1.3,1.4)
% First the curve, so the axes show through
\parametricplot[linecolor=gray,linewidth=2pt,plotstyle=curve]%
{0.23}{1}{1 t t mul sub sqrt dup dup 1 add exch 1 sub neg div ln 2 div
exch sub t}
% Now the axes and their labels
\psline{->}(0,0)(1.2,0)
\psline{->}(0,0)(0,1.2)
\uput[r](1.2,0){$x$}
\uput[u](0,1.2){$y$}
\uput[l](0,0){$T_0$}
\uput[l](0,1){$W_0$}
\uput[ur](0.217,0.666){$W$}
\uput[d](0.962,-0.1){$T$}
\pcline[linecolor=darkgray,linewidth=2pt,plotstyle=curve]%
{-}(0.217,0.666)(0.962,0)
\mput*{$L$}
\pcline[linecolor=black,linewidth=1pt]{<->}(0.217,-0.07)(0.962,-0.07)
\mput*{$\sqrt{L^2-y^2}$}
\pcline[linecolor=black,linewidth=1pt]{<->}(0.217,0)(0.217,0.666)
\mput*{$y$}
\psarcn{->}(0.962,0){1}{180}{139}
\uput[l](0.75,0.06){$\alpha$}
\end{pspicture}
\end{figure}
%----------------------------------------------------------------------------

%============================================================================
\begin{example}[The Tractrix]
\label{ode exam:tractrix}

\problem
A tractor is connected by a taut chain of length $L$ to a weight.  The
tractor begins to move in a direction at right angles to the line joining
the tractor and the weight.  It proceeds in that direction in a straight
line dragging the weight.  What curve does the weight trace out?

\solution
The geometry of this problem is illustrated in figure \ref{ode
fig:tractrix}.  The tractor is the point $T$ and the weight the point $W$.
At time $t=0$, the tractor is at the origin, $(0,0)$, and it moves along the 
$x$ axis.  Since the weight is initially at right angles to the direction of
the tractor's motion, the weight is at $(0,L)$ at time $t=0$. Let the curve 
traced out by the weight be given by $y(x)$, where $y(0)=L$.

The key observation is that the weight moves in the direction of the chain.
Therefore, the chain is always a tangent to the curve $y(x)$.  So if the
weight is at the point $(x,y)$ when the chain is at an angle $\alpha$ to
the $x$ axis, the gradient of the tangent of $y(x)$ satisfies
$$\der{y}{x}=-\tan\alpha=-\frac{y}{\sqrt{L^2-y^2}}$$
This has been obtained by using figure \ref{ode fig:tractrix} and
remembering that the chain's length is $L$.


Therefore the differential equation to be solved is
$$\der{y}{x}=-\frac{y}{\sqrt{L^2-y^2}}$$
with the initial condition
$$y(0)=L$$

The \ODE is separable, so
$$\int^y\frac{\sqrt{L^2-y^2}}{y}\,dy =-\int^x dx+C=-x+C$$
To solve this integral, make the substitution $z^2=L^2-y^2$.  Then
$z\,dz=-y\,dy$.
\begin{eqnarray*}
x&=&C-\int^y\frac{\sqrt{L^2-y^2}}{y}\,dy \\
&=&C+\int^{\sqrt{L^2-y^2}}\frac{z^2}{L^2-y^2}\,dz \\
&=&C+\int^{\sqrt{L^2-y^2}} \left(-1 +\frac{L^2}{L^2-z^2}\right)\,dz \\
&=&C+\int^{\sqrt{L^2-y^2}}\left[ 
-1 +\frac{L}{2}\left(\frac{1}{L-z}+\frac{1}{L+z}\right)\right]\,dz \\
&=&C-\sqrt{L^2-y^2}+\frac{L}{2}\left(-\ln\left(L-\sqrt{L^2-y^2}\right)
+\ln\left(L+\sqrt{L^2-y^2}\right)\right) \\
&=&C-\sqrt{L^2-y^2}+\frac{L}{2}\ln\left(\frac{L+\sqrt{L^2-y^2}}
{L-\sqrt{L^2-y^2}}\right)
\end{eqnarray*}
Applying the initial conditions of $y(0)=L$ shows that
$$0=C-0+\frac{L}{2}\ln\left(\frac{L}{L}\right)=C+0$$
so that $C=0$.

Therefore the equation of the tractrix is
$$x=\frac{L}{2}\ln\left(\frac{L+\sqrt{L^2-y^2}}{L-\sqrt{L^2-y^2}}\right)
-\sqrt{L^2-y^2}$$
\end{example}
%============================================================================

%============================================================================
\begin{exercise}
Now try the tractrix problem again with the tractor heading off at an angle
$\theta$ (which is greater than $\pi/2$) to the initial direction of the
chain.
\end{exercise}
%============================================================================

%----------------------------------------------------------------------------
\begin{figure}\centering
\caption{The curve $y(x)$ is traced out by the coyote as it runs towards the
roadrunner in the pursuit curve problem of example 
\protect\ref{ode exam:pursuit}. (For this example, $\lambda=1/2$ so the coyote
catches the roadrunner)}
\label{ode fig:pursuit}

\psset{xunit=5cm,yunit=5cm}
\begin{pspicture}(-0.25,-0.25)(1.3,1.4)
% First the curve, so the axes show through
\psplot[linecolor=gray,linewidth=2pt,plotstyle=curve]%
{0}{1}{x sqrt x 3 div 1 sub mul 0.666666 add}
% Now the axes and their labels
\psline{->}(0,0)(1.2,0)
\psline{->}(0,0)(0,1.2)
\uput[r](1.2,0){$x$}
\uput[u](0,1.2){$y$}
\uput[ur](0.36,0.13866667){$(x(t),y(t))$}
\rput[r](-0.05,0.5){roadrunner}
\uput[ur](0.36,0.25){coyote}
\pcline[linecolor=darkgray,linewidth=2pt]{*->}(0,0)(0,0.15)
\Aput*{$v$}
\pcline[linecolor=darkgray,linewidth=2pt]{*->}(0,0.33067)(0,0.48067)
\Aput*{$v$}
\uput[l](0,0.33067){$(0,vt)$}
\pcline[linecolor=black,linewidth=1pt,linestyle=dashed]{-}(0,0.33067)(0.36,0.138667)
\pcline[linecolor=black,linewidth=2pt]{*->}(0.36,0.130667)(0.228,0.2091)
\Aput*{$V$}
\pcline[linecolor=black,linewidth=2pt]{*->}(1,0)(0.85,0)
\Bput*{$V$}
\pcline[linecolor=black,linewidth=1pt]{<->}(0,-0.07)(1,-0.07)
\mput*{$L$}
\end{pspicture}
\end{figure}
%----------------------------------------------------------------------------

%============================================================================
\begin{example}[The Pursuit Curve]
\label{ode exam:pursuit}
\problem A roadrunner commences running at constant speed $v$ in a northerly
direction.  A coyote a distance $L$ away to the east scents the roadrunner
and immediately gives chase at constant speed $V$, always running directly
at the roadrunner.  What path does the coyote chase out?

\solution
The diagram in figure \ref{ode fig:pursuit} shows the roadrunner running up
the $y$ axis at speed $v$, starting at the origin at time $t=0$.  At time
$t=0$, the coyote is on the $x$ axis at the point $(L,0)$, and moves towards
the roadrunner with speed $V$.  Let the function $y(x)$ describe the path of
the coyote.

Since the coyote always runs directly towards the roadrunner, the coyote's
velocity vector, which is the tangent to the pursuit curve, intercepts the
$y$ axis at the point $(0,vt)$.  Therefore
$$\der{y}{x}=\frac{y-vt}{x}$$
so that
$$x\,\der{y}{x}=y-vt$$
The time variable $t$ is eliminated by considering the arc length $s$ of the
pursuit curve.  The length of the curve at time $t$ is
$$s=Vt$$
which is the total distance travelled by the coyote at time $t$.

Using $ds^2=dx^2+dy^2$, $\der{s}{x}$ is
$$\der{s}{x}=-\sqrt{1+\left(\der{y}{x}\right)^2}$$
The negative square root has been chosen because $s$ increases as $x$
decreases.

From $s=Vt$, write $t=s/V$ and eliminate $t$ to give
$$x\,\der{y}{x}=y-\frac{v}{V}s$$
For convenience, write the ratio of the roadrunner's and coyote's speeds as
$$\lambda=\frac{v}{V}$$
so that
$$x\,\der{y}{x}=y-\lambda s$$
Differentiate both sides with respect to $x$ 
$$\dbd{x}\left(x\,\der{y}{x}\right)=\der{y}{x}-\lambda\,\der{s}{x}$$
Now substitute for $\der{s}{x}$
$$\der{y}{x}+x\dder{y}{x}=\der{y}{x}+
\lambda\sqrt{1+\left(\der{y}{x}\right)^2}$$
Therefore the equation of the pursuit curve is 
$$x\dder{y}{x}=\lambda\sqrt{1+\left(\der{y}{x}\right)^2}$$


This equation for the pursuit curve appears to be a second order \ODE.  
However make the substitution 
$$w=\der{y}{x}$$
so that the \ODE becomes a first order separable \ODE
$$x\der{w}{x}=\lambda\sqrt{1+w^2}$$
This is solved by separating and integrating to give $w$ as a function of 
$x$.  Then integrating a second time gives $y$ in terms of $x$, as required.

Separating the separable equation in $x$ and $w$ gives
$$\int^w\frac{dw}{\sqrt{1+w^2}}=\lambda \int^x\frac{dx}{x}+C$$
Make the substitution $w=\sinh u$ so that $1+w^2=\cosh^2u$ and
$$du=\frac{dw}{\sqrt{1+w^2}}$$
Therefore
$$\int^{\arcsinh w} du=\arcsinh w=\lambda \ln x+C$$
To find the value of $C$, note that at time $t=0$ when $x=L$,
$$w=\der{y}{x}=0$$
Therefore $0=\lambda \ln L+C$ so
$$\arcsinh w=\ln\left(\frac{x}{L}\right)^{\lambda}$$
Invert the equation to give $w$ as an explicit function of $x$
$$w=\sinh\left(\ln\left(\frac{x}{L}\right)^{\lambda}\right)$$
Then use the expression 
$$\sinh z=\frac{1}{2}\left(e^z-e^{-z}\right)$$
to write $w=\der{y}{x}$ as
$$\der{y}{x}=\frac{1}{2}\left[\left(\frac{x}{L}\right)^{\lambda}
-\left(\frac{L}{x}\right)^{\lambda}\right]$$

Integrating $\der{y}{x}$ with respect to $x$ gives the solution
$$y=\frac{1}{2(\lambda+1)}\frac{x^{\lambda+1}}{L^{\lambda}}
+\frac{1}{2(\lambda-1)}\frac{L^{\lambda}}{x^{\lambda-1}}+C$$
provided $\lambda\neq 1$.  The constant $C$ is determined by the initial
condition $y=0$ at $x=L$ so that
$$C=-\frac{L\lambda}{\lambda^2-1}$$

%============================================================================
\end{example}

%----------------------------------------------------------------------------
\begin{figure}\centering
\caption{The catenary is the shape made by a hanging chain, as in 
example \protect\ref{ode exam:catenary}.}
\label{ode fig:catenary}

\psset{xunit=5cm,yunit=5cm}
\begin{pspicture}(-0.6,-0.25)(1.3,1.4)
% First the curve, so the axes show through
\psplot[linecolor=gray,linewidth=2pt,plotstyle=curve]%
{-0.5}{1}{2.71828 x 1.5 mul exp 2.71828 x 1.5 mul neg exp add 2 div 1 sub 1.5 div}
% Now the axes and their labels
\psline{->}(-0.5,0)(1.2,0)
\psline{->}(0,0)(0,1.2)
\uput[d](0,0){$O$}
\uput[r](1.2,0){$x$}
\uput[u](0,1.2){$y$}
\uput[u](-0.5,0.196){$A$}
\uput[u](1,0.9){$B$}
\uput[ul](0.8,0.54){$P(x,y)$}
\pcline[linecolor=black,linewidth=2pt]{*->}(0,0)(-0.25,0)
\Aput*{$T_0$}
\pcline[linecolor=black,linewidth=2pt]{*->}(0.55,0.24)(0.55,-0.01)
\uput[d](0.55,-0.01){$W_P$}
\pcline[linecolor=black,linewidth=1pt,linestyle=dashed]{-}(0.8,0.54)(0.442,0)
\pcline[linecolor=black,linewidth=2pt]{*->}(0.8,0.54)(0.8828,0.665)
\Bput*{$T$}
\psarc{->}(0.442,0){1}{0}{56}
\uput[r](0.64,0.06){$\theta$}
\end{pspicture}
\end{figure}
%----------------------------------------------------------------------------

%============================================================================
\begin{example}[The Catenary]
\label{ode exam:catenary}

\problem
A flexible chain of mass $\rho$ per unit length hangs under gravity from
points $A$ and $B$.  Find the shape of the chain.

\solution
At the lowest point in the chain, its slope is zero.  Take this point as the
origin.  If the chain takes the shape given by $y(x)$, consider the slope of
the chain at some point $P(x,y)$, as shown in figure \ref{ode fig:catenary}.


Define $T$ as the tension in the chain at $P=(x,y)$, $T_0$ as the tension in
the chain at the origin $O=(0,0)$, and $W_P$ as the weight of the segment of 
chain from the origin to the point $P$.  The weight $W_P$ is
$$W_P=\rho g s$$
where $\rho$ is the known mass per unit length of the chain, $g$ is the
acceleration due to gravity and $s$ is the length of the arc $OP$.
The tangent to the chain at $P$ makes an angle $\theta$ with the $x$ axis.

The derivative of $s$ is given in terms of the chain's shape $y(x)$ by
$$\der{s}{x}=\sqrt{1+\left(\der{y}{x}\right)^2}$$
Here the positive square root has been chosen because $s$ increases with $x$
if $P$ is on the right hand side of the $y$ axis.  If $P$ were on the other
side of the $y$ axis, the negative square root would have been chosen.

Resolving the horizontal forces on the chain segment $OP$ shows that
$$T_0=T\cos\theta$$
and resolving vertically shows that
$$W_P=T\sin\theta$$
Dividing these two gives
$$\tan\theta=\frac{W_P}{T_0}=\frac{\rho g}{T_0}\,s$$
But $\theta$ is the angle of the tangent to the curve so
$$\der{y}{x}=\tan\theta=\frac{\rho g}{T_0}\,s$$
Write $\lambda=\rho g/T_0$ to simplify the notation, and differentiate both
sides with respect to $x$ so that
$$\dder{y}{x}=\lambda \der{s}{x}=\lambda\sqrt{1+\left(\der{y}{x}\right)^2}$$
which is the equation of the catenary.

The solution of the catenary is found by putting $p=\der{y}{x}$ and solving
the first order separable equation in $p$, then integrating $p$ with respect
to $x$ to give $y$ as a function of $x$.

With $p=\der{y}{x}$, differentiating gives $\der{p}{x}=\dder{y}{x}$ so that
the catenary's differential equation becomes
$$\der{p}{x}=\lambda\sqrt{1+p^2}$$
which is separable with
$$\int^p\frac{dp}{\sqrt{1+p^2}}=\lambda\int^x dx+C$$
Make the substitution $p=\sinh u$ so that $1+p^2=\cosh^2 u$ and $dp=\cosh
u\,du$.  Integrating gives
$$\arcsinh p=\lambda x +C$$
which is
$$\der{y}{x}=p=\sinh\left(\lambda x+C\right)$$

The constant of integration has to be chosen so that $\der{y}{x}=0$ at the
origin, because the origin was placed at the lowest point of the catenary.
Substituting in $x=0$ and $\der{y}{x}=0$ shows that the constant is $C=0$.
Therefore
$$\der{y}{x}=\sinh\lambda x$$
Integrating with respect to $x$ gives
$$y=\frac{1}{\lambda}\cosh\lambda x+C$$
Once again, the catenary passes through the origin, so $y(0)=0$.  The
constant is $C=-1/\lambda$, giving the equation of the catenary as
$$y=\frac{1}{\lambda}\left[\cosh\left(\lambda x\right)-1\right]$$

This is still not quite the complete solution because 
$\lambda=\rho g/T_0$, the tension at the origin $T_0$, has
not yet been determined.  This depends on the length of the chain and the
position of the endpoints relative to the origin. 

\parbreak

Another way of solving the catenary starts with the equation
$$\dder{y}{x}=\lambda\sqrt{1+\left(\der{y}{x}\right)^2}$$
and uses the identity
$$\dbd{y}\left[\frac{1}{2}\left(\der{y}{x}\right)^2\right]=\dder{y}{x}$$
to write the catenary's differential equation as
$$\dbd{y}\left[\frac{1}{2}\left(\der{y}{x}\right)^2\right]
=\lambda\sqrt{1+\left(\der{y}{x}\right)^2}$$
Now make the substitution
$$Q=1+\left(\der{y}{x}\right)^2$$
so that the catenary's equation becomes
$$\der{Q}{y}=2\lambda \sqrt{Q}$$
This is separable, so
\begin{eqnarray*}
\int^Q \frac{dQ}{\sqrt{Q}}&=&2\lambda \int^y y\,dy+C \\
\thus 2\sqrt{Q}&=&2\lambda y+C
\end{eqnarray*}
At the origin, $Q=1$ because $\der{y}{x}=0$.  Therefore $C=2$ and 
$$\sqrt{Q}=\lambda y+1$$
Squaring both sides gives
$$1+\left(\der{y}{x}\right)^2=(1+\lambda y)^2$$
Move the $1$ to the right-hand side and take the square root, giving
$$\der{y}{x}=\sqrt{(1+\lambda y)^2-1}$$
The positive  square root has been chosen so that $\der{y}{x}>0$ for $x>0$
(the negative root would have been chosen for the part of the catenary with
$x<0$).  This is separable, so separate and integrate
$$\int^y\frac{dy}{\sqrt{(1+\lambda y)^2-1}}=\int^x dx+C$$

Make the substitution $1+\lambda y=\cosh u$ to show that the solution is
$$\frac{1}{\lambda}\arccosh(1+\lambda y)=x+C$$
which is
$$1+\lambda y=\cosh\lambda(x+C)$$
Using the boundary condition that $y(0)=0$ shows that $C=0$ so the equation
of the catenary is
$$y=\frac{1}{\lambda}\left[\cosh\left(\lambda x\right)-1\right]$$
which is the same as the first solution.
\end{example}
%============================================================================


%----------------------------------------------------------------------------
\begin{figure}\centering
\caption{The mirror $y(x)$ is shaped so that beams of light parallel to the
$y$ axis are reflected to the origin, as described in example  
\protect\ref{ode exam:mirror}.}
\label{ode fig:mirror}

\psset{xunit=5cm,yunit=5cm}
\begin{pspicture}(-0.25,-0.35)(1.3,1.4)
% First the curve, so the axes show through
\psplot[linecolor=gray,linewidth=2pt,plotstyle=curve]{0}{1}{x x mul 0.25 sub}
% Now the axes and their labels
\psline{->}(-0.25,0)(1.2,0)
\psline{->}(0,-0.3)(0,1.2)
\uput[dl](0,0){$O$}
\uput[r](1.2,0){$x$}
\uput[u](0,1.2){$y$}
\uput[r](0.75,0.3125){$(x,y)$}
\psline[linecolor=black,linewidth=2pt]{-}(0.75,1.2)(0.75,0.3125)(0,0)
\psline[linecolor=black,linewidth=2pt]{->}(0.75,1.2)(0.75,0.8)
\psline[linecolor=black,linewidth=2pt]{->}(0.75,0.3125)(0.375,0.15625)
\psline[linecolor=black,linewidth=1pt,linestyle=dashed]{-}(0.5416666,0)(1,0.6425)
\psline[linecolor=black,linewidth=1pt,linestyle=dashed]{-}(0.75,0.3125)(0.542,0.4504)
\psarc{->}(0.5416666,0){1}{0}{56.3}
\uput[r](0.74,0.06){$\theta$}
\psarc{->}(0.75,0.3125){1}{56.3}{90}
\uput[u](0.8,0.5){$\alpha$}
\psarcn{->}(0,0){1}{90}{23}
\uput[ur](0.05,0.18){$\phi$}
\end{pspicture}
\end{figure}
%----------------------------------------------------------------------------

%============================================================================
\begin{example}[Perfect Mirror Focus]
\label{ode exam:mirror}

\problem
Light strikes a plane curve in such a manner that all beams parallel to the
$y$ axis are reflected to a single point $O$.  Determine the shape of the
curve.

\solution
This problem is described in figure \ref{ode fig:mirror}, where the point
$O$ has been placed at the origin.  Using the property of a mirror that the 
angle of incidence is equal to the angle of reflection,
$$(\pi-\phi)+2\alpha=\pi$$
Thus $\phi=2\alpha$.

With $\theta$ the angle the tangent to the mirror makes with the $x$ axis,
the geometry requires that
$$(\pi-\phi)+\alpha+\left(\frac{\pi}{2}-\theta\right)=\pi$$
which shows that $\theta=\frac{\pi}{2}-\alpha$.

The mirror's equation is given by $y(x)$, so that from 
figure~\ref{ode fig:mirror},
$$\tan\left(\frac{\pi}{2}-\phi\right)=\frac{y}{x}$$
Writing $\phi$ in terms of $\theta$ gives the condition
$$\frac{y}{x}=\tan\left(2\theta-\frac{\pi}{2}\right)$$
Now
$$\tan\left(2\theta-\frac{\pi}{2}\right)=-\cot
2\theta=-\frac{1-\tan^2\theta}{2\tan\theta}$$
Since this is equal to $y/x$ and using $\tan\theta=\der{y}{x}$, the
differential equation describing the mirror is
$$\frac{y}{x}=-\frac{1-\left(\der{y}{x}\right)^2}{2\der{y}{x}}$$
Rearrange this to give
$$2\frac{y}{x}\der{y}{x}-\left(\der{y}{x}\right)^2=-1$$
The solution is found by putting $w=x^2$ so that
$$\der{y}{w}=\der{y}{x}\der{x}{w}=\frac{1}{2x}\der{y}{x}$$
Therefore the differential equation becomes
$$4y\der{y}{w}-4w\left(\der{y}{w}\right)^2=-1$$
Divide both sides by $4\left(\der{y}{w}\right)^2$ to give
$$y\der{w}{y}-w=-\frac{1}{4}\left(\der{w}{y}\right)^2$$

This is an example of Clairaut's equation
$$y=x\der{y}{x}+f\left(\der{y}{x}\right)$$
which is described in detail in section \ref{ode sec:clairaut} and 
chapter 2.6 of Zill.  The solution is
$$y=mx+f(m)$$
where $m$ is an arbitrary constant.  Applying this to the mirror's
differential equation, the solution is
$$w=my+\frac{1}{4}m^2$$
Since $w=x^2$, this is
$$y=\frac{x^2}{m}-\frac{m}{4}$$
which is the equation of a parabola.  

\parbreak

The shape of the mirror can also be found without using the substitution
leading to Clairaut's equation.  Starting with
$$2\frac{y}{x}\der{y}{x}-\left(\der{y}{x}\right)^2=-1$$
divide by $\frac{1}{x}\left(\der{y}{x}\right)^2$ to obtain
$$x\left(\der{x}{y}\right)^2+2y\der{x}{y}-x=0$$
This is a quadratic in $\der{x}{y}$ with roots
$$\der{x}{y}=\frac{-y\pm\sqrt{x^2+y^2}}{x}$$
Rearrange this to give
$$\pm\frac{x\,dx+y\,dy}{\sqrt{x^2+y^2}}=dy$$
Now $x\,dx+y\,dy=\frac{1}{2}d(x^2+y^2)$ so
$$\pm\frac{\frac{1}{2}d(x^2+y^2)}{\sqrt{x^2+y^2}}=dy$$
Integrate both sides
$$\pm\sqrt{x^2+y^2}=y+C$$
and square so that
$$x^2+y^2=(y+C)^2=y^2+2yC+C^2$$
Make $y$ the subject
$$y=\frac{x^2}{2C}-\frac{C}{2}$$
and put $m=2C$ to give the answer in the same form as before
$$y=\frac{x^2}{m}-\frac{m}{4}$$
\end{example}
%============================================================================

%============================================================================
\begin{exercise}
Chapters 1 and 3 of Zill have problems involving applications of
first order \ODEs.
\end{exercise}
%============================================================================

%%%%%%%%%%%%%%%%%%%%%%%%%%%%%%%%%%%%%%%%%%%%%%%%%%%%%%%%%%%%%%%%%%%%%%%%%%%%%
\section{Special Nonlinear First Order O.D.E.s}
\label{ode sec:special}

%%%%%%%%%%%%%%%%%%%%%%%%%%%%%%%%%%%%%%%%%%%%%%%%%%%%%%%%%%%%%%%%%%%%%%%%%%%%%
\subsection{Bernoulli's Equation}
\label{ode sec:bernoulli}

An \ODE of the form
$$\der{y}{x}+P(x)y=f(x)y^n$$
where $n$ is any real number, is called \name{Bernoulli's equation} (named
after James Bernoulli (1654--1705)).

Note that we can already solve this equation for $n=0$ and $n=1$.  When
$n=0$, the equation is an inhomogeneous linear \ODE.  When $n=1$, the
equation is a homogeneous linear \ODE.

To solve Bernoulli's equation, make the substitution
$$w=y^{1-n}$$
to obtain a linear expression.  Since
$$y=w^{\frac{1}{1-n}}$$
differentiating $y$ with respect to $x$ gives 
$$\der{y}{x}=\frac{1}{1-n}w^{\frac{1}{1-n}-1}\der{w}{x}$$
Substitute this into the differential equation
$$\frac{1}{1-n}w^{\frac{n}{1-n}}\der{w}{x}+P(x)w^{\frac{1}{1-n}}=
f(x)w^{\frac{n}{1-n}}$$
then multiply each side by $1-n$ and divide by $w^{\frac{n}{1-n}}$.  This
results in the following linear \ODE
$$\der{w}{x}+(1-n)P(x)w=f(x)(1-n)$$
which can easily be solved.

%============================================================================
\begin{example}
To solve
$$\der{y}{x}+\frac{1}{x}y=xy^2$$
note that this is Bernoulli's equation with $P(x)=1/x$, $f(x)=x$ and $n=2$.
Try the substitution $w=y^{1-n}=y^{1-2}=1/y$ so that 
$$y=\frac{1}{w}$$
Then
$$\der{y}{x}=-\frac{1}{w^2}\der{w}{x}$$
which gets substituted into the \ODE giving
$$-\frac{1}{w^2}\der{w}{x}+\frac{1}{xw}=\frac{x}{w^2}$$
Upon rearranging, we have
$$\der{w}{x}-\frac{1}{x}w=-x$$
The integrating factor is found by solving
$$\der{I}{x}=-\frac{1}{x}I$$
so that $\ln I=- \ln x$ or $I=1/x$.  Multiply the \ODE by the integrating
factor to obtain
$$\dbd{x}\left[\frac{w}{x}\right]=-1$$
whence
$$\frac{w}{x}=-x+c$$
so that
$$w=cx-x^2$$
Expressed in terms of $y$ rather than $w$, this is
$$y=\frac{1}{cx-x^2}$$
\end{example}
%============================================================================

%============================================================================
\begin{exercise}
Exercise 2.6 of Zill, pp. 75--76, has more examples of Bernoulli's equation.
\end{exercise}
%============================================================================

%%%%%%%%%%%%%%%%%%%%%%%%%%%%%%%%%%%%%%%%%%%%%%%%%%%%%%%%%%%%%%%%%%%%%%%%%%%%%
\subsection{Ricatti's Equation}
\label{ode sec:ricatti}

An \ODE of the form
$$\der{y}{x}=P(x)+Q(x)y+R(x)y^2$$
is called \name{Ricatti's equation} (named after Count Jacobo Francesco
Ricatti (1676--1754)).  Note that
\begin{itemize}
\item For many $P(x)$, $Q(x)$ and $R(x)$ the solution cannot be expressed in
terms of elementary functions.
\item If $y_1(x)$ is a known solution of Ricatti's equation, then a family
of solutions is
$$y(x)=y_1(x)+u(x)$$
where $u(x)$ satisfies
$$\der{u}{x}-(Q+2y_1R)u=Ru^2$$
which is Bernoulli's equation with $n=2$.
\item When $P(x)=0$, Ricatti's equation reduces to a Bernoulli equation with
$n=2$.
\end{itemize}

%============================================================================
\begin{example}
To solve
$$\der{y}{x}=2-2xy+y^2$$
given that $y_1=2x$ is one solution, put
$$y=2x+u$$
Then
$$\der{y}{x}=2+\der{u}{x}$$
so that the differential equation becomes
$$2+\der{u}{x}=2-2x(2x+u)+(2x+u)^2$$
which simplifies to
$$\der{u}{x}=2xu+u^2$$
To solve this Bernoulli equation, put $w=u^{1-2}=1/u$ so that
$u=1/w$.  Substitute this into the \ODE to obtain
$$-\frac{1}{w^2}\der{w}{x}=\frac{2x}{w}+\frac{1}{w^2}$$
This is the first order linear \ODE
$$\der{w}{x}+2xw=-1$$
The integrating factor is found by solving
$$\der{I}{x}=2xI$$
hence
$$I(x)=e^{x^2}$$
Multiply the \ODE by the integrating factor to give
$$\dbd{x}\left[e^{x^2}w\right]=-e^{x^2}$$
Integrating gives
$$w=-e^{-x^2}\int^xe^{t^2}\,dt+Ce^{-x^2}$$
Use $u=1/w$ and $y=2x+u$ to give the final solution
$$y(x)=2x+\frac{e^{x^2}}{C-\int^xe^{t^2}\,dt}$$
which cannot be represented in terms of elementary functions without the 
integral.
\end{example}
%============================================================================

%============================================================================
\begin{exercise}
Exercise 2.6 of Zill, pp. 75--76, has more examples of Ricatti's equation.
\end{exercise}
%============================================================================

%%%%%%%%%%%%%%%%%%%%%%%%%%%%%%%%%%%%%%%%%%%%%%%%%%%%%%%%%%%%%%%%%%%%%%%%%%%%%
\subsection{Clairaut's Equation}
\label{ode sec:clairaut}

An \ODE of the form
$$F\left(y-x\der{y}{x},\der{y}{x}\right)=0$$
or alternatively
$$y=x\der{y}{x}+f\left(\der{y}{x}\right)$$
is a \name{Clairaut equation} (named after Alexis Claude Clairaut
(1713--1765)). 

There is a family of solutions that are straight lines, and there may also
be a second solution that is singular.

The family of solutions that are straight lines have the form
$$y=mx+f(m)$$
where each value of $m$ gives a different solution.  To see that these are
solutions of the Clairaut equation, substitute into the \ODE.
From $y=mx+f(m)$, 
$$\der{y}{x}=m$$
so that
$$x\der{y}{x}+f\left(\der{y}{x}\right)=xm+f(m)=y$$
as required.

Clairaut's equation can possess a second solution, the \name{singular 
solution}, which is not obtainable from the general solution
$y=mx+f(m)$.  It is not in general a straight line.  The singular solution
expresses the relationship between $x$ and $y$ as a pair of
parametric equations in $t$
\begin{eqnarray*}
x(t)&=&-f'(t)\\
y(t)&=&f(t)-tf'(t)
\end{eqnarray*}

To prove that this does give a solution to Clairaut's equation,
differentiate the parametric equations for $x$ and $y$ with respect to $t$
\begin{eqnarray*}
\der{x}{t}&=&-f''(t)\\
\der{y}{t}&=&f'-tf''-f'=-tf''
\end{eqnarray*}
Therefore
$$\der{y}{x}=\frac{\der{y}{t}}{\der{x}{t}}=t$$
provided $f''(t)\neq 0$.

Then since
$$y(t)=f(t)-tf'(t)=(-f'(t))t+f(t)$$
substituting $t=\der{y}{x}$ gives
$$y=x\der{y}{x}+f\left(\der{y}{x}\right)$$
which is indeed Clairaut's equation.

%============================================================================
\begin{example}
To solve
$$y=xy'+\frac{1}{(y')^2}$$
note that this is Clairaut's equation with the general solution
$$y=mx+\frac{1}{m^2}$$

Since $f(t)=1/t^2$, the singular solution is given parametrically by
\begin{eqnarray*}
x(t)&=&\frac{2}{t^3}\\
y(t)&=&\frac{1}{t^2}-t\left(-\frac{2}{t^3}\right)=\frac{3}{t^2}
\end{eqnarray*}
To eliminate $t$, express both $x$ and $y$ in terms of $1/t^2$
$$\left(\frac{x}{2}\right)^{2/3}=\frac{1}{t^2}=\frac{y}{3}$$
which is simply
$$y=3\left(\frac{x}{2}\right)^{2/3}$$

\end{example}
%============================================================================

Note that if you are asked to solve a Clairaut equation, you should always
give both solutions.

%============================================================================
\begin{exercise}
Exercise 2.6 of Zill, pp. 75--76, has more examples of Clairaut's equation.
\end{exercise}
%============================================================================

%%%%%%%%%%%%%%%%%%%%%%%%%%%%%%%%%%%%%%%%%%%%%%%%%%%%%%%%%%%%%%%%%%%%%%%%%%%%%
\section[Second Order O.D.E.s]{Second Order Ordinary Differential Equations}
\label{ode sec:soode}

An \ODE of the form
$$\dder{y}{x}+P(x)\der{y}{x}+Q(x)y=R(x)$$
is a \name{second order linear \ODE}.  If $R(x)=0$, the \ODE is
\name{homogeneous}.  If $R(x)\neq 0$, the \ODE is \name{inhomogeneous}.

A solution to a second order linear \ODE ought to require two additional
pieces of information to specify it uniquely.  In an \name{initial value 
problem}, two initial conditions on the same point are specified
$$y(a)=c\qquad \der{y}{x}(a)=d$$
provided $P$, $Q$ and $R$ are defined at $x=a$.

There are two types of \name{boundary value problems}, simple and mixed.
A \name{simple boundary condition} specifies the value of the solution at two
different points
$$y(a)=c\qquad y(b)=d$$
A \name{mixed boundary condition} specifies the value of the solution and its
derivative at two different points
$$y(a)=c\qquad \der{y}{x}(b)=d$$
Boundary value problems may have a unique solution, several solutions, or no
solution.

%============================================================================
\begin{example}
\problem
Solve 
$$y''+a^2y=0$$
subject to
$$y(0)=0\qquad y'(0)=1$$

\solution
This is an initial value problem.  The solution is
$$y(x)=\frac{1}{a}\sin ax$$
To prove that this is the solution, $y(0)=0$ and
$$y'(x)=\cos ax$$
so $y'(0)=1$, therefore the initial conditions are satisfied.  Now
$$y''(x)=-a\sin ax=-a^2\,y(x)$$
so that $y''+a^2y=0$ as required.

Note that the second solution to the differential equation is 
$$y(x)=C\cos ax$$
but this does not satisfy the initial conditions.
\end{example}
%============================================================================

%============================================================================
\begin{example}
\problem Solve 
$$y''+a^2y=0$$
subject to
$$y(0)=0\qquad y\left(\frac{2\pi}{a}\right)=0$$

\solution
This is a boundary value problem.  Note that $y=0$ is a solution.  To see
whether there are any others, the general solution to the differential
equation is
$$y(x)=C_1\sin ax + C_2\cos ax$$
Now $C_1$ and $C_2$ have to be chosen to satisfy the boundary conditions.
$y(0)=C_2$ so the boundary condition $y(0)=0$ means that $C_2=0$.
The second boundary condition is always satisfied because
$y\left(2\pi/a\right)=C_2=0$. 
Therefore, writing $C=C_1$, there are infinitely many solutions satisfying
the \ODE and both boundary conditions
$$y(x)=C\sin ax$$
\end{example}
%============================================================================

%%%%%%%%%%%%%%%%%%%%%%%%%%%%%%%%%%%%%%%%%%%%%%%%%%%%%%%%%%%%%%%%%%%%%%%%%%%%%
\subsection{Superposition of Solutions}

Two functions $y_1(x)$ and $y_2(x)$ are \name{linearly independent} on the
range $a\leq x\leq b$ if no non-zero constants $C_1$ and $C_2$ can be found
such that
$$C_1y_1(x)+C_2y_2(x)=0$$
for all $x$ in the interval $[a,b]$.


%============================================================================
\begin{theorem}
If $y_1(x)$ and $y_2(x)$ are two solutions of the homogeneous equation
$$y''+Py'+Qy=0$$
then $C_1y_1(x)+C_2y_2(x)$ is also a solution.
\end{theorem}
%============================================================================

\begin{proof}
$y_1(x)$ and $y_2(x)$ are both solutions so
$$y_1''+Py_1'+Qy_1=0$$
$$y_2''+Py_2'+Qy_2=0$$
Now multiply the first by $C_1$ and the second by $C_2$
and add, showing that
$$\ndbd{x}{2}(C_1y_1+C_2y_2)+P\dbd{x}(C_1y_1+C_2y_2)+Q(C_1y_1+C_2y_2)=0$$
Therefore $C_1y_1+C_2y_2$ satisifes the \ODE, so it is also a solution.
\end{proof}

%============================================================================
\begin{theorem}
\label{ode thm:gen sol}
If $y_1(x)$ and $y_2(x)$ are two linearly independent solutions of the 
homogeneous equation
$$y''+Py'+Qy=0$$
then $C_1y_1(x)+C_2y_2(x)$ is the most general solution.
\end{theorem}
%============================================================================

%============================================================================
\begin{example}
The second order homogeneous \ODE
$$y''+a^2y=0$$
has solutions
$$y_1(x)=\sin ax$$
and
$$y_2(x)=\cos ax$$
Now $y_1/y_2=\tan ax$, which is not constant over any interval. 
Therefore $y_1$ and $y_2$ are linearly independent and the general solution
is
$$y(x)=C_1\sin ax + C_2\cos ax$$
where $C_1$ and $C_2$ can be determined by initial or boundary conditions.
\end{example}
%============================================================================

%%%%%%%%%%%%%%%%%%%%%%%%%%%%%%%%%%%%%%%%%%%%%%%%%%%%%%%%%%%%%%%%%%%%%%%%%%%%%
\subsection{Inhomogeneous Second Order Linear O.D.E.s}

%============================================================================
\begin{theorem}
If $y_p(x)$ is any solution of the inhomogeneous \ODE
$$y''+Py'+Qy=R$$
then the general solution is
$$y(x)=C_1y_1(x)+C_2y_2(x)+y_p(x)$$
where $y_1(x)$ and $y_2(x)$ are two linearly independent solutions of the 
homogeneous \ODE
$$y''+Py'+Qy=0$$
\end{theorem}
%============================================================================

\begin{proof}
Let $y(x)$ be the general solution of the inhomogeneous \ODE and $y_p(x)$
be any particular solution of the inhomogeneous \ODE.  Form the function
$Y=y-y_p$.  Then
\begin{eqnarray*}
Y''+PY'+QY&=&(y''+Py'+Qy)-(y_p''+Py_p'+Qy_p) \\
&=&R-R\\
&=&0
\end{eqnarray*}
So $Y$ satisfies the homogeneous \ODE.  By theorem \ref{ode thm:gen sol}, 
the general form of $Y$ is
$$Y=C_1y_1+C_2y_2$$
where $y_1$ and $y_2$ are linearly independent solutions of the homogeneous
\ODE.  Therefore
$$y=Y+y_p=C_1y_1+C_2y_2+y_p$$
\end{proof}

This shows that any solution of an inhomogeneous linear \ODE can be written 
as the sum of the particular solution to the inhomogeneous \ODE and the
general solution of the homogeneous \ODE.

%============================================================================
\begin{example}
To solve
$$y''+a^2y=1$$
note that a particular solution of this \ODE is
$$y_p(x)=\frac{1}{a^2}$$
We already know that $y_1=\sin ax$ and $y_2=\cos ax$ are linearly
independent solutions of
$$y''+a^2y=0$$
Therefore the general solution of the inhomogeneous \ODE is
$$y(x)=C_1\sin ax + C_2\cos ax + \frac{1}{a^2}$$
\end{example}
%============================================================================


%============================================================================
\begin{exercise}
Repeat this example with $R(x)$ equal to $1+x$, $1+x^2$ and $e^{\lambda x}$.
\end{exercise}
%============================================================================

%%%%%%%%%%%%%%%%%%%%%%%%%%%%%%%%%%%%%%%%%%%%%%%%%%%%%%%%%%%%%%%%%%%%%%%%%%%%%
\subsection{Reduction of Order}

The method of \name{reduction of order} is a way of obtaining a second 
solution $y_2(x)$ of 
$$y''+Py'+Qy=0$$
given a first solution $y_1(x)$.

To derive the method, put $y_2=uy_1$ and substitute into the \ODE
Then using product rule
$$y_2'=u'y_1+uy_1'$$
and
$$y_2''=u''y_1+2u'y_1'+uy_1''$$
These show that
$$y_2''+Py_2'+Qy_2=(u''+Pu')y_1+2u'y_1'+u(y_1''+Py_1'+Qy_1)$$
Now $y_1''+Py_1'+Qy_1=0$ because $y_1$ is a solution of the \ODE.  If
$y_2$ is to be a solution of the \ODE too, then $y_2$ must satisfy
$$y_2''+Py_2'+Qy_2=0$$
But
$$y_2''+Py_2'+Qy_2=(u''+Pu')y_1+2u'y_1'$$
so for $y_2$ to be a solution of the \ODE, 
$$(u''+Pu')y_1+2u'y_1'=0$$
Rearranging this shows that $u$ must satisfy
$$u''y_1+(Py_1+2y_1')u'=0$$
Divide by $y_1$ and put $w=u'$.  Then
$$w'+\left(P+2\frac{y_1'}{y_1}\right)w=0$$
This is a linear first order \ODE which can be solved by finding the
integrating factor to give $w(x)$.  Then $u'(x)=w(x)$ so determine $u(x)$ by
integrating $w(x)$.  Finally construct the second solution
$$y_2(x)=u(x)y_1(x)$$

Note that one solution of the second order homogeneous \ODE is $u$ equal
to any constant.  This can always be neglected because it would not produce
a linearly independent $y_2(x)$.

%============================================================================
\begin{example}
\problem Find a second solution of 
$$x^2y''+2xy'-6y=0$$
given that $y_1=x^2$ is a solution.

\solution Look for a second solution of the form $y_2=ux^2$.  Then
$$y_2'=2xu+x^2u'$$
and
$$y_2''=2u+4xu'+x^2u''$$
Substituting $y_2$ into the \ODE gives
$$0=x^2y_2''+2xy_2'-6y_2=x^4u''+6x^3u'+u(2x^2+4x^2-6x^2)$$
As expected, the last term is zero, leaving
$$x^4u''+6x^3u'=0$$
if $y_2$ is a second solution.  Introduce $w=u'$ so that
$$x^4w'+6x^3w=0$$
So
$$w'+\frac{6}{x}w=0$$ which means that
$$u'=w=\frac{C}{x^6}$$
Therefore
$$u=\int^x \frac{C}{x^6}\,dx+A=\frac{D}{x^5}+A$$
where the constant $D=-\frac{C}{5}$.  Thus
$$y_2=ux^2=\frac{D}{x^3}+Ax^2$$
The $Ax^2$ term is an arbitrary multiple of $y_1$ so the second linearly
independent solution is
$$y_2(x)=\frac{1}{x^3}$$
\end{example}
%============================================================================

%============================================================================
\begin{exercise}
Exercise 4.2 of Zill, pp. 158--159, has more examples of reduction of order.
\end{exercise}
%============================================================================

%%%%%%%%%%%%%%%%%%%%%%%%%%%%%%%%%%%%%%%%%%%%%%%%%%%%%%%%%%%%%%%%%%%%%%%%%%%%%
\subsection{Homogeneous Second Order O.D.E.s with Constant Coefficients}

A homogeneous second order linear \ODE with \name{constant coefficients}
has the form
$$ay''+by'+cy=0$$
where $a$, $b$ and $c$ are constants.

The method of solution involves substituting 
$$y=e^{mx}$$
to obtain
$$(am^2+bm+c)e^{mx}=0$$
Thus $e^{mx}$ is a solution provided $m$ is chosen so that
$$am^2+bm+c=0$$
or, using the general quadratic formula,
$$m=\frac{-b\pm\sqrt{b^2-4ac}}{2a}$$

There are three cases that can arise, depending on whether $b^2-4ac<0$,
$b^2-4ac=0$ or $b^2-4ac>0$.

If $b^2-4ac>0$, there are two real roots, $m_1$ and $m_2$, and the general
solution is the sum of two exponentials
$$y(x)=C_1e^{m_1x}+C_2e^{m_2x}$$

If $b^2-4ac=0$, there is only one real root, $m_1=-b/2a$ so
$$y_1(x)=e^{-\frac{b}{2a}x}$$
The second solution can be found using reduction of order.  Put
$$y_2=ue^{-\frac{b}{2a}x}$$
so that
$$y_2'=\left(u'-\frac{b}{2a}u\right)\,e^{-\frac{b}{2a}x}$$
and
$$y_2''=\left(u''-\frac{b}{a}u'+\frac{b^2}{4a^2}u\right)\,e^{-\frac{b}{2a}x}$$
Therefore
$$ay_2''+by_2'+cy_2=\left(au''+u\left(\frac{b^2}{4a}-\frac{b^2}{2a}+c\right)
\right)\,e^{-\frac{b}{2a}x}$$
The coefficient of $u$ can be written
$$\frac{b^2}{4a}-\frac{b^2}{2a}+c=-\frac{b^2-4ac}{4a}$$
which is zero because $b^2-4ac=0$.  Therefore
$$u''=0$$
so that $u=x$, which means
$$y_2(x)=x\,e^{-\frac{b}{2a}x}$$
This gives the complete solution as
$$y=(C_1+C_2x)\,e^{-\frac{b}{2a}x}$$
for the case when $b^2-4ac=0$.

If $b^2-4ac<0$, there are two imaginary roots $m_1$ and $m_2$ that are 
complex conjugates, where
$$m_1=\frac{-b+i\sqrt{4ac-b^2}}{2a}$$
Write $m_1$ in the form
$$m_1=\alpha+i\beta$$
so that the second root is
$$m_2=\alpha-i\beta$$
The general solution is
$$y=C_1e^{(\alpha+i\beta)x}+C_2e^{(\alpha-i\beta)x}
=e^{\alpha x}\left(C_1e^{i\beta x}+C_2e^{-i\beta x}\right)$$
Now $e^{i\theta}=\cos\theta+i\sin\theta$ so that
$$y(x)=e^{\alpha x}\left((C_1+C_2)\cos\beta x+i(C_1-C_2)\sin\beta x\right)$$
Write $A=C_1+C_2$ and $B=i(C_1-C_2)$, giving an alternative form of the
solution
$$y(x)=e^{\alpha x}\left(A\cos\beta x+B\sin\beta x\right)$$

%============================================================================
\begin{example}
To solve
$$y''-y'-6y=0$$
try $e^{mx}$ in the \ODE to get
$$(m^2-m-6)e^{mx}=0$$
Therefore $e^{mx}$ is a solution if
$$m^2-m-6=(m-3)(m+2)=0$$
The two roots are $m_1=3$ and $m_2=-2$ so the general solution is
$$y(x)=C_1e^{3x}+C_2e^{-2x}$$
\end{example}
%============================================================================

%============================================================================
\begin{example}
To solve
$$y''+2y'+y=0$$
try $e^{mx}$ in the \ODE to get
$$(m^2+2m+1)e^{mx}=0$$
Therefore $e^{mx}$ is a solution if
$$m^2+2m+1=0$$
There is only one root, $m=-1$, so the two solutions are
$$y_1=e^{-x}\qquad\mbox{and}\qquad y_2=xe^{-x}$$
giving the general solution
$$y(x)=(C_1+C_2x)e^{-x}$$
\end{example}
%============================================================================

%============================================================================
\begin{example}
To solve
$$y''+a^2y=0$$
try $e^{mx}$ in the \ODE to get
$$(m^2+a^2)e^{mx}=0$$
Therefore $e^{mx}$ is a solution if
$$m=\pm ia$$
The general solution is
$$y(x)=C_1e^{iax}+C_2e^{-iax}$$
or alternatively
$$y(x)=A\sin ax + B\cos ax$$
\end{example}
%============================================================================

%============================================================================
\begin{example}
To solve
$$y''+4y'-y=0$$
try $e^{mx}$ in the \ODE to get
$$(m^2+4m-1)e^{mx}=0$$
Therefore $e^{mx}$ is a solution if
$$m^2+4m-1=0$$
The two roots are
$$m=\frac{-4\pm\sqrt{4^2-4(-1)}}{2}=-2\pm\sqrt{5}$$
so the general solution is
$$y(x)=C_1e^{-(2+\sqrt{5})x}+C_2e^{-(2-\sqrt{5})x}
=e^{-2x}\left(C_1e^{-\sqrt{5}x}+C_2e^{\sqrt{5}x}\right)$$
\end{example}
%============================================================================

%============================================================================
\begin{example}
To solve
$$2y''+2y'+y=0$$
try $e^{mx}$ in the \ODE to get
$$(2m^2+2m+1)e^{mx}=0$$
Therefore $e^{mx}$ is a solution if
$$m=\frac{-2\pm\sqrt{4-8}}{4}=-\frac{1}{2}(1\pm i)$$
The general solution is
$$y(x)=e^{-x/2}\left(A\cos\frac{x}{2} + B\sin\frac{x}{2}\right)$$
\end{example}
%============================================================================

%============================================================================
\begin{exercise}
Exercise 4.3 of Zill, pp. 167--168, has more examples of constant
coefficient \ODEs.
\end{exercise}
%============================================================================

%%%%%%%%%%%%%%%%%%%%%%%%%%%%%%%%%%%%%%%%%%%%%%%%%%%%%%%%%%%%%%%%%%%%%%%%%%%%%
\subsection{Inhomogeneous Constant Coefficient O.D.E.s}

The solution of an \ODE of the form
$$ay''+by'+cy=R(x)$$
is the sum of the solutions of the homogeneous equation and the particular
solution due to $R(x)$.

%============================================================================
\begin{figure}
\caption{This table shows the form of the particular solution $y_p(x)$ for a given
$R(x)$ in the method of undetermined coefficients.}
\label{ode fig:muc}
$$\begin{array}{c|c}
R(x) & y_p(x) \\
\hline
\hbox{constant} & A \\
e^{\lambda x} & Ae^{\lambda x} \\
e^{\lambda x}\left[E\cos\mu x+F\sin\mu x\right] & e^{\lambda x}
\left[A\cos\mu x+B\sin\mu x\right] \\
a_0+a_1x+\cdots+a_nx^n & A_0+A_1x+\cdots+A_nx^n \\
e^{\lambda x}[a_0+a_1x+\cdots+a_nx^n] & e^{\lambda x}[A_0+A_1x+\cdots+A_nx^n] 
\end{array}$$
\end{figure}
%============================================================================

In the \name{method of undetermined coefficients}, the particular solution
$y_p(x)$ is found by assuming a general expression for $y_p(x)$ depending on the 
form of $R(x)$ and solving for any unknowns so that the \ODE is satisfied.
The appropriate forms of $y_p(x)$ for $R(x)$ are given in figure
\ref{ode fig:muc}. 

%============================================================================
\begin{example}
The solution of 
$$y''+a^2y=R(x)$$
depends on the form of $R(x)$.

\begin{itemize}
\item When $R(x)=1$, try $y_p=C$.  Then $y_p''=0$ so $C=1/a^2$.  Thus
$$y_p(x)=\frac{1}{a^2}$$

\item When $R(x)=1+x$, try $y_p=A+Bx$ so that $y_p''=0$.  Therefore
$$a^2(A+Bx)=1+x$$
Equating coefficients of $x^0$ and $x^1$ shows that $a^2A=1$ and $a^2B=1$. 
Thus
$$y_p(x)=\frac{1+x}{a^2}$$

\item When $R(x)=1+x^2$, try $y_p=A+Bx+Cx^2$.  Then $y_p''=2C$.
Therefore
$$2C+a^2(A+Bx+Cx^2)=1+x^2$$
Equating coefficients of $x^0$, $x^1$ and $x^2$ shows that
$$2C+a^2A=1\qquad a^2B=0\qquad a^2C=1$$
Therefore 
$$A=\frac{1}{a^2}-\frac{2}{a^4}\qquad B=0\qquad C=\frac{1}{a^2}$$
so the particular solution is
$$y_p(x)=\frac{1}{a^2}-\frac{2}{a^4}+\frac{x^2}{a^2}$$

\item When $R(x)=e^{\lambda x}$, try $y_p=Ae^{\lambda x}$.  Then
$y_p''=A\lambda^2e^{\lambda x}$.  Therefore
$$(\lambda^2+a^2)Ae^{\lambda x}=e^{\lambda x}$$
so that
$$A=\frac{1}{\lambda^2+a^2}$$
giving the particular solution
$$y_p=\frac{e^{\lambda x}}{\lambda^2+a^2}$$

\item When $R(x)=\cos\lambda x$, try $y_p=A\cos\lambda x+B\sin\lambda x$.
Therefore 
$$y_p''=-\lambda^2(A\cos\lambda x+B\sin\lambda x)=-\lambda^2 y_p$$
so that
$$(a^2-\lambda^2)(A\cos\lambda x+B\sin\lambda x)=\cos\lambda x$$
Equating coefficients of $\cos\lambda x$ and $\sin\lambda x$ shows that
$$(a^2-\lambda^2)A=1\qquad (a^2-\lambda^2)B=0$$
so that
$$A=\frac{1}{a^2-\lambda^2}\qquad B=0$$
and the particular solution is
$$y_p=\frac{\cos\lambda x}{a^2-\lambda^2}$$
(What happens here as $a\to\lambda$?)
\end{itemize}
\end{example}
%============================================================================

Note that this method fails whenever
$$R(x)=e^{m_1x}\mbox{\ or\ } e^{m_2x}$$
where $m_1$ and $m_2$ are the roots of 
$$am^2+bm+c=0$$
There is a general method of obtaining $y_p(x)$ when $y_1(x)$ and
$y_2(x)$ are known.  Section 4.7 of Zill (pp. 194--203) describes this
method, called \name{variation of parameters}.

%============================================================================
\begin{example}
When solving 
$$y''+a^2y=\cos ax$$
note that $R(x)=\cos ax$ is a solution of the homogeneous equation, whose
general solution is
$$y(x)=A\cos ax +B\sin ax$$
In cases like this, the method of undetermined coefficients has to be
modified to try a particular solution of the form
$$y_p(x)=u(x)\cos ax$$
Then
$$y_p'(x)=u'(x)\cos ax-au(x)\sin ax$$
and
$$y_p''(x)=u''(x)\cos ax-2au'(x)\sin ax-a^2u(x)\cos ax$$
so that
$$\cos ax=y_p''+a^2y_p=u''(x)\cos ax-2au'(x)\sin ax$$
which gives the differential equation for $u(x)$
$$u''-2au'\tan ax=1$$
which can be solved to determine $u$, hence determine $y_p$.
\end{example}
%============================================================================

%============================================================================
\begin{exercise}
Exercise 4.4 of Zill, pp. 179--181, has more examples of the method of
undetermined coefficients.
\end{exercise}
%============================================================================

%%%%%%%%%%%%%%%%%%%%%%%%%%%%%%%%%%%%%%%%%%%%%%%%%%%%%%%%%%%%%%%%%%%%%%%%%%%%%
\subsection{Superposition of Inhomogeneous Equations}

If $y_{p1}(x)$ is a particular solution of
$$y''+Py'+Qy=R_1$$ 
and $y_{p2}(x)$ is a particular solution of
$$y''+Py'+Qy=R_2$$
then $C_1y_{p1}(x)+C_2y_{p2}(x)$ is a particular solution of
$$y''+Py'+Qy=C_1R_1+C_2R_2$$


%============================================================================
\begin{example}
The equation
$$y_{p1}(x)=\frac{e^{\lambda x}}{\lambda^2+a^2}$$
is a particular solution of
$$y''+a^2y=e^{\lambda x}$$
and
$$y_{p2}(x)=\frac{1}{a^2}-\frac{2}{a^4}+\frac{x^2}{a^2}$$
is a particular solution of
$$y''+a^2y=1+x^2$$
Thus
$$y_p(x)=2\left(\frac{1}{a^2}-\frac{2}{a^4}+\frac{x^2}{a^2}\right)+
\frac{3e^{\lambda x}}{\lambda^2+a^2}$$
is a particular solution of
$$y''+a^2y=2(1+x^2)+3e^{\lambda x}$$
\end{example}
%============================================================================

%%%%%%%%%%%%%%%%%%%%%%%%%%%%%%%%%%%%%%%%%%%%%%%%%%%%%%%%%%%%%%%%%%%%%%%%%%%%%
\section{Free and Forced Vibrations}

Let $x(t)$ describe the response of an oscillating system over time, and write
$$\dot{x}=\der{x}{t}\qquad\ddot{x}=\der{\dot{x}}{t}$$
By Newton's second law, the system's acceleration $\ddot{x}$ multiplied by 
its mass $m$ is equal to the sum of the forces acting on it.   The three 
forces acting on the system are a restoring force of magnitude $-kx$,
resistance or drag of $-\lambda\dot{x}$ and an externally applied force 
$F(t)$.  This gives the second order differential equation
$$m\ddot{x}=-kx-\lambda\dot{x}+F(t)$$

This can be rewritten in the standard form for a second order differential 
equation as
$$\ddot{x}+2p\dot{x}+\wo^2x=f(t)$$
where $p=\lambda/2m$ is the friction term ($p\geq 0$), $\wo=\sqrt{k/m}$ is
the natural frequency ($\wo>0$) and $f(t)=F(t)/m$ is the forcing function
which drives the system.

This is a second order linear \ODE with constant coefficients.
When $f(t)=0$, this equation describes \name{free vibration}.  
When $f(t)\neq 0$, there is an external force driving the system, so the 
equation describes \name{forced vibration}.

%%%%%%%%%%%%%%%%%%%%%%%%%%%%%%%%%%%%%%%%%%%%%%%%%%%%%%%%%%%%%%%%%%%%%%%%%%%%%
\subsection{Free Vibration}

Since $f(t)=0$ for free vibration, the general equation of motion becomes
$$\ddot{x}+2p\dot{x}+\wo^2x=0$$
The nature of the solution depends on the value of the friction constant
$p$.

%%%%%%%%%%%%%%%%%%%%%%%%%%%%%%%%%%%%%%%%%%%%%%%%%%%%%%%%%%%%%%%%%%%%%%%%%%%%%
% Undamped
\subsubsection{Case 1: $p=0$}

The general solution of the differential equation
$$\ddot{x}+\wo^2x=0$$
is
$$x(t)=A\cos\wo t+B\sin\wo t$$
where $A$ and $B$ are arbitrary constants.  This can be written
$$x(t)=\sqrt{A^2+B^2}\left(\frac{A}{\sqrt{A^2+B^2}}\cos\wo t
+\frac{B}{\sqrt{A^2+B^2}}\sin\wo t\right)$$
Since the magnitudes of $A/\sqrt{A^2+B^2}$ and $B/\sqrt{A^2+B^2}$ are
less than or equal to one, and the sum of their squares is one, write
$$\frac{A}{\sqrt{A^2+B^2}}=\cos\phi\qquad\mbox{and}\qquad
\frac{B}{\sqrt{A^2+B^2}}=\sin\phi$$
so the angle $\phi$ is given by $\tan\phi=B/A$.  Then 
\begin{eqnarray*}
\frac{A}{\sqrt{A^2+B^2}}\cos\wo t
+\frac{B}{\sqrt{A^2+B^2}}\sin\wo t
&=&\cos\phi\cos\wo t + \sin\phi\sin\wo t \\&=& \cos(\wo t-\phi)
\end{eqnarray*}
This gives the simpler expression for $x(t)$
$$x(t)=\sqrt{A^2+B^2}\cos\left(\wo t-\phi\right)$$
which is a sinusoid with amplitude $\sqrt{A^2+B^2}$, period $T=2\pi/\wo$ 
and a phase shift of $\phi$.

The system's behaviour is \name{undamped} because there is no friction
to reduce $x(t)$'s amplitude over time.  This is shown in 
figure~\ref{ode fig:ffv.undamped}.


%----------------------------------------------------------------------------
\begin{figure}[t]
\caption{Typical time response $x(t)$ for free vibrations in an undamped
system ($p=0$).}\label{ode fig:ffv.undamped}

\begin{center}

\mbox{}\par

\setlength{\unitlength}{1.7cm}
\begin{pspicture}(0,-1.5)(4.25,1.7)
\psset{xunit=1.7cm,yunit=1.7cm}
% First the curve, so the axes show through
\psplot[linecolor=gray,linewidth=1.5pt,plotstyle=curve]%
{0}{4}{x 0.2 add 180 mul sin 0.8 mul}
% Now the axes and their labels
\psline{<->}(0,-1)(0,1)
\psline{->}(0,0)(4.25,0)
\rput[l](4.3,0){$t$}
\rput[b](0,1.05){$x(t)$}
% Useful points on the curve
% Amplitude
\rput[r](-0.1,0.8){$\sqrt{A^2+B^2}$}
\psline[linewidth=1.5pt](0,0.8)(0.05,0.8)
\rput[r](-0.1,-0.8){$-\sqrt{A^2+B^2}$}
\psline[linewidth=1.5pt](0,-0.8)(0.05,-0.8)
% Phase
\rput[t](0.3,-0.05){$\frac{\phi}{\wo}$}
\psline[linewidth=1.5pt](0.3,0)(0.3,0.05)
% Period
\pcline[offset=0pt]{|-|}(1.3,-0.6)(3.3,-0.6)
\mput*{$T=\frac{2\pi}{\wo}$}
\end{pspicture}
\end{center}
\end{figure}
%----------------------------------------------------------------------------

%%%%%%%%%%%%%%%%%%%%%%%%%%%%%%%%%%%%%%%%%%%%%%%%%%%%%%%%%%%%%%%%%%%%%%%%%%%%%
% Underdamped
\subsubsection{Case 2: $p$ is small and positive}

When $p$ is small and positive, the system is damped because friction reduces
the vibration's amplitude over time.  The differential equation
$$\ddot{x}+2p\dot{x}+\wo^2x=0$$
is solved by assuming $x(t)$ has the form $x(t)=e^{\alpha t}$.  This gives
the characteristic equation
$$\alpha^2+2p\alpha+\wo^2=0$$
whose roots are
$$\alpha=-p\pm\sqrt{p^2-\wo^2}=-p\pm i\sqrt{\wo^2-p^2}$$
Since we are considering the case when $p$ is small and positive, the square
root $\sqrt{p^2-\wo^2}$ is imaginary for $0<p<\wo$.  Writing
$$q=\wo\sqrt{1-\left(\frac{p}{\wo}\right)^2}$$
gives the general solution
\begin{eqnarray*}
x(t)&=&e^{-pt}\left(A\cos qt + B \sin qt\right) \\
&=&Ke^{-pt}\cos\left(qt-\phi\right)
\end{eqnarray*}
where the constants $\phi$ and $K$ are related to the arbitrary constants of
integration by
\begin{eqnarray*}
\tan\phi&=&\frac{B}{A}\\
K&=&\sqrt{A^2+B^2}
\end{eqnarray*}
A typical time response $x(t)$ for underdamped free vibration is shown in 
figure~\ref{ode fig:ffv.underdamped}.

%----------------------------------------------------------------------------
\begin{figure}[t]
\caption{Typical time response $x(t)$ for free vibrations in an underdamped
system ($0<p<\wo$).}\label{ode fig:ffv.underdamped}

\begin{center}

\mbox{}\par

\setlength{\unitlength}{1.7cm}
\begin{pspicture}(0,-1.5)(4.25,1.7)
\psset{xunit=1.7cm,yunit=1.7cm}
% First the curve, so the axes show through
% Do the curve's bounds
\psplot[linestyle=dashed]{0}{4}{2.718 x -0.5 mul exp }
\psplot[linestyle=dashed]{0}{4}{2.718 x -0.5 mul exp neg}
% And now the curve
\psplot[linecolor=gray,linewidth=1.5pt,plotstyle=curve]%
{0}{4}{x 0.2 add 180 mul sin 2.718 x -0.5 mul exp mul}
% Now the axes and their labels
\psline{<->}(0,-1)(0,1)
\psline{->}(0,0)(4.25,0)
\rput[l](4.3,0){$t$}
\rput[b](0,1.05){$x(t)$}
% Useful points on the curve
% Asymptote
\rput[bl](1.5,0.5){$Ke^{-pt}$}
% Period
\pcline[offset=0pt]{|-|}(0.8,-0.8)(2.8,-0.8)
\mput*{$T=\frac{2\pi}{q}$}
\end{pspicture}
\end{center}
\end{figure}
%----------------------------------------------------------------------------

%%%%%%%%%%%%%%%%%%%%%%%%%%%%%%%%%%%%%%%%%%%%%%%%%%%%%%%%%%%%%%%%%%%%%%%%%%%%%
% Critically damped
\subsubsection{Case 3: $p=\wo$}

When $p=\wo$, the characteristic equation has only one solution, and the
system is said to be \name{critically damped}.

This gives $x_1(t)=e^{-pt}$ as one of the solutions of the homogeneous
differential equation.  The second solution has the form $x_2(t)=tx_1(t)$ so
the general solution for critically damped free vibration is
$$x(t)=\left(A+Bt\right)e^{-pt}$$
A typical $x(t)$ for critically damped free vibration is shown in 
figure~\ref{ode fig:ffv.critically damped}.

%----------------------------------------------------------------------------
\begin{figure}[t]
\caption{Typical time response $x(t)$ for free vibrations in a critically 
damped system ($p=\wo$).}\label{ode fig:ffv.critically damped}

\begin{center}

\mbox{}\par

\setlength{\unitlength}{1.7cm}
\begin{pspicture}(0,-1.5)(4.25,1.7)
\psset{xunit=1.7cm,yunit=1.7cm}
% First the curve, so the axes show through
\psplot[linecolor=gray,linewidth=1.5pt,plotstyle=curve]%
{0}{4}{x 0.2 add 2.718 x -1.5 mul exp mul 2 mul}
% Now the axes and their labels
\psline{<->}(0,-1)(0,1)
\psline{->}(0,0)(4.25,0)
\rput[l](4.3,0){$t$}
\rput[b](0,1.05){$x(t)$}
% Useful points on the curve
% Peak
\rput[t](0.466,-0.05){$\frac{1}{p}-\frac{A}{B}$}
\psline[linewidth=1.5pt](0.466,0)(0.466,0.05)
\end{pspicture}
\end{center}
\end{figure}
%----------------------------------------------------------------------------

%%%%%%%%%%%%%%%%%%%%%%%%%%%%%%%%%%%%%%%%%%%%%%%%%%%%%%%%%%%%%%%%%%%%%%%%%%%%%
% Overdamped 
\subsubsection{Case 4: $p>\wo$}

When $p>\wo$, the system is said to be \name{overdamped}.  The 
characteristic equation has two solutions
\begin{eqnarray*}
\alpha_1&=&-p+\sqrt{p^2-\wo^2}\\
\alpha_2&=&-p-\sqrt{p^2-\wo^2}
\end{eqnarray*}
These are both real and negative, giving solutions which decay with time
without oscillating
$$x(t)=Ae^{\alpha_1t}+Be^{\alpha_2t}$$
Since $\alpha_2$ is more negative than $\alpha_1$, the $e^{\alpha_2t}$ term
decays to zero more quickly than the $e^{\alpha_1t}$ term.  So for large
$t$, $x(t)$ goes asymptotically like $Ae^{\alpha_1t}$.
A typical $x(t)$ for overdamped free vibration is shown in 
figure~\ref{ode fig:ffv.overdamped}.

%----------------------------------------------------------------------------
\begin{figure}[t]
\caption{Typical time response $x(t)$ for free vibrations in an overdamped
system ($p>\wo$).}\label{ode fig:ffv.overdamped}

\begin{center}

\mbox{}\par

\setlength{\unitlength}{1.7cm}
\begin{pspicture}(0,-1.5)(4.25,1.7)
\psset{xunit=1.7cm,yunit=1.7cm}
% First the curve, so the axes show through
\psplot[linecolor=gray,linewidth=1.5pt,plotstyle=curve]%
{0}{4}{2.718 x -2 mul exp 2 mul 2.718 x -1 mul exp sub}
% Now the axes and their labels
\psline{<->}(0,-1)(0,1)
\psline{->}(0,0)(4.25,0)
\rput[l](4.3,0){$t$}
\rput[b](0,1.05){$x(t)$}
\end{pspicture}
\end{center}
\end{figure}
%----------------------------------------------------------------------------

Note that for all three cases when damping is present (when $p>0$),
$x(t)\rightarrow 0$ as $t\rightarrow\infty$.  That is, the solutions are
\name{transient}, decaying to zero over time.

%%%%%%%%%%%%%%%%%%%%%%%%%%%%%%%%%%%%%%%%%%%%%%%%%%%%%%%%%%%%%%%%%%%%%%%%%%%%%
\subsection{Forced Vibration}

Since $f(t)\neq 0$ for forced vibration, solutions of the differential
equation
$$\ddot{x}+2p\dot{x}+\wo^2x=f(t)$$
have the form
$$x(t)=x_h(t)+x_p(t)$$
The solution to the homogeneous equation, $x_h(t)$, is given by one of the
four cases of free vibration.  For $p>0$, the homogeneous solution is
transient and tends to zero as $t\rightarrow\infty$.  The particular
solution $x_p(t)$ depends on the forcing function $f(t)$, and this is the 
dominant contribution to $x(t)$ as $t\rightarrow\infty$.

%%%%%%%%%%%%%%%%%%%%%%%%%%%%%%%%%%%%%%%%%%%%%%%%%%%%%%%%%%%%%%%%%%%%%%%%%%%%%
\subsubsection{Resonance}

When the forcing function is sinusoidal, for example $f(t)=\sin\omega t$,
the differential equation is
$$\ddot{x}+2p\dot{x}+\wo^2x=\sin\omega t$$
When $p>0$, the homogeneous solution is transient and tends to zero for
large $t$.  The particular solution $x_p(t)$ is non-transient, and has the
general form
$$x_p(t)=A\cos\omega t+B\sin\omega t$$
so that the first and second derivatives of $x_p(t)$ are
\begin{eqnarray*}
\dot{x}_p(t)&=&\omega\left(-A\sin\omega t+B\cos\omega t\right)\\
\ddot{x}_p(t)&=&-\omega^2\left(A\cos\omega t+B\sin\omega t\right)
\end{eqnarray*}
Substituting these into the differential equation shows that
$$\left(\wo^2-\omega^2\right)\left(A\cos\omega t+B\sin\omega t\right)
+2p\omega\left(-A\sin\omega t+B\cos\omega t\right)=\sin\omega t$$
Equating coefficients of $\sin\omega t$ and $\cos\omega t$ gives two
simultaneous equations
\begin{eqnarray*}
\left(\wo^2-\omega^2\right)A+2p\omega B&=&0 \\
\left(\wo^2-\omega^2\right)B-2p\omega A&=&1
\end{eqnarray*}
whose solutions are 
\begin{eqnarray*}
A&=&\frac{-2p\omega}{\left(\wo^2-\omega^2\right)^2+4p^2\omega^2} \\
B&=&\frac{\wo^2-\omega^2}{\left(\wo^2-\omega^2\right)^2+4p^2\omega^2}
\end{eqnarray*}
The $\cos\omega t$ and $\sin\omega t$ parts of $x_p(t)$ can be combined into
a single sinusoid using the same trick as before.
Written this way, the particular solution is
\begin{eqnarray*}
x_p(t)&=&\sqrt{A^2+B^2}\left(
\frac{A}{\sqrt{A^2+B^2}}\cos\omega t +
\frac{B}{\sqrt{A^2+B^2}}\sin\omega t\right)\\
&=&\alpha\left(-\sin\theta\cos\omega t +\cos\theta\sin\omega t\right)\\
&=&\alpha\sin\left(\omega t-\theta\right)
\end{eqnarray*}
The amplitude $\alpha$ is called the \name{amplification factor}
\begin{eqnarray*}
\alpha&=&\sqrt{A^2+B^2}\\
&=&\frac{1}{\sqrt{\left(\wo^2-\omega^2\right)^2+4p^2\omega^2}}
\end{eqnarray*}
and the phase shift $\theta$ is called the \name{phase-lag}
$$\tan\theta=-\frac{A}{B}=\frac{2p\omega}{\wo^2-\omega^2}$$
Note that this definition of $\theta$ is slightly different from the
corresponding definition of $\phi$ in the equation for
undamped free vibration.

%%%%%%%%%%%%%%%%%%%%%%%%%%%%%%%%%%%%%%%%%%%%%%%%%%%%%%%%%%%%%%%%%%%%%%%%%%%%%
\subsubsection{Amplification Factor}

The amplitude of the system's response to a sinusoidal forcing function
depends on the relationship between the natural frequency of the undamped
free vibration, $\wo$, the frequency of the forcing function, $\omega$, and
the damping coefficient, $p$.

For constant $\wo$ and $p$, the amplitude factor reaches its maximum when
$d\alpha/d\omega=0$.  Since
$$\der{\alpha}{\omega}=-\frac{-2\omega\left(\wo^2-\omega^2\right)
+4p^2\omega}{\left(\left(\wo^2-\omega^2\right)^2+4p^2\omega^2\right)^{3/2}}$$
$\alpha$ is at its maximum when the numerator of $d\alpha/d\omega$ is zero.
This occurs when 
$$\omega=0\qquad\mbox{and}\qquad \omega^2-\wo^2+2p^2=0$$
If $p<\wo/\sqrt{2}$, the local maxima of $\alpha$ occur when $\omega=0$
and when 
$$\omega=\wo\sqrt{1-2\left(\frac{p}{\wo}\right)^2}$$
If $p>\wo/\sqrt{2}$, the only peak in the resonance curve is at $\omega=0$.

The amplification factor is plotted as a function of $\omega$ for a number
of different values of $p$ in figure~\ref{ode fig:ffv.resonance curves}.
The $\omega$ axis has been scaled by $1/\wo$ and the $\alpha$ axis has been
scaled by $\wo^2$ to make the curves dimensionless.  Note that as $p$
increases from zero to $\wo/\sqrt{2}$, the peak of the resonance curve 
moves from $\omega/\wo=1$ back towards the origin along the dashed line.  
Also note that the peak of the resonance curve tends to $\infty$ as 
$p\rightarrow 0$.

%----------------------------------------------------------------------------
\begin{figure}[t]
\caption{Resonance curves showing the amplitude factor $\alpha$ as a
function of frequency $\omega$. The maxima of the resonance curves occur
when $\omega/\wo=\protect\sqrt{1-2\left(p/\wo\right)^2}$ for 
$p/\wo<1/\protect\sqrt{2}$.}\label{ode fig:ffv.resonance curves}

\begin{center}

\setlength{\unitlength}{2cm}
\begin{pspicture}(-0.3,-0.3)(6,6.6)
\psset{xunit=3cm,yunit=1.5cm}
% The asymptote for maxima at y=1/sqrt{1-x^4}
%\psline[linestyle=dashed,dash=3pt 3pt]{-}(1,0)(1,4)
\psplot[linestyle=dashed,dash=3pt 3pt,plotstyle=curve]{0}{0.984}%
{1 x dup mul dup mul sub sqrt 1 exch div}
% First the curve, so the axes show through
% p/wo=2
\psplot[linecolor=gray,linewidth=1pt,plotstyle=curve]{0}{2}%
{1 x x mul sub dup mul 4 2 dup mul mul x dup mul mul add sqrt 1 exch div}
% p/wo=1/sqrt{2}
\psplot[linecolor=gray,linewidth=1.5pt,plotstyle=curve]{0}{2}%
{1 x x mul sub dup mul 4 0.707 dup mul mul x dup mul mul add sqrt 1 exch div}
% p/wo=0.5
\psplot[linecolor=gray,linewidth=1pt,plotstyle=curve]{0}{2}%
{1 x x mul sub dup mul 4 0.5 dup mul mul x dup mul mul add sqrt 1 exch div}
% p/wo=0.25
\psplot[linecolor=gray,linewidth=1pt,plotstyle=curve]{0}{2}%
{1 x x mul sub dup mul 4 0.25 dup mul mul x dup mul mul add sqrt 1 exch div}
% p/wo=0.15
\psplot[linecolor=gray,linewidth=1pt,plotstyle=curve]{0}{2}%
{1 x x mul sub dup mul 4 0.15 dup mul mul x dup mul mul add sqrt 1 exch div}
% p/wo=0
\psplot[linecolor=gray,linewidth=1.5pt,plotstyle=curve]{0}{0.866}%
{1 x x mul sub dup mul 4 0 dup mul mul x dup mul mul add sqrt 1 exch div}
\psplot[linecolor=gray,linewidth=1.5pt,plotstyle=curve]{1.118}{2}%
{1 x x mul sub dup mul 4 0 dup mul mul x dup mul mul add sqrt 1 exch div}
% Now the axes and their labels
\psaxes[tickstyle=top,linewidth=1pt]{->}(2,4)
\rput[l](2.05,0){$\displaystyle\frac{\omega}{\wo}$}
\rput[b](0,4.1){$\alpha\wo^2$}
% Some labels showing p/wo
\rput(0.9,1.2){$0<\frac{p}{\wo}<\frac{1}{\sqrt{2}}$}
\rput(0.6,0.4){$\frac{p}{\wo}>\frac{1}{\sqrt{2}}$}
%
\rput[l](1.8,1.5){\rnode{A}{$\frac{p}{\wo}=\frac{1}{\sqrt{2}}$}}
\rput(1.5,0.406){\rnode{B}{}}\ncline{->}{A}{B}
%
\rput[l](1.8,3){\rnode{A}{$\frac{p}{\wo}=0$}}
\rput(1.155,3){\rnode{B}{}}\ncline{->}{A}{B}
\end{pspicture}
\end{center}
\end{figure}
%----------------------------------------------------------------------------

%----------------------------------------------------------------------------
\begin{figure}[t]
\caption{LRC series resonant circuit.}\label{ode fig:ffv.LRC}

\begin{center}

\setlength{\unitlength}{1cm}
\begin{pspicture}(-1,0)(3.5,3.5)
\psset{xunit=1cm,yunit=1cm}
% Lines around the circuit
\psline{-}(0,2)(0,2.5)(1,2.5)
\psline{-}(2,2.5)(2.5,2.5)(2.5,2)
\psline{-}(2.5,1)(2.5,0.5)(1.667,0.5)
\psline{-}(1.333,0.5)(0,0.5)(0,1)
% The resistor
\pszigzag[linearc=0,coilwidth=0.333,coilarm=0]{-}(1,2.5)(2,2.5)
\rput[b](1.5,2.833){$R$}
% The inductor
%\pscoil[coilwidth=0.333,coilheight=0.75,coilarm=0]{-}(2.5,1)(2.5,2)
\psarc{-}(2.5,1.125){0.125}{270}{450}
\psarc{-}(2.5,1.375){0.125}{270}{450}
\psarc{-}(2.5,1.625){0.125}{270}{450}
\psarc{-}(2.5,1.875){0.125}{270}{450}
\rput[l](2.85,1.5){$L$}
% The capacitor
\psline{-}(1.333,0.167)(1.333,0.833)
\psline{-}(1.667,0.167)(1.667,0.833)
\rput[t](1.5,0.067){$C$}
\rput[tl](1.767,0.4){$\scriptscriptstyle +Q$}
\rput[tr](1.233,0.4){$\scriptscriptstyle -Q$}
% The voltage source
\pscircle(0,1.5){0.5}
\rput(0,1.5){$E(t)$}
\rput[br](-0.1,2.1){$\scriptscriptstyle +$}
\rput[tr](-0.1,0.9){$\scriptscriptstyle -$}
% Finally the arrow with I(t)
\psline{->}(0.2,2.7)(0.8,2.7)
\rput[b](0.5,2.9){$I(t)$}
\end{pspicture}
\end{center}
\end{figure}
%----------------------------------------------------------------------------

%============================================================================
\begin{example}
A series LRC circuit consists of a resistor $R$, inductor $L$ and capacitor $C$ in
series across a time-varying voltage source $E(t)$.  This is illustrated in
figure~\ref{ode fig:ffv.LRC}.

Summing voltage drops around the loop gives the equation
$$-E(t)+I(t)R+L\der{I(t)}{t}+\frac{Q(t)}{C}=0$$
Since $I(t)=\dbd{t}Q(t)$, differentiating with respect to $t$ gives
$$L\dder{I}{t}+R\der{I}{t}+\frac{1}{C}I=\der{E}{t}$$
Dividing by $L$ gives the standard second order differential equation
for forced vibration 
$$\dder{I}{t}+\frac{R}{L}\der{I}{t}+\frac{1}{LC}I=\frac{1}{L}\der{E}{t}$$
which is the force vibration differential equation with $p=R/2L$, $\wo=1/\sqrt{LC}$ 
and $f(t)=\frac{1}{L}\der{E}{t}$.

If a voltage of frequency $\omega$
$$E(t)=V\cos\omega t$$
is applied, the current $I(t)$ 
reaches its maximum when the LRC circuit is tuned to $\omega$, provided that 
$p/\wo<1/\sqrt{2}$.  If so, $\omega$ is given by
$$\omega=\wo\sqrt{1-2\left(\frac{p}{\wo}\right)^2}=
\frac{1}{\sqrt{LC}}\sqrt{1-\frac{R^2C}{2L}}$$
\end{example}
%============================================================================

%============================================================================
\begin{exercise}
Exercises 5.1 to 5.4 and the review exercises of Zill, pp. 219--251, have
more examples of free and forced vibration.
\end{exercise}
%============================================================================
