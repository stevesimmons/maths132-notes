%%%%%%%%%%%%%%%%%%%%%%%%%%%%%%%%%%%%%%%%%%%%%%%%%%%%%%%%%%%%%%%%%%%%%%%%%%%
%
%			Mathematics 132 Course Notes
%
%			 Department of Mathematics,
%   			  University of Melbourne
%
%		Stephen Simmons			Lee White
%
% 8 Feb-96 SS: Updated with corrections from semester 2, 1995
%
%%%%%%%%%%%%%%%%%%% Copyright (C) 1995-96 Stephen Simmons %%%%%%%%%%%%%%%%%

\chapter{Dynamics of Systems of Particles}
\label{mpd chp}

%%%%%%%%%%%%%%%%%%%%%%%%%%%%%%%%%%%%%%%%%%%%%%%%%%%%%%%%%%%%%%%%%%%%%%%%%%%%%
\section{Centre of Mass and Momentum}

Consider a system of particles where $\vect{r}_i$ is the position vector of
the $i^{\rm th}$ particle whose mass is $m_i$.  Then the position vector of
the \name{centre of mass} of the system of particles is
$$\vect{r}_{\rm cm}=\frac{\ds\sum_i m_i\vect{r}_i}{\ds\sum_i m_i}$$
The total mass of the system of particles is
$$M=\sum_i m_i$$

%============================================================================
\begin{example}
The centre of mass of a two particle system is
$$\vect{r}_{\rm cm}=\frac{m_1\vect{r}_1+m_2\vect{r}_2}{m_1+m_2}$$
If, for example, the two particles are the Sun and a planet,
$$\vect{r}_{\rm cm}=\frac{M_s\vect{r}_s+m_p\vect{r}_p}{M_s+m_p}$$
\end{example}
%============================================================================

The linear momentum of the $i^{\rm th}$ particle is
$$\vect{p}_i=m_i\dot{\vect{r}}_i$$
so the total linear momentum of the system of particles is
$$\vect{P}=\sum_i\vect{p}_i=\sum_im_i\dot{\vect{r}}_i$$
If the total mass of the system is constant with time, this can be written 
as the total mass times the velocity of the centre of mass
$$\vect{P}=\dbd{t}\sum_im_i\vect{r}_i=M\dbd{t}\left(\frac{1}{M}
\sum_im_i\vect{r}_i\right)=M\dot{\vect{r}}_{\rm cm}$$

Newton's second law applies to the system as a whole.  Newton's second law
for an individual particle is
$$m_i\ddot{\vect{r}}_i=\vect{F}_i+\sum_{j\neq i}\vect{F}_{ij}$$
where $\vect{F}_i$ is the total external force on the $i^{\rm th}$ particle
and $\vect{F}_{ij}$ is the force exerted on the $i^{\rm th}$ particle by the
$j^{\rm th}$ particle (note that by Newton's third law,
$\vect{F}_{ij}=-\vect{F}_{ji}$).

%============================================================================
\begin{theorem}
Newton's second law for the system of particles is
$$M\ddot{\vect{r}}_{\rm cm}=\sum_i\vect{F}_i$$
\end{theorem}
\begin{proof}
Start with
$$M\vect{r}_{\rm cm}=\sum_i m_i\vect{r}_i$$
and differentiate twice with respect to time to give
$$M\ddot{\vect{r}}_{\rm cm}=\sum_i m_i\ddot{\vect{r}}_i$$
Then from Newton's second law for the $i^{\rm th}$ particle,
$$M\ddot{\vect{r}}_{\rm cm}
=\sum_i\left(\vect{F}_i+\sum_{j\neq i}\vect{F}_{ij}\right)
=\sum_i\vect{F}_i+\sum_i\sum_{j\neq i}\vect{F}_{ij}$$
The second summation is zero because $\vect{F}_{ij}=-\vect{F}_{ji}$, leaving
$$M\ddot{\vect{r}}_{\rm cm}=\sum_i\vect{F}_i$$
as required.
\end{proof}
%============================================================================

Note that when no net external force acts on the system
\typeout{LRW says this should be 'no nett external force', but I think
'net' is correct.}
$$\sum_i\vect{F}_i=0$$
and
$$M\ddot{\vect{r}}_{\rm cm}=0$$
so that $\vect{P}=M\dot{\vect{r}}_{\rm cm}$ is constant.  Therefore a system
of particles under no net external force has constant linear momentum
$\vect{P}$.

%%%%%%%%%%%%%%%%%%%%%%%%%%%%%%%%%%%%%%%%%%%%%%%%%%%%%%%%%%%%%%%%%%%%%%%%%%%
\section{Kinetic and Potential Energy}

The \name{kinetic energy} of a system is the sum of the kinetic energies of
the individual particles
$$KE=\sum_i\frac{1}{2}m_i\dot{\vect{r}}_i\cdot\dot{\vect{r}}_i$$
The \name{potential energy} of external conservative forces is
$$PE_{\rm ext}=\sum_iV_{\rm ext}(\vect{r}_i)$$
where the potential fuction $V_{\rm ext}(\vect{r})$ is defined in terms of
the external force by
$$-\grad V_{\rm ext}(\vect{r})=F(\vect{r})$$

Internal forces are usually central forces between each pair of particles so
that there is associated with each pair of particles an internal potential
energy $V_{\rm int}(r_{ij})$ which is a function of separation distance
$r_{ij}=\left|\vect{r}_i-\vect{r}_j\right|$ between the pair.  The internal
potential energy is
$$PE_{\rm int}=\frac{1}{2}\sum_i\sum_j V_{\rm int}(r_{ij})$$
Here the summation counts each pair of particles twice, hence the factor of
$1/2$.

The \name{total mechanical energy} of the system is
\begin{eqnarray*}
E&=&KE+PE_{\rm ext}+PE_{\rm int}\\
&=&\sum_i\frac{1}{2}m_i\dot{\vect{r}}_i\cdot\dot{\vect{r}}_i+
\sum_iV_{\rm ext}(\vect{r}_i)+\frac{1}{2}\sum_i\sum_j V_{\rm int}(r_{ij})
\end{eqnarray*}

%============================================================================
\begin{theorem}
In the absence of external extraneous forces
$$\der{E}{t}=0$$
so $E$ is a constant of the motion of the system and mechanical energy is
conserved.
\end{theorem}
%============================================================================

%============================================================================
\begin{example}[Reduced Mass of a Two Body System]
When no external force acts, $\dot{\vect{r}}_{\rm cm}$ is constant.  A
coordinate system with the centre of mass at the origin would therefore be an
inertial frame of reference.

If $\vect{r}_1$ and $\vect{r}_2$ are the positions of two
particles relative to this coordinate system, then
$$m_1\vect{r}_1+m_2\vect{r}_2=0$$
Define the position vector of particle 1 with respect to particle 2 as
$$\vect{r}=\vect{r}_1-\vect{r}_2
=\vect{r}_1\left(1+\frac{m_1}{m_2}\right)$$
Newton's second law holds in inertial frames of reference, so that
$$m_1\ddot{\vect{r}}_1=\vect{F}_{12}$$
If the force that particle 2 exerts on particle 1 is a function of the
distance between them and directed along the line of their centres
$$\vect{F}_{12}=f(r)\rvect$$
then
$$m_1\ddot{\vect{r}}_1=f(r)\rvect$$
Substituting
$$\ddot{\vect{r}}_1=\frac{m_2}{m_1+m_2}\vect{r}$$
we have
$$\mu\ddot{\vect{r}}=f(r)\rvect$$
where $\mu$ is the \name{reduced mass} of the system
$$\mu=\frac{m_1m_2}{m_1+m_2}$$
This is equivalent to motion in a central force field of a particle of mass
$\mu$.
\end{example}
%============================================================================

%============================================================================
\begin{example}[Sun-Planet System]
It was incorrect to assume that the Sun stays fixed at the origin as a
planet orbits it.  Instead we should use the centre of mass of the
Sun/planet system as the origin.  In this case, the position vector of the
planet relative to the Sun is
$$\vect{r}=\vect{r}_p-\vect{r}_s$$
and therefore
$$\mu\ddot{\vect{r}}=-\frac{GM_sm_p}{r^2}\rvect$$
where
$$\mu=\frac{M_sm_p}{M_s+m_p}\approx m_p$$
when $M_s\gg m_p$.

Our previous calculations are all valid if we replace $GM_s$ by
$G(M_s+m_p)$.
\end{example}
%============================================================================

%%%%%%%%%%%%%%%%%%%%%%%%%%%%%%%%%%%%%%%%%%%%%%%%%%%%%%%%%%%%%%%%%%%%%%%%%%%%%
\section{Torque}

The \name{angular momentum} of the $i^{\rm th}$ particle about the origin is
$$\vect{L}_i=\vect{r}_i\times\vect{p}_i
=\vect{r}_i\times(m_i\dot{\vect{r}}_i)$$
The angular momentum of the system of particles about the origin is
$$\vect{L}=\sum_i\vect{L}_i=\sum_i\vect{r}_i\times\vect{p}_i$$

%============================================================================
\begin{theorem}
In an inertial frame of reference
$$\der{\vect{L}}{t}=\vect{N}$$
where
$$\vect{N}=\sum_i\vect{r}_i\times\vect{F}_i$$
is the \name{total external torque} on the system of particles about the
origin.
\end{theorem}
\begin{proof}
From the definition of $\vect{L}$
$$\der{\vect{L}}{t}=\dbd{t}\sum_i\vect{r}_i\times m_i\dot{\vect{r}}_i$$
If the masses $m_i$ are constant with time, $\der{m_i}{t}=0$ so
$$\der{\vect{L}}{t}=\sum_i\vect{r}_i\times m_i\ddot{\vect{r}}_i$$
Now $m_i\ddot{\vect{r}}_i$ can be written in terms of the internal and
external forces to give
$$\der{\vect{L}}{t}=\sum_i\vect{r}_i\times\left(\vect{F}_i+\sum_j
\vect{F}_{ij}\right)$$
The second summation is zero because it is composed of pairs of the form
$$\vect{r}_i\times\vect{F}_{ij}+\vect{r}_j\times\vect{F}_{ji}
=(\vect{r}_i-\vect{r}_j)\times\vect{F}_{ij}=0$$
which are zero if $\vect{F}_{ij}$ lies along the line of the two particles'
centres.  

Therefore
$$\der{\vect{L}}{t}=\sum_i\vect{r}_i\times\vect{F}_i=\vect{N}$$
\end{proof}
%============================================================================

Note that when there is no external torque, $\vect{N}=0$ so $\vect{L}$ is
constant and angular momentum is conserved.  $\vect{N}=0$ when all of 
the external forces are zero, hence angular momentum is conserved when no
external forces act on the system.

%============================================================================
\begin{example}
\label{mpd ex:L and N}
Angular momentum and torque depend on the origin of the coordinate system.

Define $\bar{\vect{r}}_i$ to be the position of the $i^{\rm th}$ particle 
relative to a coordinate system fixed on the centre of mass.
$$\bar{\vect{r}}_i=\vect{r}_i-\vect{r}_{\rm cm}$$
Therefore
$$\vect{r}_i=\bar{\vect{r}}_i+\vect{r}_{\rm cm}$$
and
$$\dot{\vect{r}}_i=\dot{\bar{\vect{r}}}_i+\dot{\vect{r}}_{\rm cm}$$

Now the angular momentum is
$$\vect{L}=\sum_i\vect{r}_i\times(m_i\dot{\vect{r}}_i)
=\sum_i m_i(\bar{\vect{r}}_i+\vect{r}_{\rm cm})
\times(\dot{\bar{\vect{r}}}_i+\dot{\vect{r}}_{\rm cm})$$
Expanding the cross-product of the sums gives the sum of four
cross-products. 
\begin{eqnarray*}
\vect{L}&=&
\left(\sum_i m_i\right)\vect{r}_{\rm cm}\times\dot{\vect{r}}_{\rm cm}
+\vect{r}_{\rm cm}\times\left(\sum_im_i\dot{\bar{\vect{r}}}_i\right)\\
&&{}
+\left(\sum_i m_i\bar{\vect{r}}_i\right)\times\dot{\vect{r}}_{\rm cm}
+\sum_i \bar{\vect{r}}_i\times m_i\dot{\bar{\vect{r}}}_i
\end{eqnarray*}
Now $\sum_i m_i\bar{\vect{r}}_i=0$ from the definition of centre of mass
$$\sum_i m_i\bar{\vect{r}}_i=\sum_im_i\vect{r}_i-\sum_im_i\vect{r}_{\rm cm}
=M\vect{r}_{\rm cm}-M\vect{r}_{\rm cm}=0$$
Differentiating shows that $\sum_i m_i\dot{\bar{\vect{r}}}_i=0$, which
leaves
$$\vect{L}=
\left(\sum_i m_i\right)\vect{r}_{\rm cm}\times\dot{\vect{r}}_{\rm cm}
+\sum_i \bar{\vect{r}}_i\times m_i\dot{\bar{\vect{r}}}_i$$
which is the sum of the angular momentum of the centre of mass about the 
origin and the angular momentum of the system about the centre of mass
$$\vect{L}=
\vect{r}_{\rm cm}\times(M\dot{\vect{r}}_{\rm cm})
+\sum_i \bar{\vect{r}}_i\times (m_i\dot{\bar{\vect{r}}}_i)$$
\end{example}
%============================================================================


The \name{kinetic energy} of the system $T$ is the sum of the kinetic
energies of the individual particles
$$T=\frac{1}{2}\sum_im_i\dot{\vect{r}}_i\cdot\dot{\vect{r}}_i=\sum_i T_i$$
where $T_i=\frac{1}{2}m_i\dot{\vect{r}}_i\cdot\dot{\vect{r}}_i$ is the
kinetic energy of the $i^{\rm th}$ particle.

%============================================================================
\begin{example}
The kinetic energy of a system is the sum of the kinetic energy due to the
translation of the centre of mass plus the kinetic energy due to the system's
motion about the centre of mass.

To see this, write
$$\dot{\vect{r}}_i=\dot{\bar{\vect{r}}}_i+\dot{\vect{r}}_{\rm cm}$$

Now the kinetic energy is
$$T=\frac{1}{2}\sum_i m_i
(\dot{\bar{\vect{r}}}_i+\dot{\vect{r}}_{\rm cm})\cdot
(\dot{\bar{\vect{r}}}_i+\dot{\vect{r}}_{\rm cm})$$
Expanding the dot-product of the sums gives the sum of four
dot-products. 
\begin{eqnarray}
\vect{T}&=&
\frac{1}{2}\left(\sum_i m_i\right)\dot{\vect{r}}_{\rm cm}\cdot\dot{\vect{r}}_{\rm cm}
+\frac{1}{2}\dot{\vect{r}}_{\rm cm}\cdot\left(\sum_im_i\dot{\bar{\vect{r}}}_i\right)
\\
&&{}+\frac{1}{2}\left(\sum_i m_i\dot{\bar{\vect{r}}}_i\right)\cdot\dot{\vect{r}}_{\rm cm}
+\frac{1}{2}\sum_i \dot{\bar{\vect{r}}}_i\cdot m_i\dot{\bar{\vect{r}}}_i
\end{eqnarray}
As in the previous example, $\sum_i m_i\dot{\bar{\vect{r}}}_i=0$, which
leaves
$$\vect{T}
=\frac{1}{2}\left(\sum_i m_i\right)\dot{\vect{r}}_{\rm cm}\cdot\dot{\vect{r}}_{\rm cm}
+\frac{1}{2}\sum_i \dot{\bar{\vect{r}}}_i\cdot m_i\dot{\bar{\vect{r}}}_i$$
which is the sum of the kinetic energy of translation of the system as a
whole and the kinetic energy of the motion of the system about the 
centre of mass
$$\vect{T}
=\frac{1}{2}M\dot{\vect{r}}_{\rm cm}\cdot\dot{\vect{r}}_{\rm cm}
+\frac{1}{2}\sum_i m_i\dot{\bar{\vect{r}}}_i\cdot \dot{\bar{\vect{r}}}_i$$
\end{example}
%============================================================================

From example \ref{mpd ex:L and N}, the angular momentum $\vect{L}$ relative
to an inertial origin can be written as
$$\vect{L}=\vect{L}_{\rm cm}+\vect{L}_{\rm r}$$
where $\vect{L}_{\rm r}$ is the system's angular momentum relative to the
centre of mass.  $\vect{L}$ is related to the total external torque relative
to the origin according to
$$\vect{N}=\der{\vect{L}}{t}=\sum_i\vect{r}_i\times\vect{F}_i$$
The following theorem shows that a similar result applies to the angular
momentum about the centre of mass, even though the centre of mass may not be
an inertial reference frame.

%============================================================================
\begin{theorem}
The total external torque relative to the centre of mass is
$$\vect{N}_{\rm r}=\der{\vect{L}_{\rm r}}{t}
=\sum_i \bar{\vect{r}}_i\times\vect{F}_i$$
\end{theorem}
\begin{proof}
From the definition of
$$\vect{L}_{\rm cm}=\vect{r}_{\rm cm}\times M\dot{\vect{r}}_{\rm cm}$$
differentiate to give
$$\der{\vect{L}_{\rm cm}}{t}
=\vect{r}_{\rm cm}\times M\ddot{\vect{r}}_{\rm cm}
=\vect{r}_{\rm cm}\times \sum_i\vect{F}_i$$
But
$$\vect{L}=\vect{L}_{\rm cm}+\vect{L}_{\rm r}$$
so
$$\der{\vect{L}}{t}=\der{\vect{L}_{\rm cm}}{t}+\der{\vect{L}_{\rm r}}{t}$$
Therefore
$$\der{\vect{L}_{\rm r}}{t}=\der{\vect{L}}{t}-\der{\vect{L}_{\rm cm}}{t}$$
This is
$$\der{\vect{L}_{\rm r}}{t}
=\sum_i\vect{r}_i\times\vect{F}_i-\vect{r}_{\rm cm}\times\sum_i\vect{F}_i
=\sum_i(\vect{r}_i-\vect{r}_{\rm cm})\times\vect{F}_i$$
which gives
$$\der{\vect{L}_{\rm r}}{t}=\sum_i \bar{\vect{r}}_i\times\vect{F}_i$$
as required.
\end{proof}
%============================================================================
