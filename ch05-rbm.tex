%%%%%%%%%%%%%%%%%%%%%%%%%%%%%%%%%%%%%%%%%%%%%%%%%%%%%%%%%%%%%%%%%%%%%%%%%%%
%
%			Mathematics 132 Course Notes
%
%			 Department of Mathematics,
%   			  University of Melbourne
%
%		Stephen Simmons			Lee White
%
% 8 Feb-96 SS: Updated with corrections from semester 2, 1995
%
%%%%%%%%%%%%%%%%%%% Copyright (C) 1995-96 Stephen Simmons %%%%%%%%%%%%%%%%%

%%%%%%%%%%%%%%%%%%%%%%%%%%%%%%%%%%%%%%%%%%%%%%%%%%%%%%%%%%%%%%%%%%%%%%%%%%%%%
\chapter{Rigid Body Motion}
\label{rbm chp}

A system of particles behaves as a rigid body if, during the motion of the
system, the particles do not change their relative internal positions.  Even
in solids, the particles vibrate about their equilibrium positions, so there
is always internal vibrational kinetic energy.

%%%%%%%%%%%%%%%%%%%%%%%%%%%%%%%%%%%%%%%%%%%%%%%%%%%%%%%%%%%%%%%%%%%%%%%%%%%%%
\section{Continuous Rigid Bodies}

If the body is large compared with the internal particle separations, we can
regard it as a continuum with negligible error.  The mass density of the
material in the body then becomes the \name{mass per unit volume}, which is
denoted by the symbol $\rho$.

This is a meaningful concept provided we are interested in large volumes
containing many particles.

The position of the centre of mass of a system of particles
$$\vect{r}_{\rm cm}=\frac{1}{M}\sum_im_i\vect{r}_i$$
becomes an integral
$$\vect{r}_{\rm cm}=\frac{1}{M}\int_V \rho\,\vect{r}\,dV$$
where the integral is taken over the whole of the volume of the rigid body,
and $\rho(\vect{r})$ is the function giving the body's mass per unit volume
as a function of position $\vect{r}$.  The total mass of the body is
$$M=\int_V \rho\,dV$$

If the body is uniform so that $\rho$ is constant, the formula for the
centre of mass becomes
$$\vect{r}_{\rm cm}=\frac{1}{V}\int_V \vect{r}\,dV$$

In a Cartesian coordinate system, $dV=dx\,dy\,dz$ and
$$\vect{r}_{\rm cm}=\frac{1}{M}\int\!\!\int\!\!\int \rho(x,y,z)
\vect{r}(x,y,z)\,dx\,dy\,dz$$
or, for a uniform body
$$\vect{r}_{\rm cm}=\frac{1}{V}\int\!\!\int\!\!\int 
\vect{r}(x,y,z)\,dx\,dy\,dz$$

The centres of mass of the individual coordinates for a uniform body are
$$x_{\rm cm}=\frac{1}{V}\int_V x\,dV$$
with similar expressions for $y_{\rm cm}$ and $z_{\rm cm}$.

For a thin sheet, called a \name{lamina}, the volume element is $dV=dA\,t$ where
$dA$ is the element of area on the sheet and $t$ is the thickness of the
sheet.  Then
$$\vect{r}_{\rm cm}=\frac{\ds\int_A\bar{\rho}\vect{r}\,dA}
{\ds\int_A\bar{\rho}\,dA}$$
where $\bar{\rho}=\rho t$ is the mass per unit area of the lamina.  The
total mass of the sheet is 
$$M=\int_A\bar{\rho}\,dA$$

If the rigid body has a plane of symmetry, $\vect{r}_{\rm cm}$ lies in that
plane.  If the rigid body has a line of symmetry, $\vect{r}_{\rm cm}$ lies
on that line.


%============================================================================
\begin{example}
For a solid hemisphere with its centre at the origin, radius $a$ and $z\geq0$
$$x_{\rm cm}=y_{\rm cm}=0\qquad z_{\rm cm}=\frac{3a}{8}$$
For a hemispherical shell with its centre at the origin, radius $a$ 
and $z\geq0$
$$x_{\rm cm}=y_{\rm cm}=0\qquad z_{\rm cm}=\frac{a}{2}$$
For a semicircular line of radius $a$
$$x_{\rm cm}=y_{\rm cm}=0\qquad z_{\rm cm}=\frac{2a}{\pi}$$
For a semicircular lamina of radius $a$
$$x_{\rm cm}=y_{\rm cm}=0\qquad z_{\rm cm}=\frac{4a}{3\pi}$$
For an isosceles triangular lamina with height $h$
$$x_{\rm cm}=y_{\rm cm}=0\qquad z_{\rm cm}=\frac{h}{3}$$
\end{example}
%============================================================================

%----------------------------------------------------------------------------
\begin{figure}\centering
\caption{This semicircular lamina in the $xz$ plane has its centre at the
origin and has radius $a$.  The shaded region is the strip from $z$ to
$z+dz$.}
\label{rbm fig:scl}

\psset{unit=5cm}
\begin{pspicture}(-0.7,-0.3)(0.7,0.75)
% The shaded region
\psclip{\psarc[linewidth=1pt,linestyle=none]{-}(0,0){0.5}{0}{180}}
	\psframe*[fillcolor=lightgray,linecolor=gray](-0.5,0.27)(0.5,0.3)
\endpsclip
% Now the circle
\psarc[linecolor=darkgray,linewidth=2pt]{-}(0,0){0.5}{0}{180}
% Axes
\psline{->}(-0.6,0)(0.6,0)
\psline{->}(0,0)(0,0.6)
\uput[r](0.6,0){$x$}
\uput[u](0,0.6){$z$}
\SpecialCoor
\uput[d](-0.5,0){$-a$}
\uput[d](0.5,0){$a$}
\uput[dl](0,0.5){$a$}
\pcline{<->}(0,-0.1)(!0.27 0.25 exch dup mul sub sqrt -0.1)
\Bput{$\sqrt{a^2-z^2}$}
% Dashed lines
\psline[linecolor=black,linestyle=dashed]{-}(0,-0.2)(0,0) 
\psline[linecolor=black,linestyle=dashed]{-}%
(!0.27 0.25 exch dup mul sub sqrt 0.27)(!0.27 0.25 exch dup mul sub sqrt -0.2)
\uput[dl](0,0.27){$z$}
% Arrows to indicate width of dz
\psline{->}(-0.15,0.2)(-0.15,0.27) 
\psline{->}(-0.15,0.37)(-0.15,0.3) 
\uput[d](-0.15,0.2){$dz$}
\end{pspicture}
\end{figure}
%----------------------------------------------------------------------------

%============================================================================
\begin{example}[Semicircular Lamina]
\problem
Show that the centre of mass of the semicircular lamina of radius $a$ in the
half-plane $z>0$ is
$$z_{\rm cm}=\frac{4a}{3\pi}$$

\solution
As shown in figure \ref{rbm fig:scl}, the mass of 
the strip of width $dz$ of the semicircular lamina is
$$w(z)=\bar{\rho}\cdot2\sqrt{a^2-z^2}\,dz$$
The $z$ axis is a line of symmetry, so the centre of mass of the strip is on
the $z$ axis at $z$ ($x=y=0$).  Therefore the centre of mass of the whole
body is
$$z_{\rm cm}=\frac{1}{M}\int_0^a 2\bar{\rho}z\sqrt{a^2-z^2}\,dz$$
The mass of the lamina is
$$M=\bar{\rho}\cdot\frac{\pi a^2}{2}$$
so the location of the centre of mass becomes
$$z_{\rm cm}=\frac{4}{\pi a^2}\int_0^a z\sqrt{a^2-z^2}\,dz$$
Make the substitution
$$u^2=1-\frac{z^2}{a^2}$$
so that
$$z_{\rm cm}=\frac{4a}{\pi}\int_0^1 u^2\,du=\frac{4a}{3\pi}$$
\end{example}
%============================================================================


%----------------------------------------------------------------------------
\begin{figure}\centering
\caption{The hemispherical shell has its centre at the
origin and has radius $a$.  The shaded region is the circumferential strip
from $\theta$ to $\theta+d\theta$.}
\label{rbm fig:hss}

\psset{unit=5cm}
\begin{pspicture}(-0.7,-0.3)(0.7,0.75)
% Now the circle
\psarc[linecolor=darkgray,linewidth=2pt]{-}(0,0){0.5}{0}{180}
% The shaded region
\psclip{\psarc[linewidth=1pt,linestyle=none]{-}(0,0){0.5}{0}{180}}
	\psframe*[fillcolor=lightgray,linecolor=gray](-0.5,0.27)(0.5,0.3)
\endpsclip
% Axes
\psline{->}(-0.6,0)(0.6,0)
\psline{->}(0,0)(0,0.6)
\uput[r](0.6,0){$x$}
\uput[u](0,0.6){$z$}
\SpecialCoor
\uput[d](-0.5,0){$-a$}
\uput[d](0.5,0){$a$}
\uput[dl](0,0.5){$a$}
\pcline{<->}(0,-0.1)(!0.27 0.25 exch dup mul sub sqrt -0.1)
\Bput{$a\cos\theta$}
% Dashed lines
\psline[linecolor=black,linestyle=dashed]{-}(0,-0.2)(0,0) 
\psline[linecolor=black,linestyle=dashed]{-}%
(!0.27 0.25 exch dup mul sub sqrt 0.27)(!0.27 0.25 exch dup mul sub sqrt -0.2)
% Dark lines for ad\theta
\psline[linecolor=black,linewidth=2pt]{-}%
(!0.27 0.25 exch dup mul sub sqrt 0.27)(!0.3 0.25 exch dup mul sub sqrt 0.3)
\psline[linecolor=black,linewidth=2pt]{-}%
(!0.27 0.25 exch dup mul sub sqrt neg 0.27)(!0.3 0.25 exch dup mul sub sqrt
neg 0.3)
% Arrows to indicate width of ad\theta
\rput[B]{(0.0208,0.03)}(!0.27 0.25 exch dup mul sub sqrt neg 0.27){
	\psline{->}(-0.15,0.05)(0,0.05) 
	\psline{<-}(0.0365,0.05)(0.1865,0.05) 
}
\uput[ul](!0.27 0.25 exch dup mul sub sqrt neg 0.27){$a\,d\theta$}
% theta and dtheta
\psline[linestyle=dashed]{-}(0,0)(!0.25 0.27 dup mul sub sqrt 0.27)
\psarc{->}(0,0){0.2}{0}{(!0.25 0.27 dup mul sub sqrt 0.27)}
\psline[linestyle=dashed]{-}(0,0)(!0.25 0.3 dup mul sub sqrt 0.3)
\uput[ur](0.2,0){$\theta$}
\rput[B](!0.25 0.3 dup mul sub sqrt 0.3 0.6 mul exch 0.6 mul exch)%
{$d\theta$}
\end{pspicture}
\end{figure}
%----------------------------------------------------------------------------

%============================================================================
\begin{example}[Hemispherical Shell]
\problem
Show that the centre of mass of a hemispherical shell of radius $a$ with
$z>0$ is
$$z_{\rm cm}=\frac{a}{2}$$

\solution
As shown in figure \ref{rbm fig:hss}, the strip from an angle $\theta$ to 
$\theta+d\theta$ has $z$ coordinate
$$z=a\sin\theta$$
The surface area of this strip is
$$dA=2\pi(a\cos\theta)\,a\,d\theta=2\pi a^2\cos\theta\,d\theta$$
From symmetry, the centre of mass of this strip is at $z=a\sin\theta$. 
Therefore, the centre of mass of the hemispherical shell has
$x_{\rm cm}=y_{\rm cm}=0$ and
$$z_{\rm cm}=\frac{1}{M}\int_A\bar{\rho}z\,dA=\frac{1}{M}\int_0^{\pi/2}
2\pi a^3\bar{\rho}\cos\theta\sin\theta\,d\theta$$
The mass of the hemispherical shell is
$$M=\bar{\rho}\cdot 2\pi a^2$$
so the location of the centre of mass becomes
$$z_{\rm cm}=a\int_0^{\pi/2}\cos\theta\sin\theta\,d\theta$$
Make the substitution
$$u=\sin\theta$$
so that
$$z_{\rm cm}=a\int_0^1 u\,du=\frac{a}{2}$$
\end{example}
%============================================================================

%%%%%%%%%%%%%%%%%%%%%%%%%%%%%%%%%%%%%%%%%%%%%%%%%%%%%%%%%%%%%%%%%%%%%%%%%%%%%
\section{Rotation About a Fixed Axis}

%----------------------------------------------------------------------------
\begin{figure}\centering
\caption{Here is the point at $(x,y,z)$ rotating at constant speed
$\omega=\dot{\phi}$ about the $z$-axis.}
\label{rbm fig:rafi}

\psset{unit=8cm}
\begin{pspicture}(-0.25,-0.35)(0.5,0.55)
% First the xyz axes
\psline{->}(0,0)(0,0.4)
\uput[u](0,0.4){$z$}
\psline{->}(0,0)(-0.15,-0.25)
\uput[dl](-0.15,-0.25){$x$}
\psline{->}(0,0)(0.4,-0.133)
\uput[dr](0.4,-0.133){$y$}
\uput[l](0,0){$O$}
% The point and vector to it
\qdisk(0.1,0){5pt}
\psline[linewidth=3pt]{->}(0,0)(0.1,0)
\uput[r](0.11,0){$(x,y,z)$}
\uput[u](0.05,0.05){$\vect{r}$}
% Construction lines
\psline[linestyle=dashed]{-}(-0.1,-0.2)(0.1,-0.267) 
\psline[linestyle=dashed]{-}(0,0)(0.1,-0.267) 
\psline[linestyle=dashed]{-}(0.2,-0.067)(0.1,-0.267) 
\psline[linestyle=dashed]{-}(0.1,0)(0,0.3) 
\psline[linestyle=dashed]{-}(0.1,0)(0.1,-0.267) 
% Angles
\SpecialCoor
\psarc{->}(0,0){0.1}{(-0.1,-0.2)}{(0.1,-0.267)}
\psarc{->}(0,0.38){0.05}{180}{360}
\uput[r](0.05,0.38){$\omega$}
\uput[d](-0.05,-0.1){$\phi$}
% The velocity v
\psline[linewidth=2pt]{->}(0.1,0)(0.2,0.05)
\uput[r](0.2,0.05){$\vect{v}$}
\end{pspicture}
\end{figure}
%----------------------------------------------------------------------------

Consider a body rotating about the $z$ axis through a fixed origin $O$ with
angular speed
$$\omega=\dot{\phi}$$
The mass at point $\vect{r}=(x,y,z)$ is moving with velocity $\vect{v}$ in 
the $xy$ plane through the point, where $\vect{v}$ satisfies
$$\left|\vect{v}\right|=\omega\sqrt{x^2+y^2}$$

Now the point is moving in uniform circular motion, so $\vect{v}$ is also 
given by
$$\vect{v}=\dot{\vect{r}}=\dot{x}\ivect+\dot{y}\jvect$$
Using
$$\dot{x}=-\left|\vect{v}\right|\sin\phi=-\omega y$$
and
$$\dot{y}=\left|\vect{v}\right|\cos\phi=\omega x$$
the velocity can be written
$$\vect{v}=\boldsymbol{\omega}\times\vect{r}$$
where the angular velocity vector is
$$\boldsymbol{\omega}=\omega\kvect$$

For a discrete body, the rotational kinetic energy is
$$T_{\rm rot}=\sum_i\frac{1}{2}m_i\left|\vect{v}_i\right|^2
=\frac{1}{2}\sum_im_i\left(\omega\sqrt{x_i^2+y_i^2}\right)^2$$
This can be written as
$$T_{\rm rot}=\frac{1}{2}I_z\omega^2$$
where $I_z$ is the \name{moment of inertia} of the body about the $z$ axis
through $O$
$$I_z=\sum_i m_i\left(x_i^2+y_i^2\right)$$

For a continuum solid, the same formula holds
$$T_{\rm rot}=\frac{1}{2}I_z\omega^2$$
when now the moment of inertia is
$$I_z=\int_V \rho\,(x^2+y^2) \,dV$$

When the body is rotating about the $z$ axis, the velocity is
$$\vect{v}=\dot{\vect{r}}=\dot{x}\ivect+\dot{y}\jvect$$
so the angular momentum about $O$
$$\vect{L}=\sum_i\vect{r}_i\times(m_i\dot{\vect{r}}_i)$$
has only a component in the $z$ direction
$$L_z=\sum_i m_i(x_i\dot{y}_i-y_i\dot{x}_i)$$
Using $\dot{x}=-\omega y$ and $\dot{y}=\omega x$, this becomes
$$L_z=\sum_i m_i(x_i^2+y_i^2)\omega=I_z\omega$$

The \name{angular equation of motion} for a rigid body can be found from the
$z$ component of $\dbd{t}\vect{L}=\vect{N}$
$$\dbd{t}(I_z\omega)=N_z$$
For a rigid body, $I_z$ is constant so that
$$I_z\der{\omega}{t}=N_z$$

This leads to simple relations between the equations of motion for
translation along the $z$ axis and rotation around the $z$ axis.  Making the
connections
$$\left.\begin{array}{c}m \\ v_z \\ \dot{v}_z \end{array}\right\}\iff
\left\{\begin{array}{c}I_z \\ \omega \\ \dot{\omega} \end{array}\right.$$
shows how linear and angular momentum obey similar relationships
$$p_z=mv_z\qquad\iff\qquad L_z=I_z\omega$$
as do force and torque
$$F_z=m\dot{v}_z\qquad\iff\qquad N_z=I_z\dot{\omega}$$
and translational and rotational kinetic energy
$$\frac{1}{2}mv_z^2\qquad\iff\qquad\frac{1}{2}I_z\omega^2$$
 
%============================================================================
\begin{example}[Thin Rod about Centre of Mass]

The $z$ component of the moment of inertia of a thin rod of length $l$ 
about its centre with the rod lying along the $x$ axis is given by
$$I_z=\int_{-l/2}^{l/2}\int\!\!\int \rho\,(x^2+y^2)\,dz\,dy\,dx$$
Since the rod is thin and lies along the $x$ axis, this becomes
$$I_z=\int_{-l/2}^{l/2}\int\!\!\int \rho x^2\,dz\,dy\,dx=\bar{\rho}
\int_{-l/2}^{l/2}x^2\,dx$$
where $\bar{\rho}$ is the rod's mass per unit length
$$\bar{\rho}=\rho\int\!\!\int dz\,dy=\frac{M}{l}$$
Therefore
$$I_z=\frac{M}{l}\int_{-l/2}^{l/2}x^2\,dx=\frac{M}{l}\left.\frac{x^3}{3}
\right|_{-l/2}^{l/2}$$
so that
$$I_z=\frac{1}{12}Ml^2$$
\end{example}
%============================================================================

%============================================================================
\begin{example}[Laminar Annulus about Centre of Mass]

Consider an annulus of radius $R$ and thickness $h$ with $h\ll R$, lying in
the $xy$ plane with the $z$ axis at its centre.  The element of unit area is
$$dA=Rh\,d\theta$$
and the mass per unit area is
$$\bar{\rho}=\frac{M}{2\pi Rh}$$
Therefore the moment of inertia about the $z$ axis is
$$I_z=\int \bar{\rho}\,(x^2+y^2)\,dA=\bar{\rho}\int_0^{2\pi}Rh(R^2)\,d\theta
=(2\pi Rh\bar{\rho})R^2$$
Since $M=2\pi Rh\bar{\rho}$, the moment of inertia is
$$I_z=MR^2$$
\end{example}
%============================================================================

%============================================================================
\begin{example}[Circular Disc about the Centre of Mass]

From the previous example, the moment of inertia of the annulus from $r$ to
$r+dr$ is
$$dI_z=(2\pi r\,dr\,\bar{\rho})r^2$$
Therefore
$$I_z=2\pi\bar{\rho}\int_0^ar^3\,dr=2\pi\bar{\rho}\frac{a^4}{4}$$
But $\bar{\rho}=M/\pi a^2$ so the moment of inertia becomes
$$I_z=\frac{1}{2}Ma^2$$
\end{example}
%============================================================================

%============================================================================
\begin{exercise}
Show that the moments of inertia of a sphere and a thin spherical shell,
both of radius $a$, are respectively
$$I_z=\frac{2}{5}Ma^2\qquad\mbox{and}\qquad I_z=\frac{2}{3}Ma^2$$
\end{exercise}
%============================================================================

%%%%%%%%%%%%%%%%%%%%%%%%%%%%%%%%%%%%%%%%%%%%%%%%%%%%%%%%%%%%%%%%%%%%%%%%%%%%%
\subsection{Moments of Inertia about Other Axes}

The moments of inertia of a body about the three Cartesian axes are
$$I_z=\sum_im_i(x_i^2+y_i^2)$$
$$I_x=\sum_im_i(y_i^2+z_i^2)$$
$$I_y=\sum_im_i(x_i^2+z_i^2)$$
for bodies composed of discrete particles, and
$$I_z=\int_V\rho\,(x^2+y^2)\,dV$$
$$I_x=\int_V\rho\,(y^2+z^2)\,dV$$
$$I_y=\int_V\rho\,(x^2+z^2)\,dV$$
for continuous bodies.

For a laminar body lying in the $xy$ plane
$$I_z=\int_V\bar{\rho}\,(x^2+y^2)\,dV$$
$$I_x=\int_V\bar{\rho}\,y^2\,dV$$
$$I_y=\int_V\bar{\rho}\,x^2\,dV$$
because the $z$ component is zero.  Adding these shows that
$$I_z=I_x+I_y$$
which is known as the \name{perpendicular axis theorem}. In other words,
the moment of inertia of any plane body about an axis normal to the body is
equal to the sum of the moments of inertia about any two perpendicular axes
passing through the given axis and lying in the plane.

%============================================================================
\begin{example}
The moment of inertia in the $z$ direction of a circular plate lying in the
$xy$ plane is
$$I_z=\frac{1}{2}Ma^2$$
if the origin is at the centre of the circular plate.  

By symmetry, $I_x=I_y$ and by the perpendicular axis theorem,
$$I_z=I_x+I_y$$ 
so that
$$I_x=I_y=\frac{1}{2}I_z=\frac{1}{4}Ma^2$$
\end{example}
%============================================================================

%============================================================================
\begin{example}
To find the moment of inertia $I_z$ of a flat square lying in the $xy$
plane, whose size is $l$ by $l$ and mass is $M$, note that looking along the
$x$ or the $y$ axes, the plate looks like a rod of length $l$.  Therefore
$I_x$ and $I_y$ are the moments of inertia of a rod of length $l$ and mass
$M$ measured about the centre of the rod
$$I_x=I_y=\frac{1}{12}Ml^2$$
Then by the perpendicular axis theorem,
$$I_z=I_x+I_y=\frac{1}{6}Ml^2$$
\end{example}
%============================================================================

The following theorem, the \name{parallel axis theorem} describes how the
moment of inertia changes when measured relative to an axis that is
parallel to the original axis.

%============================================================================
\begin{theorem}[Parallel Axis Theorem]
Suppose a body has a moment of inertia $I_z$ measured through the origin and 
$I_{z,\rm cm}$ measured relative to the centre of mass.  If the
perpendicular distance from the centre of mass to the $z$ axis is $l$, 
the two moments of inertia are related by
$$I_z=Ml^2+I_{z,\rm cm}$$
\end{theorem}
\begin{proof}
The moment of inertia through the origin is
$$I_z=\sum_i m_i(x_i^2+y_i^2)$$
and the moment of inertia through the centre of mass is
$$I_{z,\rm cm}=\sum_i m_i(\bar{x}_i^2+\bar{y}_i^2)$$
where the coordinates relative to the origin $\vect{r}_i$ are related to the
coordinates relative to the centre of mass $\bar{\vect{r}}_i$ according to
$$\vect{r}_i=\vect{r}_{\rm cm}+\bar{\vect{r}}_i$$
where $\vect{r}_{\rm cm}$ is the position of the centre of mass relative to
the origin.  Therefore
$$I_z=\sum_i m_i(x_i^2+y_i^2)
=\sum_i m_i\left((x_{\rm cm}+\bar{x}_i)^2+(y_{\rm cm}+\bar{y}_i)^2\right)$$
Expanding the squares shows that
\begin{eqnarray*}
I_z&=&\sum_i m_i(x_{\rm cm}^2+y_{\rm cm}^2)
+2\left(\sum_i m_i\bar{x}_i\right)x_{\rm cm}\\
&&{}+2\left(\sum_i m_i\bar{y}_i\right)y_{\rm cm}
+\sum_i m_i(\bar{x}_i^2+\bar{y}_i^2)
\end{eqnarray*}
The second and third summations are zero.  Writing 
$l^2=x_{\rm cm}^2+y_{\rm cm}^2$, $M=\sum_i m_i$ and noting that the fourth 
summation is the moment of inertia about the centre of mass shows that
$$I_z=Ml^2+I_{z,\rm cm}$$
as required.
\end{proof}
%============================================================================

%============================================================================
\begin{example}
To find the moment of inertia $I_z$ of a circular plate in the $xy$ plane
relative to an axis through the edge of the plate, note that the moment of
inertia through the centre of the plate is
$$I_{z,\rm cm}=\frac{1}{2}Ma^2$$
Measuring the moment of inertia relative to an axis through the edge of the
plate shifts the axis a distance $l=a$.  Therefore the moment of inertia is
$$I_z=Ml^2+I_{z,\rm cm}=Ma^2+\frac{1}{2}Ma^2=\frac{3}{2}Ma^2$$
\end{example}
%============================================================================

%============================================================================
\begin{example}
The moment of inertia $I_x$ for the plate in the previous example can be
found by noting that the moment of inertia through the centre of mass of the
plate is
$$I_{x,\rm cm}=\frac{1}{4}Ma^2$$
Measuring the moment of inertia relative to an axis through the edge of the
plate shifts the axis a distance $l=a$.  Therefore the moment of inertia is
$$I_x=Ml^2+I_{x,\rm cm}=Ma^2+\frac{1}{4}Ma^2=\frac{5}{4}Ma^2$$
\end{example}
%============================================================================

%%%%%%%%%%%%%%%%%%%%%%%%%%%%%%%%%%%%%%%%%%%%%%%%%%%%%%%%%%%%%%%%%%%%%%%%%%%%%
\section{Gravitational Torque}

If the force $\vect{F}_i$ on the $i^{\rm th}$ particle is the gravitational 
force $m_i\vect{g}$, then the
total external gravitational torque about the origin is
$$\vect{N}=\sum_i\vect{r}_i\times\vect{F}_i$$
This can be written
$$\vect{N}=\sum_i\vect{r}_i\times(m_i\vect{g})
=\sum_i(m_i\vect{r}_i)\times\vect{g}$$
In terms of the centre of mass, this becomes
$$\vect{N}=\sum_i(m_i\vect{r}_i)\times\vect{g}=M\vect{r}_{\rm cm}
\times\vect{g}$$
Therefore the gravitional torque is equal to the torque acting on the total
weight of the body concentrated at the centre of mass.

%----------------------------------------------------------------------------
\begin{figure}\centering
\caption{This is a compound pendulum of mass $M$ with centre of mass at $C$. 
It pivots about the point $P$ which is a distance $l$ from the centre of 
mass.  The $z$ axis points out of the page.}
\label{rbm fig:cp}

\psset{unit=5cm}
\begin{pspicture}(-0.3,-1.3)(1.3,0.3)
% The outline of the pendulum
\psccurve[linecolor=darkgray,linewidth=2pt](-0.2,0.2)(0.1,0.2)(0.3,-0.1)(0.7,-0.4)(0.9,-0.2)%
(1.1,-0.4)(1.2,-0.8)(0.9,-1.1)(0.5,-0.9)(0.1,-0.6)(-0.1,-0.1)
% Axes
\psline{->}(0,0)(1,0)
\uput[r](1,0){$y$}
\psline{->}(0,0)(0,-1)
\uput[d](0,-1){$x$}
\qdisk(0,0){3pt}
\uput[ul](0,0){$P$}
% Centre of mass
\pcline{->}(0,0)(0.6,-0.6)
\Aput{$l$}
\uput[ur](0.6,-0.6){$C$}
\psline{->}(0.6,-0.6)(0.6,-0.9)
\uput[d](0.6,-0.9){$Mg$}
% The angle theta
\psarc{->}(0,0){0.2}{270}{315}
\uput[dr](0,-0.2){$\theta$}
\end{pspicture}
\end{figure}
%----------------------------------------------------------------------------

%============================================================================
\begin{example}[The Compound Pendulum]
The equation of motion of the compound pendulum about the pivot, 
shown in figure \ref{rbm fig:cp}, is
$$I_z\dot{\omega}=N_z$$
where $\dot{\omega}=\ddot{\theta}$, the torque is $\vect{N}=\vect{l}\times
M\vect{g}$ so that
$$N_z=-Mgl\sin\theta$$
and the moment of inertia about the pivot is
$$I_z=Ml^2+I_{z,\rm cm}$$
Therefore
$$(Ml^2+I_{z,\rm cm})\ddot{\theta}=-Mgl\sin\theta$$
which gives
$$\ddot{\theta}=-\frac{g}{l+I_{z,\rm cm}/Ml}\sin\theta$$

For small amplitude oscillations, $\left|\theta\right|\ll 1$ so that
$\sin\theta\approx\theta$.  The equation of motion is approximately
$$\ddot{\theta}=-\frac{g}{l+I_{z,\rm cm}/Ml}\theta$$
The solution is
$$\theta(t)=\theta_0\cos\omega t$$
where $\theta(0)=\theta_0$ and $\dot{\theta}(0)=0$.  The angular frequency
of oscillation is
$$\omega=\sqrt{\frac{g}{l+I_{z,\rm cm}/Ml}}$$
and the period is
$$T=\frac{2\pi}{\omega}=2\pi\sqrt{\frac{l+I_{z,\rm cm}/Ml}{g}}$$
which is independent of the initial conditions.

For large oscillations, the equation of motion is
$$\ddot{\theta}=-\omega_0^2\sin\theta$$
where
$$\omega_0^2=\frac{g}{l+I_{z,\rm cm}/Ml}$$
Using $\ddot{\theta}=\dbd{\theta}\left(\frac{1}{2}\dot{\theta}^2\right)$,
integrate to obtain
$$\frac{1}{2}\dot{\theta}^2=\omega_0^2\cos\theta+C$$
With $\theta(0)=\theta_0$ and the pendulum starting from rest so that
$\dot{\theta}(0)=0$, the solution is
$$\dot{\theta}^2=2\omega_0^2(\cos\theta-\cos\theta_0)$$
The integral of this is a bit complicated and cannot be obtained in terms of 
ordinary functions.  However an expression for the period $T$ can be
obtained.

Taking the square root of the equation for $\dot{\theta}^2$ 
$$\dot{\theta}=\pm\sqrt{2}\omega_0\sqrt{\cos\theta-\cos\theta_0}$$
During the period $0<t<\frac{T}{4}$, the pendulum swings from
$\theta=\theta_0$ to $\theta=0$ so $\dot{\theta}<0$.  Thus for these times,
the negative square root is appropriate and
$$\dot{\theta}=-\sqrt{2}\omega_0\sqrt{\cos\theta-\cos\theta_0}$$
which is a separable equation.

The solution is
$$T=\frac{\sqrt{8}}{\omega_0}\int^{\theta_0}_0\frac{d\theta}{\sqrt{\cos\theta
-\cos\theta_0}}$$
which is an elliptic integral.  Denoting the elliptic integral as
$f(\theta_0)$, the solution is
$$T=\frac{\sqrt{8}}{\omega_0}f(\theta_0)$$
which is a function of the initial conditions.
\end{example}
%============================================================================

%%%%%%%%%%%%%%%%%%%%%%%%%%%%%%%%%%%%%%%%%%%%%%%%%%%%%%%%%%%%%%%%%%%%%%%%%%%%%
\section{Rolling}

%----------------------------------------------------------------------------
\begin{figure}\centering
\caption{This diagram shows a body rolling on a plane inclined at an angle
$\theta$.  The dashed line indicates the path taken by the rolling body's
centre of mass.}
\label{rbm fig:rb}

\psset{unit=5cm}
\begin{pspicture}(-1.3,-0.1)(0.3,0.8)
\SpecialCoor
% First the ground
\psline[linecolor=darkgray,linewidth=2pt]{-}(1.1;150)(0;150)(-1;0)
% The rolling body
\rput[b]{330}(0.5;150){
	\begin{pspicture}(0,-0.2)(0,-0.2)
		\psline[linestyle=dashed]{-}(-0.4,0)(0.4,0) 
		\psline(0,0)(0.2;135)
		\psarc{->}(0,0){0.1}{0}{135}
		\uput[u](0.1;66){\rput[l]{*0}{$\phi$}}	
		\pscircle[linecolor=gray,linewidth=2pt](0,0){0.2}
		% The weight vector
		\psline{->}(0,0)(0.3;300)
		\uput[d](0.32;300){\rput[B]{30}{$Mg$}}
	\end{pspicture}
}
\rput[b]{330}(0.5;150){
	\begin{pspicture}(0,0)(0,0)
		\psline[linecolor=black]{->}(0,0)(-0.2,0) 
		\uput[d](-0.2,0){\rput[r]{30}{$F_f$}}
		\psline[linecolor=black]{->}(0,0)(0,0.1) 
		\uput[l](0,0.05){\rput[b]{30}{$R$}}
	\end{pspicture}
}
% The angle theta
\psarcn{->}(0,0){0.2}{180}{150}
\uput[l](-0.2,0.05){$\theta$}
\end{pspicture}
\end{figure}
%----------------------------------------------------------------------------

For the body rolling down the slope illustrated in figure \ref{rbm fig:rb}, 
the forces on the body lead to the following equations of motion
$$M\ddot{x}_{\rm cm}=Mg\sin\theta-F_f$$
$$M\ddot{y}_{\rm cm}=-Mg\cos\theta+R$$
The body stays in contact with the slope so that
$$y_{\rm cm}=a$$ 
and
$$\ddot{y}_{\rm cm}=0$$ 
so that the reaction is
$$R=Mg\cos\theta$$
The torque on the body about the centre of mass is
$$I_{\rm cm}\dot{\omega}=N_z=-F_fa$$
since all other forces have zero torque about the centre of mass.  The
velocity of the point of the rolling body in contact with the slope is
$$\dot{x}_{\rm cm}=-a\omega$$ 
so that
$$\omega=-\frac{\dot{x}_{\rm cm}}{a}\qquad\mif\qquad
\dot{\omega}=-\frac{\ddot{x}_{\rm cm}}{a}$$
Substituting into the torque equation gives
$$I_{\rm cm}\left(-\frac{\ddot{x}_{\rm cm}}{a}\right)=-F_fa$$
so that
$$F_f=\frac{I_{\rm cm}}{a^2}\ddot{x}_{\rm cm}$$
Substituting into the force equation shows that
$$\ddot{x}_{\rm cm}=\frac{g\sin\theta}{1+I_{\rm cm}/Ma^2}$$
which is a constant.  Thus
$$\dot{\omega}=-\frac{\ddot{x}_{\rm cm}}{a}=
\frac{g\sin\theta}{a\left(1+I_{\rm cm}/Ma^2\right)}$$
which is also a constant.

If the point of contact slips on the inclined plane, then
$$F_f=\mu R$$
where $\mu$ is the coefficient of sliding friction.  Therefore the force
equation becomes
$$M\ddot{x}_{\rm cm}=Mg\sin\theta-\mu Mg\cos\theta$$
or
$$\ddot{x}_{\rm cm}=g\left(\sin\theta-\mu\cos\theta\right)$$
which is a constant.  To calculate the angular velocity, use the torque
equation
$$I_{\rm cm}\dot{\omega}=-F_fa$$
so
$$\dot{\omega}=-\frac{a\mu Mg\cos\theta}{I_{\rm cm}}$$

%----------------------------------------------------------------------------
\begin{figure}\centering
\caption{Initially, this billiard ball is spinning with an angular velocity
of $\omega_0$ and is released with zero forward velocity.   At the time
shown in this figure, the ball is spinning with angular velocity $\omega$
and has a forward velocity of $\vect{v}_{\rm cm}$.}
\label{rbm fig:sbb}

\psset{unit=5cm}
\begin{pspicture}(-0.8,-0.6)(0.6,0.6)
% Billiard ball
\pscircle[linecolor=gray,linewidth=2pt](0,0){0.5}
% Axes
\psline{->}(-0.6,-0.55)(0.6,-0.55)
\uput[r](0.6,-0.55){$x$}
\psline{->}(-0.6,-0.55)(-0.6,0.2)
\uput[u](-0.6,0.2){$y$}
% Reaction and friction
\psline{->}(0,-0.5)(0,-0.3)
\uput[r](0,-0.3){$R$}
\psline{->}(0,-0.5)(0.3,-0.5)
\uput[r](0.3,-0.5){$F_f$}
% Gravity and velocity
\psline{->}(0,0)(0.3,0)
\uput[r](0.3,0){$\vect{v}_{\rm cm}$}
\psline{->}(0,0)(0,-0.2)
\uput[r](0,-0.2){$Mg$}
% The angle omega
\psarcn{->}(0,0){0.4}{180}{135}
\SpecialCoor
\uput[ur](0.4;135){$\omega$}
\end{pspicture}
\end{figure}
%----------------------------------------------------------------------------

%============================================================================
\begin{example}[Slipping Billiard Ball]
\problem
A billiard ball rotating with angular speed $\omega_0$ is released with
zero forward velocity such that it slips on the table (as shown in figure
\ref{rbm fig:sbb}), where the coefficient of friction is $\mu$.  When does 
it start to roll?

\solution
The reaction is $R=Mg$ and initially, the ball is slipping so that
$$F_f=\mu R$$
The force equation gives
$$M\ddot{x}_{\rm cm}=F_f=\mu Mg$$
so that
$$\ddot{x}_{\rm cm}=\mu g\qquad\dot{x}_{\rm cm}=\mu gt\qquad
x_{\rm cm}=\frac{\mu gt^2}{2}$$
which is valid up to the point when rolling starts and slipping stops as
then $F_f\neq \mu R$.

The torque equation about the centre of mass is
$$I_{\rm cm}\dot{\omega}=+\mu Mga$$
so that
$$\omega=\frac{\mu Mga}{I_{\rm cm}}t-\omega_0$$
Slipping stops when $\omega$ slows down to the point when
$$-\frac{\dot{x}_{\rm cm}}{a}=\omega$$
At this time, $t_{\rm roll}$,
$$-\frac{\mu gt_{\rm roll}}{a}=\frac{\mu Mga}{I_{\rm cm}}t_{\rm roll}
-\omega_0$$
so the time when pure rolling starts is
$$t_{\rm roll}=\frac{a\omega_0}{\mu g\left(1+Ma^2/I_{\rm cm}\right)}$$
which is a distance
$$x_{\rm roll}=\frac{a^2\omega_0^2}{2\mu g\left(1+Ma^2/I_{\rm cm}\right)^2}$$
from the point when it is first released.

After this point, we have
$$\ddot{x}_{\rm cm}=\frac{F_f}{M}$$
Now
$$I_{\rm cm}\dot{\omega}=F_fa$$
but the  ball is rotating without slipping so
$$\dot{\omega}=-\frac{\ddot{x}_{\rm cm}}{a}$$
so that
$$F_f=-\frac{I_{\rm cm}\ddot{x}_{\rm cm}}{a^2}$$
This implies that
$$\ddot{x}_{\rm cm}\left(1+\frac{I_{\rm cm}}{Ma^2}\right)=0$$
which means that the ball's velocity is constant after pure rolling has 
started.  This implies that $F_f$ does no work on a rolling body.
\end{example}
%============================================================================

%----------------------------------------------------------------------------
\begin{figure}\centering
\caption{This shows the forces on a horizontal plank of length $l$ and mass
$M$ at the instant that the right-hand end is dropped.  The dropped plank
pivots about the point $P$.}
\label{rbm fig:dl}

\psset{unit=5cm}
\begin{pspicture}(-0.4,-0.4)(1.4,0.4)
% The ladder and its support
\psline[linecolor=gray,linewidth=2pt]{-}(0,0)(1,0)
\pspolygon*[linecolor=darkgray](0,0)(-0.05,-0.1)(0.05,-0.1)
% Gravity
\psline{->}(0.5,0)(0.5,-0.2)
\uput[d](0.5,-0.2){$Mg$}
% l/2 labels
\uput[u](0.25,0){$l/2$}
\uput[u](0.75,0){$l/2$}
% Reaction and unit vectors
\psline{->}(0,0)(0,0.2)
\uput[u](0,0.2){$R$}
\psline{->}(1,0)(1,0.2)
\uput[u](1,0.2){$\tvect$}
\psline{->}(1,0)(1.2,0)
\uput[r](1.2,0){$\rvect$}
% The pivot point P
\uput[l](0,0){$P$}
\end{pspicture}
\end{figure}
%----------------------------------------------------------------------------

%============================================================================
\begin{example}
\problem
Two people hold a plank of mass $M$ and length $l$ at each end.  One person
lets their end go.  Show that the load supported by the other person drops
from $Mg/2$ to $Mg/4$ instantly.  Show that the
acceleration of the free end is $3g/2$.

\solution
A diagram of the force is shown in figure \ref{rbm fig:dl}.  The plank
rotates about $P$ according to 
$$I_P\dot{\omega}=-Mg\frac{l}{2}$$
With $R$ the reaction from the person holding the plank at $P$, the
equation of rotation about the plank's centre of mass is
$$I_{\rm cm}\dot{\omega}=-R\frac{l}{2}$$
so that
$$\frac{I_{\rm cm}}{I_P}=\frac{R}{Mg}$$
Using the parallel axis theorem to write $I_P=I_{\rm cm}+
M\left(l/2\right)^2$ gives
$$R=Mg\frac{I_{\rm cm}}{I_{\rm cm}+M\left(l/2\right)^2}$$
But the moment of inertia about the centre of mass is
$$I_{\rm cm}=\frac{1}{12}Ml^2$$ 
so the reaction is
$$R=Mg\frac{\frac{1}{12}}{\frac{1}{12}+\frac{1}{4}}=\frac{Mg}{4}$$

To work out the acceleration of the free end, use
$$\ddot{\vect{r}}=\left(\ddot{r}-r\dot{\theta}^2\right)\rvect
+\left(r\ddot{\theta}+2\dot{r}\dot{\theta}\right)\tvect$$
Here $r=l$ so $\dot{r}=\ddot{r}=0$.  Initially, $\omega=\dot{\theta}=0$ so
the acceleration is
$$\ddot{\vect{r}}=l\ddot{\theta}\tvect=l\dot{\omega}\tvect$$
Using 
$$I_{\rm cm}\dot{\omega}=-R\frac{l}{2}$$
shows that
$$\ddot{\vect{r}}=-l\frac{R\frac{l}{2}}{I_{\rm cm}}\tvect=
-\frac{Mgl^2}{8I_{\rm cm}}\tvect=-\frac{3}{2}g\tvect$$
so the initial acceleration of the free end is $3g/2$ as required.
\end{example}
%============================================================================


%----------------------------------------------------------------------------
\begin{figure}\centering
\caption{This shows the forces on a thin rod of mass $M$ and length $l$ 
falling over on a rough table.}
\label{rbm fig:fr}

\psset{unit=5cm}
\begin{pspicture}(-0.4,-0.4)(1.4,1.4)
% The rod
\psline[linecolor=gray,linewidth=2pt]{-}(0,0)(1,1)
% Gravity
\psline{->}(0.5,0.5)(0.5,0.3)
\uput[d](0.5,0.3){$Mg$}
% l/2 labels
\uput[ul](0.25,0.25){$l/2$}
\uput[ul](0.75,0.75){$l/2$}
% Reaction and friction
\psline{->}(0,0)(0,0.2)
\uput[u](0,0.2){$R$}
\psline{->}(0,0)(0.2,0)
\uput[r](0.2,0){$F$}
% The pivot point P
\uput[dl](0,0){$P$}
\end{pspicture}
\end{figure}
%----------------------------------------------------------------------------

%============================================================================
\begin{example}
\problem
A thin rod of mass $M$ and length $l$ falls over from an initial vertical
position on a rough table.  Solve the subsequent motion.

\solution
A diagram of the falling rod is given in figure \ref{rbm fig:fr}.  The rod's
equation of angular motion about $P$ is
$$I_P\ddot{\theta}=-Mg\frac{l}{2}\cos\theta$$
which can be written as
$$I_P\dbd{\theta}\left(\frac{1}{2}\dot{\theta}^2\right)=-Mg\frac{l}{2}
\cos\theta$$
Integrating and using the initial condition $\dot{\theta}=0$ at $\theta=\pi/2$
gives
$$\dot{\theta}^2=\frac{Mgl}{I_P}(1-\sin\theta)$$
Now from the parallel axes theorem, 
$$I_P=M\left(\frac{l}{2}\right)^2+\frac{1}{12}Ml^2=\frac{1}{3}Ml^2$$
so
$$\dot{\theta}^2=\left(\frac{3g}{l}\right)(1-\sin\theta)$$
Also, substituting for $I_P$ in the initial equation of angular motion
about $P$ gives
$$\ddot{\theta}=-\left(\frac{3g}{2l}\right)\cos\theta$$
The equations of motion of the centre of mass are
\begin{eqnarray*}
M\ddot{x}_{cm}&=&F\\
M\ddot{y}_{cm}&=&R-Mg
\end{eqnarray*}
If no slipping initially, then we have
\begin{eqnarray*}
x_{cm}&=&\frac{l}{2}\cos\theta\\
y_{cm}&=&\frac{l}{2}\sin\theta
\end{eqnarray*}
Therefore
$$\dot{x}_{cm}=-\frac{l}{2}\sin\theta\dot{\theta}$$
and
\begin{eqnarray*}
\ddot{x}_{cm}&=&-\frac{l}{2}\sin\theta\ddot{\theta}
-\frac{l}{2}\cos\theta\dot{\theta}^2 \\
&=&-\frac{l}{2}\left[
\sin\theta\left(-\frac{3g}{2l}\cos\theta\right)
+\cos\theta\left(\frac{3g}{l}(1-\sin\theta)\right)\right]\\
&=&\frac{3g}{4}\cos\theta(3\sin\theta-2)
\end{eqnarray*}
Similarly for $y_{cm}$,
$$\dot{y}_{cm}=\frac{l}{2}\cos\theta\dot{\theta}$$
and
\begin{eqnarray*}
\ddot{y}_{cm}&=&-\frac{l}{2}\sin\theta\dot{\theta}^2
+\frac{l}{2}\cos\theta\ddot{\theta} \\
&=&-\frac{l}{2}\left[
\sin\theta\left(\frac{3g}{l}(1-\sin\theta)\right)
-\cos\theta\left(-\frac{3g}{2l}\cos\theta\right)\right]\\
&=&-\frac{3g}{4}(1+2\sin\theta-3\sin^2\theta)
\end{eqnarray*}
Substituting into the equations of motion,
\begin{eqnarray*}
F &=& M\ddot{x}_{cm}\\
&=&\frac{3Mg}{4}\cos\theta(3\sin\theta-2)
\end{eqnarray*}
and
\begin{eqnarray*}
R&=&Mg-M\ddot{y}_{cm}\\
&=&\frac{Mg}{4}(1-6\sin\theta+9\sin^2\theta)
\end{eqnarray*}
Slipping starts at the angle $\theta^*$ where
$$F=\mu R$$
so that $\mu$ is given in terms of $\theta^*$ as
$$\mu=
\frac{3\cos\theta^*(3\sin\theta^*-2)}{1-6\sin\theta^*+9\sin^2\theta^*}$$
\end{example}
%============================================================================
