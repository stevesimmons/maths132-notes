% LaTeX2e version of Maths 132 Notes
%%%%%%%%%%%%%%%%%%%%%%%%%%%%%%%%%%%%%%%%%%%%%%%%%%%%%%%%%%%%%%%%%%%%%%%%%%%
%
%			Mathematics 132 Course Notes
%
%			 Department of Mathematics,
%   			  University of Melbourne
%
%		Stephen Simmons			Lee White
%
% 8-Feb-96 SS: \conj{x} changed from \overline{x} to {x}^*
% 9-Feb-96 SS: Make gray in diagrams lighter to improve photocopies
%
%%%%%%%%%%%%%%%%%%% Copyright (C) 1995-96 Stephen Simmons %%%%%%%%%%%%%%%%%
\documentclass[12pt,a4paper]{book}

% If not present on your system, these LaTeX2e packages can be found at
% ftp.cs.rmit.edu.au in a subdirectory of tex-archive.  Use the 
% 'site index xxxx' command to find where 'xxxx' lives in the archive.

\usepackage{amsmath}   % Was amstex
\usepackage{amssymb}
\usepackage{theorem}		% Flexible definitions of theorem styles
\usepackage{xspace}
\usepackage{picinpar}		% Used only for 1 picture in chapter 7

% Use the PSTricks macro package developed by Tim Van Zandt.  This is
% used for almost every diagram in the notes.
\input{pstricks}
\input{pst-node}       
\input{pst-plot}
\input{pst-coil}
%\input{pstricks.bug}
% PSTricks defines gray as 0.5.  This comes out too dark when the notes
% are photocopied, so lighten it to 0.6.
\newgray{gray}{0.6}

% This is a PostScript header file which adds support for drawing 
% phase portraits, as used in chapter 7.
\pstheader{phasport.pro}

% Page setup
\oddsidemargin=-1cm
\marginparsep=3mm
\marginparwidth=3.5cm

% Simple abbreviations
\newcommand{\ODE}{o.d.e.\xspace}
\newcommand{\ODEs}{o.d.e.s\xspace}
\newcommand{\parbreak}{\par\mbox{}\par}
% Maths things
\newcommand{\arcsinh}{\operatorname{arcsinh}}
\newcommand{\arccosh}{\operatorname{arccosh}}
\newcommand{\nn}{\nonumber}
\newcommand{\ds}{\displaystyle}
\newcommand{\mif}{\;\Longrightarrow\;} 		% Like \iff
\newcommand{\thus}{\;\therefore\;}		% therefore (.:) with space
\newcommand{\grad}{\nabla}
\newcommand{\del}{\partial}
\newcommand{\conj}[1]{{#1}^*}			% Used to be \overline, not *
% Derivatives
\newcommand{\der}[2]{\frac{d#1}{d#2}}
\newcommand{\dder}[2]{\frac{d^2#1}{d{#2}^2}}
\newcommand{\nder}[3]{\frac{d^{#3}#1}{d{#2}^{#3}}}
\newcommand{\dbd}[1]{\frac{d}{d{#1}}}
\newcommand{\ndbd}[2]{\frac{d^{#2}}{d{#1}^{#2}}}
% Other bits
\newcommand{\wo}{\omega_0}
% Cartesian unit vectors
\newcommand{\ivect}{\hat{\vect{i}}}
\newcommand{\jvect}{\hat{\vect{j}}}
\newcommand{\kvect}{\hat{\vect{k}}}
% Polar unit vectors
\newcommand{\rvect}{\hat{\vect{r}}}
\newcommand{\tvect}{\hat{\boldsymbol{\theta}}}
% Intrinsic unit vectors
\newcommand{\Tvect}{\hat{\vect{T}}}
\newcommand{\Nvect}{\hat{\vect{N}}}

% Vectors
\DeclareMathAlphabet{\mathbfit}{OT1}{cmr}{bx}{it}
\SetMathAlphabet{\mathbfit}{normal}{OT1}{cmr}{bx}{it}
\newcommand{\vect}[1]{\mathbfit{#1}}


% This gathers together the definition of the phase portraits used 
% throughout chapter 7.  The first parameter is the bits of PostScript
% which draw the phase portrait, and the second parameter is any additional
% PSTricks TeX commands for overlaying things on top of the basic phase
% portrait.
\def\phaseportrait#1#2{
{\small
\psset{unit=1.3cm}
\begin{pspicture}(-1.4,-1.4)(1.6,1.6)
%\psframe(-1.4,-1.4)(1.6,1.6)
\pscustom[linecolor=gray,linewidth=1pt]{
  \code{
    % First scale to make the PS coordinate system the same as PSTricks'
    % The first number on the next line of code should be the same as that in
    % the \psset{unit=1.3cm} above.  If changing the scale, change this too!
    1.3 2.54 div 72 mul dup dup scale CLW exch div SLW
    % Now the phase portrait, using the macros in PHASPORT.PRO
    clsDict begin
      #1
      clsend
    end
  }
}
% Draw the axes and their labels last so they show through
\psset{linewidth=1.2pt,linecolor=black}
\psline{->}(-1.2,0)(1.2,0)
\psline{->}(0,-1.2)(0,1.2)
\uput[r](1.2,0){$y_1$}
\uput[u](0,1.2){$y_2$}
#2
\end{pspicture}
}
}
% A couple of the phase portraits have z instead of y for the axes' labels.
\def\zphaseportrait#1#2{
{\small
\psset{unit=1.3cm}
\begin{pspicture}(-1.4,-1.4)(1.6,1.6)
%\psframe(-1.4,-1.4)(1.6,1.6)
\pscustom[linecolor=gray,linewidth=1pt]{
  \code{
    % First scale to make the PS coordinate system the same as PSTricks'
    % The first number on the next line of code should be the same as that in
    % the \psset{unit=1.3cm} above.  If changing the scale, change this too!
    1.3 2.54 div 72 mul dup dup scale CLW exch div SLW
    % Now the phase portrait, using the macros in PHASPORT.PRO
    clsDict begin
      #1
      clsend
    end
  }
}
% Draw the axes and their labels last so they show through
\psset{linewidth=1.2pt,linecolor=black}
\psline{->}(-1.2,0)(1.2,0)
\psline{->}(0,-1.2)(0,1.2)
\uput[r](1.2,0){$z_1$}
\uput[u](0,1.2){$z_2$}
#2
\end{pspicture}
}
}
%%%%%%%%%%%%%%%%%%%%%%%%%%%%%%%%%%%%%%%%%%%%%%%%%%%%%%%%%%%%%%%%%%%%%%%%%%%%%
% Define the visual layout of the notes
%
% This is set up as follows:
%
% Definitions:			\textbf
% Names of theorems, etc.:	\textbf
% Theorem body:			\textit
% Proofs:			\textrm
% Example title:		\textbf
% Example statement problem:	\textrm
% Example solution:		\textsl
% Exercise:			\textsf

% Emphasise a definition with \name...
\newcommand{\name}[1]{\textbf{#1}}

\theoremstyle{break}
\theoremheaderfont{\upshape\bfseries}

% Theorems numbered within chapters
{\theorembodyfont{\rmfamily\itshape}\newtheorem{theorem}{THEOREM}[chapter]}

% Proofs not numbered
\def\finishproof{\hspace*{\fill}\rule{2mm}{2mm}}
\newenvironment{proof}{\par\medskip\noindent\textbf{Proof:}\par\rmfamily}
	{\finishproof\par\medskip}

% Examples numbered within chapters, and have a black box at their end
{\theorembodyfont{\rmfamily\slshape}\newtheorem{example}{EXAMPLE}[chapter]}
\let\newendexample=\endexample
\def\endexample{\finishproof\newendexample\smallskip}

% For examples with a separate problem statment then solution, use these
% to change the type style.
\newcommand{\problem}{\upshape}
\newcommand{\solution}{\noindent \textbf{Solution:} \slshape}

% Exercises
\newenvironment{exercise}{\par\medskip\centering\par\sffamily}
	{\par\par\medskip}

%%%%%%%%%%%%%%%%%%%%%%%%%%%%%%%%%%%%%%%%%%%%%%%%%%%%%%%%%%%%%%%%%%%%%%%%%%%%%
% Start of document

\begin{document}
\frontmatter
\tableofcontents
%%%%%%%%%%%%%%%%%%%%%%%%%%%%%%%%%%%%%%%%%%%%%%%%%%%%%%%%%%%%%%%%%%%%%%%%%%%
%
%			Mathematics 132 Course Notes
%
%			 Department of Mathematics,
%   			  University of Melbourne
%
%		Stephen Simmons			Lee White
%
%%%%%%%%%%%%%%%%%%%%% Copyright (C) 1995 Stephen Simmons %%%%%%%%%%%%%%%%%%

\chapter{Introduction}
\label{pre chp}

Occasional references are made throughout these notes to exercises in the 
textbooks by Zill and Fowles.  These books are:

\begin{itemize}
\item Zill, D. G., {\em A First Course in Differential Equations}, PWS
Publishing Company, fifth edition, 1993.
\item Fowles, G. R., {\em Analytical Mechanics}, Holt Rinehart Winston,
third edition, 1977.
\end{itemize}

While the exact page references in these notes refer to the editions listed 
above, earlier editions cover much the same material with slightly different 
page and section numbering.

	
\mainmatter
%%%%%%%%%%%%%%%%%%%%%%%%%%%%%%%%%%%%%%%%%%%%%%%%%%%%%%%%%%%%%%%%%%%%%%%%%%%
%
%			Mathematics 132 Course Notes
%
%			 Department of Mathematics,
%   			  University of Melbourne
%
%		Stephen Simmons			Lee White
%
% 8 Feb-96 SS: Updated with corrections from semester 2, 1995
%
%%%%%%%%%%%%%%%%%%% Copyright (C) 1995-96 Stephen Simmons %%%%%%%%%%%%%%%%%

\chapter{Ordinary Differential Equations}
\label{ode chp}

An equation of the form
$$f\left(x,y,\der{y}{x},\dder{y}{x},\ldots,\nder{y}{x}{n}\right)=0$$
involving $y(x)$ and its derivatives is an \name{ordinary differential
equation} (\ODE).  The \name{order} of an \ODE is the order of the
highest derivative that appears.  An \ODE is \name{linear} if it can be
written in the form
$$a_n(x)\nder{y}{x}{n}+a_{n-1}(x)\nder{y}{x}{n-1}+\cdots+a_1(x)\der{y}{x}
+a_0(x)y=F(x)$$
A \name{solution} of an \ODE is a function $y(x)$ which when substituted
into the \ODE renders it an identity.  An \name{integral} of an \ODE is
an implicit relationship connecting $y$ and $x$ which when substituted
into the \ODE  renders it an identity (this is sometimes called an
\name{implicit solution}).

%============================================================================
\begin{example}
An integral or implicit solution of the \ODE
$$\der{y}{x}=-\frac{x}{y}$$
is
$$y^2+x^2=C$$
\end{example}
%============================================================================

%%%%%%%%%%%%%%%%%%%%%%%%%%%%%%%%%%%%%%%%%%%%%%%%%%%%%%%%%%%%%%%%%%%%%%%%%%%%%
\section{First Order O.D.E.s}
\label{ode sec:foode}

The general \name{first order \ODE} is written
$$\der{y}{x}=f(x,y)$$
Note that $F(x,y,\der{y}{x})=0$ may not be able to be inverted to obtain
$\der{y}{x}$ in terms of $x$ or $y$, or there may be a multiplicity of
solutions, for example, as occurs in
$$\left(\der{y}{x}\right)^2-g(x,y)=0$$

The \name{initial value problem} is to find the solution of an \ODE
subject to the \name{initial condition}
$$y(x_0)=y_0$$

%%%%%%%%%%%%%%%%%%%%%%%%%%%%%%%%%%%%%%%%%%%%%%%%%%%%%%%%%%%%%%%%%%%%%%%%%%%%%
\subsection{Separable Equations}
\label{ode sec:sep}

An \ODE of the form
$$\der{y}{x}=\frac{g(x)}{h(y)}$$
is called \name{separable}.  Separable \ODEs are solved by considering the 
indefinite integral
$$G(x)=\int^x g(x)\,dx$$
From the fundamental theorem of calculus,
$$\der{G}{x}=g(x)$$
Similarly, the function 
$$H(x)=\int^x h(x)\,dx$$
obeys
$$\der{H}{x}=h(x)$$
Thus, using chain rule,
$$\dbd{x}H\left(y(x)\right)=\der{H}{y}\,\der{y}{x}=h(y)\,\der{y}{x}=g(x)$$
Using $g(x)=\der{G}{x}$ shows that
$$\dbd{x}H\left(y(x)\right)=\dbd{x} G(x)$$
Integrating both sides with respect to $x$ shows that
$$H\left(y(x)\right)=G(x)+C$$
Therefore the general solution is
$$\int^y h(y)\,dy =\int^x g(x)\,dx +C$$
This is in the form of an integral or implicit solution of the \ODE.

%============================================================================
\begin{example}
For these examples, $h(y)=1$ so that the separable equation has the form
$$\der{y}{x}=g(x)$$

\begin{eqnarray*}
\der{y}{x}=x^n		&\mif&	y(x)=\frac{x^{n+1}}{n+1}+C	\\
\der{y}{x}=\sin ax	&\mif&	y(x)=-\frac{1}{a}\cos ax+C	\\
\der{y}{x}=B\,e^{ax}	&\mif&	y(x)=\frac{B}{a}\,e^{ax}+C	\\
\der{y}{x}=\tan ax	&\mif&	y(x)=-\frac{1}{a}\ln\left|\cos ax\right|+C
\end{eqnarray*}
These are all examples of \name{explicit solutions} because the solution
$y(x)$ is given as a function of $x$.
\end{example}
%============================================================================

Note that $C$, the \name{constant of integration}, is determined by the
problem's initial conditions.

%============================================================================
\begin{example}
To solve
$$\der{y}{x}=\frac{y}{1+x}$$
note that $h(y)=1/y$ and $g(x)=1/(1+x)$.  Then
$$\frac{dy}{y}=\frac{dx}{1+x}$$
Integrating both sides gives
$$\int^y \frac{dy}{y}=\int^x\frac{dx}{1+x}+C$$
which means that
$$\ln\left|y\right|=\ln\left|1+x\right|+C$$
Then by writing $C=\ln\left|A\right|$ and removing the logs, the solution is
$$y=A(1+x)$$
\end{example}
%============================================================================

%============================================================================
\begin{example}
To solve
$$\der{y}{x}=-\frac{x}{y}$$
rewrite the \ODE as
$$y\,dy=-x\,dx$$
Integrating both sides gives the solution
$$y^2=-x^2+C$$
which is the equation for a circle of radius $\sqrt{C}$
$$x^2+y^2=C$$
\end{example}
%============================================================================

%============================================================================
\begin{example}
To solve
$$\der{y}{x}=y^2-a^2$$
rewrite it as
$$\frac{dy}{y^2-a^2}=dx$$
Integrating both sides gives
$$\int^y\frac{dy}{y^2-a^2}=x+C$$
Using partial fractions, the integrand is equal to
$$\frac{1}{y^2-a^2}=\frac{1}{2a}\left(\frac{1}{y-a}-\frac{1}{y+a}\right)$$
Therefore
\begin{eqnarray*}
\int^y\frac{dy}{y^2-a^2}
&=&\frac{1}{2a}\int^y \frac{1}{y-a}-\frac{1}{y+a}\,dy	\\
&=&\frac{1}{2a}\ln\left|\frac{y-a}{y+a}\right|		\\
&=&x+C
\end{eqnarray*}
Now writing $\left|A\right|=e^{2aC}$ gives the solution
$$\frac{y-a}{y+a}=A\,e^{2ax}$$
which as a function $y(x)$ is
$$y=a\frac{1+A\,e^{2ax}}{1-A\,e^{2ax}}$$
\end{example}
%============================================================================


%============================================================================
\begin{example}
The \name{Doomsday Model} is
$$\der{P}{t}=kP$$
with initial condition $P(0)=P_0$.  The general solution is
\begin{eqnarray*}
\int^P\frac{dP}{P}&=&k\int^t dt +C	\\
\thus\ln P&=&kt+C
\end{eqnarray*}
At $t=0$, the initial condition can be used to determine $C$
$$\ln P_0=0+C$$
so that
$$\ln\frac{P}{P_0}=kt$$
or
$$P(t)=P_0\,e^{kt}$$
\end{example}
%============================================================================

%============================================================================
\begin{example}
The \name{logistic equation} is
$$\der{P}{t}=kP(1-bP)$$
with initial condition $P(0)=P_0$.  The solution is found using partial
fractions
\begin{eqnarray*}
\int^P\frac{dP}{P(1-bP)}&=&k\int^t dt +C	\\
\thus \ln \frac{P}{1-bP}&=&kt+C
\end{eqnarray*}
At $t=0$, the initial condition can be used to determine $C$, giving the
solution
$$\frac{P(t)}{1-bP(t)}=\frac{P_0}{1-bP_0}\,e^{kt}$$
which can easily be rearranged to give an explicit solution for $P(t)$ in
terms of $t$.
\end{example}
%============================================================================

%============================================================================
\begin{exercise}
Exercise 2.2 of Zill, pp. 44--46, has more examples of separable \ODEs.
\end{exercise}
%============================================================================

%%%%%%%%%%%%%%%%%%%%%%%%%%%%%%%%%%%%%%%%%%%%%%%%%%%%%%%%%%%%%%%%%%%%%%%%%%%%%
\subsection{Homogeneous Equations}

If a function $f(x,y)$ satisfies
$$f(tx,ty)=t^n\,f(x,y)$$
for all $x$ and $y$ and for some constant real number $n$, then $f(x,y)$ is
said to be a \name{homogeneous} function of degree $n$.

%============================================================================
\begin{example}
Here are some examples of functions that are homogeneous or inhomogeneous:
\begin{eqnarray*}
ax^2+bxy+cy^2 		&&  \mbox{homogenous ($n=2$)}	\\
\ds\frac{ax+by}{cx+dy}	&&  \mbox{homogenous ($n=0$)}	\\
\cos(ax^2+by^2)		&&  \mbox{inhomogenous}	\\
\exp\left(\ds\frac{ax^2+by^2}{cx^2+dy^2}\right)	
			&&  \mbox{homogenous ($n=0$)}	
\end{eqnarray*}
\end{example}
%============================================================================

An \ODE of the form
$$N(x,y)\,\der{y}{x}+M(x,y)=0$$
where $M(x,y)$ and $N(x,y)$ are homogeneous functions of the same degree $n$
is called a \name{homogeneous} \ODE.  Homogeneous \ODEs can be turned into 
separable \ODEs with the substitution
$$u=\frac{y}{x}$$
When using this substitution, note what happens to $M(x,y)$, $N(x,y)$ and
$\der{y}{x}$:
$$M(x,y)=M(x,xu)=x^n\,M(1,u)$$
$$N(x,y)=N(x,xu)=x^n\,N(1,u)$$
$$\ds\der{y}{x}=\ds\der{(xu)}{x}=x\ds\der{u}{x}+u$$
Now, substituting these into the \ODE, we have
$$0=x^n\,M(1,u)+x^n\,N(1,u)\left(x\der{u}{x}+u\right)$$
which can be rearranged to give
$$\left(\frac{N(1,u)}{M(1,u)+uN(1,u)}\right)\der{u}{x}=-\frac{1}{x}$$
which is a separable \ODE in $u$ and $x$.

Note that sometimes it may be simpler to use the alternative substitution
$$v=\frac{x}{y}$$
which gives a different separable equation.

Never try to remember these final equations; learn the principle and
rederive each time starting with the substitution $u=y/x$ or $v=x/y$.

%============================================================================
\begin{example}
To solve
$$\der{y}{x}=\frac{x+3y}{3x+y}=\frac{1+3y/x}{3+y/x}$$
put $u=y/x$ so that $y=xu$ and
$$x\,\der{u}{x}+u=\der{y}{x}$$
Therefore
$$x\,\der{u}{x}+u=\frac{1+3u}{3+u}$$
so that
$$x\,\der{u}{x}=\frac{1+3u}{3+u}-u=\frac{1-u^2}{3+u}$$
Separating the terms involving $x$ and $u$,
$$\int^u\frac{3+u}{1-u^2}\,du=\int^x\frac{dx}{x}+C$$
Using partial fractions,
\begin{eqnarray*}
\int^u \frac{1}{1+u}+\frac{2}{1-u}\,du&=&\ln\left|x\right|+C\\
\thus \ln\left|\frac{1+u}{(1-u)^2}\right|&=&\ln\left|x\right|+C
\end{eqnarray*}
Writing the constant of integration as $C=\ln\left|A\right|$ gives the solution
$$\frac{1+u}{(1-u)^2}=Ax$$
\end{example}
%============================================================================

%============================================================================
\begin{exercise}
Exercise 2.3 of Zill, pp. 52--53, has more examples of homogeneous \ODEs.
\end{exercise}
%============================================================================

%%%%%%%%%%%%%%%%%%%%%%%%%%%%%%%%%%%%%%%%%%%%%%%%%%%%%%%%%%%%%%%%%%%%%%%%%%%%%
\subsection{Linear First Order O.D.E.s}

An \ODE of the form
$$\der{y}{x}+p(x)\,y=f(x)$$
is a \name{linear} first order \ODE.  When $f(x)=0$, the \ODE is
\name{homogeneous}.  When $f(x)\neq 0$, the \ODE is \name{inhomogeneous}.

One \name{trivial solution} of an homogeneous linear \ODE is always
$y=0$.  This usually need not be considered since the initial conditions will
usually be non-zero.

To solve a linear first order \ODE, use the \name{integrating factor},
$I(x)$.  $I(x)$ is defined to be any solution of 
$$\der{I}{x}=p(x)\,I$$
This is a separable equation which is solved using
$$\int^I\frac{dI}{I}=\int^x p(x)\,dx+C$$
so that
$$\ln I=\int^x p(x)\,dx+C$$
Set $C=0$ so that the integrating factor is
$$I(x)=\exp\left(\int^x p(x)\,dx\right)$$
Note that $y=1/I(x)$ is the solution of the homogeneous equation
$$\der{y}{x}+p(x)\,y=0$$
To prove this, substitute $y=1/I(x)$ so that
$$\der{y}{x}=-\frac{1}{I^2}\,\der{I}{x}$$
Now substituting this into the homogeneous equation shows that
$$\der{y}{x}+py=-\frac{1}{I^2}\left(\der{I}{x}-pI\right)=0$$
where the term in parentheses is zero from the definition of the integrating
factor.

Now, to complete the solution of the \ODE
$$\der{y}{x}+p(x)\,y=f(x)$$
multiply both sides by $I(x)$ to give
$$I(x)\,\der{y}{x}+p(x)I(x)\,y=I(x)f(x)$$
But from the definition of the integrating factor, $p(x)I(x)=\der{I}{x}$, 
so that
$$I(x)\,\der{y}{x}+\der{I}{x}\,y=I(x)f(x)$$
Using product rule, the left hand side is seen to be 
$\dbd{x}\left(Iy\right)$, hence
$$\dbd{x}\left[I(x)\,y\right]=I(x)f(x)$$
Integrating to remove the derivative gives the solution
$$I(x)\,y=\int^x I(x)f(x)\,dx+C$$
which is
$$y=C\frac{1}{I(x)}+\frac{1}{I(x)}\int^x I(x)f(x)\,dx$$
The first part of this solution is $C$ times the solution of the homogeneous
\ODE, and the second part is a \name{particular solution} which exists
because $f(x)\neq 0$.


%============================================================================
\begin{example}
To solve the \ODE
$$x^2\,\der{y}{x}+xy=1$$
divide by $x^2$ to turn it into standard form
$$\der{y}{x}+\frac{1}{x}\,y=\frac{1}{x^2}$$

The integrating factor is 
$$I(x)=x$$
found by solving
$$\der{I}{x}=\frac{1}{x}\,I$$
and setting the constant in
$$\int^x \frac{dI}{I}=\int^x\frac{dx}{x}+C$$
to zero.

Now solve the \ODE by multiplying it by the integrating factor
$$x\,\der{y}{x}+y=\frac{1}{x}$$
Using product rule, the left hand side is
$$x\,\der{y}{x}+y=\dbd{x}\left(xy\right)$$
so that 
$$xy=\ln\left|x\right|+C$$
Therefore the solution is
$$y(x)=\frac{\ln\left|x\right|}{x}+\frac{C}{x}$$
\end{example}
%============================================================================

%============================================================================
\begin{example}
To solve the \ODE
$$\left(1-\cos x\right)\der{y}{x}+2y\sin x-\tan x=0$$
put it in standard form, so that
$$\der{y}{x}+\left(\frac{2\sin x}{1-\cos x}\right)y=\frac{\tan x}{1-\cos x}$$
To find the integrating factor, solve $\der{I}{x}=p(x) I$.
\begin{eqnarray*}
\int^I\frac{dI}{I}&=&\int^x\frac{2\sin x}{1-\cos{x}}\,dx	\\
\thus \ln I&=&2\ln(1-\cos x)
\end{eqnarray*}
where the constant of integration has been set to zero.  Therefore
$$I(x)=\left(1-\cos x\right)^2$$

Now, solve the \ODE by multiplying by the integrating factor.  This gives
$$\dbd{x}(Iy)=(1-\cos x)^2\frac{\tan x}{1-\cos x}=(1-\cos x)\tan x$$
Integrate with respect to $x$
\begin{eqnarray*}
Iy&=&\int^x(1-\cos x)\tan x\,dx+C \\
&=&\int^x (\tan x -\sin x)\,dx+C \\
&=&-\int^x \frac{d(\cos x)}{\cos x}+\cos x+C \\
&=&-\ln\left|\cos x\right|+\cos x+C
\end{eqnarray*}
Dividing both sides by $I(x)=(1-\cos x)^2$ gives the solution
$$y(x)=\frac{C}{(1-\cos x)^2}
+\frac{\cos x-\ln\left|\cos x\right|}{(1-\cos x)^2}$$
which is of the form of $C$ times the homogeneous solution plus the
particular solution due to $f(x)$.
\end{example}
%============================================================================

%============================================================================
\begin{example}
Solving
$$\der{y}{x}=\frac{y}{y^3+x}$$
is a little harder because the \ODE is neither linear, homogeneous nor 
separable.  But by inverting it and rearranging, it becomes a linear \ODE
in $y$
$$\der{x}{y}=\frac{y^3+x}{y}\mif\der{x}{y}-\frac{1}{y}\,x=y^2$$
The solution gives $x$ as a function of $y$.  The integrating factor $I(y)$ 
satisfies
$$\der{I}{y}=-\frac{1}{y}I$$
so that
$$I(y)=\frac{1}{y}$$
Multiply both sides of the \ODE by $I(y)$ and use product rule on the
left hand side to give
$$\dbd{y}\left(\frac{x}{y}\right)=y$$
Integrate with respect to $y$ 
$$\frac{x}{y}=\int^y y\,dy+C$$
to show that the solution is
$$x=Cy+\frac{y^3}{2}$$
Once again, this takes the form of $C$ times the homogeneous solution 
plus the particular solution due to $f(x)$.
\end{example}
%============================================================================

%============================================================================
\begin{exercise}
Exercise 2.5 of Zill, pp. 69--71, has more examples of linear \ODEs.
\end{exercise}
%============================================================================

%%%%%%%%%%%%%%%%%%%%%%%%%%%%%%%%%%%%%%%%%%%%%%%%%%%%%%%%%%%%%%%%%%%%%%%%%%%%%
\section{Applications of First Order O.D.E.s}


%----------------------------------------------------------------------------
\begin{figure}\centering
\caption{The curve $y(x)$ is traced out by the weight $W$ dragged by a tractor
$T$ in the tractrix problem of example \protect\ref{ode exam:tractrix}.}
\label{ode fig:tractrix}

\psset{xunit=5cm,yunit=5cm}
\begin{pspicture}(-0.25,-0.25)(1.3,1.4)
% First the curve, so the axes show through
\parametricplot[linecolor=gray,linewidth=2pt,plotstyle=curve]%
{0.23}{1}{1 t t mul sub sqrt dup dup 1 add exch 1 sub neg div ln 2 div
exch sub t}
% Now the axes and their labels
\psline{->}(0,0)(1.2,0)
\psline{->}(0,0)(0,1.2)
\uput[r](1.2,0){$x$}
\uput[u](0,1.2){$y$}
\uput[l](0,0){$T_0$}
\uput[l](0,1){$W_0$}
\uput[ur](0.217,0.666){$W$}
\uput[d](0.962,-0.1){$T$}
\pcline[linecolor=darkgray,linewidth=2pt,plotstyle=curve]%
{-}(0.217,0.666)(0.962,0)
\mput*{$L$}
\pcline[linecolor=black,linewidth=1pt]{<->}(0.217,-0.07)(0.962,-0.07)
\mput*{$\sqrt{L^2-y^2}$}
\pcline[linecolor=black,linewidth=1pt]{<->}(0.217,0)(0.217,0.666)
\mput*{$y$}
\psarcn{->}(0.962,0){1}{180}{139}
\uput[l](0.75,0.06){$\alpha$}
\end{pspicture}
\end{figure}
%----------------------------------------------------------------------------

%============================================================================
\begin{example}[The Tractrix]
\label{ode exam:tractrix}

\problem
A tractor is connected by a taut chain of length $L$ to a weight.  The
tractor begins to move in a direction at right angles to the line joining
the tractor and the weight.  It proceeds in that direction in a straight
line dragging the weight.  What curve does the weight trace out?

\solution
The geometry of this problem is illustrated in figure \ref{ode
fig:tractrix}.  The tractor is the point $T$ and the weight the point $W$.
At time $t=0$, the tractor is at the origin, $(0,0)$, and it moves along the 
$x$ axis.  Since the weight is initially at right angles to the direction of
the tractor's motion, the weight is at $(0,L)$ at time $t=0$. Let the curve 
traced out by the weight be given by $y(x)$, where $y(0)=L$.

The key observation is that the weight moves in the direction of the chain.
Therefore, the chain is always a tangent to the curve $y(x)$.  So if the
weight is at the point $(x,y)$ when the chain is at an angle $\alpha$ to
the $x$ axis, the gradient of the tangent of $y(x)$ satisfies
$$\der{y}{x}=-\tan\alpha=-\frac{y}{\sqrt{L^2-y^2}}$$
This has been obtained by using figure \ref{ode fig:tractrix} and
remembering that the chain's length is $L$.


Therefore the differential equation to be solved is
$$\der{y}{x}=-\frac{y}{\sqrt{L^2-y^2}}$$
with the initial condition
$$y(0)=L$$

The \ODE is separable, so
$$\int^y\frac{\sqrt{L^2-y^2}}{y}\,dy =-\int^x dx+C=-x+C$$
To solve this integral, make the substitution $z^2=L^2-y^2$.  Then
$z\,dz=-y\,dy$.
\begin{eqnarray*}
x&=&C-\int^y\frac{\sqrt{L^2-y^2}}{y}\,dy \\
&=&C+\int^{\sqrt{L^2-y^2}}\frac{z^2}{L^2-y^2}\,dz \\
&=&C+\int^{\sqrt{L^2-y^2}} \left(-1 +\frac{L^2}{L^2-z^2}\right)\,dz \\
&=&C+\int^{\sqrt{L^2-y^2}}\left[ 
-1 +\frac{L}{2}\left(\frac{1}{L-z}+\frac{1}{L+z}\right)\right]\,dz \\
&=&C-\sqrt{L^2-y^2}+\frac{L}{2}\left(-\ln\left(L-\sqrt{L^2-y^2}\right)
+\ln\left(L+\sqrt{L^2-y^2}\right)\right) \\
&=&C-\sqrt{L^2-y^2}+\frac{L}{2}\ln\left(\frac{L+\sqrt{L^2-y^2}}
{L-\sqrt{L^2-y^2}}\right)
\end{eqnarray*}
Applying the initial conditions of $y(0)=L$ shows that
$$0=C-0+\frac{L}{2}\ln\left(\frac{L}{L}\right)=C+0$$
so that $C=0$.

Therefore the equation of the tractrix is
$$x=\frac{L}{2}\ln\left(\frac{L+\sqrt{L^2-y^2}}{L-\sqrt{L^2-y^2}}\right)
-\sqrt{L^2-y^2}$$
\end{example}
%============================================================================

%============================================================================
\begin{exercise}
Now try the tractrix problem again with the tractor heading off at an angle
$\theta$ (which is greater than $\pi/2$) to the initial direction of the
chain.
\end{exercise}
%============================================================================

%----------------------------------------------------------------------------
\begin{figure}\centering
\caption{The curve $y(x)$ is traced out by the coyote as it runs towards the
roadrunner in the pursuit curve problem of example 
\protect\ref{ode exam:pursuit}. (For this example, $\lambda=1/2$ so the coyote
catches the roadrunner)}
\label{ode fig:pursuit}

\psset{xunit=5cm,yunit=5cm}
\begin{pspicture}(-0.25,-0.25)(1.3,1.4)
% First the curve, so the axes show through
\psplot[linecolor=gray,linewidth=2pt,plotstyle=curve]%
{0}{1}{x sqrt x 3 div 1 sub mul 0.666666 add}
% Now the axes and their labels
\psline{->}(0,0)(1.2,0)
\psline{->}(0,0)(0,1.2)
\uput[r](1.2,0){$x$}
\uput[u](0,1.2){$y$}
\uput[ur](0.36,0.13866667){$(x(t),y(t))$}
\rput[r](-0.05,0.5){roadrunner}
\uput[ur](0.36,0.25){coyote}
\pcline[linecolor=darkgray,linewidth=2pt]{*->}(0,0)(0,0.15)
\Aput*{$v$}
\pcline[linecolor=darkgray,linewidth=2pt]{*->}(0,0.33067)(0,0.48067)
\Aput*{$v$}
\uput[l](0,0.33067){$(0,vt)$}
\pcline[linecolor=black,linewidth=1pt,linestyle=dashed]{-}(0,0.33067)(0.36,0.138667)
\pcline[linecolor=black,linewidth=2pt]{*->}(0.36,0.130667)(0.228,0.2091)
\Aput*{$V$}
\pcline[linecolor=black,linewidth=2pt]{*->}(1,0)(0.85,0)
\Bput*{$V$}
\pcline[linecolor=black,linewidth=1pt]{<->}(0,-0.07)(1,-0.07)
\mput*{$L$}
\end{pspicture}
\end{figure}
%----------------------------------------------------------------------------

%============================================================================
\begin{example}[The Pursuit Curve]
\label{ode exam:pursuit}
\problem A roadrunner commences running at constant speed $v$ in a northerly
direction.  A coyote a distance $L$ away to the east scents the roadrunner
and immediately gives chase at constant speed $V$, always running directly
at the roadrunner.  What path does the coyote chase out?

\solution
The diagram in figure \ref{ode fig:pursuit} shows the roadrunner running up
the $y$ axis at speed $v$, starting at the origin at time $t=0$.  At time
$t=0$, the coyote is on the $x$ axis at the point $(L,0)$, and moves towards
the roadrunner with speed $V$.  Let the function $y(x)$ describe the path of
the coyote.

Since the coyote always runs directly towards the roadrunner, the coyote's
velocity vector, which is the tangent to the pursuit curve, intercepts the
$y$ axis at the point $(0,vt)$.  Therefore
$$\der{y}{x}=\frac{y-vt}{x}$$
so that
$$x\,\der{y}{x}=y-vt$$
The time variable $t$ is eliminated by considering the arc length $s$ of the
pursuit curve.  The length of the curve at time $t$ is
$$s=Vt$$
which is the total distance travelled by the coyote at time $t$.

Using $ds^2=dx^2+dy^2$, $\der{s}{x}$ is
$$\der{s}{x}=-\sqrt{1+\left(\der{y}{x}\right)^2}$$
The negative square root has been chosen because $s$ increases as $x$
decreases.

From $s=Vt$, write $t=s/V$ and eliminate $t$ to give
$$x\,\der{y}{x}=y-\frac{v}{V}s$$
For convenience, write the ratio of the roadrunner's and coyote's speeds as
$$\lambda=\frac{v}{V}$$
so that
$$x\,\der{y}{x}=y-\lambda s$$
Differentiate both sides with respect to $x$ 
$$\dbd{x}\left(x\,\der{y}{x}\right)=\der{y}{x}-\lambda\,\der{s}{x}$$
Now substitute for $\der{s}{x}$
$$\der{y}{x}+x\dder{y}{x}=\der{y}{x}+
\lambda\sqrt{1+\left(\der{y}{x}\right)^2}$$
Therefore the equation of the pursuit curve is 
$$x\dder{y}{x}=\lambda\sqrt{1+\left(\der{y}{x}\right)^2}$$


This equation for the pursuit curve appears to be a second order \ODE.  
However make the substitution 
$$w=\der{y}{x}$$
so that the \ODE becomes a first order separable \ODE
$$x\der{w}{x}=\lambda\sqrt{1+w^2}$$
This is solved by separating and integrating to give $w$ as a function of 
$x$.  Then integrating a second time gives $y$ in terms of $x$, as required.

Separating the separable equation in $x$ and $w$ gives
$$\int^w\frac{dw}{\sqrt{1+w^2}}=\lambda \int^x\frac{dx}{x}+C$$
Make the substitution $w=\sinh u$ so that $1+w^2=\cosh^2u$ and
$$du=\frac{dw}{\sqrt{1+w^2}}$$
Therefore
$$\int^{\arcsinh w} du=\arcsinh w=\lambda \ln x+C$$
To find the value of $C$, note that at time $t=0$ when $x=L$,
$$w=\der{y}{x}=0$$
Therefore $0=\lambda \ln L+C$ so
$$\arcsinh w=\ln\left(\frac{x}{L}\right)^{\lambda}$$
Invert the equation to give $w$ as an explicit function of $x$
$$w=\sinh\left(\ln\left(\frac{x}{L}\right)^{\lambda}\right)$$
Then use the expression 
$$\sinh z=\frac{1}{2}\left(e^z-e^{-z}\right)$$
to write $w=\der{y}{x}$ as
$$\der{y}{x}=\frac{1}{2}\left[\left(\frac{x}{L}\right)^{\lambda}
-\left(\frac{L}{x}\right)^{\lambda}\right]$$

Integrating $\der{y}{x}$ with respect to $x$ gives the solution
$$y=\frac{1}{2(\lambda+1)}\frac{x^{\lambda+1}}{L^{\lambda}}
+\frac{1}{2(\lambda-1)}\frac{L^{\lambda}}{x^{\lambda-1}}+C$$
provided $\lambda\neq 1$.  The constant $C$ is determined by the initial
condition $y=0$ at $x=L$ so that
$$C=-\frac{L\lambda}{\lambda^2-1}$$

%============================================================================
\end{example}

%----------------------------------------------------------------------------
\begin{figure}\centering
\caption{The catenary is the shape made by a hanging chain, as in 
example \protect\ref{ode exam:catenary}.}
\label{ode fig:catenary}

\psset{xunit=5cm,yunit=5cm}
\begin{pspicture}(-0.6,-0.25)(1.3,1.4)
% First the curve, so the axes show through
\psplot[linecolor=gray,linewidth=2pt,plotstyle=curve]%
{-0.5}{1}{2.71828 x 1.5 mul exp 2.71828 x 1.5 mul neg exp add 2 div 1 sub 1.5 div}
% Now the axes and their labels
\psline{->}(-0.5,0)(1.2,0)
\psline{->}(0,0)(0,1.2)
\uput[d](0,0){$O$}
\uput[r](1.2,0){$x$}
\uput[u](0,1.2){$y$}
\uput[u](-0.5,0.196){$A$}
\uput[u](1,0.9){$B$}
\uput[ul](0.8,0.54){$P(x,y)$}
\pcline[linecolor=black,linewidth=2pt]{*->}(0,0)(-0.25,0)
\Aput*{$T_0$}
\pcline[linecolor=black,linewidth=2pt]{*->}(0.55,0.24)(0.55,-0.01)
\uput[d](0.55,-0.01){$W_P$}
\pcline[linecolor=black,linewidth=1pt,linestyle=dashed]{-}(0.8,0.54)(0.442,0)
\pcline[linecolor=black,linewidth=2pt]{*->}(0.8,0.54)(0.8828,0.665)
\Bput*{$T$}
\psarc{->}(0.442,0){1}{0}{56}
\uput[r](0.64,0.06){$\theta$}
\end{pspicture}
\end{figure}
%----------------------------------------------------------------------------

%============================================================================
\begin{example}[The Catenary]
\label{ode exam:catenary}

\problem
A flexible chain of mass $\rho$ per unit length hangs under gravity from
points $A$ and $B$.  Find the shape of the chain.

\solution
At the lowest point in the chain, its slope is zero.  Take this point as the
origin.  If the chain takes the shape given by $y(x)$, consider the slope of
the chain at some point $P(x,y)$, as shown in figure \ref{ode fig:catenary}.


Define $T$ as the tension in the chain at $P=(x,y)$, $T_0$ as the tension in
the chain at the origin $O=(0,0)$, and $W_P$ as the weight of the segment of 
chain from the origin to the point $P$.  The weight $W_P$ is
$$W_P=\rho g s$$
where $\rho$ is the known mass per unit length of the chain, $g$ is the
acceleration due to gravity and $s$ is the length of the arc $OP$.
The tangent to the chain at $P$ makes an angle $\theta$ with the $x$ axis.

The derivative of $s$ is given in terms of the chain's shape $y(x)$ by
$$\der{s}{x}=\sqrt{1+\left(\der{y}{x}\right)^2}$$
Here the positive square root has been chosen because $s$ increases with $x$
if $P$ is on the right hand side of the $y$ axis.  If $P$ were on the other
side of the $y$ axis, the negative square root would have been chosen.

Resolving the horizontal forces on the chain segment $OP$ shows that
$$T_0=T\cos\theta$$
and resolving vertically shows that
$$W_P=T\sin\theta$$
Dividing these two gives
$$\tan\theta=\frac{W_P}{T_0}=\frac{\rho g}{T_0}\,s$$
But $\theta$ is the angle of the tangent to the curve so
$$\der{y}{x}=\tan\theta=\frac{\rho g}{T_0}\,s$$
Write $\lambda=\rho g/T_0$ to simplify the notation, and differentiate both
sides with respect to $x$ so that
$$\dder{y}{x}=\lambda \der{s}{x}=\lambda\sqrt{1+\left(\der{y}{x}\right)^2}$$
which is the equation of the catenary.

The solution of the catenary is found by putting $p=\der{y}{x}$ and solving
the first order separable equation in $p$, then integrating $p$ with respect
to $x$ to give $y$ as a function of $x$.

With $p=\der{y}{x}$, differentiating gives $\der{p}{x}=\dder{y}{x}$ so that
the catenary's differential equation becomes
$$\der{p}{x}=\lambda\sqrt{1+p^2}$$
which is separable with
$$\int^p\frac{dp}{\sqrt{1+p^2}}=\lambda\int^x dx+C$$
Make the substitution $p=\sinh u$ so that $1+p^2=\cosh^2 u$ and $dp=\cosh
u\,du$.  Integrating gives
$$\arcsinh p=\lambda x +C$$
which is
$$\der{y}{x}=p=\sinh\left(\lambda x+C\right)$$

The constant of integration has to be chosen so that $\der{y}{x}=0$ at the
origin, because the origin was placed at the lowest point of the catenary.
Substituting in $x=0$ and $\der{y}{x}=0$ shows that the constant is $C=0$.
Therefore
$$\der{y}{x}=\sinh\lambda x$$
Integrating with respect to $x$ gives
$$y=\frac{1}{\lambda}\cosh\lambda x+C$$
Once again, the catenary passes through the origin, so $y(0)=0$.  The
constant is $C=-1/\lambda$, giving the equation of the catenary as
$$y=\frac{1}{\lambda}\left[\cosh\left(\lambda x\right)-1\right]$$

This is still not quite the complete solution because 
$\lambda=\rho g/T_0$, the tension at the origin $T_0$, has
not yet been determined.  This depends on the length of the chain and the
position of the endpoints relative to the origin. 

\parbreak

Another way of solving the catenary starts with the equation
$$\dder{y}{x}=\lambda\sqrt{1+\left(\der{y}{x}\right)^2}$$
and uses the identity
$$\dbd{y}\left[\frac{1}{2}\left(\der{y}{x}\right)^2\right]=\dder{y}{x}$$
to write the catenary's differential equation as
$$\dbd{y}\left[\frac{1}{2}\left(\der{y}{x}\right)^2\right]
=\lambda\sqrt{1+\left(\der{y}{x}\right)^2}$$
Now make the substitution
$$Q=1+\left(\der{y}{x}\right)^2$$
so that the catenary's equation becomes
$$\der{Q}{y}=2\lambda \sqrt{Q}$$
This is separable, so
\begin{eqnarray*}
\int^Q \frac{dQ}{\sqrt{Q}}&=&2\lambda \int^y y\,dy+C \\
\thus 2\sqrt{Q}&=&2\lambda y+C
\end{eqnarray*}
At the origin, $Q=1$ because $\der{y}{x}=0$.  Therefore $C=2$ and 
$$\sqrt{Q}=\lambda y+1$$
Squaring both sides gives
$$1+\left(\der{y}{x}\right)^2=(1+\lambda y)^2$$
Move the $1$ to the right-hand side and take the square root, giving
$$\der{y}{x}=\sqrt{(1+\lambda y)^2-1}$$
The positive  square root has been chosen so that $\der{y}{x}>0$ for $x>0$
(the negative root would have been chosen for the part of the catenary with
$x<0$).  This is separable, so separate and integrate
$$\int^y\frac{dy}{\sqrt{(1+\lambda y)^2-1}}=\int^x dx+C$$

Make the substitution $1+\lambda y=\cosh u$ to show that the solution is
$$\frac{1}{\lambda}\arccosh(1+\lambda y)=x+C$$
which is
$$1+\lambda y=\cosh\lambda(x+C)$$
Using the boundary condition that $y(0)=0$ shows that $C=0$ so the equation
of the catenary is
$$y=\frac{1}{\lambda}\left[\cosh\left(\lambda x\right)-1\right]$$
which is the same as the first solution.
\end{example}
%============================================================================


%----------------------------------------------------------------------------
\begin{figure}\centering
\caption{The mirror $y(x)$ is shaped so that beams of light parallel to the
$y$ axis are reflected to the origin, as described in example  
\protect\ref{ode exam:mirror}.}
\label{ode fig:mirror}

\psset{xunit=5cm,yunit=5cm}
\begin{pspicture}(-0.25,-0.35)(1.3,1.4)
% First the curve, so the axes show through
\psplot[linecolor=gray,linewidth=2pt,plotstyle=curve]{0}{1}{x x mul 0.25 sub}
% Now the axes and their labels
\psline{->}(-0.25,0)(1.2,0)
\psline{->}(0,-0.3)(0,1.2)
\uput[dl](0,0){$O$}
\uput[r](1.2,0){$x$}
\uput[u](0,1.2){$y$}
\uput[r](0.75,0.3125){$(x,y)$}
\psline[linecolor=black,linewidth=2pt]{-}(0.75,1.2)(0.75,0.3125)(0,0)
\psline[linecolor=black,linewidth=2pt]{->}(0.75,1.2)(0.75,0.8)
\psline[linecolor=black,linewidth=2pt]{->}(0.75,0.3125)(0.375,0.15625)
\psline[linecolor=black,linewidth=1pt,linestyle=dashed]{-}(0.5416666,0)(1,0.6425)
\psline[linecolor=black,linewidth=1pt,linestyle=dashed]{-}(0.75,0.3125)(0.542,0.4504)
\psarc{->}(0.5416666,0){1}{0}{56.3}
\uput[r](0.74,0.06){$\theta$}
\psarc{->}(0.75,0.3125){1}{56.3}{90}
\uput[u](0.8,0.5){$\alpha$}
\psarcn{->}(0,0){1}{90}{23}
\uput[ur](0.05,0.18){$\phi$}
\end{pspicture}
\end{figure}
%----------------------------------------------------------------------------

%============================================================================
\begin{example}[Perfect Mirror Focus]
\label{ode exam:mirror}

\problem
Light strikes a plane curve in such a manner that all beams parallel to the
$y$ axis are reflected to a single point $O$.  Determine the shape of the
curve.

\solution
This problem is described in figure \ref{ode fig:mirror}, where the point
$O$ has been placed at the origin.  Using the property of a mirror that the 
angle of incidence is equal to the angle of reflection,
$$(\pi-\phi)+2\alpha=\pi$$
Thus $\phi=2\alpha$.

With $\theta$ the angle the tangent to the mirror makes with the $x$ axis,
the geometry requires that
$$(\pi-\phi)+\alpha+\left(\frac{\pi}{2}-\theta\right)=\pi$$
which shows that $\theta=\frac{\pi}{2}-\alpha$.

The mirror's equation is given by $y(x)$, so that from 
figure~\ref{ode fig:mirror},
$$\tan\left(\frac{\pi}{2}-\phi\right)=\frac{y}{x}$$
Writing $\phi$ in terms of $\theta$ gives the condition
$$\frac{y}{x}=\tan\left(2\theta-\frac{\pi}{2}\right)$$
Now
$$\tan\left(2\theta-\frac{\pi}{2}\right)=-\cot
2\theta=-\frac{1-\tan^2\theta}{2\tan\theta}$$
Since this is equal to $y/x$ and using $\tan\theta=\der{y}{x}$, the
differential equation describing the mirror is
$$\frac{y}{x}=-\frac{1-\left(\der{y}{x}\right)^2}{2\der{y}{x}}$$
Rearrange this to give
$$2\frac{y}{x}\der{y}{x}-\left(\der{y}{x}\right)^2=-1$$
The solution is found by putting $w=x^2$ so that
$$\der{y}{w}=\der{y}{x}\der{x}{w}=\frac{1}{2x}\der{y}{x}$$
Therefore the differential equation becomes
$$4y\der{y}{w}-4w\left(\der{y}{w}\right)^2=-1$$
Divide both sides by $4\left(\der{y}{w}\right)^2$ to give
$$y\der{w}{y}-w=-\frac{1}{4}\left(\der{w}{y}\right)^2$$

This is an example of Clairaut's equation
$$y=x\der{y}{x}+f\left(\der{y}{x}\right)$$
which is described in detail in section \ref{ode sec:clairaut} and 
chapter 2.6 of Zill.  The solution is
$$y=mx+f(m)$$
where $m$ is an arbitrary constant.  Applying this to the mirror's
differential equation, the solution is
$$w=my+\frac{1}{4}m^2$$
Since $w=x^2$, this is
$$y=\frac{x^2}{m}-\frac{m}{4}$$
which is the equation of a parabola.  

\parbreak

The shape of the mirror can also be found without using the substitution
leading to Clairaut's equation.  Starting with
$$2\frac{y}{x}\der{y}{x}-\left(\der{y}{x}\right)^2=-1$$
divide by $\frac{1}{x}\left(\der{y}{x}\right)^2$ to obtain
$$x\left(\der{x}{y}\right)^2+2y\der{x}{y}-x=0$$
This is a quadratic in $\der{x}{y}$ with roots
$$\der{x}{y}=\frac{-y\pm\sqrt{x^2+y^2}}{x}$$
Rearrange this to give
$$\pm\frac{x\,dx+y\,dy}{\sqrt{x^2+y^2}}=dy$$
Now $x\,dx+y\,dy=\frac{1}{2}d(x^2+y^2)$ so
$$\pm\frac{\frac{1}{2}d(x^2+y^2)}{\sqrt{x^2+y^2}}=dy$$
Integrate both sides
$$\pm\sqrt{x^2+y^2}=y+C$$
and square so that
$$x^2+y^2=(y+C)^2=y^2+2yC+C^2$$
Make $y$ the subject
$$y=\frac{x^2}{2C}-\frac{C}{2}$$
and put $m=2C$ to give the answer in the same form as before
$$y=\frac{x^2}{m}-\frac{m}{4}$$
\end{example}
%============================================================================

%============================================================================
\begin{exercise}
Chapters 1 and 3 of Zill have problems involving applications of
first order \ODEs.
\end{exercise}
%============================================================================

%%%%%%%%%%%%%%%%%%%%%%%%%%%%%%%%%%%%%%%%%%%%%%%%%%%%%%%%%%%%%%%%%%%%%%%%%%%%%
\section{Special Nonlinear First Order O.D.E.s}
\label{ode sec:special}

%%%%%%%%%%%%%%%%%%%%%%%%%%%%%%%%%%%%%%%%%%%%%%%%%%%%%%%%%%%%%%%%%%%%%%%%%%%%%
\subsection{Bernoulli's Equation}
\label{ode sec:bernoulli}

An \ODE of the form
$$\der{y}{x}+P(x)y=f(x)y^n$$
where $n$ is any real number, is called \name{Bernoulli's equation} (named
after James Bernoulli (1654--1705)).

Note that we can already solve this equation for $n=0$ and $n=1$.  When
$n=0$, the equation is an inhomogeneous linear \ODE.  When $n=1$, the
equation is a homogeneous linear \ODE.

To solve Bernoulli's equation, make the substitution
$$w=y^{1-n}$$
to obtain a linear expression.  Since
$$y=w^{\frac{1}{1-n}}$$
differentiating $y$ with respect to $x$ gives 
$$\der{y}{x}=\frac{1}{1-n}w^{\frac{1}{1-n}-1}\der{w}{x}$$
Substitute this into the differential equation
$$\frac{1}{1-n}w^{\frac{n}{1-n}}\der{w}{x}+P(x)w^{\frac{1}{1-n}}=
f(x)w^{\frac{n}{1-n}}$$
then multiply each side by $1-n$ and divide by $w^{\frac{n}{1-n}}$.  This
results in the following linear \ODE
$$\der{w}{x}+(1-n)P(x)w=f(x)(1-n)$$
which can easily be solved.

%============================================================================
\begin{example}
To solve
$$\der{y}{x}+\frac{1}{x}y=xy^2$$
note that this is Bernoulli's equation with $P(x)=1/x$, $f(x)=x$ and $n=2$.
Try the substitution $w=y^{1-n}=y^{1-2}=1/y$ so that 
$$y=\frac{1}{w}$$
Then
$$\der{y}{x}=-\frac{1}{w^2}\der{w}{x}$$
which gets substituted into the \ODE giving
$$-\frac{1}{w^2}\der{w}{x}+\frac{1}{xw}=\frac{x}{w^2}$$
Upon rearranging, we have
$$\der{w}{x}-\frac{1}{x}w=-x$$
The integrating factor is found by solving
$$\der{I}{x}=-\frac{1}{x}I$$
so that $\ln I=- \ln x$ or $I=1/x$.  Multiply the \ODE by the integrating
factor to obtain
$$\dbd{x}\left[\frac{w}{x}\right]=-1$$
whence
$$\frac{w}{x}=-x+c$$
so that
$$w=cx-x^2$$
Expressed in terms of $y$ rather than $w$, this is
$$y=\frac{1}{cx-x^2}$$
\end{example}
%============================================================================

%============================================================================
\begin{exercise}
Exercise 2.6 of Zill, pp. 75--76, has more examples of Bernoulli's equation.
\end{exercise}
%============================================================================

%%%%%%%%%%%%%%%%%%%%%%%%%%%%%%%%%%%%%%%%%%%%%%%%%%%%%%%%%%%%%%%%%%%%%%%%%%%%%
\subsection{Ricatti's Equation}
\label{ode sec:ricatti}

An \ODE of the form
$$\der{y}{x}=P(x)+Q(x)y+R(x)y^2$$
is called \name{Ricatti's equation} (named after Count Jacobo Francesco
Ricatti (1676--1754)).  Note that
\begin{itemize}
\item For many $P(x)$, $Q(x)$ and $R(x)$ the solution cannot be expressed in
terms of elementary functions.
\item If $y_1(x)$ is a known solution of Ricatti's equation, then a family
of solutions is
$$y(x)=y_1(x)+u(x)$$
where $u(x)$ satisfies
$$\der{u}{x}-(Q+2y_1R)u=Ru^2$$
which is Bernoulli's equation with $n=2$.
\item When $P(x)=0$, Ricatti's equation reduces to a Bernoulli equation with
$n=2$.
\end{itemize}

%============================================================================
\begin{example}
To solve
$$\der{y}{x}=2-2xy+y^2$$
given that $y_1=2x$ is one solution, put
$$y=2x+u$$
Then
$$\der{y}{x}=2+\der{u}{x}$$
so that the differential equation becomes
$$2+\der{u}{x}=2-2x(2x+u)+(2x+u)^2$$
which simplifies to
$$\der{u}{x}=2xu+u^2$$
To solve this Bernoulli equation, put $w=u^{1-2}=1/u$ so that
$u=1/w$.  Substitute this into the \ODE to obtain
$$-\frac{1}{w^2}\der{w}{x}=\frac{2x}{w}+\frac{1}{w^2}$$
This is the first order linear \ODE
$$\der{w}{x}+2xw=-1$$
The integrating factor is found by solving
$$\der{I}{x}=2xI$$
hence
$$I(x)=e^{x^2}$$
Multiply the \ODE by the integrating factor to give
$$\dbd{x}\left[e^{x^2}w\right]=-e^{x^2}$$
Integrating gives
$$w=-e^{-x^2}\int^xe^{t^2}\,dt+Ce^{-x^2}$$
Use $u=1/w$ and $y=2x+u$ to give the final solution
$$y(x)=2x+\frac{e^{x^2}}{C-\int^xe^{t^2}\,dt}$$
which cannot be represented in terms of elementary functions without the 
integral.
\end{example}
%============================================================================

%============================================================================
\begin{exercise}
Exercise 2.6 of Zill, pp. 75--76, has more examples of Ricatti's equation.
\end{exercise}
%============================================================================

%%%%%%%%%%%%%%%%%%%%%%%%%%%%%%%%%%%%%%%%%%%%%%%%%%%%%%%%%%%%%%%%%%%%%%%%%%%%%
\subsection{Clairaut's Equation}
\label{ode sec:clairaut}

An \ODE of the form
$$F\left(y-x\der{y}{x},\der{y}{x}\right)=0$$
or alternatively
$$y=x\der{y}{x}+f\left(\der{y}{x}\right)$$
is a \name{Clairaut equation} (named after Alexis Claude Clairaut
(1713--1765)). 

There is a family of solutions that are straight lines, and there may also
be a second solution that is singular.

The family of solutions that are straight lines have the form
$$y=mx+f(m)$$
where each value of $m$ gives a different solution.  To see that these are
solutions of the Clairaut equation, substitute into the \ODE.
From $y=mx+f(m)$, 
$$\der{y}{x}=m$$
so that
$$x\der{y}{x}+f\left(\der{y}{x}\right)=xm+f(m)=y$$
as required.

Clairaut's equation can possess a second solution, the \name{singular 
solution}, which is not obtainable from the general solution
$y=mx+f(m)$.  It is not in general a straight line.  The singular solution
expresses the relationship between $x$ and $y$ as a pair of
parametric equations in $t$
\begin{eqnarray*}
x(t)&=&-f'(t)\\
y(t)&=&f(t)-tf'(t)
\end{eqnarray*}

To prove that this does give a solution to Clairaut's equation,
differentiate the parametric equations for $x$ and $y$ with respect to $t$
\begin{eqnarray*}
\der{x}{t}&=&-f''(t)\\
\der{y}{t}&=&f'-tf''-f'=-tf''
\end{eqnarray*}
Therefore
$$\der{y}{x}=\frac{\der{y}{t}}{\der{x}{t}}=t$$
provided $f''(t)\neq 0$.

Then since
$$y(t)=f(t)-tf'(t)=(-f'(t))t+f(t)$$
substituting $t=\der{y}{x}$ gives
$$y=x\der{y}{x}+f\left(\der{y}{x}\right)$$
which is indeed Clairaut's equation.

%============================================================================
\begin{example}
To solve
$$y=xy'+\frac{1}{(y')^2}$$
note that this is Clairaut's equation with the general solution
$$y=mx+\frac{1}{m^2}$$

Since $f(t)=1/t^2$, the singular solution is given parametrically by
\begin{eqnarray*}
x(t)&=&\frac{2}{t^3}\\
y(t)&=&\frac{1}{t^2}-t\left(-\frac{2}{t^3}\right)=\frac{3}{t^2}
\end{eqnarray*}
To eliminate $t$, express both $x$ and $y$ in terms of $1/t^2$
$$\left(\frac{x}{2}\right)^{2/3}=\frac{1}{t^2}=\frac{y}{3}$$
which is simply
$$y=3\left(\frac{x}{2}\right)^{2/3}$$

\end{example}
%============================================================================

Note that if you are asked to solve a Clairaut equation, you should always
give both solutions.

%============================================================================
\begin{exercise}
Exercise 2.6 of Zill, pp. 75--76, has more examples of Clairaut's equation.
\end{exercise}
%============================================================================

%%%%%%%%%%%%%%%%%%%%%%%%%%%%%%%%%%%%%%%%%%%%%%%%%%%%%%%%%%%%%%%%%%%%%%%%%%%%%
\section[Second Order O.D.E.s]{Second Order Ordinary Differential Equations}
\label{ode sec:soode}

An \ODE of the form
$$\dder{y}{x}+P(x)\der{y}{x}+Q(x)y=R(x)$$
is a \name{second order linear \ODE}.  If $R(x)=0$, the \ODE is
\name{homogeneous}.  If $R(x)\neq 0$, the \ODE is \name{inhomogeneous}.

A solution to a second order linear \ODE ought to require two additional
pieces of information to specify it uniquely.  In an \name{initial value 
problem}, two initial conditions on the same point are specified
$$y(a)=c\qquad \der{y}{x}(a)=d$$
provided $P$, $Q$ and $R$ are defined at $x=a$.

There are two types of \name{boundary value problems}, simple and mixed.
A \name{simple boundary condition} specifies the value of the solution at two
different points
$$y(a)=c\qquad y(b)=d$$
A \name{mixed boundary condition} specifies the value of the solution and its
derivative at two different points
$$y(a)=c\qquad \der{y}{x}(b)=d$$
Boundary value problems may have a unique solution, several solutions, or no
solution.

%============================================================================
\begin{example}
\problem
Solve 
$$y''+a^2y=0$$
subject to
$$y(0)=0\qquad y'(0)=1$$

\solution
This is an initial value problem.  The solution is
$$y(x)=\frac{1}{a}\sin ax$$
To prove that this is the solution, $y(0)=0$ and
$$y'(x)=\cos ax$$
so $y'(0)=1$, therefore the initial conditions are satisfied.  Now
$$y''(x)=-a\sin ax=-a^2\,y(x)$$
so that $y''+a^2y=0$ as required.

Note that the second solution to the differential equation is 
$$y(x)=C\cos ax$$
but this does not satisfy the initial conditions.
\end{example}
%============================================================================

%============================================================================
\begin{example}
\problem Solve 
$$y''+a^2y=0$$
subject to
$$y(0)=0\qquad y\left(\frac{2\pi}{a}\right)=0$$

\solution
This is a boundary value problem.  Note that $y=0$ is a solution.  To see
whether there are any others, the general solution to the differential
equation is
$$y(x)=C_1\sin ax + C_2\cos ax$$
Now $C_1$ and $C_2$ have to be chosen to satisfy the boundary conditions.
$y(0)=C_2$ so the boundary condition $y(0)=0$ means that $C_2=0$.
The second boundary condition is always satisfied because
$y\left(2\pi/a\right)=C_2=0$. 
Therefore, writing $C=C_1$, there are infinitely many solutions satisfying
the \ODE and both boundary conditions
$$y(x)=C\sin ax$$
\end{example}
%============================================================================

%%%%%%%%%%%%%%%%%%%%%%%%%%%%%%%%%%%%%%%%%%%%%%%%%%%%%%%%%%%%%%%%%%%%%%%%%%%%%
\subsection{Superposition of Solutions}

Two functions $y_1(x)$ and $y_2(x)$ are \name{linearly independent} on the
range $a\leq x\leq b$ if no non-zero constants $C_1$ and $C_2$ can be found
such that
$$C_1y_1(x)+C_2y_2(x)=0$$
for all $x$ in the interval $[a,b]$.


%============================================================================
\begin{theorem}
If $y_1(x)$ and $y_2(x)$ are two solutions of the homogeneous equation
$$y''+Py'+Qy=0$$
then $C_1y_1(x)+C_2y_2(x)$ is also a solution.
\end{theorem}
%============================================================================

\begin{proof}
$y_1(x)$ and $y_2(x)$ are both solutions so
$$y_1''+Py_1'+Qy_1=0$$
$$y_2''+Py_2'+Qy_2=0$$
Now multiply the first by $C_1$ and the second by $C_2$
and add, showing that
$$\ndbd{x}{2}(C_1y_1+C_2y_2)+P\dbd{x}(C_1y_1+C_2y_2)+Q(C_1y_1+C_2y_2)=0$$
Therefore $C_1y_1+C_2y_2$ satisifes the \ODE, so it is also a solution.
\end{proof}

%============================================================================
\begin{theorem}
\label{ode thm:gen sol}
If $y_1(x)$ and $y_2(x)$ are two linearly independent solutions of the 
homogeneous equation
$$y''+Py'+Qy=0$$
then $C_1y_1(x)+C_2y_2(x)$ is the most general solution.
\end{theorem}
%============================================================================

%============================================================================
\begin{example}
The second order homogeneous \ODE
$$y''+a^2y=0$$
has solutions
$$y_1(x)=\sin ax$$
and
$$y_2(x)=\cos ax$$
Now $y_1/y_2=\tan ax$, which is not constant over any interval. 
Therefore $y_1$ and $y_2$ are linearly independent and the general solution
is
$$y(x)=C_1\sin ax + C_2\cos ax$$
where $C_1$ and $C_2$ can be determined by initial or boundary conditions.
\end{example}
%============================================================================

%%%%%%%%%%%%%%%%%%%%%%%%%%%%%%%%%%%%%%%%%%%%%%%%%%%%%%%%%%%%%%%%%%%%%%%%%%%%%
\subsection{Inhomogeneous Second Order Linear O.D.E.s}

%============================================================================
\begin{theorem}
If $y_p(x)$ is any solution of the inhomogeneous \ODE
$$y''+Py'+Qy=R$$
then the general solution is
$$y(x)=C_1y_1(x)+C_2y_2(x)+y_p(x)$$
where $y_1(x)$ and $y_2(x)$ are two linearly independent solutions of the 
homogeneous \ODE
$$y''+Py'+Qy=0$$
\end{theorem}
%============================================================================

\begin{proof}
Let $y(x)$ be the general solution of the inhomogeneous \ODE and $y_p(x)$
be any particular solution of the inhomogeneous \ODE.  Form the function
$Y=y-y_p$.  Then
\begin{eqnarray*}
Y''+PY'+QY&=&(y''+Py'+Qy)-(y_p''+Py_p'+Qy_p) \\
&=&R-R\\
&=&0
\end{eqnarray*}
So $Y$ satisfies the homogeneous \ODE.  By theorem \ref{ode thm:gen sol}, 
the general form of $Y$ is
$$Y=C_1y_1+C_2y_2$$
where $y_1$ and $y_2$ are linearly independent solutions of the homogeneous
\ODE.  Therefore
$$y=Y+y_p=C_1y_1+C_2y_2+y_p$$
\end{proof}

This shows that any solution of an inhomogeneous linear \ODE can be written 
as the sum of the particular solution to the inhomogeneous \ODE and the
general solution of the homogeneous \ODE.

%============================================================================
\begin{example}
To solve
$$y''+a^2y=1$$
note that a particular solution of this \ODE is
$$y_p(x)=\frac{1}{a^2}$$
We already know that $y_1=\sin ax$ and $y_2=\cos ax$ are linearly
independent solutions of
$$y''+a^2y=0$$
Therefore the general solution of the inhomogeneous \ODE is
$$y(x)=C_1\sin ax + C_2\cos ax + \frac{1}{a^2}$$
\end{example}
%============================================================================


%============================================================================
\begin{exercise}
Repeat this example with $R(x)$ equal to $1+x$, $1+x^2$ and $e^{\lambda x}$.
\end{exercise}
%============================================================================

%%%%%%%%%%%%%%%%%%%%%%%%%%%%%%%%%%%%%%%%%%%%%%%%%%%%%%%%%%%%%%%%%%%%%%%%%%%%%
\subsection{Reduction of Order}

The method of \name{reduction of order} is a way of obtaining a second 
solution $y_2(x)$ of 
$$y''+Py'+Qy=0$$
given a first solution $y_1(x)$.

To derive the method, put $y_2=uy_1$ and substitute into the \ODE
Then using product rule
$$y_2'=u'y_1+uy_1'$$
and
$$y_2''=u''y_1+2u'y_1'+uy_1''$$
These show that
$$y_2''+Py_2'+Qy_2=(u''+Pu')y_1+2u'y_1'+u(y_1''+Py_1'+Qy_1)$$
Now $y_1''+Py_1'+Qy_1=0$ because $y_1$ is a solution of the \ODE.  If
$y_2$ is to be a solution of the \ODE too, then $y_2$ must satisfy
$$y_2''+Py_2'+Qy_2=0$$
But
$$y_2''+Py_2'+Qy_2=(u''+Pu')y_1+2u'y_1'$$
so for $y_2$ to be a solution of the \ODE, 
$$(u''+Pu')y_1+2u'y_1'=0$$
Rearranging this shows that $u$ must satisfy
$$u''y_1+(Py_1+2y_1')u'=0$$
Divide by $y_1$ and put $w=u'$.  Then
$$w'+\left(P+2\frac{y_1'}{y_1}\right)w=0$$
This is a linear first order \ODE which can be solved by finding the
integrating factor to give $w(x)$.  Then $u'(x)=w(x)$ so determine $u(x)$ by
integrating $w(x)$.  Finally construct the second solution
$$y_2(x)=u(x)y_1(x)$$

Note that one solution of the second order homogeneous \ODE is $u$ equal
to any constant.  This can always be neglected because it would not produce
a linearly independent $y_2(x)$.

%============================================================================
\begin{example}
\problem Find a second solution of 
$$x^2y''+2xy'-6y=0$$
given that $y_1=x^2$ is a solution.

\solution Look for a second solution of the form $y_2=ux^2$.  Then
$$y_2'=2xu+x^2u'$$
and
$$y_2''=2u+4xu'+x^2u''$$
Substituting $y_2$ into the \ODE gives
$$0=x^2y_2''+2xy_2'-6y_2=x^4u''+6x^3u'+u(2x^2+4x^2-6x^2)$$
As expected, the last term is zero, leaving
$$x^4u''+6x^3u'=0$$
if $y_2$ is a second solution.  Introduce $w=u'$ so that
$$x^4w'+6x^3w=0$$
So
$$w'+\frac{6}{x}w=0$$ which means that
$$u'=w=\frac{C}{x^6}$$
Therefore
$$u=\int^x \frac{C}{x^6}\,dx+A=\frac{D}{x^5}+A$$
where the constant $D=-\frac{C}{5}$.  Thus
$$y_2=ux^2=\frac{D}{x^3}+Ax^2$$
The $Ax^2$ term is an arbitrary multiple of $y_1$ so the second linearly
independent solution is
$$y_2(x)=\frac{1}{x^3}$$
\end{example}
%============================================================================

%============================================================================
\begin{exercise}
Exercise 4.2 of Zill, pp. 158--159, has more examples of reduction of order.
\end{exercise}
%============================================================================

%%%%%%%%%%%%%%%%%%%%%%%%%%%%%%%%%%%%%%%%%%%%%%%%%%%%%%%%%%%%%%%%%%%%%%%%%%%%%
\subsection{Homogeneous Second Order O.D.E.s with Constant Coefficients}

A homogeneous second order linear \ODE with \name{constant coefficients}
has the form
$$ay''+by'+cy=0$$
where $a$, $b$ and $c$ are constants.

The method of solution involves substituting 
$$y=e^{mx}$$
to obtain
$$(am^2+bm+c)e^{mx}=0$$
Thus $e^{mx}$ is a solution provided $m$ is chosen so that
$$am^2+bm+c=0$$
or, using the general quadratic formula,
$$m=\frac{-b\pm\sqrt{b^2-4ac}}{2a}$$

There are three cases that can arise, depending on whether $b^2-4ac<0$,
$b^2-4ac=0$ or $b^2-4ac>0$.

If $b^2-4ac>0$, there are two real roots, $m_1$ and $m_2$, and the general
solution is the sum of two exponentials
$$y(x)=C_1e^{m_1x}+C_2e^{m_2x}$$

If $b^2-4ac=0$, there is only one real root, $m_1=-b/2a$ so
$$y_1(x)=e^{-\frac{b}{2a}x}$$
The second solution can be found using reduction of order.  Put
$$y_2=ue^{-\frac{b}{2a}x}$$
so that
$$y_2'=\left(u'-\frac{b}{2a}u\right)\,e^{-\frac{b}{2a}x}$$
and
$$y_2''=\left(u''-\frac{b}{a}u'+\frac{b^2}{4a^2}u\right)\,e^{-\frac{b}{2a}x}$$
Therefore
$$ay_2''+by_2'+cy_2=\left(au''+u\left(\frac{b^2}{4a}-\frac{b^2}{2a}+c\right)
\right)\,e^{-\frac{b}{2a}x}$$
The coefficient of $u$ can be written
$$\frac{b^2}{4a}-\frac{b^2}{2a}+c=-\frac{b^2-4ac}{4a}$$
which is zero because $b^2-4ac=0$.  Therefore
$$u''=0$$
so that $u=x$, which means
$$y_2(x)=x\,e^{-\frac{b}{2a}x}$$
This gives the complete solution as
$$y=(C_1+C_2x)\,e^{-\frac{b}{2a}x}$$
for the case when $b^2-4ac=0$.

If $b^2-4ac<0$, there are two imaginary roots $m_1$ and $m_2$ that are 
complex conjugates, where
$$m_1=\frac{-b+i\sqrt{4ac-b^2}}{2a}$$
Write $m_1$ in the form
$$m_1=\alpha+i\beta$$
so that the second root is
$$m_2=\alpha-i\beta$$
The general solution is
$$y=C_1e^{(\alpha+i\beta)x}+C_2e^{(\alpha-i\beta)x}
=e^{\alpha x}\left(C_1e^{i\beta x}+C_2e^{-i\beta x}\right)$$
Now $e^{i\theta}=\cos\theta+i\sin\theta$ so that
$$y(x)=e^{\alpha x}\left((C_1+C_2)\cos\beta x+i(C_1-C_2)\sin\beta x\right)$$
Write $A=C_1+C_2$ and $B=i(C_1-C_2)$, giving an alternative form of the
solution
$$y(x)=e^{\alpha x}\left(A\cos\beta x+B\sin\beta x\right)$$

%============================================================================
\begin{example}
To solve
$$y''-y'-6y=0$$
try $e^{mx}$ in the \ODE to get
$$(m^2-m-6)e^{mx}=0$$
Therefore $e^{mx}$ is a solution if
$$m^2-m-6=(m-3)(m+2)=0$$
The two roots are $m_1=3$ and $m_2=-2$ so the general solution is
$$y(x)=C_1e^{3x}+C_2e^{-2x}$$
\end{example}
%============================================================================

%============================================================================
\begin{example}
To solve
$$y''+2y'+y=0$$
try $e^{mx}$ in the \ODE to get
$$(m^2+2m+1)e^{mx}=0$$
Therefore $e^{mx}$ is a solution if
$$m^2+2m+1=0$$
There is only one root, $m=-1$, so the two solutions are
$$y_1=e^{-x}\qquad\mbox{and}\qquad y_2=xe^{-x}$$
giving the general solution
$$y(x)=(C_1+C_2x)e^{-x}$$
\end{example}
%============================================================================

%============================================================================
\begin{example}
To solve
$$y''+a^2y=0$$
try $e^{mx}$ in the \ODE to get
$$(m^2+a^2)e^{mx}=0$$
Therefore $e^{mx}$ is a solution if
$$m=\pm ia$$
The general solution is
$$y(x)=C_1e^{iax}+C_2e^{-iax}$$
or alternatively
$$y(x)=A\sin ax + B\cos ax$$
\end{example}
%============================================================================

%============================================================================
\begin{example}
To solve
$$y''+4y'-y=0$$
try $e^{mx}$ in the \ODE to get
$$(m^2+4m-1)e^{mx}=0$$
Therefore $e^{mx}$ is a solution if
$$m^2+4m-1=0$$
The two roots are
$$m=\frac{-4\pm\sqrt{4^2-4(-1)}}{2}=-2\pm\sqrt{5}$$
so the general solution is
$$y(x)=C_1e^{-(2+\sqrt{5})x}+C_2e^{-(2-\sqrt{5})x}
=e^{-2x}\left(C_1e^{-\sqrt{5}x}+C_2e^{\sqrt{5}x}\right)$$
\end{example}
%============================================================================

%============================================================================
\begin{example}
To solve
$$2y''+2y'+y=0$$
try $e^{mx}$ in the \ODE to get
$$(2m^2+2m+1)e^{mx}=0$$
Therefore $e^{mx}$ is a solution if
$$m=\frac{-2\pm\sqrt{4-8}}{4}=-\frac{1}{2}(1\pm i)$$
The general solution is
$$y(x)=e^{-x/2}\left(A\cos\frac{x}{2} + B\sin\frac{x}{2}\right)$$
\end{example}
%============================================================================

%============================================================================
\begin{exercise}
Exercise 4.3 of Zill, pp. 167--168, has more examples of constant
coefficient \ODEs.
\end{exercise}
%============================================================================

%%%%%%%%%%%%%%%%%%%%%%%%%%%%%%%%%%%%%%%%%%%%%%%%%%%%%%%%%%%%%%%%%%%%%%%%%%%%%
\subsection{Inhomogeneous Constant Coefficient O.D.E.s}

The solution of an \ODE of the form
$$ay''+by'+cy=R(x)$$
is the sum of the solutions of the homogeneous equation and the particular
solution due to $R(x)$.

%============================================================================
\begin{figure}
\caption{This table shows the form of the particular solution $y_p(x)$ for a given
$R(x)$ in the method of undetermined coefficients.}
\label{ode fig:muc}
$$\begin{array}{c|c}
R(x) & y_p(x) \\
\hline
\hbox{constant} & A \\
e^{\lambda x} & Ae^{\lambda x} \\
e^{\lambda x}\left[E\cos\mu x+F\sin\mu x\right] & e^{\lambda x}
\left[A\cos\mu x+B\sin\mu x\right] \\
a_0+a_1x+\cdots+a_nx^n & A_0+A_1x+\cdots+A_nx^n \\
e^{\lambda x}[a_0+a_1x+\cdots+a_nx^n] & e^{\lambda x}[A_0+A_1x+\cdots+A_nx^n] 
\end{array}$$
\end{figure}
%============================================================================

In the \name{method of undetermined coefficients}, the particular solution
$y_p(x)$ is found by assuming a general expression for $y_p(x)$ depending on the 
form of $R(x)$ and solving for any unknowns so that the \ODE is satisfied.
The appropriate forms of $y_p(x)$ for $R(x)$ are given in figure
\ref{ode fig:muc}. 

%============================================================================
\begin{example}
The solution of 
$$y''+a^2y=R(x)$$
depends on the form of $R(x)$.

\begin{itemize}
\item When $R(x)=1$, try $y_p=C$.  Then $y_p''=0$ so $C=1/a^2$.  Thus
$$y_p(x)=\frac{1}{a^2}$$

\item When $R(x)=1+x$, try $y_p=A+Bx$ so that $y_p''=0$.  Therefore
$$a^2(A+Bx)=1+x$$
Equating coefficients of $x^0$ and $x^1$ shows that $a^2A=1$ and $a^2B=1$. 
Thus
$$y_p(x)=\frac{1+x}{a^2}$$

\item When $R(x)=1+x^2$, try $y_p=A+Bx+Cx^2$.  Then $y_p''=2C$.
Therefore
$$2C+a^2(A+Bx+Cx^2)=1+x^2$$
Equating coefficients of $x^0$, $x^1$ and $x^2$ shows that
$$2C+a^2A=1\qquad a^2B=0\qquad a^2C=1$$
Therefore 
$$A=\frac{1}{a^2}-\frac{2}{a^4}\qquad B=0\qquad C=\frac{1}{a^2}$$
so the particular solution is
$$y_p(x)=\frac{1}{a^2}-\frac{2}{a^4}+\frac{x^2}{a^2}$$

\item When $R(x)=e^{\lambda x}$, try $y_p=Ae^{\lambda x}$.  Then
$y_p''=A\lambda^2e^{\lambda x}$.  Therefore
$$(\lambda^2+a^2)Ae^{\lambda x}=e^{\lambda x}$$
so that
$$A=\frac{1}{\lambda^2+a^2}$$
giving the particular solution
$$y_p=\frac{e^{\lambda x}}{\lambda^2+a^2}$$

\item When $R(x)=\cos\lambda x$, try $y_p=A\cos\lambda x+B\sin\lambda x$.
Therefore 
$$y_p''=-\lambda^2(A\cos\lambda x+B\sin\lambda x)=-\lambda^2 y_p$$
so that
$$(a^2-\lambda^2)(A\cos\lambda x+B\sin\lambda x)=\cos\lambda x$$
Equating coefficients of $\cos\lambda x$ and $\sin\lambda x$ shows that
$$(a^2-\lambda^2)A=1\qquad (a^2-\lambda^2)B=0$$
so that
$$A=\frac{1}{a^2-\lambda^2}\qquad B=0$$
and the particular solution is
$$y_p=\frac{\cos\lambda x}{a^2-\lambda^2}$$
(What happens here as $a\to\lambda$?)
\end{itemize}
\end{example}
%============================================================================

Note that this method fails whenever
$$R(x)=e^{m_1x}\mbox{\ or\ } e^{m_2x}$$
where $m_1$ and $m_2$ are the roots of 
$$am^2+bm+c=0$$
There is a general method of obtaining $y_p(x)$ when $y_1(x)$ and
$y_2(x)$ are known.  Section 4.7 of Zill (pp. 194--203) describes this
method, called \name{variation of parameters}.

%============================================================================
\begin{example}
When solving 
$$y''+a^2y=\cos ax$$
note that $R(x)=\cos ax$ is a solution of the homogeneous equation, whose
general solution is
$$y(x)=A\cos ax +B\sin ax$$
In cases like this, the method of undetermined coefficients has to be
modified to try a particular solution of the form
$$y_p(x)=u(x)\cos ax$$
Then
$$y_p'(x)=u'(x)\cos ax-au(x)\sin ax$$
and
$$y_p''(x)=u''(x)\cos ax-2au'(x)\sin ax-a^2u(x)\cos ax$$
so that
$$\cos ax=y_p''+a^2y_p=u''(x)\cos ax-2au'(x)\sin ax$$
which gives the differential equation for $u(x)$
$$u''-2au'\tan ax=1$$
which can be solved to determine $u$, hence determine $y_p$.
\end{example}
%============================================================================

%============================================================================
\begin{exercise}
Exercise 4.4 of Zill, pp. 179--181, has more examples of the method of
undetermined coefficients.
\end{exercise}
%============================================================================

%%%%%%%%%%%%%%%%%%%%%%%%%%%%%%%%%%%%%%%%%%%%%%%%%%%%%%%%%%%%%%%%%%%%%%%%%%%%%
\subsection{Superposition of Inhomogeneous Equations}

If $y_{p1}(x)$ is a particular solution of
$$y''+Py'+Qy=R_1$$ 
and $y_{p2}(x)$ is a particular solution of
$$y''+Py'+Qy=R_2$$
then $C_1y_{p1}(x)+C_2y_{p2}(x)$ is a particular solution of
$$y''+Py'+Qy=C_1R_1+C_2R_2$$


%============================================================================
\begin{example}
The equation
$$y_{p1}(x)=\frac{e^{\lambda x}}{\lambda^2+a^2}$$
is a particular solution of
$$y''+a^2y=e^{\lambda x}$$
and
$$y_{p2}(x)=\frac{1}{a^2}-\frac{2}{a^4}+\frac{x^2}{a^2}$$
is a particular solution of
$$y''+a^2y=1+x^2$$
Thus
$$y_p(x)=2\left(\frac{1}{a^2}-\frac{2}{a^4}+\frac{x^2}{a^2}\right)+
\frac{3e^{\lambda x}}{\lambda^2+a^2}$$
is a particular solution of
$$y''+a^2y=2(1+x^2)+3e^{\lambda x}$$
\end{example}
%============================================================================

%%%%%%%%%%%%%%%%%%%%%%%%%%%%%%%%%%%%%%%%%%%%%%%%%%%%%%%%%%%%%%%%%%%%%%%%%%%%%
\section{Free and Forced Vibrations}

Let $x(t)$ describe the response of an oscillating system over time, and write
$$\dot{x}=\der{x}{t}\qquad\ddot{x}=\der{\dot{x}}{t}$$
By Newton's second law, the system's acceleration $\ddot{x}$ multiplied by 
its mass $m$ is equal to the sum of the forces acting on it.   The three 
forces acting on the system are a restoring force of magnitude $-kx$,
resistance or drag of $-\lambda\dot{x}$ and an externally applied force 
$F(t)$.  This gives the second order differential equation
$$m\ddot{x}=-kx-\lambda\dot{x}+F(t)$$

This can be rewritten in the standard form for a second order differential 
equation as
$$\ddot{x}+2p\dot{x}+\wo^2x=f(t)$$
where $p=\lambda/2m$ is the friction term ($p\geq 0$), $\wo=\sqrt{k/m}$ is
the natural frequency ($\wo>0$) and $f(t)=F(t)/m$ is the forcing function
which drives the system.

This is a second order linear \ODE with constant coefficients.
When $f(t)=0$, this equation describes \name{free vibration}.  
When $f(t)\neq 0$, there is an external force driving the system, so the 
equation describes \name{forced vibration}.

%%%%%%%%%%%%%%%%%%%%%%%%%%%%%%%%%%%%%%%%%%%%%%%%%%%%%%%%%%%%%%%%%%%%%%%%%%%%%
\subsection{Free Vibration}

Since $f(t)=0$ for free vibration, the general equation of motion becomes
$$\ddot{x}+2p\dot{x}+\wo^2x=0$$
The nature of the solution depends on the value of the friction constant
$p$.

%%%%%%%%%%%%%%%%%%%%%%%%%%%%%%%%%%%%%%%%%%%%%%%%%%%%%%%%%%%%%%%%%%%%%%%%%%%%%
% Undamped
\subsubsection{Case 1: $p=0$}

The general solution of the differential equation
$$\ddot{x}+\wo^2x=0$$
is
$$x(t)=A\cos\wo t+B\sin\wo t$$
where $A$ and $B$ are arbitrary constants.  This can be written
$$x(t)=\sqrt{A^2+B^2}\left(\frac{A}{\sqrt{A^2+B^2}}\cos\wo t
+\frac{B}{\sqrt{A^2+B^2}}\sin\wo t\right)$$
Since the magnitudes of $A/\sqrt{A^2+B^2}$ and $B/\sqrt{A^2+B^2}$ are
less than or equal to one, and the sum of their squares is one, write
$$\frac{A}{\sqrt{A^2+B^2}}=\cos\phi\qquad\mbox{and}\qquad
\frac{B}{\sqrt{A^2+B^2}}=\sin\phi$$
so the angle $\phi$ is given by $\tan\phi=B/A$.  Then 
\begin{eqnarray*}
\frac{A}{\sqrt{A^2+B^2}}\cos\wo t
+\frac{B}{\sqrt{A^2+B^2}}\sin\wo t
&=&\cos\phi\cos\wo t + \sin\phi\sin\wo t \\&=& \cos(\wo t-\phi)
\end{eqnarray*}
This gives the simpler expression for $x(t)$
$$x(t)=\sqrt{A^2+B^2}\cos\left(\wo t-\phi\right)$$
which is a sinusoid with amplitude $\sqrt{A^2+B^2}$, period $T=2\pi/\wo$ 
and a phase shift of $\phi$.

The system's behaviour is \name{undamped} because there is no friction
to reduce $x(t)$'s amplitude over time.  This is shown in 
figure~\ref{ode fig:ffv.undamped}.


%----------------------------------------------------------------------------
\begin{figure}[t]
\caption{Typical time response $x(t)$ for free vibrations in an undamped
system ($p=0$).}\label{ode fig:ffv.undamped}

\begin{center}

\mbox{}\par

\setlength{\unitlength}{1.7cm}
\begin{pspicture}(0,-1.5)(4.25,1.7)
\psset{xunit=1.7cm,yunit=1.7cm}
% First the curve, so the axes show through
\psplot[linecolor=gray,linewidth=1.5pt,plotstyle=curve]%
{0}{4}{x 0.2 add 180 mul sin 0.8 mul}
% Now the axes and their labels
\psline{<->}(0,-1)(0,1)
\psline{->}(0,0)(4.25,0)
\rput[l](4.3,0){$t$}
\rput[b](0,1.05){$x(t)$}
% Useful points on the curve
% Amplitude
\rput[r](-0.1,0.8){$\sqrt{A^2+B^2}$}
\psline[linewidth=1.5pt](0,0.8)(0.05,0.8)
\rput[r](-0.1,-0.8){$-\sqrt{A^2+B^2}$}
\psline[linewidth=1.5pt](0,-0.8)(0.05,-0.8)
% Phase
\rput[t](0.3,-0.05){$\frac{\phi}{\wo}$}
\psline[linewidth=1.5pt](0.3,0)(0.3,0.05)
% Period
\pcline[offset=0pt]{|-|}(1.3,-0.6)(3.3,-0.6)
\mput*{$T=\frac{2\pi}{\wo}$}
\end{pspicture}
\end{center}
\end{figure}
%----------------------------------------------------------------------------

%%%%%%%%%%%%%%%%%%%%%%%%%%%%%%%%%%%%%%%%%%%%%%%%%%%%%%%%%%%%%%%%%%%%%%%%%%%%%
% Underdamped
\subsubsection{Case 2: $p$ is small and positive}

When $p$ is small and positive, the system is damped because friction reduces
the vibration's amplitude over time.  The differential equation
$$\ddot{x}+2p\dot{x}+\wo^2x=0$$
is solved by assuming $x(t)$ has the form $x(t)=e^{\alpha t}$.  This gives
the characteristic equation
$$\alpha^2+2p\alpha+\wo^2=0$$
whose roots are
$$\alpha=-p\pm\sqrt{p^2-\wo^2}=-p\pm i\sqrt{\wo^2-p^2}$$
Since we are considering the case when $p$ is small and positive, the square
root $\sqrt{p^2-\wo^2}$ is imaginary for $0<p<\wo$.  Writing
$$q=\wo\sqrt{1-\left(\frac{p}{\wo}\right)^2}$$
gives the general solution
\begin{eqnarray*}
x(t)&=&e^{-pt}\left(A\cos qt + B \sin qt\right) \\
&=&Ke^{-pt}\cos\left(qt-\phi\right)
\end{eqnarray*}
where the constants $\phi$ and $K$ are related to the arbitrary constants of
integration by
\begin{eqnarray*}
\tan\phi&=&\frac{B}{A}\\
K&=&\sqrt{A^2+B^2}
\end{eqnarray*}
A typical time response $x(t)$ for underdamped free vibration is shown in 
figure~\ref{ode fig:ffv.underdamped}.

%----------------------------------------------------------------------------
\begin{figure}[t]
\caption{Typical time response $x(t)$ for free vibrations in an underdamped
system ($0<p<\wo$).}\label{ode fig:ffv.underdamped}

\begin{center}

\mbox{}\par

\setlength{\unitlength}{1.7cm}
\begin{pspicture}(0,-1.5)(4.25,1.7)
\psset{xunit=1.7cm,yunit=1.7cm}
% First the curve, so the axes show through
% Do the curve's bounds
\psplot[linestyle=dashed]{0}{4}{2.718 x -0.5 mul exp }
\psplot[linestyle=dashed]{0}{4}{2.718 x -0.5 mul exp neg}
% And now the curve
\psplot[linecolor=gray,linewidth=1.5pt,plotstyle=curve]%
{0}{4}{x 0.2 add 180 mul sin 2.718 x -0.5 mul exp mul}
% Now the axes and their labels
\psline{<->}(0,-1)(0,1)
\psline{->}(0,0)(4.25,0)
\rput[l](4.3,0){$t$}
\rput[b](0,1.05){$x(t)$}
% Useful points on the curve
% Asymptote
\rput[bl](1.5,0.5){$Ke^{-pt}$}
% Period
\pcline[offset=0pt]{|-|}(0.8,-0.8)(2.8,-0.8)
\mput*{$T=\frac{2\pi}{q}$}
\end{pspicture}
\end{center}
\end{figure}
%----------------------------------------------------------------------------

%%%%%%%%%%%%%%%%%%%%%%%%%%%%%%%%%%%%%%%%%%%%%%%%%%%%%%%%%%%%%%%%%%%%%%%%%%%%%
% Critically damped
\subsubsection{Case 3: $p=\wo$}

When $p=\wo$, the characteristic equation has only one solution, and the
system is said to be \name{critically damped}.

This gives $x_1(t)=e^{-pt}$ as one of the solutions of the homogeneous
differential equation.  The second solution has the form $x_2(t)=tx_1(t)$ so
the general solution for critically damped free vibration is
$$x(t)=\left(A+Bt\right)e^{-pt}$$
A typical $x(t)$ for critically damped free vibration is shown in 
figure~\ref{ode fig:ffv.critically damped}.

%----------------------------------------------------------------------------
\begin{figure}[t]
\caption{Typical time response $x(t)$ for free vibrations in a critically 
damped system ($p=\wo$).}\label{ode fig:ffv.critically damped}

\begin{center}

\mbox{}\par

\setlength{\unitlength}{1.7cm}
\begin{pspicture}(0,-1.5)(4.25,1.7)
\psset{xunit=1.7cm,yunit=1.7cm}
% First the curve, so the axes show through
\psplot[linecolor=gray,linewidth=1.5pt,plotstyle=curve]%
{0}{4}{x 0.2 add 2.718 x -1.5 mul exp mul 2 mul}
% Now the axes and their labels
\psline{<->}(0,-1)(0,1)
\psline{->}(0,0)(4.25,0)
\rput[l](4.3,0){$t$}
\rput[b](0,1.05){$x(t)$}
% Useful points on the curve
% Peak
\rput[t](0.466,-0.05){$\frac{1}{p}-\frac{A}{B}$}
\psline[linewidth=1.5pt](0.466,0)(0.466,0.05)
\end{pspicture}
\end{center}
\end{figure}
%----------------------------------------------------------------------------

%%%%%%%%%%%%%%%%%%%%%%%%%%%%%%%%%%%%%%%%%%%%%%%%%%%%%%%%%%%%%%%%%%%%%%%%%%%%%
% Overdamped 
\subsubsection{Case 4: $p>\wo$}

When $p>\wo$, the system is said to be \name{overdamped}.  The 
characteristic equation has two solutions
\begin{eqnarray*}
\alpha_1&=&-p+\sqrt{p^2-\wo^2}\\
\alpha_2&=&-p-\sqrt{p^2-\wo^2}
\end{eqnarray*}
These are both real and negative, giving solutions which decay with time
without oscillating
$$x(t)=Ae^{\alpha_1t}+Be^{\alpha_2t}$$
Since $\alpha_2$ is more negative than $\alpha_1$, the $e^{\alpha_2t}$ term
decays to zero more quickly than the $e^{\alpha_1t}$ term.  So for large
$t$, $x(t)$ goes asymptotically like $Ae^{\alpha_1t}$.
A typical $x(t)$ for overdamped free vibration is shown in 
figure~\ref{ode fig:ffv.overdamped}.

%----------------------------------------------------------------------------
\begin{figure}[t]
\caption{Typical time response $x(t)$ for free vibrations in an overdamped
system ($p>\wo$).}\label{ode fig:ffv.overdamped}

\begin{center}

\mbox{}\par

\setlength{\unitlength}{1.7cm}
\begin{pspicture}(0,-1.5)(4.25,1.7)
\psset{xunit=1.7cm,yunit=1.7cm}
% First the curve, so the axes show through
\psplot[linecolor=gray,linewidth=1.5pt,plotstyle=curve]%
{0}{4}{2.718 x -2 mul exp 2 mul 2.718 x -1 mul exp sub}
% Now the axes and their labels
\psline{<->}(0,-1)(0,1)
\psline{->}(0,0)(4.25,0)
\rput[l](4.3,0){$t$}
\rput[b](0,1.05){$x(t)$}
\end{pspicture}
\end{center}
\end{figure}
%----------------------------------------------------------------------------

Note that for all three cases when damping is present (when $p>0$),
$x(t)\rightarrow 0$ as $t\rightarrow\infty$.  That is, the solutions are
\name{transient}, decaying to zero over time.

%%%%%%%%%%%%%%%%%%%%%%%%%%%%%%%%%%%%%%%%%%%%%%%%%%%%%%%%%%%%%%%%%%%%%%%%%%%%%
\subsection{Forced Vibration}

Since $f(t)\neq 0$ for forced vibration, solutions of the differential
equation
$$\ddot{x}+2p\dot{x}+\wo^2x=f(t)$$
have the form
$$x(t)=x_h(t)+x_p(t)$$
The solution to the homogeneous equation, $x_h(t)$, is given by one of the
four cases of free vibration.  For $p>0$, the homogeneous solution is
transient and tends to zero as $t\rightarrow\infty$.  The particular
solution $x_p(t)$ depends on the forcing function $f(t)$, and this is the 
dominant contribution to $x(t)$ as $t\rightarrow\infty$.

%%%%%%%%%%%%%%%%%%%%%%%%%%%%%%%%%%%%%%%%%%%%%%%%%%%%%%%%%%%%%%%%%%%%%%%%%%%%%
\subsubsection{Resonance}

When the forcing function is sinusoidal, for example $f(t)=\sin\omega t$,
the differential equation is
$$\ddot{x}+2p\dot{x}+\wo^2x=\sin\omega t$$
When $p>0$, the homogeneous solution is transient and tends to zero for
large $t$.  The particular solution $x_p(t)$ is non-transient, and has the
general form
$$x_p(t)=A\cos\omega t+B\sin\omega t$$
so that the first and second derivatives of $x_p(t)$ are
\begin{eqnarray*}
\dot{x}_p(t)&=&\omega\left(-A\sin\omega t+B\cos\omega t\right)\\
\ddot{x}_p(t)&=&-\omega^2\left(A\cos\omega t+B\sin\omega t\right)
\end{eqnarray*}
Substituting these into the differential equation shows that
$$\left(\wo^2-\omega^2\right)\left(A\cos\omega t+B\sin\omega t\right)
+2p\omega\left(-A\sin\omega t+B\cos\omega t\right)=\sin\omega t$$
Equating coefficients of $\sin\omega t$ and $\cos\omega t$ gives two
simultaneous equations
\begin{eqnarray*}
\left(\wo^2-\omega^2\right)A+2p\omega B&=&0 \\
\left(\wo^2-\omega^2\right)B-2p\omega A&=&1
\end{eqnarray*}
whose solutions are 
\begin{eqnarray*}
A&=&\frac{-2p\omega}{\left(\wo^2-\omega^2\right)^2+4p^2\omega^2} \\
B&=&\frac{\wo^2-\omega^2}{\left(\wo^2-\omega^2\right)^2+4p^2\omega^2}
\end{eqnarray*}
The $\cos\omega t$ and $\sin\omega t$ parts of $x_p(t)$ can be combined into
a single sinusoid using the same trick as before.
Written this way, the particular solution is
\begin{eqnarray*}
x_p(t)&=&\sqrt{A^2+B^2}\left(
\frac{A}{\sqrt{A^2+B^2}}\cos\omega t +
\frac{B}{\sqrt{A^2+B^2}}\sin\omega t\right)\\
&=&\alpha\left(-\sin\theta\cos\omega t +\cos\theta\sin\omega t\right)\\
&=&\alpha\sin\left(\omega t-\theta\right)
\end{eqnarray*}
The amplitude $\alpha$ is called the \name{amplification factor}
\begin{eqnarray*}
\alpha&=&\sqrt{A^2+B^2}\\
&=&\frac{1}{\sqrt{\left(\wo^2-\omega^2\right)^2+4p^2\omega^2}}
\end{eqnarray*}
and the phase shift $\theta$ is called the \name{phase-lag}
$$\tan\theta=-\frac{A}{B}=\frac{2p\omega}{\wo^2-\omega^2}$$
Note that this definition of $\theta$ is slightly different from the
corresponding definition of $\phi$ in the equation for
undamped free vibration.

%%%%%%%%%%%%%%%%%%%%%%%%%%%%%%%%%%%%%%%%%%%%%%%%%%%%%%%%%%%%%%%%%%%%%%%%%%%%%
\subsubsection{Amplification Factor}

The amplitude of the system's response to a sinusoidal forcing function
depends on the relationship between the natural frequency of the undamped
free vibration, $\wo$, the frequency of the forcing function, $\omega$, and
the damping coefficient, $p$.

For constant $\wo$ and $p$, the amplitude factor reaches its maximum when
$d\alpha/d\omega=0$.  Since
$$\der{\alpha}{\omega}=-\frac{-2\omega\left(\wo^2-\omega^2\right)
+4p^2\omega}{\left(\left(\wo^2-\omega^2\right)^2+4p^2\omega^2\right)^{3/2}}$$
$\alpha$ is at its maximum when the numerator of $d\alpha/d\omega$ is zero.
This occurs when 
$$\omega=0\qquad\mbox{and}\qquad \omega^2-\wo^2+2p^2=0$$
If $p<\wo/\sqrt{2}$, the local maxima of $\alpha$ occur when $\omega=0$
and when 
$$\omega=\wo\sqrt{1-2\left(\frac{p}{\wo}\right)^2}$$
If $p>\wo/\sqrt{2}$, the only peak in the resonance curve is at $\omega=0$.

The amplification factor is plotted as a function of $\omega$ for a number
of different values of $p$ in figure~\ref{ode fig:ffv.resonance curves}.
The $\omega$ axis has been scaled by $1/\wo$ and the $\alpha$ axis has been
scaled by $\wo^2$ to make the curves dimensionless.  Note that as $p$
increases from zero to $\wo/\sqrt{2}$, the peak of the resonance curve 
moves from $\omega/\wo=1$ back towards the origin along the dashed line.  
Also note that the peak of the resonance curve tends to $\infty$ as 
$p\rightarrow 0$.

%----------------------------------------------------------------------------
\begin{figure}[t]
\caption{Resonance curves showing the amplitude factor $\alpha$ as a
function of frequency $\omega$. The maxima of the resonance curves occur
when $\omega/\wo=\protect\sqrt{1-2\left(p/\wo\right)^2}$ for 
$p/\wo<1/\protect\sqrt{2}$.}\label{ode fig:ffv.resonance curves}

\begin{center}

\setlength{\unitlength}{2cm}
\begin{pspicture}(-0.3,-0.3)(6,6.6)
\psset{xunit=3cm,yunit=1.5cm}
% The asymptote for maxima at y=1/sqrt{1-x^4}
%\psline[linestyle=dashed,dash=3pt 3pt]{-}(1,0)(1,4)
\psplot[linestyle=dashed,dash=3pt 3pt,plotstyle=curve]{0}{0.984}%
{1 x dup mul dup mul sub sqrt 1 exch div}
% First the curve, so the axes show through
% p/wo=2
\psplot[linecolor=gray,linewidth=1pt,plotstyle=curve]{0}{2}%
{1 x x mul sub dup mul 4 2 dup mul mul x dup mul mul add sqrt 1 exch div}
% p/wo=1/sqrt{2}
\psplot[linecolor=gray,linewidth=1.5pt,plotstyle=curve]{0}{2}%
{1 x x mul sub dup mul 4 0.707 dup mul mul x dup mul mul add sqrt 1 exch div}
% p/wo=0.5
\psplot[linecolor=gray,linewidth=1pt,plotstyle=curve]{0}{2}%
{1 x x mul sub dup mul 4 0.5 dup mul mul x dup mul mul add sqrt 1 exch div}
% p/wo=0.25
\psplot[linecolor=gray,linewidth=1pt,plotstyle=curve]{0}{2}%
{1 x x mul sub dup mul 4 0.25 dup mul mul x dup mul mul add sqrt 1 exch div}
% p/wo=0.15
\psplot[linecolor=gray,linewidth=1pt,plotstyle=curve]{0}{2}%
{1 x x mul sub dup mul 4 0.15 dup mul mul x dup mul mul add sqrt 1 exch div}
% p/wo=0
\psplot[linecolor=gray,linewidth=1.5pt,plotstyle=curve]{0}{0.866}%
{1 x x mul sub dup mul 4 0 dup mul mul x dup mul mul add sqrt 1 exch div}
\psplot[linecolor=gray,linewidth=1.5pt,plotstyle=curve]{1.118}{2}%
{1 x x mul sub dup mul 4 0 dup mul mul x dup mul mul add sqrt 1 exch div}
% Now the axes and their labels
\psaxes[tickstyle=top,linewidth=1pt]{->}(2,4)
\rput[l](2.05,0){$\displaystyle\frac{\omega}{\wo}$}
\rput[b](0,4.1){$\alpha\wo^2$}
% Some labels showing p/wo
\rput(0.9,1.2){$0<\frac{p}{\wo}<\frac{1}{\sqrt{2}}$}
\rput(0.6,0.4){$\frac{p}{\wo}>\frac{1}{\sqrt{2}}$}
%
\rput[l](1.8,1.5){\rnode{A}{$\frac{p}{\wo}=\frac{1}{\sqrt{2}}$}}
\rput(1.5,0.406){\rnode{B}{}}\ncline{->}{A}{B}
%
\rput[l](1.8,3){\rnode{A}{$\frac{p}{\wo}=0$}}
\rput(1.155,3){\rnode{B}{}}\ncline{->}{A}{B}
\end{pspicture}
\end{center}
\end{figure}
%----------------------------------------------------------------------------

%----------------------------------------------------------------------------
\begin{figure}[t]
\caption{LRC series resonant circuit.}\label{ode fig:ffv.LRC}

\begin{center}

\setlength{\unitlength}{1cm}
\begin{pspicture}(-1,0)(3.5,3.5)
\psset{xunit=1cm,yunit=1cm}
% Lines around the circuit
\psline{-}(0,2)(0,2.5)(1,2.5)
\psline{-}(2,2.5)(2.5,2.5)(2.5,2)
\psline{-}(2.5,1)(2.5,0.5)(1.667,0.5)
\psline{-}(1.333,0.5)(0,0.5)(0,1)
% The resistor
\pszigzag[linearc=0,coilwidth=0.333,coilarm=0]{-}(1,2.5)(2,2.5)
\rput[b](1.5,2.833){$R$}
% The inductor
%\pscoil[coilwidth=0.333,coilheight=0.75,coilarm=0]{-}(2.5,1)(2.5,2)
\psarc{-}(2.5,1.125){0.125}{270}{450}
\psarc{-}(2.5,1.375){0.125}{270}{450}
\psarc{-}(2.5,1.625){0.125}{270}{450}
\psarc{-}(2.5,1.875){0.125}{270}{450}
\rput[l](2.85,1.5){$L$}
% The capacitor
\psline{-}(1.333,0.167)(1.333,0.833)
\psline{-}(1.667,0.167)(1.667,0.833)
\rput[t](1.5,0.067){$C$}
\rput[tl](1.767,0.4){$\scriptscriptstyle +Q$}
\rput[tr](1.233,0.4){$\scriptscriptstyle -Q$}
% The voltage source
\pscircle(0,1.5){0.5}
\rput(0,1.5){$E(t)$}
\rput[br](-0.1,2.1){$\scriptscriptstyle +$}
\rput[tr](-0.1,0.9){$\scriptscriptstyle -$}
% Finally the arrow with I(t)
\psline{->}(0.2,2.7)(0.8,2.7)
\rput[b](0.5,2.9){$I(t)$}
\end{pspicture}
\end{center}
\end{figure}
%----------------------------------------------------------------------------

%============================================================================
\begin{example}
A series LRC circuit consists of a resistor $R$, inductor $L$ and capacitor $C$ in
series across a time-varying voltage source $E(t)$.  This is illustrated in
figure~\ref{ode fig:ffv.LRC}.

Summing voltage drops around the loop gives the equation
$$-E(t)+I(t)R+L\der{I(t)}{t}+\frac{Q(t)}{C}=0$$
Since $I(t)=\dbd{t}Q(t)$, differentiating with respect to $t$ gives
$$L\dder{I}{t}+R\der{I}{t}+\frac{1}{C}I=\der{E}{t}$$
Dividing by $L$ gives the standard second order differential equation
for forced vibration 
$$\dder{I}{t}+\frac{R}{L}\der{I}{t}+\frac{1}{LC}I=\frac{1}{L}\der{E}{t}$$
which is the force vibration differential equation with $p=R/2L$, $\wo=1/\sqrt{LC}$ 
and $f(t)=\frac{1}{L}\der{E}{t}$.

If a voltage of frequency $\omega$
$$E(t)=V\cos\omega t$$
is applied, the current $I(t)$ 
reaches its maximum when the LRC circuit is tuned to $\omega$, provided that 
$p/\wo<1/\sqrt{2}$.  If so, $\omega$ is given by
$$\omega=\wo\sqrt{1-2\left(\frac{p}{\wo}\right)^2}=
\frac{1}{\sqrt{LC}}\sqrt{1-\frac{R^2C}{2L}}$$
\end{example}
%============================================================================

%============================================================================
\begin{exercise}
Exercises 5.1 to 5.4 and the review exercises of Zill, pp. 219--251, have
more examples of free and forced vibration.
\end{exercise}
%============================================================================
	
%%%%%%%%%%%%%%%%%%%%%%%%%%%%%%%%%%%%%%%%%%%%%%%%%%%%%%%%%%%%%%%%%%%%%%%%%%%
%
%			Mathematics 132 Course Notes
%
%			 Department of Mathematics,
%   			  University of Melbourne
%
%		Stephen Simmons			Lee White
%
% 8 Feb-96 SS: Updated with corrections from semester 2, 1995
%
%%%%%%%%%%%%%%%%%%% Copyright (C) 1995-96 Stephen Simmons %%%%%%%%%%%%%%%%%

\chapter{Vector Functions}
\label{vf chp}

A vector function $\vect{F}(t)$
$$\vect{F}(t)=F_x(t)\ivect+F_y(t)\jvect+F_z(t)\kvect$$
where $F_x(t)$, $F_y(t)$ and $F_z(t)$ are scalar functions of $t$, describes 
a trajectory in space as the parameter $t$ is varied.

Since the three unit vectors $\ivect$, $\jvect$ and $\kvect$ are constant
vectors, the first and second derivatives of $\vect{F}(t)$ are
\begin{eqnarray*}
\der{\vect{F}}{t}=\dot{\vect{F}}
&=&\lim_{\delta t\to 0}\frac{\vect{F}(t+\delta t)-\vect{F}(t)}{\delta t} \\
&=&\dot{F_x}\ivect+\dot{F_y}\jvect+\dot{F_z}\kvect
\end{eqnarray*}
and
\begin{eqnarray*}
\dder{\vect{F}}{t}=\ddot{\vect{F}}
&=&\lim_{\delta t\to 0}\frac{\dot{\vect{F}}(t+\delta t)
-\dot{\vect{F}}(t)}{\delta t} \\
&=&\ddot{F_x}\ivect+\ddot{F_y}\jvect+\ddot{F_z}\kvect
\end{eqnarray*}

In the limit as $\delta t\to 0$, $\vect{F}(t+\delta t)-\vect{F}(t)$ becomes
tangential to the trajectory curve so that $\dot{\vect{F}}(t)$ is the tangent 
at $t$.

If $t$ is time and $\vect{r}(t)$ is the position of a particle at time $t$
$$\vect{r}(t)=x(t)\ivect+y(t)\jvect+z(t)\kvect$$
where $x(t)$, $y(t)$ and $z(t)$ are the particle's Cartesian coordinates, 
the velocity of the particle is
$$\dot{\vect{r}}(t)=\dot{x}(t)\ivect+\dot{y}(t)\jvect+\dot{z}(t)\kvect$$
and the acceleration is 
$$\ddot{\vect{r}}(t)=\ddot{x}(t)\ivect+\ddot{y}(t)\jvect+\ddot{z}(t)\kvect$$


%%%%%%%%%%%%%%%%%%%%%%%%%%%%%%%%%%%%%%%%%%%%%%%%%%%%%%%%%%%%%%%%%%%%%%%%%%%%%
\section{Differentiation of Sums and Products}

Let $\vect{F}(t)$ and $\vect{G}(t)$ be vector functions and $f(t)$ be a
scalar function.  Then
$$\dbd{t}\left(\vect{F}+\vect{G}\right)=\der{\vect{F}}{t}+\der{\vect{G}}{t}$$
$$\dbd{t}\left(\vect{F}\cdot\vect{G}\right)=\der{\vect{F}}{t}\cdot\vect{G}
+\vect{F}\cdot\der{\vect{G}}{t}$$
$$\dbd{t}\left(\vect{F}\times\vect{G}\right)=\der{\vect{F}}{t}\times\vect{G}
+\vect{F}\times\der{\vect{G}}{t}$$
$$\dbd{t}\left(f\vect{F}\right)=\der{f}{t}\vect{F}+f\der{\vect{F}}{t}$$

%============================================================================
\begin{example}
A particle moving with \name{constant rectilinear motion} has Cartesian 
coordinates
\begin{eqnarray*}
x(t)&=&at+b\\
y(t)&=&ct+d\\
z(t)&=&et+f
\end{eqnarray*}
so the position of the particle can be written
$$\vect{r}(t)=(at+b)\ivect+(ct+d)\jvect+(et+f)\kvect=\vect{r}_0+\vect{v}t$$
where
\begin{eqnarray*}
\vect{r}_0&=&b\ivect+d\jvect+f\kvect \\
\vect{v}&=&a\ivect+c\jvect+e\kvect 
\end{eqnarray*}
are constant vectors.

Then the velocity is
$$\dot{\vect{r}}=\dot{x}\ivect+\dot{y}\jvect+\dot{z}\kvect
=a\ivect+c\jvect+e\kvect=\vect{v}$$
and the acceleration is
$$\ddot{\vect{r}}=\ddot{x}\ivect+\ddot{y}\jvect+\ddot{z}\kvect
=\vect{0}$$
\end{example}
%============================================================================

%============================================================================
\begin{example}
A particle moving with \name{uniform circular motion} has Cartesian 
coordinates
\begin{eqnarray*}
x(t)&=&x_0+a\cos\omega t\\
y(t)&=&y_0+a\sin\omega t\\
z(t)&=&0
\end{eqnarray*}
so the position of the particle can be written
$$\vect{r}(t)=\vect{r}_0+a\cos\omega t\ivect+a\sin\omega t\jvect$$
where
$$\vect{r}_0=x_0\ivect+y_0\jvect$$
Note that
$$\left|\vect{r}-\vect{r}_0\right|
=\left|a\cos\omega t\ivect+a\sin\omega t\jvect\right|=a$$

The velocity is
$$\dot{\vect{r}}(t)=-a\omega\sin\omega t\ivect+a\omega\cos\omega t\jvect$$
so the speed is constant
$$\left|\dot{\vect{r}}\right|=a\omega$$
and that the velocity is orthogonal to $\vect{r}-\vect{r}_0$
$$\dot{\vect{r}}\cdot\left(\vect{r}-\vect{r}_0\right)=0$$
This orthogonality can be proved by direct substitution, or alternatively,
by using 
$$\left(\vect{r}-\vect{r}_0\right)\cdot\left(\vect{r}-\vect{r}_0\right)
=\left|\vect{r}-\vect{r}_0\right|^2=a^2$$
Then differentiating with respect to time, 
$$0=\dbd{t}a^2=\dbd{t}\left|\vect{r}-\vect{r}_0\right|^2
=2\left(\dbd{t}(\vect{r}-\vect{r}_0)\right)
\cdot\left(\vect{r}-\vect{r}_0\right)$$
Since $\dot{\vect{r}}_0=0$, 
$$\dot{\vect{r}}\cdot\left(\vect{r}-\vect{r}_0\right)=0$$
as required.

The acceleration is
$$\ddot{\vect{r}}(t)=-a\omega^2\cos\omega t\ivect
-a\omega^2\sin\omega t\jvect=-\omega^2 (\vect{r}-\vect{r}_0)$$
Note that the acceleration is inwards along the radius of the circle and
that the magnitude of the acceleration is constant
$$\left|\ddot{\vect{r}}\right|=\omega^2 a$$
Also, the acceleration and velocity are at right angles
$$\dot{\vect{r}}\cdot\ddot{\vect{r}}=0$$
This can be shown by direct substitution or by differentiating 
$$\dot{\vect{r}}\cdot\dot{\vect{r}}=\left|\dot{\vect{r}}\right|^2
=a^2\omega^2$$
with respect to $t$ to give
$$\dot{\vect{r}}\cdot\ddot{\vect{r}}+\ddot{\vect{r}}\cdot\dot{\vect{r}}=0$$
so that
$$\dot{\vect{r}}\cdot\ddot{\vect{r}}=0$$
\end{example}
%============================================================================

%============================================================================
\begin{exercise}
Problems and more examples can be found in chapter 1 of Fowles.
\end{exercise}
%============================================================================

%%%%%%%%%%%%%%%%%%%%%%%%%%%%%%%%%%%%%%%%%%%%%%%%%%%%%%%%%%%%%%%%%%%%%%%%%%%%%
\section{Polar Coordinates}

%----------------------------------------------------------------------------
\begin{figure}\centering
\caption{The unit vectors for polar coordinates $(r,\theta)$ are $\rvect$ in 
the radial direction and $\tvect$ in the transverse direction.}
\label{vf fig:polar}

\psset{xunit=5cm,yunit=5cm}
\begin{pspicture}(-0.25,-0.35)(1.3,1.4)
% First the curve, so the axes show through
\psplot[linecolor=gray,linewidth=2pt,plotstyle=curve]{0.5}{1}%
{1 1 x x mul sub sqrt sub}
% Now the axes and their labels
\psline{->}(-0.25,0)(1.2,0)
\psline{->}(0,-0.3)(0,1.2)
\uput[dl](0,0){$O$}
\uput[r](1.2,0){$x$}
\uput[u](0,1.2){$y$}
\pcline[linecolor=black,linewidth=1pt]{->}(0,0)(0.707,0.293)
\Aput{$r$}

\rput[bl]{22.5}(0.707,0.293){
	\psline{->}(0,0)(0.15,0) \rput[l]{*0}(0.16,0){$\rvect$}
	\psline{->}(0,0)(0,0.15) \rput[b]{*0}(0,0.16){$\tvect$}
}
\uput[dr](0.707,0.293){$P$}
\psarc{->}(0,0){1}{0}{22.5}
\uput[r](0.2,0.05){$\theta$}
\end{pspicture}
\end{figure}
%----------------------------------------------------------------------------

Consider a point $P$ at $(x,y)$ so that $\vect{r}=x\ivect+y\jvect$.
As shown in figure \ref{vf fig:polar}, $\vect{r}$ can be written in terms of
$r$ and $\theta$, where $r$ is the particle's distance from the origin and
$\theta$ is the angle that $\vect{r}$ makes with the positive $x$ axis.
\begin{eqnarray*}
x&=&r\cos\theta\\
y&=&r\sin\theta
\end{eqnarray*}
Then the velocity can also be written in terms of $r$ and $\theta$
\begin{eqnarray*}
\dot{x}&=&\dot{r}\cos\theta-r\dot{\theta}\sin\theta\\
\dot{y}&=&\dot{r}\sin\theta+r\dot{\theta}\cos\theta
\end{eqnarray*}
so that
\begin{eqnarray*}
\dot{\vect{r}}
&=&\dot{x}\ivect+\dot{y}\jvect\\
&=&(\dot{r}\cos\theta-r\dot{\theta}\sin\theta)\ivect+
(\dot{r}\sin\theta+r\dot{\theta}\cos\theta)\jvect \\
&=&\dot{r}(\cos\theta\ivect+\sin\theta\jvect)+r\dot{\theta}(-\sin\theta\ivect
+\cos\theta\jvect)
\end{eqnarray*}
Therefore
$$\dot{\vect{r}}=\dot{r}\rvect+r\dot{\theta}\tvect$$
where $\rvect$ is the \name{radial unit vector} in the direction of
$r$ increasing
$$\rvect=\cos\theta\ivect+\sin\theta\jvect=\frac{\vect{r}}{r}$$
Note that $\left|\hat{\vect{r}}\right|=1$ because $\hat{\vect{r}}$ is a unit
vector.  $\hat{\vect{r}}$ is not a constant vector (like $\ivect$) but
changes direction as $\theta$ changes.

$\tvect$ is the \name{transverse unit vector} in the direction 
of $\theta$ increasing
$$\tvect=-\sin\theta\ivect+\cos\theta\jvect$$
which is orthogonal to $\rvect$ 
$$\tvect\cdot\rvect=0$$
Note that $\left|\hat{\vect{\theta}}\right|=1$ because $\hat{\vect{\theta}}$ 
is a unit vector.  $\hat{\vect{\theta}}$ is not a constant vector but
changes direction as $\theta$ changes.

In the expression for velocity in polar coordinates,
$$\dot{\vect{r}}=\dot{r}\rvect+r\dot{\theta}\tvect$$
$\dot{r}$ is called the \name{radial velocity} and $r\dot{\theta}$ is called
the \name{transverse velocity}.

The derivative of the radial unit vector is
$$\der{\rvect}{t}=\dbd{t}(\cos\theta\ivect+\sin\theta\jvect)
=-\dot{\theta}\sin\theta\ivect+\dot{\theta}\cos\theta\jvect$$
which can be written
$$\dot{\rvect}=\dot{\theta}\tvect$$

The derivative of the transverse unit vector is
$$\der{\tvect}{t}=\dbd{t}(-\sin\theta\ivect+\cos\theta\jvect)
=-\dot{\theta}\cos\theta\ivect-\dot{\theta}\sin\theta\jvect$$
which can be written
$$\dot{\tvect}=-\dot{\theta}\rvect$$

Taking the derivative of the polar form of the particle's velocity gives
\begin{eqnarray*}
\ddot{\vect{r}}
&=&\dbd{t}\left(\dot{r}\rvect+r\dot{\theta}\tvect\right)\\
&=&\ddot{r}\rvect+\dot{r}\dot{\rvect}+\dot{r}\dot{\theta}\tvect
   +r\ddot{\theta}\tvect+r\dot{\theta}\dot{\tvect} \\
&=&\ddot{r}\rvect+\dot{r}\dot{\theta}\tvect
+\dot{r}\dot{\theta}\tvect+r\ddot{\theta}\tvect-r\dot{\theta}^2\rvect
\end{eqnarray*}
Gathering terms shows that acceleration in polar coordinates is
$$\ddot{\vect{r}}=\left(\ddot{r}-r\dot{\theta}^2\right)\rvect
+\left(r\ddot{\theta}+2\dot{r}\dot{\theta}\right)\tvect$$
where $\ddot{r}-r\dot{\theta}^2$ is called the \name{radial acceleration}
and $r\ddot{\theta}+2\dot{r}\dot{\theta}$ is called the \name{transverse
acceleration}.

%============================================================================
\begin{example}
For a particle moving in a circular orbit around the origin, $r=a$ so
$$\dot{\vect{r}}=a\dot{\theta}\tvect$$
so that the velocity is always transverse, and
$$\ddot{\vect{r}}=-a\dot{\theta}^2\rvect+a\ddot{\theta}\tvect$$
where $\dot{\theta}$ is the \name{instantaneous angular velocity} and
$\ddot{\theta}$ is the \name{instantaneous angular acceleration}.

If we write $v=a\dot{\theta}=\left|\dot{\vect{r}}\right|$ as the speed of
the particle, then the radial acceleration is $-a\dot{\theta}^2=-v^2/a$.
\end{example}
%============================================================================


%============================================================================
\begin{example}
For a particle moving with rectilinear motion through the origin, $\theta$
is constant so that $\rvect$ is constant.  Therefore
\begin{eqnarray*}
\vect{r}&=&r\rvect \\
\dot{\vect{r}}&=&\dot{r}\rvect \\
\ddot{\vect{r}}&=&\ddot{r}\rvect 
\end{eqnarray*}
These equations for rectilinear motion in polar coordinates are only simple
if the particle's trajectory includes the origin, otherwise $\theta$ changes
along the trajectory.
\end{example}
%============================================================================

%%%%%%%%%%%%%%%%%%%%%%%%%%%%%%%%%%%%%%%%%%%%%%%%%%%%%%%%%%%%%%%%%%%%%%%%%%%%%
\section{Intrinsic Coordinates}

%%%%%%%%%%%%%%%%%%%%%%%%%%%%%%%%%%%%%%%%%%%%%%%%%%%%%%%%%%%%%%%%%%%%%%%%%%%%%
\subsection{Two-Dimensional Intrinsic Coordinates}
\label{vf sec:2D ic}

%----------------------------------------------------------------------------
\begin{figure}\centering
\caption{The unit vectors for intrinsic coordinates $(s,\psi)$ are $\Tvect$ 
in the tangential direction and $\Nvect$ in the normal direction.  $s$ is
the distance along the curve and $\psi$ the angle the tangent makes with the
$x$ axis.}
\label{vf fig:2D intrinsic}

\psset{xunit=5cm,yunit=5cm}
\begin{pspicture}(-0.25,-0.35)(1.3,1.4)
% First the curve, so the axes show through
\psplot[linecolor=gray,linewidth=2pt,plotstyle=curve]{0.25}{1}%
{1 1 x x mul sub sqrt sub}
\psplot[linecolor=gray,linewidth=2pt,plotstyle=curve,arrows=>->]{0.35}{0.55}%
{1 1 x x mul sub sqrt sub}
\rput[B](0.45,0.135){$s$}
\uput*[r](0.2,0.05){$\theta$}
% Now the axes and their labels
\psline{->}(-0.25,0)(1.2,0)
\psline{->}(0,-0.3)(0,1.2)
\uput[dl](0,0){$O$}
\uput[r](1.2,0){$x$}
\uput[u](0,1.2){$y$}
\pcline[linecolor=black,linewidth=1pt]{->}(0,0)(0.707,0.293)
\Aput{$r$}
\psline[linecolor=black,linewidth=1pt,linestyle=dashed]{-}(0.414,0)(0.707,0.293)
\psarc{->}(0.414,0){1}{0}{45}
\uput[r](0.6,0.05){$\psi$}
\rput[bl]{22.5}(0.707,0.293){
	\psline{->}(0,0)(0.15,0) \rput[l]{*0}(0.16,0){$\rvect$}
	\psline{->}(0,0)(0,0.15) \rput[b]{*0}(0,0.16){$\tvect$}
}
\rput[bl]{45}(0.707,0.293){
	\psline[linecolor=black]{->}(0,0)(0.15,0) 
	\rput[bl]{*0}(0.16,0){$\Tvect$}
	\psline[linecolor=black]{->}(0,0)(0,0.15) 
	\rput[br]{*0}(0,0.16){$\Nvect$}
}
\uput[dr](0.707,0.293){$P$}
\psarc{->}(0,0){1}{0}{22.5}
\end{pspicture}
\end{figure}
%----------------------------------------------------------------------------

The point $P$ at $(x,y)$ in Cartesian coordinates or $(r,\theta)$ is polar
coordinates can also be described by the \name{intrinsic coordinates} 
$(s,\psi)$ shown in figure \ref{vf fig:2D intrinsic}.

$s$ is arc length measured along the trajectory from some specified starting
point and $\psi$ is the angle the tangent to the trajectory makes with the
positive $x$ axis.

The associated unit vectors are $\Tvect$, the unit vector along the tangent
in the direction of increasing $s$
$$\Tvect=\cos\psi\ivect+\sin\psi\jvect$$
and $\Nvect$, the unit vector transverse to $\Tvect$ in the direction of
increasing $\psi$
$$\Nvect=-\sin\psi\ivect+\cos\psi\jvect$$

To find expressions for velocity and acceleration in intrinsic coordinates,
let $\vect{\delta r}$ be the difference in $\vect{r}$ as the particle moves a
distance $\delta s$ along the curve 
$$\vect{\delta r}=\vect{r}(s+\delta s)-\vect{r}(s)$$
As $\delta s\to 0$, $\vect{\delta r}$ becomes tangential to the curve so
that the magnitude of $\vect{\delta r}$ tends to $\delta s$
$$\frac{\left|\vect{\delta r}\right|}{\delta s}\to 1$$
and $\vect{\delta r}$ becomes a tangent to the curve
$$\frac{\vect{\delta r}}{\delta s}\to \Tvect$$
This implies that
$$\der{\vect{r}}{s}=\Tvect$$
Therefore
$$\dbd{t}\left(\vect{r}\left(s(t)\right)\right)=\der{\vect{r}}{s}\der{s}{t}
=\dot{s}\Tvect$$
so the expression for velocity in intrinsic coordinates is
$$\dot{\vect{r}}=\dot{s}\Tvect$$

To find the expression for acceleration, 
$$\ddot{\vect{r}}=\dbd{t}\left(\dot{s}\Tvect\right)
=\ddot{s}\Tvect+\dot{s}\dot{\Tvect}$$
Now $\Tvect=\cos\psi\ivect+\sin\psi\jvect$ so
$$\dot{\Tvect}=\dot{\psi}\left(-\sin\psi\ivect+\cos\psi\jvect\right)
=\dot{\psi}\Nvect$$
Therefore the expression for acceleration in intrinsic coordinates is
$$\ddot{\vect{r}}=\ddot{s}\Tvect+\dot{s}\dot{\psi}\Nvect$$

Note that from the definitions of $\Tvect$ and $\Nvect$,
$$\der{\Tvect}{s}=\der{\psi}{s}\Nvect$$
on the curve $\psi(s)$. $\der{\psi}{s}$ is the \name{curvature} of the curve
at $P(s)$ and
$$\der{\psi}{s}=\frac{1}{\rho(s)}$$
where $\rho(s)$ is the \name{radius of curvature} of the curve at the point
$P(s)$.

%%%%%%%%%%%%%%%%%%%%%%%%%%%%%%%%%%%%%%%%%%%%%%%%%%%%%%%%%%%%%%%%%%%%%%%%%%%%%
\subsection{Three-Dimensional Intrinsic Coordinates}


When the particle follows a trajectory through three-dimensional space, 
$s$ is the arc length measured along the curve from some specified point.
As for two-dimensional intrinsic coordinates,
$$\der{\vect{r}}{s}=\Tvect$$
and
$$\der{\Tvect}{s}=\kappa \Nvect=\frac{1}{\rho}\Nvect$$
where $\Nvect$ is the \name{principal normal} to the curve (transverse to
$\Tvect$ in the plane of the local curve).  Here $\kappa$ is the 
\name{curvature} of the curve at $P(s)$
$$\kappa=\left|\der{\Tvect}{s}\right|$$
and $\rho$ is the \name{radius of curvature} at $P(s)$.

Velocity in three-dimensional intrinsic coordinates is given by
$$\dot{\vect{r}}=\dbd{t}\vect{r}(s(t))=\der{\vect{r}}{s}\dot{s}$$
so that
$$\dot{\vect{r}}=\dot{s}\Tvect$$

Acceleration in three-dimensional intrinsic coordinates is 
$$\ddot{\vect{r}}=\dbd{t}(\dot{s}\Tvect)=\ddot{s}\Tvect+\dot{s}
\der{\Tvect}{t}$$
Now $\dbd{t}\Tvect=\der{\Tvect}{s}\dot{s}$ so that acceleration can be
written
$$\ddot{\vect{r}}=\ddot{s}\Tvect+\frac{\dot{s}^2}{\rho(s)}\Nvect$$





        
%%%%%%%%%%%%%%%%%%%%%%%%%%%%%%%%%%%%%%%%%%%%%%%%%%%%%%%%%%%%%%%%%%%%%%%%%%%
%
%			Mathematics 132 Course Notes
%
%			 Department of Mathematics,
%   			  University of Melbourne
%
%		Stephen Simmons			Lee White
%
% 8 Feb-96 SS: Updated with corrections from semester 2, 1995
%
%%%%%%%%%%%%%%%%%%% Copyright (C) 1995-96 Stephen Simmons %%%%%%%%%%%%%%%%%

%%%%%%%%%%%%%%%%%%%%%%%%%%%%%%%%%%%%%%%%%%%%%%%%%%%%%%%%%%%%%%%%%%%%%%%%%%%%%
\chapter{Single Particle Dynamics}
\label{spd chp}

A \name{particle} is a moving geometric point at which matter is
concentrated.  \name{Newton's Laws of Motion} describe how particles behave:
\begin{description}
\item[First Law] A particle moves at constant velocity $\dot{\vect{r}}$ 
relative to an inertial frame of reference unless acted upon by a force.

\item[Second Law] If a particle is acted upon by a force $\vect{F}$, it will
accelerate relative to an inertial frame of reference such that
$$\vect{F}=m\ddot{\vect{r}}$$
The constant of proportionality $m$ is the \name{inertial mass} of the
particle.

\item[Third Law] When two particles act on one another, the two forces
acting on the particles are equal in magnitude and opposite in direction and
along the line of their centres
$$\vect{F}_{12}=-\vect{F}_{21}$$
\end{description}

Particles are also affected by \name{Newton's postulate of gravitation},
which says that any two particles in the universe attract each other with a
gravitational force of magnitude
$$\frac{Gm_1m_2}{r^2}$$
along the line of their centres, where $r$ is the distance between the
particles, $m_1$ and $m_2$ are their inertial masses and $G$ is the
\name{gravitational constant}
$$G=6.67\times 10^{-11}\rm\,m^3kg^{-1}s^{-2}$$

The gravitational force between a uniform sphere of mass $M$ and a particle
of mass $m$ is
$$\vect{F}_p=-\frac{GmM}{r^2}\rvect$$
This is derived in Fowles, pp. 134--136.

The near gravitational field of the Earth can be found using
$$\vect{F}_p=-\frac{GmM}{r^2}\rvect$$
where $M$ is the mass of the Earth and $r$ is the distance of the particle
from the centre of the Earth.  Write $r=R+h$ where $R$ is the radius of the
Earth and $h$ is the height of the particle above the surface of the Earth. 
Then
$$\vect{F}_p=-\frac{GmM}{(R+h)^2}\rvect$$
If the particle is close to the Earth's surface,
$$\vect{F}_p=-\frac{GM}{R^2}m\rvect + O\left(\frac{h}{R}\right)$$
Since $h\ll R$, the $h/R$ terms can be neglected so that
$$\vect{F}_p=-\frac{GM}{R^2}m\rvect=-gm\rvect$$
where $g$ is gravitational acceleration at the Earth's surface
$$g=\frac{GM}{R^2}=9.81\rm\,ms^{-2}$$
This simplification is valid as long as $h\ll R$ and it assumes that the Earth
is composed of uniform spherical shells.

%%%%%%%%%%%%%%%%%%%%%%%%%%%%%%%%%%%%%%%%%%%%%%%%%%%%%%%%%%%%%%%%%%%%%%%%%%%%%
\section{Rectilinear Motion}

If the particle moves in a straight line, the force must always be parallel
to the direction of its motion.  Suppose that the force is constant
$$m\ddot{\vect{r}}=\vect{F}$$
Choose the $x$ axis to lie in the direction of $\vect{F}$.  Then
$\vect{F}=F\ivect$ and
\begin{eqnarray*}
m\ddot{x}&=&F\\
m\ddot{y}&=&0\\
m\ddot{z}&=&0
\end{eqnarray*}
Integrating gives the particle's velocities 
\begin{eqnarray*}
\dot{x}(t)&=&\frac{F}{m}t+\dot{x}(0)\\
\dot{y}(t)&=&\dot{y}(0)=0\\
\dot{z}(t)&=&\dot{z}(0)=0
\end{eqnarray*}
because there is no motion in the $y$ and $z$ directions.

Integrating again gives the particle's position
\begin{eqnarray*}
x(t)&=&\frac{1}{2}\frac{F}{m}t^2+\dot{x}(0)t+x(0)\\
y(t)&=&y(0)\\
z(t)&=&z(0)
\end{eqnarray*}


Using this, note that
\begin{eqnarray*}
2\frac{F}{m}\left[x(t)-x(0)\right]
&=&\left(\frac{F}{m}t\right)^2+2\frac{F}{m}t\dot{x}(0) \\
&=&\left(\dot{x}(t)-\dot{x}(0)\right)^2
	+2\dot{x}(0)\left(\dot{x}(t)-\dot{x}(0)\right)\\
&=&\dot{x}^2(t)-\dot{x}^2(0)
\end{eqnarray*}
This can be written as
$$F\left[x(t)-x(0)\right]=\frac{1}{2}m\dot{x}^2(t)-\frac{1}{2}m\dot{x}^2(0)$$
which shows that the work done is equal to the change in kinetic energy.

Alternatively, this can be obtained using
$$\ddot{x}=\der{\dot{x}}{t}=\der{\dot{x}}{x}\der{x}{t}
=\dot{x}\der{\dot{x}}{x}=\frac{1}{2}\dbd{x}\left(\dot{x}^2\right)$$
so that
$$\frac{1}{2}\dbd{x}\left(\dot{x}^2\right)=\frac{F}{m}$$
Integrating both sides with respect to $x$ shows that
$$\dot{x}^2(t)-\dot{x}^2(0)=\frac{2F}{m}\left(x(t)-x(0)\right)$$

In the \name{general rectilinear problem}, only one coordinate is important.
If the $z$ axis lies along the line of motion,
$$\vect{r}=z\kvect$$
and
$$\vect{F}=F\kvect$$
The equation of motion becomes a scalar differential equation in $z$
$$m\ddot{z}=F(z,\dot{z},t)$$

%============================================================================
\begin{example}
\label{spd ex:dropped}

\problem A particle is dropped from rest at a height $z_0$ close to the
Earth's surface, with air resistance proportional to the particle's speed.
Find the particle's height $z(t)$ and its terminal velocity 
$\dot{z}_{\rm ter}$.

\solution
Air resistance is proportional to speed so the force on the particle is
$$\vect{F}=-mg\kvect-\lambda\dot{z}\kvect$$
Therefore the equation of motion is a second order linear \ODE with 
constant coefficients
$$m\ddot{z}=-mg-\lambda\dot{z}$$
with $\dot{z}=0$ and $z=z_0$ when $t=0$.

In standard form, this is
$$\ddot{z}+\frac{\lambda}{m}\dot{z}=-g$$
Try a homogeneous solution of the form $z=e^{\alpha t}$.  This gives
$$\ddot{z}+\frac{\lambda}{m}\dot{z}
=\left[\alpha^2+\frac{\lambda}{m}\alpha\right]e^{\alpha t}=0$$
so that
$$z=A+Be^{-\frac{\lambda}{m}t}+z_p$$
where the particular solution $z_p$ is yet to be determined.

The homogeneous solution with $\alpha=0$ has the same form as the
inhomogeneous term in the differential equation, thus a constant is not an
acceptable particular solution.  Instead try
$$z_p(t)=ct$$
Substituting into the differential equation shows that
$$c=-\frac{mg}{\lambda}$$
so that 
$$z_p(t)=-\frac{mg}{\lambda}t$$
giving the general solution
$$z=A+Be^{-\frac{\lambda}{m}t}-\frac{mg}{\lambda}t$$

When $t=0$, $z_0=A+B$ and $\dot{z}=0=-\lambda B/m-mg/\lambda$,
so the constants are
$$B=-\left(\frac{m}{\lambda}\right)^2g$$
and
$$A=z_0-B=z_0+\left(\frac{m}{\lambda}\right)^2g$$
giving the solution
$$z(t)=z_0+\left(\frac{m}{\lambda}\right)^2g\left[
1-e^{-\frac{\lambda}{m}t}-\frac{\lambda}{m}t\right]$$

The speed of the particle is
$$\dot{z}(t)=-\frac{mg}{\lambda}\left[1-e^{-\frac{\lambda}{m}t}\right]$$
which reaches a terminal velocity of 
$$\dot{z}_{\rm ter}=-\frac{mg}{\lambda}$$
as $t\to\infty$.

\parbreak Alternatively the terminal velocity can be obtained directly 
from the \ODE by noting
that as the terminal velocity is approached, $\dot{z}$ becomes constant so
that $\ddot{z}$ tends to zero.  Therefore the differential equation
$$m\ddot{z}=-mg-\lambda\dot{z}$$
becomes
$$0=-mg-\lambda\dot{z}_{\rm ter}$$
so that the terminal velocity is
$$\dot{z}_{\rm ter}=-\frac{mg}{\lambda}$$

To see what happens in the absence of air resistance, take the limit as
$\lambda\to 0$.  Now 
\begin{eqnarray*}
\lim_{\lambda\to 0} z(t)
&=&z_0+\lim_{\lambda\to 0}\left(\frac{m}{\lambda}\right)^2g\left(
1-e^{-\frac{\lambda}{m}t}-\frac{\lambda}{m}t\right) \\
&=&z_0+g\lim_{\alpha\to 0}\frac{1}{\alpha^2}
	\left(1-e^{-\alpha t}-\alpha t\right) \\
&=&z_0+g\lim_{\alpha\to 0}\frac{1}{\alpha^2}
	\left[(1-\alpha t)-\left(1-\alpha t +\frac{\alpha^2t^2}{2}-\cdots
\right)\right] \\
&=&z_0-\frac{gt^2}{2}
\end{eqnarray*}
which is the free-fall limit.
\end{example}
%============================================================================

%============================================================================
\begin{example}
\problem Find the height reached by a particle launched vertically upwards 
with speed $v$ from the Earth's surface with air resistance proportional 
to the square of the particle's speed.

\solution The equation of motion is 
$$m\ddot{z}=-mg-\lambda\dot{z}^2$$
subject to $z=0$ and $\dot{z}=v$ at $t=0$.

The differential equation is
$$\ddot{z}=-g-\frac{\lambda}{m}\dot{z}^2$$
But $\ddot{z}=\dbd{z}\left(\frac{1}{2}\dot{z}^2\right)$ so put 
$w=\dot{z}^2/2$ in the equation of motion to give
$$\der{w}{z}=-g-\frac{2\lambda}{m}w$$
or
$$\der{w}{z}+\frac{2\lambda}{m}w=-g$$
$w(z)$ is the sum of the homogeneous solution and a particular solution.  
For the homogeneous solution, trying $w=e^{\alpha z}$ shows that
$\alpha=-2\lambda/m$.  For the particular solution, try 
$$w_p=A$$
Substituting into the \ODE shows that
$$\frac{2\lambda}{m}A=-g$$
so that the general solution is
$$w(z)=Ce^{-\frac{2\lambda}{m}z}-\frac{mg}{2\lambda}$$

At $z=0$, $w=v^2/2$ so
$$\frac{1}{2}v^2=C-\frac{mg}{2\lambda}$$
Therefore
$$w(z)=\left(\frac{v^2}{2}+\frac{mg}{2\lambda}\right)
\,e^{-\frac{2\lambda}{m}z}-\frac{mg}{2\lambda}$$

At the greatest height reached, the velocity is zero, hence
$$w(z_{\rm max})=0$$
Therefore
$$0=\left(\frac{v^2}{2}+\frac{mg}{2\lambda}\right)
\,e^{-\frac{2\lambda}{m}z_{\rm max}}-\frac{mg}{2\lambda}$$
so that
$$z_{\rm max}=\frac{m}{2\lambda}\ln\left[1+\frac{\lambda v^2}{mg}\right]$$

To find the greatest height reached when there is no air resistance,
take the limit as $\lambda\to 0$.  Since, for small $x$,
$$\frac{1}{1+x}=1-x+x^2-x^3+\cdots$$
integrating term-by-term gives
$$\ln(1+x)=\int^x \frac{dx}{1+x}=x-\frac{x^2}{x}+\frac{x^3}{3}-\cdots$$
Therefore
$$z_{\rm max}=\frac{m}{2\lambda}\left[
\frac{\lambda v^2}{mg}-\frac{1}{2}\left(\frac{\lambda v^2}{mg}\right)^2
+\frac{1}{3}\left(\frac{\lambda v^2}{mg}\right)^3-\cdots\right]$$
This can be written in the form
$$z_{\rm max}=\frac{v^2}{2g}+O(\lambda)$$
where $O(\lambda)$ indicates terms of order $\lambda$ and higher.  In the
limit as $\lambda\to 0$, the $O(\lambda)$ terms tend to zero, leaving
$$z_{\rm max}=\frac{v^2}{2g}$$
\end{example}
%============================================================================

%%%%%%%%%%%%%%%%%%%%%%%%%%%%%%%%%%%%%%%%%%%%%%%%%%%%%%%%%%%%%%%%%%%%%%%%%%%%%
\section{Motion in a Plane}

For this analysis of projectile motion, ignore the Earth's rotation and
assume that the height of the projectile is always much smaller than the
radius of the Earth.

%----------------------------------------------------------------------------
\begin{figure}\centering
\caption{Motion of a projectile with air resistance $\vect{F}_{\rm res}$. 
The projectile's initial velocity is $v$ at an angle $\alpha$ to the
horizontal.}
\label{spd fig:pm I}

\psset{unit=5cm}
\begin{pspicture}(-0.25,-0.35)(1.65,1.2)
% First the curve, so the axes show through
\psplot[linecolor=gray,linewidth=2pt,plotstyle=curve]{0}{1.2}{2 x mul 
1.2 x sub mul}
% Now the axes and their labels
\psline{<->}(0,0.8)(0,0)(1.4,0)
\uput[dl](0,0){$O$}
\uput[r](1.4,0){$x$}
\uput[u](0,0.8){$z$}
% The velocity at the origin
\psline{->}(0,0)(0.154,0.369)
\uput[u](0.154,0.369){$v$}
\psarc{->}(0,0){0.2}{0}{67}
\uput[r](0.17,0.07){$\alpha$}
% The vector r(t)
\pcline{->}(0,0)(0.9,0.54)
\Bput{$\vect{r}(t)$}
% The weight
\psline{->}(0.9,0.54)(0.9,0.29)
\uput[d](0.9,0.29){$mg$}
% The air resistance
\rput[b]{-50.2}(0.9,0.54){
	\psline[linecolor=black,linestyle=dashed]{-}(0,0)(0.25,0) 
	\rput[b]{*0}(0,0.05){$P$}
	\psline[linecolor=black]{->}(0,0)(-0.25,0) 
	\rput[br]{*0}(-0.26,0){$\vect{F}_{\rm res}$}
}
\end{pspicture}
\end{figure}
%----------------------------------------------------------------------------

From figure \ref{spd fig:pm I}, the equation of motion is
$$m\ddot{\vect{r}}=-mg\kvect+\vect{F}_{\rm res}$$
where the air resistance has magnitude $\phi(v)$ and is directed in the
opposite direction to the projectile's motion
$$\vect{F}_{\rm res}=-\phi(v) \hat{\dot{\vect{r}}}
=-\frac{\phi(v)}{v}\dot{\vect{r}}$$
where the projectile's speed is
$$v=\left|\dot{\vect{r}}\right|=\sqrt{\dot{x}^2+\dot{z}^2}$$

The equation of motion can be written as separate equations for the $x$ and
$z$ directions
$$\ddot{x}=-\frac{\phi(v)}{mv}\dot{x}$$
and
$$\ddot{z}=-g-\frac{\phi(v)}{mv}\dot{z}$$
The solution of these depend on the form of the air resistance function
$\phi(v)$.


%============================================================================
\begin{example}
When $\phi(v)=0$ and there is no air resistance, the equations of motion
become
$$\ddot{x}=0\qquad\mbox{and}\qquad\ddot{z}=-g$$
subject to the initial conditions $x(0)=z(0)=0$, $\dot{x}(0)=v\cos\alpha$ 
and $\dot{z}(0)=v\sin\alpha$.

The solution is 
$$x(t)=vt\cos\alpha$$
$$z(t)=vt\sin\alpha-\frac{1}{2}gt^2$$
Eliminating $t$ from these two equations gives the equation of the
projectile's trajectory
$$z=x\tan\alpha-\frac{gx^2}{2v^2\cos^2\alpha}$$
which is a parabola.  Factorising $z$ as
$$z=\frac{x}{\cos\alpha}\left(\sin\alpha-\frac{gx}{2v^2\cos\alpha}\right)$$
shows that $z=0$ when $x=0$ and when $x=2v^2\sin\alpha\cos\alpha/g$
so that the horizontal range is
$$x_{\rm max}=\frac{v^2}{g}\sin 2\alpha$$
This range is maximised when $\sin 2\alpha=1$, which is an elevation of 
$\alpha=\pi/4$.  

The maximum height occurs when $\dot{z}(t)=0$, which occurs when
$$t=\frac{v\sin\alpha}{g}$$
Therefore the maximum height as a function of the projectile's initial
elevation is
$$z_{\rm max}(\alpha)=\frac{v^2\sin^2\alpha}{2g}$$
which occurs when $x$ is equal to half the range.  Note that
$$z_{\rm max}\left(\frac{\pi}{4}\right)=\frac{v^2}{4g}$$
which is equal to one-quarter of the maximum range.
\end{example}
%============================================================================

%============================================================================
\begin{exercise}
Now try the problem again assuming that the ground is at an angle $\beta$
so that when the projectile lands, $x$ and $z$ must satisfy
$$z=x\tan\beta$$
Combine this with the equation for the projectile's motion to find the $x$
and $z$ coordinates of the point when the projectile lands.
\end{exercise}
%============================================================================


%============================================================================
\begin{example}
\label{spd ex:prop sp}

When $\phi(v)=\lambda v$ and the air resistance is proportional to the
projectile's speed, the equations of motion become
$$\ddot{x}=-\frac{\lambda}{m}\dot{x}$$
and
$$\ddot{z}=-g-\frac{\lambda}{m}\dot{z}$$
subject to the initial conditions $x(0)=z(0)=0$, $\dot{x}(0)=v\cos\alpha$ 
and $\dot{z}(0)=v\sin\alpha$.

This has already been solved for $z$ with different boundary conditions
in example \ref{spd ex:dropped}, so the solution is
$$z(t)=A+Be^{-\frac{\lambda}{m}t}-\frac{m}{\lambda}gt$$
The boundary conditions are
$$z(0)=0=A+B$$
and
$$\dot{z}(0)=v\sin\alpha=-\frac{\lambda}{m}B-\frac{mg}{\lambda}$$
Therefore
$$A=-B=\left(\frac{m}{\lambda}\right)^2g+\frac{m}{\lambda}v\sin\alpha$$
so the $z$ solution is
$$z(t)=\left[\left(\frac{m}{\lambda}\right)^2g+\frac{m}{\lambda}v\sin\alpha
\right]\left(1-e^{-\frac{\lambda}{m}t}\right)-\frac{m}{\lambda}gt$$

The general solution for $x$ is
$$x(t)=C+De^{-\frac{\lambda}{m}t}$$
The boundary conditions are
$$x(0)=0=C+D$$ 
and
$$\dot{x}(0)=v\cos\alpha=-\frac{\lambda}{m}D$$
Therefore
$$C=-D=\frac{m}{\lambda}v\cos\alpha$$
so the $x$ solution is
$$x(t)=\frac{m}{\lambda}v\cos\alpha\left(1-e^{-\frac{\lambda}{m}t}\right)$$

Eliminating $t$ from the equations for $x(t)$ and $z(t)$ gives
$$z=\left(\frac{v\sin\alpha+mg/\lambda}{v\cos\alpha}\right)x
+\left(\frac{m}{\lambda}\right)^2g
\ln\left(1-\frac{\lambda x}{mv\cos\alpha}\right)$$
which is plotted in figure \ref{spd fig:pm II}.
\end{example}
%============================================================================

%----------------------------------------------------------------------------
\begin{figure}\centering
\caption{For projectile motion with air resistance, three points $x_1$,
$x_2$ and $x_3$ can be defined.  $x_1$ is the horizontal position of the
projectile's greatest elevation, $x_2$ the point at which the projectile
hits the ground, and $x_3$ the position of the vertical asymptote if the
projectile were to continue falling through the ground.}
\label{spd fig:pm II}

\psset{unit=5cm}
\begin{pspicture}(-0.25,-0.5)(1.65,1.2)
% First the curve, so the axes show through
\psplot[linecolor=gray,linewidth=2pt,plotstyle=curve]{0}{1.175}{1 x 1.2 div
sub ln 0.4 mul x add 1.4 mul} 
% Now the axes and their labels
\psline{->}(0,0)(1.4,0)
\psline{->}(0,-0.5)(0,0.8)
\uput[l](0,0){$O$}
\uput[r](1.4,0){$x$}
\uput[u](0,0.8){$z$}
% The velocity at the origin
\psline{->}(0,0)(0.29,0.27)
\uput[u](0.29,0.27){$v$}
\psarc{->}(0,0){0.2}{0}{43}
\uput[r](0.17,0.07){$\alpha$}
% The intercepts				
\psline[linecolor=black,linestyle=dashed]{-}(0.8,0)(0.8,0.5) 
\uput[d](0.8,0){$x_1$}
\psline[linecolor=black,linestyle=dashed]{-}(1.2,-0.5)(1.2,0.5) 
\uput[ur](1.2,0){$x_3$}
\uput[dl](1.1,0){$x_2$}
\end{pspicture}
\end{figure}
%----------------------------------------------------------------------------

%============================================================================
\begin{exercise}
Find expressions for the points $x_1$ and $x_3$ in figure \ref{spd fig:pm
II}.
\end{exercise}
%============================================================================

%============================================================================
\begin{example}
The general expression for air resistance is $\phi(v)=\lambda v^m$ for some
power $m$.  The equation of motion is
$$m\ddot{\vect{r}}=-mg\kvect-\phi(v)\Tvect$$
The velocity of the projectile is just the speed in the tangential direction
$\dot{\vect{r}}=\dot{s}\Tvect$ so
$$\left|\dot{\vect{r}}\right|=v=\dot{s}$$
From
$$\ddot{\vect{r}}=\ddot{s}\Tvect+\dot{s}\dot{\psi}\Nvect$$
the equation of motion in the $\Tvect$ direction is
\begin{eqnarray*}
\ddot{s}&=&-g\kvect\cdot\Tvect-\frac{\phi(\dot{s})}{m} \\
&=&-g\sin\psi-\frac{\phi(\dot{s})}{m}
\end{eqnarray*}
and in the $\Nvect$ direction,
\begin{eqnarray*}
\dot{s}\dot{\psi}&=&-g\kvect\cdot\Nvect\\
&=&-g\cos\psi
\end{eqnarray*}
The remainder of the solution depends on the form of $\phi(\dot{s})$.
\end{example}
%============================================================================

%----------------------------------------------------------------------------
\begin{figure}\centering
\caption{Motion of the missile for the Death Star problem.}
\label{spd fig:ds}

\psset{unit=5cm}
\begin{pspicture}(-0.25,-0.65)(1.45,0.9)
% First the curve, so the axes show through
\psplot[linecolor=gray,linewidth=2pt,plotstyle=curve]{0}{1.175}{1 x 1.2 div
sub ln 0.4 mul x add 1.4 mul} 
% Now the axes and their labels
\psline{->}(0,0)(0.4,0)
\psline{->}(0,0)(0,0.4)
\uput[r](0.4,0){$x$}
\uput[u](0,0.4){$z$}
% The velocity at the origin
\psline{->}(0,0)(0.29,0.27)
\uput[u](0.29,0.27){$v$}
\psarc{->}(0,0){0.2}{0}{43}
\uput[r](0.17,0.07){$\alpha$}
% The Death Star
\psline[linecolor=black,linewidth=2pt]{-}(0,-0.3)(1,-0.3)(1,-0.6) 
\psline[linecolor=black,linewidth=2pt]{-}(1.2,-0.6)(1.2,-0.3)(1.4,-0.3)
% The D, H and d labels with arrows
\pcline{<->}(0,-0.55)(1.2,-0.55)
\Bput{$D$}
\pcline{<->}(1,-0.35)(1.2,-0.35)
\Aput{$d$}
\pcline{<->}(0,-0.3)(0,0)
\Aput{$H$}
% The weight
\psline{->}(0.95,0.45)(0.95,0.2)
\uput[d](0.95,0.2){$mg$}
% The air resistance
\rput[br]{-40.3}(0.95,0.45){
	\psline[linecolor=black]{->}(0,0)(-0.25,0) 
	\rput[br]{*0}(-0.26,0){$\vect{F}_{\rm res}=-\lambda\dot{\vect{r}}$}
}
\end{pspicture}
\end{figure}
%----------------------------------------------------------------------------

%============================================================================
\begin{example}[The Death Star]

\problem From the diagram in figure \ref{spd fig:ds}, you have to fire a 
missile	that falls down a narrow chute of width $d$.  The missile is
launched at an initial angle $\alpha$ with speed $v$ from a height $H$ and
a distance $D$ from the far edge of the chute.  Given that air resistance is
proportional to speed, calculate the maximum value of $D$.  Also calculate
the minimum height $H$ from which the missile can be released at this
maximum distance $D$ and still fall down the chute.

\solution
With air resistance proportional to speed, the equation of motion is
$$m\ddot{\vect{r}}=-mg\kvect-\lambda \dot{\vect{r}}$$
This is identical to the situation considered in example \ref{spd ex:prop sp}
with solution
$$z=\left(\frac{v\sin\alpha+mg/\lambda}{v\cos\alpha}\right)x
+\left(\frac{m}{\lambda}\right)^2g
\ln\left(1-\frac{\lambda x}{mv\cos\alpha}\right)$$

The maximum horizontal range is the furthest $x$ distance reached as
$z\to-\infty$.  This occurs when
$$1-\frac{\lambda x}{mv\cos\alpha}=0$$
or when
$$x=\frac{mv}{\lambda}\cos\alpha$$

The maximum range is maximised when $\alpha=0$, so this is achieved when the
missile is fired horizontally, and 
$$D=x_{\rm max}=\frac{mv}{\lambda}$$
Putting $\alpha=0$, the trajectory becomes
$$z=\left(\frac{mg}{\lambda v}\right)x
+\left(\frac{m}{\lambda}\right)^2g\ln\left(1-\frac{\lambda x}{mv}\right)$$

The trajectory with minimum height just grazes the inner lip of the chute
and the asymptote is the chute's back wall.  Therefore the trajectory
passes through
$$z=-H\qquad\mbox{at}\qquad x=D-d$$
Substituting this into the trajectory with $\alpha=0$ gives
$$-H=\left(\frac{mg}{\lambda v}\right)(D-d)
+\left(\frac{m}{\lambda}\right)^2g\ln\left(1-\frac{\lambda (D-d)}{mv}\right)$$
which gives
$$H=\frac{gD^2}{v^2}\left[-\ln\left(\frac{d}{D}\right)+\frac{d}{D}-1\right]$$
\end{example}
%============================================================================

%----------------------------------------------------------------------------
\begin{figure}\centering
\caption{This shows a projectile $P$ launched from the equator $E$ of an airless
planet with speed $v$ at an angle $\alpha$ to the vertical towards the
north.  $\vect{F}_{\rm gr}$ is the gravitational attraction.}
\label{spd fig:ap}

\psset{unit=5cm}
\begin{pspicture}(-0.6,-0.6)(1.2,0.9)
% Here is the Earth with its axes
\pscircle(0,0){0.5}
\psline{->}(-0.6,0)(1,0)
\psline{->}(0,-0.6)(0,0.6)
\uput[dr](0.5,0){$E$}
\uput[u](0,0.6){$N$}
% The parabola describes the projectile's trajectory
\parametricplot[linecolor=gray,linewidth=2pt,plotstyle=curve,arrows=->]%
{-0.3}{0.2}{1 t t mul 0.18 div sub t 0.3 add}
% The velocity at the origin
\SpecialCoor
\psline{->}(0.5,0)(0.8,0.1)
\uput[r](0.8,0.1){$v$}
\psarc{->}(0.5,0){0.2}{0}{(0.3,0.1)}
\uput[ur](0.7,0){$\alpha$}
% The point P
\uput[r](1,0.3){$P$}
\psarc{->}(0,0){0.2}{0}{(1,0.3)}
\uput[ur](0.2,0){$\theta$}
\pcline{->}(0,0)(1,0.3)
\Aput{$\vect{r}(t)$}
\pcline{->}(1,0.3)(0.8,0.24)
\Bput{$\vect{F}_{gr}$}
% Radius of the Earth
\pcline{->}(0,0)(0.5;300)\Aput{$R$}
\end{pspicture}
\end{figure}
%----------------------------------------------------------------------------

%============================================================================
\begin{example}
\problem
A projectile is fired from the equator $E$ of an airless planet of radius
$R$ northwards with initial speed $v$ and at an angle $\alpha$ to the
vertical.  The acceleration due to gravity at the surface is $g$.
This is illustrated in figure \ref{spd fig:ap}. 

\begin{description}
\item[(a)] Show that the projectile's motion satisfies
$$\ddot{r}-r\dot{\theta}^2=-\frac{gR^2}{r^2}$$
and
$$r\ddot{\theta}+2\dot{r}\dot{\theta}=0$$

\item[(b)] Hence show that 
$$\dot{\theta}=\frac{Rv\sin\alpha}{r^2}$$

\item[(c)] Also show that the maximum height of the projectile is given by
$$\frac{r}{R}=\frac{1}{2(\lambda-1)}\left(
\sqrt{\lambda^2-4(\lambda-1)\sin^2\alpha}+\lambda\right)$$
where $\lambda=2gR/v^2$ provided $\lambda>1$.
\end{description}

\solution
With $m$ the projectile's mass and $M$ the planet's mass, the force due to 
gravity is
$$\vect{F}=-\frac{GmM}{r^2}\rvect$$
At the surface, when $r=R$, the force due to gravity is
$$\vect{F}=-mg\rvect=-\frac{GmM}{R^2}\rvect$$
so that
$$\vect{F}=-\frac{mgR^2}{r^2}\rvect$$

There is no air resistance, so from Newton's second law,
$$m\ddot{\vect{r}}=\vect{F}$$
Choosing a polar coordinate system,
$$\ddot{\vect{r}}=(\ddot{r}-r\dot{\theta}^2)\rvect
+(r\ddot{\theta}+2\dot{r}\dot{\theta})\tvect=-\frac{gR^2}{r^2}\rvect$$
Equating radial and polar components yields
$$\ddot{r}-r\dot{\theta}^2=-\frac{gR^2}{r^2}$$
and
$$r\ddot{\theta}+2\dot{r}\dot{\theta}=0$$
as required.


From the initial conditions, $r(0)=R$ and $\theta(0)=0$.  Initially,
$$\dot{\vect{r}}(0)=v\cos\alpha\rvect+v\sin\alpha\tvect$$
Using
$$\dot{\vect{r}}=\dot{r}\rvect+r\dot{\theta}\tvect$$
shows that
$$\dot{r}(0)=v\cos\alpha$$
and
$$R\dot{\theta}(0)=v\sin\alpha$$

To show that
$$\dot{\theta}=\frac{Rv\sin\alpha}{r^2}$$
multiply
$$r\ddot{\theta}+2\dot{r}\dot{\theta}=0$$
by $r$ and use product rule to show that
$$\dbd{t}(r^2\dot{\theta})=0$$
Integrating shows that
$$r^2\dot{\theta}=C$$
for some constant $C$.  At $t=0$, $r=R$ and
$\dot{\theta}=v\sin\alpha/R$.  Therefore
$$r^2\dot{\theta}=R^2\frac{v}{R}\sin\alpha$$
so that
$$\dot{\theta}=\frac{Rv\sin\alpha}{r^2}$$

To show that the maximum height of the projectile is given by
$$\frac{r}{R}=\frac{1}{2(\lambda-1)}\left(
\sqrt{\lambda^2-4(\lambda-1)\sin^2\alpha}+\lambda\right)$$
where $\lambda=2gR/v^2$ provided $\lambda>1$, we know that
$$\dot{\theta}=\frac{Rv\sin\alpha}{r^2}$$
and
$$\ddot{r}-r\dot{\theta}^2=-\frac{gR^2}{r^2}$$
Substituting for $\dot{\theta}$ shows that
$$\ddot{r}=\frac{v^2R^2\sin^2\alpha}{r^3}-\frac{gR^2}{r^2}$$
Using the usual identity,
$$\ddot{r}=\dbd{r}\left(\frac{1}{2}\dot{r}^2\right)=
\frac{v^2R^2\sin^2\alpha}{r^3}-\frac{gR^2}{r^2}$$
Integrating with respect to $r$ gives
$$\frac{1}{2}\dot{r}^2=
-\frac{1}{2}\frac{v^2R^2\sin^2\alpha}{r^2}+\frac{gR^2}{r}+C$$
At time $t=0$, $r=R$ and $\dot{r}=v\cos\alpha$ so
$$\frac{1}{2}v^2\cos^2\alpha=
-\frac{1}{2}\frac{v^2R^2\sin^2\alpha}{R^2}+\frac{gR^2}{R}+C$$
which determines the constant
$$C=\frac{1}{2}v^2-gR$$
Therefore, a first integral of the radial equation is
$$\frac{1}{2}\dot{r}^2=\frac{1}{2}v^2-gR
-\frac{1}{2}\frac{v^2R^2\sin^2\alpha}{r^2}+\frac{gR^2}{r}$$
The maximum height occurs when $\dot{r}=0$, thus
$$0=\frac{1}{2}v^2-gR
-\frac{1}{2}\frac{v^2R^2\sin^2\alpha}{r^2}+\frac{gR^2}{r}$$
Divide by $v^2/2$ and rearrange to give the quadratic
$$\left(\frac{r}{R}\right)^2(\lambda-1)-\lambda\frac{r}{R}+\sin^2\alpha=0$$
Therefore
$$\frac{r}{R}=\frac{1}{2(\lambda-1)}\left(
\lambda\pm\sqrt{\lambda^2-4(\lambda-1)\sin^2\alpha}\right)$$
To determine which of the two solutions is appropriate, note that as
$v\to0$, $r/R$ must tend to $1$.  As $v\to0$, $\lambda\to\infty$,
showing that the positive solution is needed.  This gives the maximum 
height as
$$\frac{r}{R}=\frac{1}{2(\lambda-1)}\left(
\sqrt{\lambda^2-4(\lambda-1)\sin^2\alpha}+\lambda\right)$$
as required.

Note that if $\lambda<1$, then $\sqrt{\lambda^2-4(\lambda-1)\sin^2\alpha}
>\lambda$ so the negative solution must be chosen to ensure that $r/R>0$.
In this case,
$$\frac{r}{R}=\frac{1}{2(1-\lambda)}\left(
\sqrt{\lambda^2-4(\lambda-1)\sin^2\alpha}-\lambda\right)<1$$
so no maximum height is reached.  When $\lambda=1$, the speed of the projectile
is exactly equal to the escape velocity of the planet.
\end{example}
%============================================================================


%%%%%%%%%%%%%%%%%%%%%%%%%%%%%%%%%%%%%%%%%%%%%%%%%%%%%%%%%%%%%%%%%%%%%%%%%%%%%
\section{Momentum and Torque}

\name{Linear momentum} $\vect{p}$ is defined as
$$\vect{p}=m\dot{\vect{r}}$$
From Newton's second law, in an inertial reference frame,
$$\vect{F}=m\ddot{\vect{r}}=\dbd{t}(m\dot{\vect{r}})$$
so that
$$\der{\vect{p}}{t}=\vect{F}$$
When $\vect{F}=0$, $\vect{p}$ is constant, so linear momentum is conserved
when no net force acts on the particle.

The \name{angular momentum} $\vect{L}$ of a particle about the point $O$
is defined as 
$$\vect{L}=\vect{r}\times\vect{p}$$
where $\vect{r}$ is the particle's position relative to $O$ and $\vect{p}$
is the particle's linear momentum.  Note that
$$\der{\vect{L}}{t}=\dbd{t}(\vect{r}\times(m\dot{\vect{r}}))
=\dot{\vect{r}}\times(m\dot{\vect{r}})+\vect{r}\times(m\ddot{\vect{r}})$$
from product rule.  The first cross-product is zero because $\dot{\vect{r}}$
and $m\dot{\vect{r}}$ are parallel.  Writing $\vect{F}=m\ddot{\vect{r}}$
shows that
$$\der{\vect{L}}{t}=\vect{r}\times \vect{F}$$
provided the reference frame is inertial.

The \name{torque} of a force $\vect{F}$ about a point $O$ is
$\vect{r}\times\vect{F}$ where $\vect{r}$ is the position vector from $O$ to
the point of application of the force.  

Therefore, torque is equal to the rate of change of angular momentum.  Angular
momentum is conserved when no torque acts.

%%%%%%%%%%%%%%%%%%%%%%%%%%%%%%%%%%%%%%%%%%%%%%%%%%%%%%%%%%%%%%%%%%%%%%%%%%%%%
\section{Central Forces}

A force $\vect{F}$ which always acts along the position vector $\vect{r}$ is
called a \name{central force}.
$$\vect{F}=f(r)\rvect$$

%============================================================================
\begin{theorem}
The trajectory of a particle acted on by a central force lies in a plane.
\end{theorem}

\begin{proof}
At any instant $t$, the two vectors $\vect{r}(t)$ and $\dot{\vect{r}}(t)$
define a plane.  The acceleration is
$$\ddot{\vect{r}}=\frac{1}{m}\vect{F}=\frac{1}{m}f(r)\rvect$$
which lies in the plane.  Thus the particle's subsequent motion is confined
to that plane.
\end{proof}
%============================================================================

Therefore central forces imply motion in a plane.

%============================================================================
\begin{theorem}
The angular momentum of a particle in a central force field is constant.
\end{theorem}

\begin{proof}
The rate of change is angular momentum is
$$\der{\vect{L}}{t}=\vect{r}\times \vect{F}=\vect{r}\times f(r)\rvect=0$$
because $\vect{F}$ is parallel to $\vect{r}$.  Therefore $\vect{L}$ is
constant.
\end{proof}
%============================================================================

The physical significance of constant angular momentum is related to the
area swept out by the particle's position vector $\vect{r}(t)$.  Suppose
that $\vect{r}(t)$ sweeps out an area $A(t)$ as it moves in the plane.  In a
short time $\delta t$, the area swept out is
$$\delta A\approx\frac{1}{2}\left|\vect{r}(t+\delta t)\times\vect{r}(t)\right|$$
Now $\vect{r}(t+\delta t)=\vect{r}(t)+\delta\vect{r}$ so
$$\vect{r}(t+\delta t)\times\vect{r}(t)=\delta\vect{r}\times\vect{r}(t)$$
Therefore
$$\der{A}{t}=\lim_{\delta t\to 0}\frac{\delta A}{\delta t}
=\lim_{\delta t\to 0}\frac{1}{2}\left|\vect{r}\times\frac{\delta\vect{r}}
{\delta t}\right|=\frac{1}{2}\left|\vect{r}\times\dot{\vect{r}}\right|$$
Writing
$$\frac{1}{2}\left|\vect{r}\times\dot{\vect{r}}\right|
=\frac{1}{2m}\left|\vect{r}\times(m\dot{\vect{r}})\right|$$
shows that
$$\der{A}{t}=\frac{1}{2m}\left|\vect{L}\right|$$
Since angular momentum is constant in a central force field, we have
$$\der{A}{t}=\rm constant$$
This is Kepler's second rule of planetary motion (1609) which was explained
by Newton as a property of central forces in his {\em Principia} of 1687.

In polar coordinates, 
$$\dot{\vect{r}}=\dot{r}\rvect+r\dot{\theta}\tvect$$
so that
$$\vect{L}=\vect{r}\times (m\dot{\vect{r}})=m(r\rvect)\times
(\dot{r}\rvect+r\dot{\theta}\tvect)$$
Therefore 
$$\vect{L}=mr^2\dot{\theta}(\rvect\times\tvect)$$
Now $\rvect\times\tvect$ is the unit vector normal to the plane in which the
particle moves, so the magnitude of the angular momentum is
$$\left|\vect{L}\right|=mr^2\left|\dot{\theta}\right|$$
Substituting this back into the equation for $\der{A}{t}$ shows that
$$\der{A}{t}=\frac{1}{2}r^2\left|\dot{\theta}\right|$$

%%%%%%%%%%%%%%%%%%%%%%%%%%%%%%%%%%%%%%%%%%%%%%%%%%%%%%%%%%%%%%%%%%%%%%%%%%%%%
\section{Orbits in a Central Force Field}

The equation of motion is
$$m\ddot{\vect{r}}=f(r)\rvect$$
where the acceleration can be expressed as
$$\ddot{\vect{r}}=\left(\ddot{r}-r\dot{\theta}^2\right)\rvect
+\left(r\ddot{\theta}+2\dot{r}\dot{\theta}\right)\tvect$$
in polar coordinates.

The radial component is
$$\ddot{r}-r\dot{\theta}^2=\frac{f(r)}{m}$$
and the transverse component is
$$r\ddot{\theta}+2\dot{r}\dot{\theta}=0$$
This can be solved for $\dot{\theta}$ by writing it as
$$\dbd{t}\dot{\theta}+2\frac{\dot{r}}{r}\dot{\theta}=0$$
The integrating factor is $r^2$ which shows that
$$\dbd{t}\left(r^2\dot{\theta}\right)=0$$
so that $r^2\dot{\theta}$ is a constant.  Let this constant be $h$ so that
$$r^2\dot{\theta}=h$$
Note that $\frac{1}{2}\left|h\right|=\der{A}{t}=\frac{1}{2m}\left|\vect{L}
\right|$.

To obtain the equation of the orbit in the form $r(\theta)$, use
$$\dot{r}=\der{r}{\theta}\dot{\theta}=\frac{h}{r^2}\der{r}{\theta}$$
and
$$\ddot{r}=\frac{h}{r^2}\dbd{\theta}
\left(\frac{h}{r^2}\der{r}{\theta}\right)$$
The radial equation of motion can then be written as
$$\frac{h^2}{r^2}\dbd{\theta}\left(\frac{1}{r^2}\der{r}{\theta}\right)
-r\left(\frac{h}{r^2}\right)^2=\frac{f(r)}{m}$$
Rearranging this gives
$$\dbd{\theta}\left(\frac{1}{r^2}\der{r}{\theta}\right)
-\frac{1}{r}=\frac{r^2f(r)}{mh^2}$$
which can be solved to give $r(\theta)$.

%%%%%%%%%%%%%%%%%%%%%%%%%%%%%%%%%%%%%%%%%%%%%%%%%%%%%%%%%%%%%%%%%%%%%%%%%%%%%
\subsection{Motion in a Gravitational Field}

Let the central force be
$$f(r)=-\frac{GMm}{r^2}$$
where $M$ is the mass of the Sun at the origin and $m$ is the mass of the
body.  We will asssume that $M\gg m$ so that the Sun may be assumed to be
fixed at the origin.

The equation for the orbit becomes
$$\dbd{\theta}\left(\frac{1}{r^2}\der{r}{\theta}\right)
-\frac{1}{r}=-\frac{GM}{h^2}$$
Since 
$$\frac{1}{r^2}\der{r}{\theta}=-\dbd{\theta}\left(\frac{1}{r}\right)$$
this can be written
$$\ndbd{\theta}{2}\left(\frac{1}{r}\right)+\frac{1}{r}=\frac{GM}{h^2}$$
The general solution is
$$\frac{1}{r}=B\cos\theta+C\sin\theta+\frac{GM}{h^2}=A\cos(\theta-\theta_0)
+\frac{GM}{h^2}$$
where $A$ and $\theta_0$ are arbitrary constants giving the orbit's
amplitude and phase angle.  In terms of $r$, this is
$$r=\frac{1}{A\cos(\theta-\theta_0)+GM/h^2}$$
$A$ and $\theta_0$ are determined by the initial conditions.  Since
$\theta_0$ determines only the orientation of the orbit in the plane, we can
set it to zero when discussing the orbit's shape.

\typeout{There is no page 74, but my written notes don't have a break.}

Define the orbit's \name{eccentricity} to be
$$e=\frac{Ah^2}{GM}$$
and $r_0$ to be the \name{perihelion distance}
$$r_0=\frac{h^2/GM}{1+e}$$
which is the smallest distance between the body and the Sun during the
orbit.  Then the orbit equation can be written
$$r=r_0\frac{1+e}{1+e\cos\theta}$$
so that $r=r_0$ when $\theta=0$ and $r>r_0$ for all other $\theta$ during
each orbit.

The shape of the orbit depends on the eccentricity $e$.  If $e=0$, the orbit
is a circle, and for $0<e<1$, the orbit is an ellipse.  Both of these are
closed, periodic orbits.  When $e=1$, the orbit is a parabola and for $e>1$,
the orbit is a hyperbola.  Both of these are open, single encounter orbits.

%%%%%%%%%%%%%%%%%%%%%%%%%%%%%%%%%%%%%%%%%%%%%%%%%%%%%%%%%%%%%%%%%%%%%%%%%%%%%
\subsection{Cartesian Equation of an Orbit}

The orbit equation can be written
$$r(1+e\cos\theta)=r_0(1+e)$$
Writing $L=r_0(1+e)$ and putting $x=r\cos\theta$, we have
$$r=L-ex$$
Squaring both sides so that $r^2=x^2+y^2$,
$$x^2+y^2=(L-ex)^2$$
thus
$$(1-e^2)x^2+2Lex+y^2=L^2$$
Completing the square gives
$$\frac{(x-x_0)^2}{L^2/(1-e^2)^2}+\frac{y^2}{L^2/(1-e^2)}=1$$
where
$$x_0=-\frac{Le}{1-e^2}$$
When $e=0$, this simplifies to the equation of a circle
$$x^2+y^2=L^2$$
When $0<e<1$, this simplifies to the equation of an ellipse
$$\frac{(x-x_0)^2}{a^2}+\frac{y^2}{b^2}=1$$
where $a=L/(1-e^2)$ and $b=L/\sqrt{1-e^2}$.  

When  $e=1$, the equation becomes that of a parabola
$$y^2=L^2-2Lx$$
and when $e>1$, it becomes the equation of a hyperbola
$$\frac{(x-x_0)^2}{a^2}-\frac{y^2}{b^2}=1$$
where $a=L/(e^2-1)$ and $b=L/\sqrt{e^2-1}$.  

%%%%%%%%%%%%%%%%%%%%%%%%%%%%%%%%%%%%%%%%%%%%%%%%%%%%%%%%%%%%%%%%%%%%%%%%%%%%%
\subsection{Kepler's Laws}

Centuries of increasingly accurate astronomical observations culminated in
Kepler's laws of planetary motion (1609):

\begin{enumerate}
\item Each planet moves in an ellipse with the Sun as a focus.
\item The radius vector sweeps out equal areas in equal times 
($\der{A}{t}$ is constant).
\item The square of the period of revolution about the Sun is proportional
to the cube of the major axis of the orbit.
\end{enumerate}

Law 2 is a consequence of gravity being a central force.  Law 1 is a
consequence of gravity being an inverse square attractive force.

%%%%%%%%%%%%%%%%%%%%%%%%%%%%%%%%%%%%%%%%%%%%%%%%%%%%%%%%%%%%%%%%%%%%%%%%%%%%%
\subsection{Elliptical Planetary Orbits}

The equations for a planet's orbit are
$$r=r_0\frac{1+e}{1+e\cos\theta}$$
or
$$\frac{(x-x_0)^2}{a^2}+\frac{y^2}{b^2}=1$$
where $a=L/(1-e^2)$ and $b=L/\sqrt{1-e^2}$ (so $a>b$) and
$x_0=-Le/(1-e^2)$. 

%----------------------------------------------------------------------------
\begin{figure}\centering
\caption{Planets move in an ellipse around the Sun.  Their closest approach
is at perihelion and their furthest distance is at aphelion.}
\label{spd fig:pm}

\psset{unit=5cm}
\begin{pspicture}(-0.8,-0.5)(0.5,0.6)
% First the curve, so the axes show through
\psellipse[linecolor=gray,linewidth=2pt,plotstyle=curve](-0.2,0)(0.5,0.3)
% Now the axes and their labels
\psline{->}(-0.8,0)(0.4,0)
\psline{->}(0,-0.35)(0,0.4)
\uput[r](0.4,0){$x$}
\uput[u](0,0.4){$y$}
% Dashed lines
\psline[linecolor=black,linestyle=dashed]{-}(-0.7,-0.5)(-0.7,0) 
\psline[linecolor=black,linestyle=dashed]{-}(-0.2,-0.5)(-0.2,0.4) 
\psline[linecolor=black,linestyle=dashed]{-}(0.3,-0.5)(0.3,0) 
% Arrows and labels
\pcline{<->}(-0.7,-0.4)(0,-0.4)\Aput{$r_1$}
\pcline{<->}(0,-0.4)(0.3,-0.4)\Aput{$r_0$}
\pcline{<->}(-0.2,-0.5)(0.3,-0.5)\Bput{$a$}
\pcline{<->}(-0.2,0)(-0.2,0.3)\Aput{$b$}
\uput[dr](-0.2,0){$x_0$}
\pcline{*-*}(0,0)(0.2,0.18)\Aput{$r$}
\psarc{->}(0,0){0.14}{0}{42}
\uput[r](0.13,0.05){$\theta$}
% Words for aphelion and perihelion
\pnode(0.3,0){PER}
\rput[bl](0.3,0.3){\rnode{PERLAB}{perihelion}}
\ncline[nodesep=3pt]{->}{PERLAB}{PER}
\pnode(-0.7,0){AP}
\rput[br](-0.6,0.3){\rnode{APLAB}{aphelion}}
\ncline[nodesep=3pt]{->}{APLAB}{AP}
\end{pspicture}
\end{figure}
%----------------------------------------------------------------------------

This elliptic orbit is shown in figure \ref{spd fig:pm}.  The \name{aphelion
distance}, $r_1$, is related to the \name{perihelion distance}, $r_0$, 
according to
$$r_1=r_0\frac{1+e}{1-e}$$
The area of the ellipse is 
$$A=\pi ab=\frac{\pi L^2}{(1-e^2)^{3/2}}$$
The rate at which the ellipse's area is swept out is
$$\der{A}{t}=\frac{h}{2}=\frac{1}{2}r^2\dot{\theta}$$
The period $T$ of the orbit is thus
$$T=\frac{A}{\der{A}{t}}=\frac{2\pi L^2}{h(1-e^2)^{3/2}}$$
so that $T^2$ is
$$T^2=\frac{4\pi^2 L^4}{h^2(1-e^2)^3}=\frac{4\pi^2a^3L}{h^2}$$
where $a=L/(1-e^2)$.  But $L=h^2/GM$ so 
$$\frac{a^3}{T^2}=\frac{GM}{4\pi^2}$$
which is the same for all planets.  This is the explanation of Kepler's
third law given by Newton.

%%%%%%%%%%%%%%%%%%%%%%%%%%%%%%%%%%%%%%%%%%%%%%%%%%%%%%%%%%%%%%%%%%%%%%%%%%%%%
\subsection{Orbit in Terms of Perihelion Parameters}

From the definition of $r_0$,
$$e=\frac{h^2}{GMr_0}-1$$
At perihelion, $r=r_0$ and $\dot{r}=0$.  The speed at perihelion $v_0$ is
$$v_0=r_0\dot{\theta}_{\rm per}$$
But $h=r^2\dot{\theta}$ so at perihelion,
$$h=r_0^2\dot{\theta}_{\rm per}=r_0v_0$$
Therefore the eccentricity can be written
$$e=\frac{r_0v_0^2}{GM}-1$$
Define $v_c$ as the perihelion speed necessary to give a circular orbit
$$v_c=\sqrt{\frac{GM}{r_0}}$$
so that another way of writing the eccentricity is
$$e=\left(\frac{v_0}{v_c}\right)^2-1$$
Then the orbit equation can be written as
$$r=\frac{r_0\left(v_0/v_c\right)^2}
{1+\left[\left(v_0/v_c\right)^2-1\right]\cos\theta}$$
At aphelion, $r=r_1$ and $\theta=\pi$ so that
$$r_1=\frac{r_0\left(v_0/v_c\right)^2}
{2-\left(v_0/v_c\right)^2}$$

%============================================================================
\begin{example}
\problem
Find the speed of a satellite executing a circular orbit around the Earth.

\solution
Assuming that the Earth's gravitational field dominates over the Sun's field,
the satellite is moving in a central force field with
$$f(r)=-\frac{GM_em}{r^2}$$
where $M_e$ is the mass of the Earth.

The critical speed for a circular orbit at $r=r_0$ is
$$v_c=\sqrt{\frac{GM_e}{r_0}}$$
Note that $g=GM_e/R_e^2$ where $R_e$ is the Earth's radius so this
critical velocity can also be written
$$v_c=\sqrt{\frac{gR_e^2}{r_0}}$$
For a satellite close to the Earth, $r_0\approx R_e$ so
$$v_c\approx\sqrt{gR_e}=7920\ \rm ms^{-1}$$
\end{example}
%============================================================================

%============================================================================
\begin{example}
\problem
Calculate the escape velocity at perihelion from the solar system.

\solution
When $e=1$, the orbit is open and so the orbiting body may escape.  Since
$e=\left(v_0/v_c\right)^2-1$, $e=1$ when $v_0=\sqrt{2}v_c$. 
Therefore the escape velocity is
$$v_c=\sqrt{\frac{2gR_e^2}{r_0}}$$
\end{example}
%============================================================================

%============================================================================
\begin{example}
\problem
A satellite in a circular orbit $r_0$ fires its rocket to increase its speed
suddenly by a factor of $\alpha$.   Calculate the new apogee distance.

\solution
The old speed was
$$v_c=\sqrt{\frac{GM_e}{r_0}}$$
The new speed 
$$v_0=(1+\alpha)v_c$$
is the perihelion speed since the satellite will immediately move to
increase $r$.  The new orbit is
$$r=\frac{r_0\left(v_0/v_c\right)^2}
{1+\left[\left(v_0/v_c\right)^2-1\right]\cos\theta}
=\frac{r_0(1+\alpha)^2}{1+\left[(1+\alpha)^2-1\right]\cos\theta}$$
The apogee distance is
$$r=\frac{r_0(1+\alpha)^2}{2-(1+\alpha)^2}$$

Note that when $\alpha=0.15$, $r_1=1.95 r_0$.  When
$\alpha=\sqrt{2}-1\approx 0.4$, $r_1=\infty$ and the satellite escapes.
\end{example}
%============================================================================

%%%%%%%%%%%%%%%%%%%%%%%%%%%%%%%%%%%%%%%%%%%%%%%%%%%%%%%%%%%%%%%%%%%%%%%%%%%%%
\subsection{Inverse Cubic Attraction}


In general, the equation of orbital motion is
$$\ndbd{\theta}{2}\left(\frac{1}{r}\right)+\frac{1}{r}=-\frac{r^2f(r)}{mh^2}$$
When the attraction follows an inverse cubic law, 
$$f(r)=-\frac{k}{r^3}$$
the orbital motion with inverse cubic attraction is found by solving
$$\ndbd{\theta}{2}\left(\frac{1}{r}\right)+\frac{1}{r}
=\frac{k}{mh^2}\frac{1}{r}$$
which may be written as
$$\ndbd{\theta}{2}\left(\frac{1}{r}\right)
-\left(\frac{k}{mh^2}-1\right)\left(\frac{1}{r}\right)=0$$
When $k/mh^2>1$, the solution is
$$\frac{1}{r}=Ae^{-\alpha\theta}+Be^{\alpha\theta}$$
where 
$$\alpha=\sqrt{\frac{k}{mh^2}-1}$$
and $A$ and $B$ are determined by the initial conditions.
If $B>0$, $1/r\to Be^{\alpha\theta}\to\infty$ and the satellite
follows a path that spirals into the planet.
If $B<0$, $1/r$ will vanish (which means that $r\to\infty$) at some
finite value of $\theta$.
If $B=0$, $1/r=Ae^{-\alpha\theta}$ so $r\to\infty$ and the satellite
escapes from the planet.

When $k/mh^2=1$, the solution is
$$\frac{1}{r}=A+B\theta$$
If $B>0$, $1/r\to\infty$ as $\theta$ increases, so the satellite is
captured in a decaying spiralling orbit.
If $B<0$, $1/r\to 0$ as $\theta$ tends to some finite
$\theta_{\infty}$ and the satellite escapes.  If $B=0$, $1/r=A$ and
the satellite is in a closed circular orbit.

When $k/mh^2<1$, the solution is
$$\frac{1}{r}=A\cos\alpha\theta+B\sin\alpha\theta
=C\cos\alpha(\theta-\theta_0)$$
where $\alpha=\sqrt{1-k/mh^2}<1$, $C=\sqrt{A^2+B^2}$ and
$\tan\alpha\theta_0=B/A$.  Since $1/r\to0$ as
$\theta\to\theta_0+\pi/2\alpha$, the satellite escapes.

%%%%%%%%%%%%%%%%%%%%%%%%%%%%%%%%%%%%%%%%%%%%%%%%%%%%%%%%%%%%%%%%%%%%%%%%%%%%%
\section{Constrained Motion}

%%%%%%%%%%%%%%%%%%%%%%%%%%%%%%%%%%%%%%%%%%%%%%%%%%%%%%%%%%%%%%%%%%%%%%%%%%%%%
\subsection{Frictionless Constrained Motion}

%----------------------------------------------------------------------------
\begin{figure}\centering
\caption{This shows a bead moving on a smooth wire.  There is no friction, 
so the only forces on the bead are gravity $mg$ and the reaction $R$.}
\label{spd fig:bsw}

\psset{unit=5cm}
\begin{pspicture}(-0.1,-0.1)(1.3,0.8)
% First the curve, so the axes show through
\psplot[linecolor=gray,linewidth=2pt,plotstyle=curve,arrows=->]{0.3}{0.45}%
{1 1 x x mul sub sqrt sub}
\uput[ul](0.45,0.11){$s$}
\psplot[linecolor=gray,linewidth=2pt,plotstyle=curve]{0.25}{0.9}%
{1 1 x x mul sub sqrt sub}
% Now the axes and their labels
\psline{->}(0,0)(1.2,0)
\psline{->}(0,0)(0,0.6)
\uput[r](1.2,0){$x$}
\uput[u](0,0.6){$y$}
\psline[linecolor=black,linewidth=1pt,linestyle=dashed]{-}(0.414,0)(0.707,0.293)
\psarc{->}(0.414,0){0.2}{0}{45}
\uput[r](0.6,0.05){$\psi$}
\rput[B]{45}(0.707,0.293){
	\qdisk(0,0){3pt}
	\psline[linecolor=black]{->}(0,0)(0.2,0) 
	\uput[d](0.2,0){\rput[t]{*0}{$\Tvect$}}
	\psline[linecolor=black]{->}(0,0)(0,0.2) 
	\uput[u](0,0.2){\rput[b]{*0}{$\Nvect$}}
	\psline[linecolor=black]{->}(0,0)(0,0.15) 
	\uput[l](0,0.15){\rput[r]{*0}{$R$}}
}
% The weight
\pcline{->}(0.707,0.293)(0.707,0.1)
\Aput{$mg$}
\end{pspicture}
\end{figure}
%----------------------------------------------------------------------------

Suppose a bead of mass $m$ slides smoothly on a wire whose shape is given by
the function $\vect{r}(s)$.  From section \ref{vf sec:2D ic}, the bead's 
velocity and acceleration in intrinsic coordinates are
$$\dot{\vect{r}}=\dot{s}\Tvect$$
$$\ddot{\vect{r}}=\ddot{s}\Tvect+\dot{s}\dot{\psi}\Nvect$$
The intrinsic coordinates are illustrated in figure \ref{spd fig:bsw}.

For a smooth wire, the only force the wire can exert on the bead is a 
\name{reaction} that is \name{normal}, $R\Nvect$ ($R$ can be either
positive or negative).

The only other force is gravity, giving the total force
$$\vect{F}=-mg\jvect+R\Nvect$$
From Newton's second law, the equation of motion is
$$m\ddot{\vect{r}}=-mg\jvect+R\Nvect$$
The tangential component is
$$m\ddot{s}=-mg\Tvect\cdot\jvect$$
and the normal component is
$$m\dot{s}\dot{\psi}=-mg\Nvect\cdot\jvect+R$$
Since $\Tvect\cdot\jvect=\sin\psi$ and $\Nvect\cdot\jvect=\cos\psi$,
$$\ddot{s}=-g\sin\psi$$
and
$$\dot{s}\dot{\psi}=-g\cos\psi+\frac{R}{m}$$

Suppose the curve can be specified in terms of $\psi=\psi(s)$.  Then
$$\dot{\psi}=\der{\psi}{s}\dot{s}$$
where $\der{\psi}{s}$ is a known function of $s$.  Substituting this into
the normal equation of motion gives
$$R=m\left[\dot{s}^2\der{\psi}{s}+g\cos\psi\right]$$
Therefore, once $\dot{s}$ is known as a function of $s$, $R(s)$ is known at
any point $s$ on the wire.  The tangential component can then be written
$$\ddot{s}=\frac{1}{2}\dbd{s}\left(\dot{s}^2\right)=-g\sin\psi$$
Since $\psi(s)$ is specified, integrate the tangential equation with respect
to $s$ to obtain
$$\dot{s}^2=-2g\int_0^s\sin\psi(s)\,ds +\left.\dot{s}^2\right|_{s=0}$$
Once $\dot{s}(s)$ is known, we can integrate again to obtain $s(t)$.

Note that it may be more convenient to solve with $\psi$ rather than with
$s$ as the variable.  For example, the tangential equation can be written
$$\der{\dot{s}^2}{\psi}=\der{(\dot{s}^2)}{s}\der{s}{\psi}=-2g\sin\psi
\der{s}{\psi}$$
where $\der{s}{\psi}$ is a known function of $\psi$.

%----------------------------------------------------------------------------
\begin{figure}\centering
\caption{This shows a bead moving on a straight wire at an angle $\alpha$ to
the horizontal.  The distance $s$ is measured from the origin.}
\label{spd fig:bstw}

\psset{unit=5cm}
\begin{pspicture}(-0.25,-0.25)(1,1.1)
% First the curve, so the axes show through
\psline[linecolor=gray,linewidth=2pt]{-}(-0.1,-0.1)(0.8,0.8)
% Now the axes and their labels
\psline{->}(-0.1,0)(0.9,0)
\psline{->}(0,-0.1)(0,0.9)
\uput[r](0.9,0){$x$}
\uput[u](0,0.9){$y$}
\psarc{->}(0,0){0.2}{0}{45}
\uput[r](0.2,0.1){$\alpha$}
\rput[B]{45}(0.6,0.6){
	\qdisk(0,0){3pt}
	\pcline[linecolor=black]{->}(0,0)(0,0.2) 
	\uput[u](0,0.2){\rput[b]{*0}{$R$}}
}
% The weight
\psline{->}(0.6,0.6)(0.6,0.4)
\uput[d](0.6,0.4){$mg$}
% The label for s
\psline[linecolor=gray,linewidth=2pt]{->}(0.2,0.2)(0.4,0.4)
\uput[ul](0.4,0.4){$s$}
\end{pspicture}
\end{figure}
%----------------------------------------------------------------------------

%============================================================================
\begin{example}
\problem
Consider a wire at an angle $\alpha$ to the horizontal with the bead on the
wire released from rest at a height $H$, as in figure \ref{spd fig:bstw}.  
Calculate the equation of motion of the bead.

\solution
In this case, $\psi=\alpha$ and
$$\der{\dot{s}^2}{s}=-2g\sin\alpha$$
Integrate to show that
$$\dot{s}^2=-2gs\sin\alpha+C$$
The bead is released from a height $H$ so when $t=0$
$$s=\frac{H}{\sin\alpha}$$
and
$$\dot{s}=0$$
Therefore $C=2gH$ and
$$\dot{s}^2=2g\left(H-s\sin\alpha\right)$$
and the normal equation is
$$R=m\left(\dot{s}^2\der{\psi}{s}+g\cos\psi\right)=mg\cos\alpha$$
because $\psi$ is independent of $s$.

Taking the square root of the expression for $\dot{s}^2$ gives
$$\dot{s}=-\sqrt{2g(H-s\sin\alpha)}$$
where the negative root has been selected because the motion is in the
direction of $s$ decreasing.  This is a separable equation
$$\frac{ds}{\sqrt{H-s\sin\alpha}}=-\sqrt{2g}\,dt$$
Integrating gives
$$\frac{-2\sqrt{H-s\sin\alpha}}{\sin\alpha}=-\sqrt{2g}t+C$$
At $t=0$, $s=H/\sin\alpha$ so $C=0$ and
$$H-s\sin\alpha=\frac{gt^2}{2}\sin^2\alpha$$
which can be rearranged to give
$$s=\frac{H}{\sin\alpha}-\frac{1}{2}gt^2\sin\alpha$$
\end{example}
%============================================================================

%----------------------------------------------------------------------------
\begin{figure}\centering
\caption{For the particle sliding down the sphere in problem
\protect\ref{spd exam:psds}, place the top of the sphere at the origin.}
\label{spd fig:psds}

\psset{unit=5cm}
\begin{pspicture}(-0.25,-1.1)(1.2,0.35)
% First the curve, so the axes show through
\psplot[linecolor=gray,linewidth=2pt,plotstyle=curve]{0}{1}{1 x x mul sub 
sqrt 1 sub}
\psplot[linecolor=gray,linewidth=2pt,plotstyle=curve,arrows=->]{0}{0.25}{1 x x mul sub 
sqrt 1 sub}
% Now the axes and their labels
\psline{->}(-0.1,0)(1,0)
\psline{->}(0,-1)(0,0.1)
\uput[r](1,0){$x$}
\uput[u](0,0.1){$y$}
% Psi
\psline[linestyle=dashed]{-}(0.414,0)(0.707,-0.293)
\psarc{->}(0.414,0){0.08}{0}{315}
\uput[u](0.414,0.08){$\psi$}
% Phi
\pcline[linestyle=dashed]{-}(0,-1)(0.707,-0.293)
\Aput{$a$}
\psarcn{->}(0,-1){0.2}{90}{45}
\uput[u](0.07,-0.8){$\phi$}
% Forces at the particle
\rput[B]{-45}(0.707,-0.293){
	\qdisk(0,0){3pt}
	\psline[linecolor=black]{->}(0,0)(0,0.2) 
	\rput[b]{*0}(0,0.21){$R$}
	\psline[linecolor=black]{->}(0,0)(0.2,0) 
	\rput[l]{*0}(0.21,0){$\Tvect$}
}
% The weight
\psline{->}(0.707,-0.293)(0.707,-0.45)
\uput[d](0.707,-0.45){$mg$}
% The label for s
\uput[d](0.25,-0.05){$s$}
\end{pspicture}
\end{figure}
%----------------------------------------------------------------------------

%============================================================================
\begin{example}
\label{spd exam:psds}

\problem 
A particle slides down the surface of a sphere of radius $a$.  Initially,
the particle is at the top of the sphere and is moving so slowly that it is
nearly at rest.  When does the particle leave the sphere?

\solution
This is illustrated in figure \ref{spd fig:psds}, from which it can be 
seen that
$$\psi=2\pi-\phi$$
The tangential equation of motion is 
$$\dbd{s}(\dot{s}^2)=-2g\sin\psi=2g\sin\phi$$
The normal equation of motion is
$$\dot{s}\dot{\psi}=-g\cos\psi+\frac{R}{m}$$
which can be written in terms of $\phi$ as
$$\dot{s}\dot{\phi}=g\cos\phi-\frac{R}{m}$$

The equation of the surface is
$$s=a\phi$$
where $s$ is measured from the top.  This shows that
$$\dot{\phi}=\frac{\dot{s}}{a}$$
and 
$$R=m\left(g\cos\phi-\frac{\dot{s}^2}{a}\right)$$
The tangential equation becomes
$$\dbd{s}(\dot{s}^2)=2g\sin\phi=2g\sin\frac{s}{a}$$
Integrating gives
$$\dot{s}^2=-2ag\cos\frac{s}{a}+C$$
But when $t=0$, $s=0$ and $\dot{s}=0$ so that $C=2ag$.  Therefore
$$\dot{s}^2=2ag\left(1-\cos\frac{s}{a}\right)=2ag\left(1-\cos\phi\right)$$
Substituting this into the equation for the reaction $R$ shows that
$$R=m\left(g\cos\phi-\frac{2ag(1-\cos\phi)}{a}\right)=mg(3\cos\phi-2)$$

From this, the reaction is zero when 
$$R=mg(3\cos\phi-2)=0$$
which occurs when
$$\cos\phi=\frac{2}{3}$$
which is point at which the particle leaves the sphere.

Note that in problems like this, there is nothing to sustain negative
reactions, unlike the bead on the wire problem.
\end{example}
%============================================================================

%----------------------------------------------------------------------------
\begin{figure}\centering
\caption{This diagram shows a bead on a parabolic wire with equation
$y=-x^2/2L$.  It has been drawn upside down so that gravity points upwards.}
\label{spd fig:bopw}

\psset{unit=5cm}
\begin{pspicture}(-0.1,-0.1)(1.3,1.3)
% Plot the curve twice, one with an arrow so the direction of s is shown
\psplot[linecolor=gray,linewidth=2pt,plotstyle=curve,arrows=->]{0}{0.4}{x x mul}
\psplot[linecolor=gray,linewidth=2pt,plotstyle=curve]{0}{1}{x x mul}
\uput[ul](0.4,0.16){$s$}
% Now the axes and their labels
\psline{->}(0,0)(1.1,0)
\psline{->}(0,0)(0,1.1)
\uput[r](1.1,0){$x$}
\uput[u](0,1.1){$y$}
% The vectors at the point (x,y)
\psline[linecolor=black,linewidth=1pt,linestyle=dashed]{-}(0.3536,0)(0.707,0.5)
\psarc{->}(0.3536,0){0.2}{0}{54.7}
\uput[r](0.53,0.05){$\psi$}
\rput[B]{54.7}(0.707,0.5){
	\qdisk(0,0){3pt}
	\psline[linecolor=black]{->}(0,0)(0.2,0) 
	\uput[d](0.2,0){\rput[t]{*0}{$\Tvect$}}
	\psline[linecolor=black]{->}(0,0)(0,0.2) 
	\uput[u](0,0.2){\rput[b]{*0}{$\Nvect$}}
	\psline[linecolor=black]{->}(0,0)(0,0.15) 
	\uput[l](0,0.15){\rput[r]{*0}{$R$}}
}
% The weight
\psline{->}(0.707,0.5)(0.707,0.7)
\uput[u](0.707,0.7){$mg$}
\uput[r](0.9,0.81){$y=\ds\frac{x^2}{2L}$}
\end{pspicture}
\end{figure}
%----------------------------------------------------------------------------

%============================================================================
\begin{example}
\label{spd ex:bopw}
\problem
Find the motion of a bead under gravity starting with speed $u$ at the
origin, threaded on a frictionless wire that has been bent into a
parabolic shape
$$y=-\frac{x^2}{2L}$$

\solution
This is illustrated upside down (so gravity points upwards) in figure
\ref{spd fig:bopw}.  From this, the tangential equation of motion is
$$m\ddot{s}=mg\sin\psi$$
and the normal equation is
$$m\dot{s}\der{\psi}{s}=mg\cos\psi+R$$

To find the equation of the  wire in intrinsic coordinates, use
$$\der{s}{x}=\sqrt{1+\left(\der{y}{x}\right)^2}$$
where the positive square root has been chosen as $s$ increases with $x$.

Since $y=x^2/2L$ in the inverted diagram, 
$$\der{y}{x}=\frac{x}{L}=\tan\psi$$
Therefore
$$\der{x}{\psi}=\frac{L}{\cos^2\psi}$$
and
$$\der{s}{\psi}=\der{s}{x}\der{x}{\psi}=\sqrt{1+tan^2\psi}
\frac{L}{\cos^2\psi}=\frac{L}{\cos^3\psi}$$
The tangential equation of motion can be written as
$$\der{\dot{s}^2}{\psi}=\der{\dot{s}^2}{s}\der{s}{\psi}=
2\ddot{s}\der{s}{\psi}=2Lg\frac{\sin\psi}{\cos^3\psi}$$
Integrating gives
$$\dot{s}^2=2gL\int^{\psi}_0\frac{sin\psi}{\cos^3\psi}\,d\psi
+\left.\dot{s}^2\right|_{\psi=0}
=\left.\frac{gL}{\cos^2\psi}\right|_0^{\psi}+u^2$$
Therefore
$$\dot{s}^2=gL\left[\frac{1}{\cos^2\psi}-1\right]+u^2$$
Now 
$$\frac{1}{\cos^2\psi}-1=\tan^2\psi=\frac{x^2}{L^2}$$
so
$$\dot{s}^2=\frac{g}{L}x^2+u^2$$

From the normal equation of motion, 
\begin{eqnarray*}
R&=&m\dot{s}^2\der{\psi}{s}-mg\cos\psi \\
&=&m\left[gL\left(\frac{1}{\cos^2\psi}-1\right)+u^2\right]
\frac{\cos^3\psi}{L}-mg\cos\psi \\
&=&m(u^2-gL)\cos^3\psi
\end{eqnarray*}
Therefore the reaction is
$$R=\frac{m(u^2-gL)}{\left(1+x^2/L^2\right)^{3/2}}$$
Note that the reaction is positive for $u>\sqrt{gL}$ and negative for
$u<\sqrt{gL}$.
\end{example}
%============================================================================

%%%%%%%%%%%%%%%%%%%%%%%%%%%%%%%%%%%%%%%%%%%%%%%%%%%%%%%%%%%%%%%%%%%%%%%%%%%%%
\subsection{Friction}

%----------------------------------------------------------------------------
\begin{figure}\centering
\caption{The three forces on a stationary mass on a rough surface are the
reaction $R$, an applied force $f_{\rm app}$ and friction $F$.  The 
direction of the frictional force $F$ opposes the motion that
would occur if friction were absent.}
\label{spd fig:fr I}

\psset{unit=5cm}
\begin{pspicture}(-0.3,-0.1)(1.3,0.9)
% First the ground
\psline[linecolor=black,linewidth=2pt]{-}(0,0.03)(1,0.03)
% The box
\pspolygon[fillstyle=solid,linewidth=2pt,fillcolor=lightgray]%
(0.3,0.05)(0.7,0.05)(0.7,0.45)(0.3,0.45)
% The reaction
\psline{->}(0.5,0.07)(0.5,0.65)\uput[u](0.5,0.65){$R$}
% Friction
\psline{->}(0.3,0.05)(-0.1,0.05)\uput[l](-0.1,0.05){$F$}
% Applied force
\psline{<-}(0.3,0.35)(-0.1,0.35)\uput[l](-0.1,0.35){$f_{\rm app}$}
\end{pspicture}
\end{figure}
%----------------------------------------------------------------------------

\name{Friction} is a force that acts to oppose the relative motion of rough
surfaces.  As shown in figure \ref{spd fig:fr I}, $f_{\rm app}$ is the force
applied to the body, $R$ is the normal reaction force of the surface on the
body and $F$ is the frictional force on the body acting along the surface in
a direction which opposes the motion that would occur if friction were
absent.

\name{Static friction} occurs provided there is no motion, so that
$$F=f_{\rm app}$$
As $f_{\rm app}$ is increased, $F$ increases.  Experiments show that there
is a critical $f_{\rm app}$ at which the body starts to move, which is
proportional to the reaction $R$.  The friction force at this point can be
written
$$F=\mu_sR$$
where $\mu_s$ is the coefficient of limiting static friction.

%----------------------------------------------------------------------------
\begin{figure}\centering
\caption{The three forces on a stationary mass on a rough plane inclined at
an angle $\alpha$ are the reaction $R$, friction $F$ and weight $mg$.}
\label{spd fig:moi}

\psset{unit=5cm}
\begin{pspicture}(-0.3,-0.1)(1.3,0.8)
% The box
\rput[B]{30}(0.45,0.3773){
	\pspolygon[fillstyle=solid,fillcolor=lightgray]%
	(-0.15,-0.1)(0.15,-0.1)(0.15,0.1)(-0.15,0.1)
	\qdisk(0,0){3pt}
	\psline[linecolor=black]{->}(0,0)(0.3,0) 
	\uput[r](0.3,0){\rput[l]{*0}{$F$}}
	\psline[linecolor=black]{->}(0,0)(0,0.3) 
	\uput[u](0,0.3){\rput[b]{*0}{$R$}}
}
% First the ground
\SpecialCoor
\psline[linecolor=black,linewidth=2pt]{-}(1.1;30)(0;30)(1;0)
% The angle alpha
\psarc{->}(0,0){0.2}{0}{30}
\uput[r](0.2,0.05){$\alpha$}
% The weight vector
\psline{->}(0.45,0.3773)(0.45,0.1773)
\uput[d](0.45,0.1773){$mg$}
\end{pspicture}
\end{figure}
%----------------------------------------------------------------------------

%============================================================================
\begin{example}
For a stationary mass on an inclined plane, such as in figure 
\ref{spd fig:moi}, the forces must add to zero.  The tangential force
component is
$$F-mg\sin\alpha=0$$
and the normal force component is
$$R-mg\cos\alpha=0$$
At a critical angle $\alpha_c$ when the mass is just about to slip, we know
that
$$F=\mu_s R$$
Therefore, eliminating $R$ from the force equations at this point,
$$\mu_s=\tan\alpha_c$$

So if $\alpha<\alpha_c$, $F<\mu_sR$ and there is no motion.  If $\alpha\geq
\alpha_c$, the mass slides down the incline.
\end{example}
%============================================================================

Once motion has started, the frictional force still acts to oppose the
motion.  Experimentally, it is observed that
$$F=\mu_k R$$
for low speeds, where $\mu_k$ is the coefficient of \name{kinetic friction}.

Note that $\mu_k\leq\mu_s$.  Friction in problems involving rolling is
slightly different; this is dealt with later.

The kinetic friction acts to oppose the motion, so it can be described by
$$\vect{F}=-\mu_k\left|R\right|\hat{\dot{\vect{r}}}$$
where $\hat{\dot{\vect{r}}}$ is the unit vector in the direction of motion.


%----------------------------------------------------------------------------
\begin{figure}\centering
\caption{This shows one quarter of a rotating floor-polisher, which is
rotating with angular speed $\omega$ and moving with velocity $v$ in the 
$y$ direction.  The elemental unit of area $dA$ extends from $r$ to $r+dr$ 
and from $\theta$ to $\theta+d\theta$.}
\label{spd fig:fp}

\psset{unit=5cm}
\begin{pspicture}(-0.1,-0.1)(1.3,1.3)
\SpecialCoor
% The circular brush
\psarc[linecolor=gray,linewidth=2pt,plotstyle=curve]{-}(0,0){1}{-5}{95}
\uput[dr](1,0){$a$}
% Now the axes and their labels
\psline{->}(-0.1,0)(1.1,0)
\psline{->}(0,-0.1)(0,1.1)
\uput[r](1.1,0){$x$}
\uput[u](0,1.1){$y$}
% Dashed lines giving the elemental area
\psarc[linecolor=black,linewidth=1pt,linestyle=dashed]{-}(0,0){0.45}{0}{90}
\psarc[linecolor=black,linewidth=1pt,linestyle=dashed]{-}(0,0){0.55}{0}{90}
\psline[linecolor=black,linewidth=1pt,linestyle=dashed]{-}(0,0)(0.55;40)
\psline[linecolor=black,linewidth=1pt,linestyle=dashed]{-}(0,0)(0.55;50)
\uput[dr](1,0){$a$}
\uput[dr](0.55,0){$r+dr$}
\uput[dl](0.45,0){$\vphantom{d}r$}
\psarc{->}(0,0){0.2}{0}{40}
\uput[r](0.18,0.05){$\theta$}
\psarc{<-}(0,0){0.2}{50}{70}
\uput[ur](0.2;45){$d\theta$}
% The elemental area
\psarc[linecolor=black,linewidth=2pt]{-}(0,0){0.45}{40}{50}
\psarc[linecolor=black,linewidth=2pt]{-}(0,0){0.55}{40}{50}
\psline[linecolor=black,linewidth=2pt]{-}(0.45;40)(0.55;40)
\psline[linecolor=black,linewidth=2pt]{-}(0.45;50)(0.55;50)
% The vectors from the elemental area
\psarc{->}(0,0){0.5}{45}{65}
\psline{->}(0.3536,0.3536)(0.3536,0.55)
\uput[ul](0.5;65){$\omega r$}
\uput[u](0.3536,0.55){$v$}
\end{pspicture}
\end{figure}
%----------------------------------------------------------------------------

%============================================================================
\begin{example}
\problem
A floor polisher of weight $w$ has a circular brush of radius $a$ which is
rotating with angular speed $\omega$ and slides over a floor with speed $v$. 
Calculate the frictional force, where $\mu$ is the coefficient of sliding
friction.  

\solution
Consider the element of the brush between $r$ to $r+dr$ and $\theta$ to
$\theta+d\theta$, as shown in figure \ref{spd fig:fp}.  The area of this 
element is 
$$dA=r\,dr\,d\theta$$
Assume that the weight is distributed uniformly over all area elements, so
that the reaction force on this element is
$$R_e=\frac{w}{\pi a^2}r\,dr\,d\theta$$
The velocity of the element is
$$\dot{\vect{r}}=v\jvect+\omega r (-\sin\theta\ivect+\cos\theta\jvect)$$
and the unit vector in this direction is
$$\hat{\dot{\vect{r}}}=\frac{v\jvect+\omega r 
(-\sin\theta\ivect+\cos\theta\jvect)}
{\sqrt{v^2+2\omega rv\cos\theta+\omega^2r^2}}$$
The frictional force on the element is
$$-\frac{\mu w r\,dr\,d\theta}{\pi a^2}\hat{\dot{\vect{r}}}$$
The frictional force on the whole annulus from $r$ to $r+dr$ is found by
integrating the frictional force on the element for $\theta$ from $0$ to
$2\pi$
$$-\int_0^{2\pi}\hat{\dot{\vect{r}}}\,d\theta\,\frac{\mu wr\,dr}{\pi a^2}$$

This is difficult to solve exactly, so suppose that $\omega r\gg v$.  Then
$$\frac{1}{\sqrt{v^2+2\omega rv\cos\theta+\omega^2r^2}}\approx
\frac{1}{\omega r}\left[1-\frac{v\cos\theta}{\omega r}\right]$$
So the frictional force on the annulus is
$$-\jvect\frac{\mu wr\,dr}{a^2}\frac{v}{\omega r}$$
and the frictional force on the whole brush is
$$-\jvect\frac{\mu wv}{a\omega}$$

If the brush is not rotating, $\omega=0$, and the frictional force must be
$$\vect{F}_0=-\jvect\mu w$$
Therefore, when the brush is rotating, the friction can be written
$$\left|\vect{F}\right|=\left|\vect{F}_0\right|\frac{v}{aw}$$
which tends to zero as $\omega\to\infty$ and the polisher becomes
essentially frictionless as the rotational speed is increased.
\end{example}
%============================================================================

%%%%%%%%%%%%%%%%%%%%%%%%%%%%%%%%%%%%%%%%%%%%%%%%%%%%%%%%%%%%%%%%%%%%%%%%%%%%%
\subsection{Constrained Motion with Friction}

The equation of motion in intrinsic coordinates with friction is
$$m(\ddot{s}\Tvect+\dot{s}\dot{\psi}\Nvect)=m\vect{g}+R\Nvect-F\Tvect$$
where the friction term is
$$F=\mu\left|R\right|\frac{\dot{s}}{\left|\dot{s}\right|}$$

%----------------------------------------------------------------------------
\begin{figure}\centering
\caption{This shows a bead on a rough parabolic wire, so friction $F$
opposes the direction of motion $s$.  As in figure \protect\ref{spd
fig:bopw}, the diagram has been drawn upside down so that gravity points 
upwards.}
\label{spd fig:bopwf}

\psset{unit=5cm}
\begin{pspicture}(-0.1,-0.1)(1.3,1.3)
% Plot the curve twice, one with an arrow so the direction of s is shown
\psplot[linecolor=gray,linewidth=2pt,plotstyle=curve,arrows=->]{0}{0.4}{x x mul}
\psplot[linecolor=gray,linewidth=2pt,plotstyle=curve]{0}{1}{x x mul}
\uput[ul](0.4,0.16){$s$}
% Now the axes and their labels
\psline{->}(0,0)(1.1,0)
\psline{->}(0,0)(0,1.1)
\uput[r](1.1,0){$x$}
\uput[u](0,1.1){$y$}
% The vectors at the point (x,y)
\psline[linecolor=black,linewidth=1pt,linestyle=dashed]{-}(0.3536,0)(0.707,0.5)
\psarc{->}(0.3536,0){0.2}{0}{54.7}
\uput[r](0.53,0.1){$\psi$}
\rput[B]{54.7}(0.707,0.5){
	\qdisk(0,0){3pt}
	\psline[linecolor=black]{->}(0,0)(0.2,0) 
	\uput[d](0.2,0){\rput[t]{*0}{$\Tvect$}}
	\psline[linecolor=black]{->}(0,0)(-0.2,0) 
	\uput[d](-0.2,0){\rput[t]{*0}{$F$}}
	\psline[linecolor=black]{->}(0,0)(0,0.2) 
	\uput[u](0,0.2){\rput[b]{*0}{$\Nvect$}}
	\psline[linecolor=black]{->}(0,0)(0,0.15) 
	\uput[l](0,0.15){\rput[r]{*0}{$R$}}
}
% The weight
\psline{->}(0.707,0.5)(0.707,0.7)
\uput[u](0.707,0.7){$mg$}
\uput[r](0.9,0.81){$y=\ds\frac{x^2}{2L}$}
\end{pspicture}
\end{figure}
%----------------------------------------------------------------------------

%============================================================================
\begin{example}
\problem
Consider the bead on the parabolic wire from example \ref{spd ex:bopw} where
the wire is no longer smooth, so friction must be included.  This is
illustrated in figure \ref{spd fig:bopwf}.  Calculate the bead's equation of
motion.

\solution
Recall from example \ref{spd ex:bopw} that
$$\der{s}{\psi}=\frac{L}{\cos^3\psi}$$
and 
$$1+\left(\frac{x}{L}\right)^2=\frac{1}{\cos^2\psi}$$

Then tangential equation of motion is
$$m\ddot{s}=mg\sin\psi-F$$
and the normal equation of motion is
$$m\dot{s}\dot{\psi}=mg\cos\psi+R$$

Assume that the initial speed is sufficiently small that $R<0$ at the start. 
We can then write
$$F=-\mu R=-\mu(m\dot{s}\dot{\psi}-mg\cos\psi)$$
Therefore
$$\frac{1}{2}\dbd{s}(\dot{s}^2)=g\sin\psi+\mu(\dot{s}\dot{\psi}-g\cos\psi)$$
so that
$$\dbd{s}(\dot{s}^2)-2\mu\der{\psi}{s}(\dot{s}^2)=2g(\sin\psi-\mu\cos\psi)$$
This is more conveniently expressed with respect to $\psi$ rather than $s$. 
So multiply by $\der{s}{\psi}$ to give
$$\dbd{\psi}(\dot{s}^2)-2\mu(\dot{s}^2)=2g\der{s}{\psi}
(\sin\psi-\mu\cos\psi)$$
The integrating factor is
$$I(\psi)=e^{-2\mu\psi}$$ 
so that
$$\dbd{\psi}\left(e^{-2\mu\psi}\dot{s}^2\right)=
\frac{2gLe^{-2\mu\psi}}{\cos^3\psi}(\sin\psi-\mu\cos\psi)$$
which gives
$$\dot{s}^2e^{-2\mu\psi}=u^2+2gL\int_0^{\psi}
\frac{e^{-2\mu\psi}}{\cos^3\psi}(\sin\psi-\mu\cos\psi)\,d\psi$$
Therefore $\dot{s}^2$ is 
$$\dot{s}^2=u^2e^{2\mu\psi}+2gLe^{2\mu\psi}\int_0^{\psi}
\frac{e^{-2\mu\psi}}{\cos^3\psi}(\sin\psi-\mu\cos\psi)\,d\psi$$
which is substantially harder to solve than when there was no friction.
\end{example}
%============================================================================

%----------------------------------------------------------------------------
\begin{figure}\centering
\caption{A light bead on a rough wire which is rotating in a horizontal
plane.}
\label{spd fig:lbrrw}

\psset{unit=5cm}
\begin{pspicture}(-0.25,-0.25)(1,1.1)
% First the curve, so the axes show through
\psline[linecolor=gray,linewidth=2pt]{-}(-0.1,-0.1)(0.8,0.8)
% Now the axes and their labels
\psline{->}(-0.1,0)(0.9,0)
\psline{->}(0,-0.1)(0,0.9)
\uput[r](0.9,0){$x$}
\uput[u](0,0.9){$y$}
\psarc{->}(0,0){0.2}{0}{45}
\uput[r](0.2,0.1){$\theta$}
\rput[B]{45}(0.6,0.6){
	\qdisk(0,0){3pt}
	\psline[linecolor=black]{->}(0,0)(0.2,0) 
	\uput[d](0.2,0){\rput[t]{*0}{$\rvect$}}
	\psline[linecolor=black]{->}(0,0)(-0.2,0) 
	\uput[d](-0.2,0){\rput[t]{*0}{$F$}}
	\psline[linecolor=black]{->}(0,0)(0,0.2) 
	\uput[u](0,0.2){\rput[b]{*0}{$\tvect$}}
	\psline[linecolor=black]{->}(0,0)(0,0.15) 
	\uput[l](0,0.15){\rput[r]{*0}{$R$}}
}
\psarc{->}(0,0){1.2}{35}{55}
\uput[r](0.9,0.9){$\dot{\theta}=\omega$}
\end{pspicture}
\end{figure}
%----------------------------------------------------------------------------

%============================================================================
\begin{example}
\problem
Consider a light bead on a rough wire that is rotating horizontally, as
shown in figure \ref{spd fig:lbrrw}.  Since the bead is light, its weight 
can be ignored.  Calculate the bead's position on the wire, $r(t)$.

\solution
The equations of motion in polar coordinates are
$$m(\ddot{r}-r\dot{\theta}^2)=-F$$
and
$$m(r\ddot{\theta}+2\dot{r}\dot{\theta})=R$$
Since the wire is rotating with angular speed $\omega$ and the bead is
constained to the wire, 
$$\dot{\theta}=\omega$$
and
$$\ddot{\theta}=0$$
Thus
$$m(\ddot{r}-r\omega^2)=-F$$
and
$$m(2\dot{r}\omega)=R$$

To find the critical rotation rate for the bead to move, at the point of
incipient motion
$$F=\mu_sR$$ 
with $\dot{r}=\ddot{r}=0$.  Then from the equations of motion, initially
$R=0$ and therefore $F=0$ initially.  

For any $\omega$, the initial acceleration is
$$\ddot{r}=r\omega^2$$
Friction only starts to act once $\dot{r}$, and hence $R$, becomes non-zero.
Therefore
$$\omega_{\rm crit}=0$$
although this has to be modified if the weight of the bead is included.


When $\omega\neq 0$ and the bead is sliding, the transverse equation shows 
that
$$F=\mu_kR=2m\omega\mu_k\dot{r}$$
The radial equation is
$$\ddot{r}-r\omega^2=-\frac{F}{m}=-2\omega\mu_k\dot{r}$$
which is
$$\ddot{r}+2\omega\mu_k\dot{r}-\omega^2r=0$$
Try $r=e^{\alpha t}$ in the differential equation to get
$$\alpha^2+2\omega\mu_k\alpha-\omega^2=0$$
which can be expressed as
$$\left(\frac{\alpha}{\omega}\right)^2+2\mu_k\left(\frac{\alpha}{\omega}
\right)-1=0$$
Thus
$$\alpha=\omega\left(\mu_k\pm\sqrt{\mu_k^2+1}\right)$$
This gives the two solutions
$$\alpha_1=-\omega\left(\mu_k+\sqrt{\mu_k^2+1}\right)<0$$
and
$$\alpha_2=\omega\left(\sqrt{\mu_k^2+1}-\mu_k\right)>0$$
so 
$$r(t)=Ae^{\alpha_1t}+Be^{\alpha_2t}$$
At $t=0$, $r=r_0$ and $\dot{r}=0$ so that
$$r_0=A+B\qquad\mbox{and}\qquad 0=\alpha_1A+\alpha_2B$$
Therefore
$$A=-\frac{\alpha_2}{\alpha_1}\left(\frac{r_0}{1-\alpha_2/\alpha_1}
\right)>0$$
and
$$B=\frac{r_0}{1-\alpha_2/\alpha_1}>0$$
This gives the solution
$$r(t)=\frac{r_0}{1-\alpha_2/\alpha_1}\left(
-\frac{\alpha_2}{\alpha_1}e^{\alpha_1t}+e^{\alpha_2t}
\right)$$
\end{example}
%============================================================================

	
%%%%%%%%%%%%%%%%%%%%%%%%%%%%%%%%%%%%%%%%%%%%%%%%%%%%%%%%%%%%%%%%%%%%%%%%%%%
%
%			Mathematics 132 Course Notes
%
%			 Department of Mathematics,
%   			  University of Melbourne
%
%		Stephen Simmons			Lee White
%
% 8 Feb-96 SS: Updated with corrections from semester 2, 1995
%
%%%%%%%%%%%%%%%%%%% Copyright (C) 1995-96 Stephen Simmons %%%%%%%%%%%%%%%%%

\chapter{Dynamics of Systems of Particles}
\label{mpd chp}

%%%%%%%%%%%%%%%%%%%%%%%%%%%%%%%%%%%%%%%%%%%%%%%%%%%%%%%%%%%%%%%%%%%%%%%%%%%%%
\section{Centre of Mass and Momentum}

Consider a system of particles where $\vect{r}_i$ is the position vector of
the $i^{\rm th}$ particle whose mass is $m_i$.  Then the position vector of
the \name{centre of mass} of the system of particles is
$$\vect{r}_{\rm cm}=\frac{\ds\sum_i m_i\vect{r}_i}{\ds\sum_i m_i}$$
The total mass of the system of particles is
$$M=\sum_i m_i$$

%============================================================================
\begin{example}
The centre of mass of a two particle system is
$$\vect{r}_{\rm cm}=\frac{m_1\vect{r}_1+m_2\vect{r}_2}{m_1+m_2}$$
If, for example, the two particles are the Sun and a planet,
$$\vect{r}_{\rm cm}=\frac{M_s\vect{r}_s+m_p\vect{r}_p}{M_s+m_p}$$
\end{example}
%============================================================================

The linear momentum of the $i^{\rm th}$ particle is
$$\vect{p}_i=m_i\dot{\vect{r}}_i$$
so the total linear momentum of the system of particles is
$$\vect{P}=\sum_i\vect{p}_i=\sum_im_i\dot{\vect{r}}_i$$
If the total mass of the system is constant with time, this can be written 
as the total mass times the velocity of the centre of mass
$$\vect{P}=\dbd{t}\sum_im_i\vect{r}_i=M\dbd{t}\left(\frac{1}{M}
\sum_im_i\vect{r}_i\right)=M\dot{\vect{r}}_{\rm cm}$$

Newton's second law applies to the system as a whole.  Newton's second law
for an individual particle is
$$m_i\ddot{\vect{r}}_i=\vect{F}_i+\sum_{j\neq i}\vect{F}_{ij}$$
where $\vect{F}_i$ is the total external force on the $i^{\rm th}$ particle
and $\vect{F}_{ij}$ is the force exerted on the $i^{\rm th}$ particle by the
$j^{\rm th}$ particle (note that by Newton's third law,
$\vect{F}_{ij}=-\vect{F}_{ji}$).

%============================================================================
\begin{theorem}
Newton's second law for the system of particles is
$$M\ddot{\vect{r}}_{\rm cm}=\sum_i\vect{F}_i$$
\end{theorem}
\begin{proof}
Start with
$$M\vect{r}_{\rm cm}=\sum_i m_i\vect{r}_i$$
and differentiate twice with respect to time to give
$$M\ddot{\vect{r}}_{\rm cm}=\sum_i m_i\ddot{\vect{r}}_i$$
Then from Newton's second law for the $i^{\rm th}$ particle,
$$M\ddot{\vect{r}}_{\rm cm}
=\sum_i\left(\vect{F}_i+\sum_{j\neq i}\vect{F}_{ij}\right)
=\sum_i\vect{F}_i+\sum_i\sum_{j\neq i}\vect{F}_{ij}$$
The second summation is zero because $\vect{F}_{ij}=-\vect{F}_{ji}$, leaving
$$M\ddot{\vect{r}}_{\rm cm}=\sum_i\vect{F}_i$$
as required.
\end{proof}
%============================================================================

Note that when no net external force acts on the system
\typeout{LRW says this should be 'no nett external force', but I think
'net' is correct.}
$$\sum_i\vect{F}_i=0$$
and
$$M\ddot{\vect{r}}_{\rm cm}=0$$
so that $\vect{P}=M\dot{\vect{r}}_{\rm cm}$ is constant.  Therefore a system
of particles under no net external force has constant linear momentum
$\vect{P}$.

%%%%%%%%%%%%%%%%%%%%%%%%%%%%%%%%%%%%%%%%%%%%%%%%%%%%%%%%%%%%%%%%%%%%%%%%%%%
\section{Kinetic and Potential Energy}

The \name{kinetic energy} of a system is the sum of the kinetic energies of
the individual particles
$$KE=\sum_i\frac{1}{2}m_i\dot{\vect{r}}_i\cdot\dot{\vect{r}}_i$$
The \name{potential energy} of external conservative forces is
$$PE_{\rm ext}=\sum_iV_{\rm ext}(\vect{r}_i)$$
where the potential fuction $V_{\rm ext}(\vect{r})$ is defined in terms of
the external force by
$$-\grad V_{\rm ext}(\vect{r})=F(\vect{r})$$

Internal forces are usually central forces between each pair of particles so
that there is associated with each pair of particles an internal potential
energy $V_{\rm int}(r_{ij})$ which is a function of separation distance
$r_{ij}=\left|\vect{r}_i-\vect{r}_j\right|$ between the pair.  The internal
potential energy is
$$PE_{\rm int}=\frac{1}{2}\sum_i\sum_j V_{\rm int}(r_{ij})$$
Here the summation counts each pair of particles twice, hence the factor of
$1/2$.

The \name{total mechanical energy} of the system is
\begin{eqnarray*}
E&=&KE+PE_{\rm ext}+PE_{\rm int}\\
&=&\sum_i\frac{1}{2}m_i\dot{\vect{r}}_i\cdot\dot{\vect{r}}_i+
\sum_iV_{\rm ext}(\vect{r}_i)+\frac{1}{2}\sum_i\sum_j V_{\rm int}(r_{ij})
\end{eqnarray*}

%============================================================================
\begin{theorem}
In the absence of external extraneous forces
$$\der{E}{t}=0$$
so $E$ is a constant of the motion of the system and mechanical energy is
conserved.
\end{theorem}
%============================================================================

%============================================================================
\begin{example}[Reduced Mass of a Two Body System]
When no external force acts, $\dot{\vect{r}}_{\rm cm}$ is constant.  A
coordinate system with the centre of mass at the origin would therefore be an
inertial frame of reference.

If $\vect{r}_1$ and $\vect{r}_2$ are the positions of two
particles relative to this coordinate system, then
$$m_1\vect{r}_1+m_2\vect{r}_2=0$$
Define the position vector of particle 1 with respect to particle 2 as
$$\vect{r}=\vect{r}_1-\vect{r}_2
=\vect{r}_1\left(1+\frac{m_1}{m_2}\right)$$
Newton's second law holds in inertial frames of reference, so that
$$m_1\ddot{\vect{r}}_1=\vect{F}_{12}$$
If the force that particle 2 exerts on particle 1 is a function of the
distance between them and directed along the line of their centres
$$\vect{F}_{12}=f(r)\rvect$$
then
$$m_1\ddot{\vect{r}}_1=f(r)\rvect$$
Substituting
$$\ddot{\vect{r}}_1=\frac{m_2}{m_1+m_2}\vect{r}$$
we have
$$\mu\ddot{\vect{r}}=f(r)\rvect$$
where $\mu$ is the \name{reduced mass} of the system
$$\mu=\frac{m_1m_2}{m_1+m_2}$$
This is equivalent to motion in a central force field of a particle of mass
$\mu$.
\end{example}
%============================================================================

%============================================================================
\begin{example}[Sun-Planet System]
It was incorrect to assume that the Sun stays fixed at the origin as a
planet orbits it.  Instead we should use the centre of mass of the
Sun/planet system as the origin.  In this case, the position vector of the
planet relative to the Sun is
$$\vect{r}=\vect{r}_p-\vect{r}_s$$
and therefore
$$\mu\ddot{\vect{r}}=-\frac{GM_sm_p}{r^2}\rvect$$
where
$$\mu=\frac{M_sm_p}{M_s+m_p}\approx m_p$$
when $M_s\gg m_p$.

Our previous calculations are all valid if we replace $GM_s$ by
$G(M_s+m_p)$.
\end{example}
%============================================================================

%%%%%%%%%%%%%%%%%%%%%%%%%%%%%%%%%%%%%%%%%%%%%%%%%%%%%%%%%%%%%%%%%%%%%%%%%%%%%
\section{Torque}

The \name{angular momentum} of the $i^{\rm th}$ particle about the origin is
$$\vect{L}_i=\vect{r}_i\times\vect{p}_i
=\vect{r}_i\times(m_i\dot{\vect{r}}_i)$$
The angular momentum of the system of particles about the origin is
$$\vect{L}=\sum_i\vect{L}_i=\sum_i\vect{r}_i\times\vect{p}_i$$

%============================================================================
\begin{theorem}
In an inertial frame of reference
$$\der{\vect{L}}{t}=\vect{N}$$
where
$$\vect{N}=\sum_i\vect{r}_i\times\vect{F}_i$$
is the \name{total external torque} on the system of particles about the
origin.
\end{theorem}
\begin{proof}
From the definition of $\vect{L}$
$$\der{\vect{L}}{t}=\dbd{t}\sum_i\vect{r}_i\times m_i\dot{\vect{r}}_i$$
If the masses $m_i$ are constant with time, $\der{m_i}{t}=0$ so
$$\der{\vect{L}}{t}=\sum_i\vect{r}_i\times m_i\ddot{\vect{r}}_i$$
Now $m_i\ddot{\vect{r}}_i$ can be written in terms of the internal and
external forces to give
$$\der{\vect{L}}{t}=\sum_i\vect{r}_i\times\left(\vect{F}_i+\sum_j
\vect{F}_{ij}\right)$$
The second summation is zero because it is composed of pairs of the form
$$\vect{r}_i\times\vect{F}_{ij}+\vect{r}_j\times\vect{F}_{ji}
=(\vect{r}_i-\vect{r}_j)\times\vect{F}_{ij}=0$$
which are zero if $\vect{F}_{ij}$ lies along the line of the two particles'
centres.  

Therefore
$$\der{\vect{L}}{t}=\sum_i\vect{r}_i\times\vect{F}_i=\vect{N}$$
\end{proof}
%============================================================================

Note that when there is no external torque, $\vect{N}=0$ so $\vect{L}$ is
constant and angular momentum is conserved.  $\vect{N}=0$ when all of 
the external forces are zero, hence angular momentum is conserved when no
external forces act on the system.

%============================================================================
\begin{example}
\label{mpd ex:L and N}
Angular momentum and torque depend on the origin of the coordinate system.

Define $\bar{\vect{r}}_i$ to be the position of the $i^{\rm th}$ particle 
relative to a coordinate system fixed on the centre of mass.
$$\bar{\vect{r}}_i=\vect{r}_i-\vect{r}_{\rm cm}$$
Therefore
$$\vect{r}_i=\bar{\vect{r}}_i+\vect{r}_{\rm cm}$$
and
$$\dot{\vect{r}}_i=\dot{\bar{\vect{r}}}_i+\dot{\vect{r}}_{\rm cm}$$

Now the angular momentum is
$$\vect{L}=\sum_i\vect{r}_i\times(m_i\dot{\vect{r}}_i)
=\sum_i m_i(\bar{\vect{r}}_i+\vect{r}_{\rm cm})
\times(\dot{\bar{\vect{r}}}_i+\dot{\vect{r}}_{\rm cm})$$
Expanding the cross-product of the sums gives the sum of four
cross-products. 
\begin{eqnarray*}
\vect{L}&=&
\left(\sum_i m_i\right)\vect{r}_{\rm cm}\times\dot{\vect{r}}_{\rm cm}
+\vect{r}_{\rm cm}\times\left(\sum_im_i\dot{\bar{\vect{r}}}_i\right)\\
&&{}
+\left(\sum_i m_i\bar{\vect{r}}_i\right)\times\dot{\vect{r}}_{\rm cm}
+\sum_i \bar{\vect{r}}_i\times m_i\dot{\bar{\vect{r}}}_i
\end{eqnarray*}
Now $\sum_i m_i\bar{\vect{r}}_i=0$ from the definition of centre of mass
$$\sum_i m_i\bar{\vect{r}}_i=\sum_im_i\vect{r}_i-\sum_im_i\vect{r}_{\rm cm}
=M\vect{r}_{\rm cm}-M\vect{r}_{\rm cm}=0$$
Differentiating shows that $\sum_i m_i\dot{\bar{\vect{r}}}_i=0$, which
leaves
$$\vect{L}=
\left(\sum_i m_i\right)\vect{r}_{\rm cm}\times\dot{\vect{r}}_{\rm cm}
+\sum_i \bar{\vect{r}}_i\times m_i\dot{\bar{\vect{r}}}_i$$
which is the sum of the angular momentum of the centre of mass about the 
origin and the angular momentum of the system about the centre of mass
$$\vect{L}=
\vect{r}_{\rm cm}\times(M\dot{\vect{r}}_{\rm cm})
+\sum_i \bar{\vect{r}}_i\times (m_i\dot{\bar{\vect{r}}}_i)$$
\end{example}
%============================================================================


The \name{kinetic energy} of the system $T$ is the sum of the kinetic
energies of the individual particles
$$T=\frac{1}{2}\sum_im_i\dot{\vect{r}}_i\cdot\dot{\vect{r}}_i=\sum_i T_i$$
where $T_i=\frac{1}{2}m_i\dot{\vect{r}}_i\cdot\dot{\vect{r}}_i$ is the
kinetic energy of the $i^{\rm th}$ particle.

%============================================================================
\begin{example}
The kinetic energy of a system is the sum of the kinetic energy due to the
translation of the centre of mass plus the kinetic energy due to the system's
motion about the centre of mass.

To see this, write
$$\dot{\vect{r}}_i=\dot{\bar{\vect{r}}}_i+\dot{\vect{r}}_{\rm cm}$$

Now the kinetic energy is
$$T=\frac{1}{2}\sum_i m_i
(\dot{\bar{\vect{r}}}_i+\dot{\vect{r}}_{\rm cm})\cdot
(\dot{\bar{\vect{r}}}_i+\dot{\vect{r}}_{\rm cm})$$
Expanding the dot-product of the sums gives the sum of four
dot-products. 
\begin{eqnarray}
\vect{T}&=&
\frac{1}{2}\left(\sum_i m_i\right)\dot{\vect{r}}_{\rm cm}\cdot\dot{\vect{r}}_{\rm cm}
+\frac{1}{2}\dot{\vect{r}}_{\rm cm}\cdot\left(\sum_im_i\dot{\bar{\vect{r}}}_i\right)
\\
&&{}+\frac{1}{2}\left(\sum_i m_i\dot{\bar{\vect{r}}}_i\right)\cdot\dot{\vect{r}}_{\rm cm}
+\frac{1}{2}\sum_i \dot{\bar{\vect{r}}}_i\cdot m_i\dot{\bar{\vect{r}}}_i
\end{eqnarray}
As in the previous example, $\sum_i m_i\dot{\bar{\vect{r}}}_i=0$, which
leaves
$$\vect{T}
=\frac{1}{2}\left(\sum_i m_i\right)\dot{\vect{r}}_{\rm cm}\cdot\dot{\vect{r}}_{\rm cm}
+\frac{1}{2}\sum_i \dot{\bar{\vect{r}}}_i\cdot m_i\dot{\bar{\vect{r}}}_i$$
which is the sum of the kinetic energy of translation of the system as a
whole and the kinetic energy of the motion of the system about the 
centre of mass
$$\vect{T}
=\frac{1}{2}M\dot{\vect{r}}_{\rm cm}\cdot\dot{\vect{r}}_{\rm cm}
+\frac{1}{2}\sum_i m_i\dot{\bar{\vect{r}}}_i\cdot \dot{\bar{\vect{r}}}_i$$
\end{example}
%============================================================================

From example \ref{mpd ex:L and N}, the angular momentum $\vect{L}$ relative
to an inertial origin can be written as
$$\vect{L}=\vect{L}_{\rm cm}+\vect{L}_{\rm r}$$
where $\vect{L}_{\rm r}$ is the system's angular momentum relative to the
centre of mass.  $\vect{L}$ is related to the total external torque relative
to the origin according to
$$\vect{N}=\der{\vect{L}}{t}=\sum_i\vect{r}_i\times\vect{F}_i$$
The following theorem shows that a similar result applies to the angular
momentum about the centre of mass, even though the centre of mass may not be
an inertial reference frame.

%============================================================================
\begin{theorem}
The total external torque relative to the centre of mass is
$$\vect{N}_{\rm r}=\der{\vect{L}_{\rm r}}{t}
=\sum_i \bar{\vect{r}}_i\times\vect{F}_i$$
\end{theorem}
\begin{proof}
From the definition of
$$\vect{L}_{\rm cm}=\vect{r}_{\rm cm}\times M\dot{\vect{r}}_{\rm cm}$$
differentiate to give
$$\der{\vect{L}_{\rm cm}}{t}
=\vect{r}_{\rm cm}\times M\ddot{\vect{r}}_{\rm cm}
=\vect{r}_{\rm cm}\times \sum_i\vect{F}_i$$
But
$$\vect{L}=\vect{L}_{\rm cm}+\vect{L}_{\rm r}$$
so
$$\der{\vect{L}}{t}=\der{\vect{L}_{\rm cm}}{t}+\der{\vect{L}_{\rm r}}{t}$$
Therefore
$$\der{\vect{L}_{\rm r}}{t}=\der{\vect{L}}{t}-\der{\vect{L}_{\rm cm}}{t}$$
This is
$$\der{\vect{L}_{\rm r}}{t}
=\sum_i\vect{r}_i\times\vect{F}_i-\vect{r}_{\rm cm}\times\sum_i\vect{F}_i
=\sum_i(\vect{r}_i-\vect{r}_{\rm cm})\times\vect{F}_i$$
which gives
$$\der{\vect{L}_{\rm r}}{t}=\sum_i \bar{\vect{r}}_i\times\vect{F}_i$$
as required.
\end{proof}
%============================================================================
	
%%%%%%%%%%%%%%%%%%%%%%%%%%%%%%%%%%%%%%%%%%%%%%%%%%%%%%%%%%%%%%%%%%%%%%%%%%%
%
%			Mathematics 132 Course Notes
%
%			 Department of Mathematics,
%   			  University of Melbourne
%
%		Stephen Simmons			Lee White
%
% 8 Feb-96 SS: Updated with corrections from semester 2, 1995
%
%%%%%%%%%%%%%%%%%%% Copyright (C) 1995-96 Stephen Simmons %%%%%%%%%%%%%%%%%

%%%%%%%%%%%%%%%%%%%%%%%%%%%%%%%%%%%%%%%%%%%%%%%%%%%%%%%%%%%%%%%%%%%%%%%%%%%%%
\chapter{Rigid Body Motion}
\label{rbm chp}

A system of particles behaves as a rigid body if, during the motion of the
system, the particles do not change their relative internal positions.  Even
in solids, the particles vibrate about their equilibrium positions, so there
is always internal vibrational kinetic energy.

%%%%%%%%%%%%%%%%%%%%%%%%%%%%%%%%%%%%%%%%%%%%%%%%%%%%%%%%%%%%%%%%%%%%%%%%%%%%%
\section{Continuous Rigid Bodies}

If the body is large compared with the internal particle separations, we can
regard it as a continuum with negligible error.  The mass density of the
material in the body then becomes the \name{mass per unit volume}, which is
denoted by the symbol $\rho$.

This is a meaningful concept provided we are interested in large volumes
containing many particles.

The position of the centre of mass of a system of particles
$$\vect{r}_{\rm cm}=\frac{1}{M}\sum_im_i\vect{r}_i$$
becomes an integral
$$\vect{r}_{\rm cm}=\frac{1}{M}\int_V \rho\,\vect{r}\,dV$$
where the integral is taken over the whole of the volume of the rigid body,
and $\rho(\vect{r})$ is the function giving the body's mass per unit volume
as a function of position $\vect{r}$.  The total mass of the body is
$$M=\int_V \rho\,dV$$

If the body is uniform so that $\rho$ is constant, the formula for the
centre of mass becomes
$$\vect{r}_{\rm cm}=\frac{1}{V}\int_V \vect{r}\,dV$$

In a Cartesian coordinate system, $dV=dx\,dy\,dz$ and
$$\vect{r}_{\rm cm}=\frac{1}{M}\int\!\!\int\!\!\int \rho(x,y,z)
\vect{r}(x,y,z)\,dx\,dy\,dz$$
or, for a uniform body
$$\vect{r}_{\rm cm}=\frac{1}{V}\int\!\!\int\!\!\int 
\vect{r}(x,y,z)\,dx\,dy\,dz$$

The centres of mass of the individual coordinates for a uniform body are
$$x_{\rm cm}=\frac{1}{V}\int_V x\,dV$$
with similar expressions for $y_{\rm cm}$ and $z_{\rm cm}$.

For a thin sheet, called a \name{lamina}, the volume element is $dV=dA\,t$ where
$dA$ is the element of area on the sheet and $t$ is the thickness of the
sheet.  Then
$$\vect{r}_{\rm cm}=\frac{\ds\int_A\bar{\rho}\vect{r}\,dA}
{\ds\int_A\bar{\rho}\,dA}$$
where $\bar{\rho}=\rho t$ is the mass per unit area of the lamina.  The
total mass of the sheet is 
$$M=\int_A\bar{\rho}\,dA$$

If the rigid body has a plane of symmetry, $\vect{r}_{\rm cm}$ lies in that
plane.  If the rigid body has a line of symmetry, $\vect{r}_{\rm cm}$ lies
on that line.


%============================================================================
\begin{example}
For a solid hemisphere with its centre at the origin, radius $a$ and $z\geq0$
$$x_{\rm cm}=y_{\rm cm}=0\qquad z_{\rm cm}=\frac{3a}{8}$$
For a hemispherical shell with its centre at the origin, radius $a$ 
and $z\geq0$
$$x_{\rm cm}=y_{\rm cm}=0\qquad z_{\rm cm}=\frac{a}{2}$$
For a semicircular line of radius $a$
$$x_{\rm cm}=y_{\rm cm}=0\qquad z_{\rm cm}=\frac{2a}{\pi}$$
For a semicircular lamina of radius $a$
$$x_{\rm cm}=y_{\rm cm}=0\qquad z_{\rm cm}=\frac{4a}{3\pi}$$
For an isosceles triangular lamina with height $h$
$$x_{\rm cm}=y_{\rm cm}=0\qquad z_{\rm cm}=\frac{h}{3}$$
\end{example}
%============================================================================

%----------------------------------------------------------------------------
\begin{figure}\centering
\caption{This semicircular lamina in the $xz$ plane has its centre at the
origin and has radius $a$.  The shaded region is the strip from $z$ to
$z+dz$.}
\label{rbm fig:scl}

\psset{unit=5cm}
\begin{pspicture}(-0.7,-0.3)(0.7,0.75)
% The shaded region
\psclip{\psarc[linewidth=1pt,linestyle=none]{-}(0,0){0.5}{0}{180}}
	\psframe*[fillcolor=lightgray,linecolor=gray](-0.5,0.27)(0.5,0.3)
\endpsclip
% Now the circle
\psarc[linecolor=darkgray,linewidth=2pt]{-}(0,0){0.5}{0}{180}
% Axes
\psline{->}(-0.6,0)(0.6,0)
\psline{->}(0,0)(0,0.6)
\uput[r](0.6,0){$x$}
\uput[u](0,0.6){$z$}
\SpecialCoor
\uput[d](-0.5,0){$-a$}
\uput[d](0.5,0){$a$}
\uput[dl](0,0.5){$a$}
\pcline{<->}(0,-0.1)(!0.27 0.25 exch dup mul sub sqrt -0.1)
\Bput{$\sqrt{a^2-z^2}$}
% Dashed lines
\psline[linecolor=black,linestyle=dashed]{-}(0,-0.2)(0,0) 
\psline[linecolor=black,linestyle=dashed]{-}%
(!0.27 0.25 exch dup mul sub sqrt 0.27)(!0.27 0.25 exch dup mul sub sqrt -0.2)
\uput[dl](0,0.27){$z$}
% Arrows to indicate width of dz
\psline{->}(-0.15,0.2)(-0.15,0.27) 
\psline{->}(-0.15,0.37)(-0.15,0.3) 
\uput[d](-0.15,0.2){$dz$}
\end{pspicture}
\end{figure}
%----------------------------------------------------------------------------

%============================================================================
\begin{example}[Semicircular Lamina]
\problem
Show that the centre of mass of the semicircular lamina of radius $a$ in the
half-plane $z>0$ is
$$z_{\rm cm}=\frac{4a}{3\pi}$$

\solution
As shown in figure \ref{rbm fig:scl}, the mass of 
the strip of width $dz$ of the semicircular lamina is
$$w(z)=\bar{\rho}\cdot2\sqrt{a^2-z^2}\,dz$$
The $z$ axis is a line of symmetry, so the centre of mass of the strip is on
the $z$ axis at $z$ ($x=y=0$).  Therefore the centre of mass of the whole
body is
$$z_{\rm cm}=\frac{1}{M}\int_0^a 2\bar{\rho}z\sqrt{a^2-z^2}\,dz$$
The mass of the lamina is
$$M=\bar{\rho}\cdot\frac{\pi a^2}{2}$$
so the location of the centre of mass becomes
$$z_{\rm cm}=\frac{4}{\pi a^2}\int_0^a z\sqrt{a^2-z^2}\,dz$$
Make the substitution
$$u^2=1-\frac{z^2}{a^2}$$
so that
$$z_{\rm cm}=\frac{4a}{\pi}\int_0^1 u^2\,du=\frac{4a}{3\pi}$$
\end{example}
%============================================================================


%----------------------------------------------------------------------------
\begin{figure}\centering
\caption{The hemispherical shell has its centre at the
origin and has radius $a$.  The shaded region is the circumferential strip
from $\theta$ to $\theta+d\theta$.}
\label{rbm fig:hss}

\psset{unit=5cm}
\begin{pspicture}(-0.7,-0.3)(0.7,0.75)
% Now the circle
\psarc[linecolor=darkgray,linewidth=2pt]{-}(0,0){0.5}{0}{180}
% The shaded region
\psclip{\psarc[linewidth=1pt,linestyle=none]{-}(0,0){0.5}{0}{180}}
	\psframe*[fillcolor=lightgray,linecolor=gray](-0.5,0.27)(0.5,0.3)
\endpsclip
% Axes
\psline{->}(-0.6,0)(0.6,0)
\psline{->}(0,0)(0,0.6)
\uput[r](0.6,0){$x$}
\uput[u](0,0.6){$z$}
\SpecialCoor
\uput[d](-0.5,0){$-a$}
\uput[d](0.5,0){$a$}
\uput[dl](0,0.5){$a$}
\pcline{<->}(0,-0.1)(!0.27 0.25 exch dup mul sub sqrt -0.1)
\Bput{$a\cos\theta$}
% Dashed lines
\psline[linecolor=black,linestyle=dashed]{-}(0,-0.2)(0,0) 
\psline[linecolor=black,linestyle=dashed]{-}%
(!0.27 0.25 exch dup mul sub sqrt 0.27)(!0.27 0.25 exch dup mul sub sqrt -0.2)
% Dark lines for ad\theta
\psline[linecolor=black,linewidth=2pt]{-}%
(!0.27 0.25 exch dup mul sub sqrt 0.27)(!0.3 0.25 exch dup mul sub sqrt 0.3)
\psline[linecolor=black,linewidth=2pt]{-}%
(!0.27 0.25 exch dup mul sub sqrt neg 0.27)(!0.3 0.25 exch dup mul sub sqrt
neg 0.3)
% Arrows to indicate width of ad\theta
\rput[B]{(0.0208,0.03)}(!0.27 0.25 exch dup mul sub sqrt neg 0.27){
	\psline{->}(-0.15,0.05)(0,0.05) 
	\psline{<-}(0.0365,0.05)(0.1865,0.05) 
}
\uput[ul](!0.27 0.25 exch dup mul sub sqrt neg 0.27){$a\,d\theta$}
% theta and dtheta
\psline[linestyle=dashed]{-}(0,0)(!0.25 0.27 dup mul sub sqrt 0.27)
\psarc{->}(0,0){0.2}{0}{(!0.25 0.27 dup mul sub sqrt 0.27)}
\psline[linestyle=dashed]{-}(0,0)(!0.25 0.3 dup mul sub sqrt 0.3)
\uput[ur](0.2,0){$\theta$}
\rput[B](!0.25 0.3 dup mul sub sqrt 0.3 0.6 mul exch 0.6 mul exch)%
{$d\theta$}
\end{pspicture}
\end{figure}
%----------------------------------------------------------------------------

%============================================================================
\begin{example}[Hemispherical Shell]
\problem
Show that the centre of mass of a hemispherical shell of radius $a$ with
$z>0$ is
$$z_{\rm cm}=\frac{a}{2}$$

\solution
As shown in figure \ref{rbm fig:hss}, the strip from an angle $\theta$ to 
$\theta+d\theta$ has $z$ coordinate
$$z=a\sin\theta$$
The surface area of this strip is
$$dA=2\pi(a\cos\theta)\,a\,d\theta=2\pi a^2\cos\theta\,d\theta$$
From symmetry, the centre of mass of this strip is at $z=a\sin\theta$. 
Therefore, the centre of mass of the hemispherical shell has
$x_{\rm cm}=y_{\rm cm}=0$ and
$$z_{\rm cm}=\frac{1}{M}\int_A\bar{\rho}z\,dA=\frac{1}{M}\int_0^{\pi/2}
2\pi a^3\bar{\rho}\cos\theta\sin\theta\,d\theta$$
The mass of the hemispherical shell is
$$M=\bar{\rho}\cdot 2\pi a^2$$
so the location of the centre of mass becomes
$$z_{\rm cm}=a\int_0^{\pi/2}\cos\theta\sin\theta\,d\theta$$
Make the substitution
$$u=\sin\theta$$
so that
$$z_{\rm cm}=a\int_0^1 u\,du=\frac{a}{2}$$
\end{example}
%============================================================================

%%%%%%%%%%%%%%%%%%%%%%%%%%%%%%%%%%%%%%%%%%%%%%%%%%%%%%%%%%%%%%%%%%%%%%%%%%%%%
\section{Rotation About a Fixed Axis}

%----------------------------------------------------------------------------
\begin{figure}\centering
\caption{Here is the point at $(x,y,z)$ rotating at constant speed
$\omega=\dot{\phi}$ about the $z$-axis.}
\label{rbm fig:rafi}

\psset{unit=8cm}
\begin{pspicture}(-0.25,-0.35)(0.5,0.55)
% First the xyz axes
\psline{->}(0,0)(0,0.4)
\uput[u](0,0.4){$z$}
\psline{->}(0,0)(-0.15,-0.25)
\uput[dl](-0.15,-0.25){$x$}
\psline{->}(0,0)(0.4,-0.133)
\uput[dr](0.4,-0.133){$y$}
\uput[l](0,0){$O$}
% The point and vector to it
\qdisk(0.1,0){5pt}
\psline[linewidth=3pt]{->}(0,0)(0.1,0)
\uput[r](0.11,0){$(x,y,z)$}
\uput[u](0.05,0.05){$\vect{r}$}
% Construction lines
\psline[linestyle=dashed]{-}(-0.1,-0.2)(0.1,-0.267) 
\psline[linestyle=dashed]{-}(0,0)(0.1,-0.267) 
\psline[linestyle=dashed]{-}(0.2,-0.067)(0.1,-0.267) 
\psline[linestyle=dashed]{-}(0.1,0)(0,0.3) 
\psline[linestyle=dashed]{-}(0.1,0)(0.1,-0.267) 
% Angles
\SpecialCoor
\psarc{->}(0,0){0.1}{(-0.1,-0.2)}{(0.1,-0.267)}
\psarc{->}(0,0.38){0.05}{180}{360}
\uput[r](0.05,0.38){$\omega$}
\uput[d](-0.05,-0.1){$\phi$}
% The velocity v
\psline[linewidth=2pt]{->}(0.1,0)(0.2,0.05)
\uput[r](0.2,0.05){$\vect{v}$}
\end{pspicture}
\end{figure}
%----------------------------------------------------------------------------

Consider a body rotating about the $z$ axis through a fixed origin $O$ with
angular speed
$$\omega=\dot{\phi}$$
The mass at point $\vect{r}=(x,y,z)$ is moving with velocity $\vect{v}$ in 
the $xy$ plane through the point, where $\vect{v}$ satisfies
$$\left|\vect{v}\right|=\omega\sqrt{x^2+y^2}$$

Now the point is moving in uniform circular motion, so $\vect{v}$ is also 
given by
$$\vect{v}=\dot{\vect{r}}=\dot{x}\ivect+\dot{y}\jvect$$
Using
$$\dot{x}=-\left|\vect{v}\right|\sin\phi=-\omega y$$
and
$$\dot{y}=\left|\vect{v}\right|\cos\phi=\omega x$$
the velocity can be written
$$\vect{v}=\boldsymbol{\omega}\times\vect{r}$$
where the angular velocity vector is
$$\boldsymbol{\omega}=\omega\kvect$$

For a discrete body, the rotational kinetic energy is
$$T_{\rm rot}=\sum_i\frac{1}{2}m_i\left|\vect{v}_i\right|^2
=\frac{1}{2}\sum_im_i\left(\omega\sqrt{x_i^2+y_i^2}\right)^2$$
This can be written as
$$T_{\rm rot}=\frac{1}{2}I_z\omega^2$$
where $I_z$ is the \name{moment of inertia} of the body about the $z$ axis
through $O$
$$I_z=\sum_i m_i\left(x_i^2+y_i^2\right)$$

For a continuum solid, the same formula holds
$$T_{\rm rot}=\frac{1}{2}I_z\omega^2$$
when now the moment of inertia is
$$I_z=\int_V \rho\,(x^2+y^2) \,dV$$

When the body is rotating about the $z$ axis, the velocity is
$$\vect{v}=\dot{\vect{r}}=\dot{x}\ivect+\dot{y}\jvect$$
so the angular momentum about $O$
$$\vect{L}=\sum_i\vect{r}_i\times(m_i\dot{\vect{r}}_i)$$
has only a component in the $z$ direction
$$L_z=\sum_i m_i(x_i\dot{y}_i-y_i\dot{x}_i)$$
Using $\dot{x}=-\omega y$ and $\dot{y}=\omega x$, this becomes
$$L_z=\sum_i m_i(x_i^2+y_i^2)\omega=I_z\omega$$

The \name{angular equation of motion} for a rigid body can be found from the
$z$ component of $\dbd{t}\vect{L}=\vect{N}$
$$\dbd{t}(I_z\omega)=N_z$$
For a rigid body, $I_z$ is constant so that
$$I_z\der{\omega}{t}=N_z$$

This leads to simple relations between the equations of motion for
translation along the $z$ axis and rotation around the $z$ axis.  Making the
connections
$$\left.\begin{array}{c}m \\ v_z \\ \dot{v}_z \end{array}\right\}\iff
\left\{\begin{array}{c}I_z \\ \omega \\ \dot{\omega} \end{array}\right.$$
shows how linear and angular momentum obey similar relationships
$$p_z=mv_z\qquad\iff\qquad L_z=I_z\omega$$
as do force and torque
$$F_z=m\dot{v}_z\qquad\iff\qquad N_z=I_z\dot{\omega}$$
and translational and rotational kinetic energy
$$\frac{1}{2}mv_z^2\qquad\iff\qquad\frac{1}{2}I_z\omega^2$$
 
%============================================================================
\begin{example}[Thin Rod about Centre of Mass]

The $z$ component of the moment of inertia of a thin rod of length $l$ 
about its centre with the rod lying along the $x$ axis is given by
$$I_z=\int_{-l/2}^{l/2}\int\!\!\int \rho\,(x^2+y^2)\,dz\,dy\,dx$$
Since the rod is thin and lies along the $x$ axis, this becomes
$$I_z=\int_{-l/2}^{l/2}\int\!\!\int \rho x^2\,dz\,dy\,dx=\bar{\rho}
\int_{-l/2}^{l/2}x^2\,dx$$
where $\bar{\rho}$ is the rod's mass per unit length
$$\bar{\rho}=\rho\int\!\!\int dz\,dy=\frac{M}{l}$$
Therefore
$$I_z=\frac{M}{l}\int_{-l/2}^{l/2}x^2\,dx=\frac{M}{l}\left.\frac{x^3}{3}
\right|_{-l/2}^{l/2}$$
so that
$$I_z=\frac{1}{12}Ml^2$$
\end{example}
%============================================================================

%============================================================================
\begin{example}[Laminar Annulus about Centre of Mass]

Consider an annulus of radius $R$ and thickness $h$ with $h\ll R$, lying in
the $xy$ plane with the $z$ axis at its centre.  The element of unit area is
$$dA=Rh\,d\theta$$
and the mass per unit area is
$$\bar{\rho}=\frac{M}{2\pi Rh}$$
Therefore the moment of inertia about the $z$ axis is
$$I_z=\int \bar{\rho}\,(x^2+y^2)\,dA=\bar{\rho}\int_0^{2\pi}Rh(R^2)\,d\theta
=(2\pi Rh\bar{\rho})R^2$$
Since $M=2\pi Rh\bar{\rho}$, the moment of inertia is
$$I_z=MR^2$$
\end{example}
%============================================================================

%============================================================================
\begin{example}[Circular Disc about the Centre of Mass]

From the previous example, the moment of inertia of the annulus from $r$ to
$r+dr$ is
$$dI_z=(2\pi r\,dr\,\bar{\rho})r^2$$
Therefore
$$I_z=2\pi\bar{\rho}\int_0^ar^3\,dr=2\pi\bar{\rho}\frac{a^4}{4}$$
But $\bar{\rho}=M/\pi a^2$ so the moment of inertia becomes
$$I_z=\frac{1}{2}Ma^2$$
\end{example}
%============================================================================

%============================================================================
\begin{exercise}
Show that the moments of inertia of a sphere and a thin spherical shell,
both of radius $a$, are respectively
$$I_z=\frac{2}{5}Ma^2\qquad\mbox{and}\qquad I_z=\frac{2}{3}Ma^2$$
\end{exercise}
%============================================================================

%%%%%%%%%%%%%%%%%%%%%%%%%%%%%%%%%%%%%%%%%%%%%%%%%%%%%%%%%%%%%%%%%%%%%%%%%%%%%
\subsection{Moments of Inertia about Other Axes}

The moments of inertia of a body about the three Cartesian axes are
$$I_z=\sum_im_i(x_i^2+y_i^2)$$
$$I_x=\sum_im_i(y_i^2+z_i^2)$$
$$I_y=\sum_im_i(x_i^2+z_i^2)$$
for bodies composed of discrete particles, and
$$I_z=\int_V\rho\,(x^2+y^2)\,dV$$
$$I_x=\int_V\rho\,(y^2+z^2)\,dV$$
$$I_y=\int_V\rho\,(x^2+z^2)\,dV$$
for continuous bodies.

For a laminar body lying in the $xy$ plane
$$I_z=\int_V\bar{\rho}\,(x^2+y^2)\,dV$$
$$I_x=\int_V\bar{\rho}\,y^2\,dV$$
$$I_y=\int_V\bar{\rho}\,x^2\,dV$$
because the $z$ component is zero.  Adding these shows that
$$I_z=I_x+I_y$$
which is known as the \name{perpendicular axis theorem}. In other words,
the moment of inertia of any plane body about an axis normal to the body is
equal to the sum of the moments of inertia about any two perpendicular axes
passing through the given axis and lying in the plane.

%============================================================================
\begin{example}
The moment of inertia in the $z$ direction of a circular plate lying in the
$xy$ plane is
$$I_z=\frac{1}{2}Ma^2$$
if the origin is at the centre of the circular plate.  

By symmetry, $I_x=I_y$ and by the perpendicular axis theorem,
$$I_z=I_x+I_y$$ 
so that
$$I_x=I_y=\frac{1}{2}I_z=\frac{1}{4}Ma^2$$
\end{example}
%============================================================================

%============================================================================
\begin{example}
To find the moment of inertia $I_z$ of a flat square lying in the $xy$
plane, whose size is $l$ by $l$ and mass is $M$, note that looking along the
$x$ or the $y$ axes, the plate looks like a rod of length $l$.  Therefore
$I_x$ and $I_y$ are the moments of inertia of a rod of length $l$ and mass
$M$ measured about the centre of the rod
$$I_x=I_y=\frac{1}{12}Ml^2$$
Then by the perpendicular axis theorem,
$$I_z=I_x+I_y=\frac{1}{6}Ml^2$$
\end{example}
%============================================================================

The following theorem, the \name{parallel axis theorem} describes how the
moment of inertia changes when measured relative to an axis that is
parallel to the original axis.

%============================================================================
\begin{theorem}[Parallel Axis Theorem]
Suppose a body has a moment of inertia $I_z$ measured through the origin and 
$I_{z,\rm cm}$ measured relative to the centre of mass.  If the
perpendicular distance from the centre of mass to the $z$ axis is $l$, 
the two moments of inertia are related by
$$I_z=Ml^2+I_{z,\rm cm}$$
\end{theorem}
\begin{proof}
The moment of inertia through the origin is
$$I_z=\sum_i m_i(x_i^2+y_i^2)$$
and the moment of inertia through the centre of mass is
$$I_{z,\rm cm}=\sum_i m_i(\bar{x}_i^2+\bar{y}_i^2)$$
where the coordinates relative to the origin $\vect{r}_i$ are related to the
coordinates relative to the centre of mass $\bar{\vect{r}}_i$ according to
$$\vect{r}_i=\vect{r}_{\rm cm}+\bar{\vect{r}}_i$$
where $\vect{r}_{\rm cm}$ is the position of the centre of mass relative to
the origin.  Therefore
$$I_z=\sum_i m_i(x_i^2+y_i^2)
=\sum_i m_i\left((x_{\rm cm}+\bar{x}_i)^2+(y_{\rm cm}+\bar{y}_i)^2\right)$$
Expanding the squares shows that
\begin{eqnarray*}
I_z&=&\sum_i m_i(x_{\rm cm}^2+y_{\rm cm}^2)
+2\left(\sum_i m_i\bar{x}_i\right)x_{\rm cm}\\
&&{}+2\left(\sum_i m_i\bar{y}_i\right)y_{\rm cm}
+\sum_i m_i(\bar{x}_i^2+\bar{y}_i^2)
\end{eqnarray*}
The second and third summations are zero.  Writing 
$l^2=x_{\rm cm}^2+y_{\rm cm}^2$, $M=\sum_i m_i$ and noting that the fourth 
summation is the moment of inertia about the centre of mass shows that
$$I_z=Ml^2+I_{z,\rm cm}$$
as required.
\end{proof}
%============================================================================

%============================================================================
\begin{example}
To find the moment of inertia $I_z$ of a circular plate in the $xy$ plane
relative to an axis through the edge of the plate, note that the moment of
inertia through the centre of the plate is
$$I_{z,\rm cm}=\frac{1}{2}Ma^2$$
Measuring the moment of inertia relative to an axis through the edge of the
plate shifts the axis a distance $l=a$.  Therefore the moment of inertia is
$$I_z=Ml^2+I_{z,\rm cm}=Ma^2+\frac{1}{2}Ma^2=\frac{3}{2}Ma^2$$
\end{example}
%============================================================================

%============================================================================
\begin{example}
The moment of inertia $I_x$ for the plate in the previous example can be
found by noting that the moment of inertia through the centre of mass of the
plate is
$$I_{x,\rm cm}=\frac{1}{4}Ma^2$$
Measuring the moment of inertia relative to an axis through the edge of the
plate shifts the axis a distance $l=a$.  Therefore the moment of inertia is
$$I_x=Ml^2+I_{x,\rm cm}=Ma^2+\frac{1}{4}Ma^2=\frac{5}{4}Ma^2$$
\end{example}
%============================================================================

%%%%%%%%%%%%%%%%%%%%%%%%%%%%%%%%%%%%%%%%%%%%%%%%%%%%%%%%%%%%%%%%%%%%%%%%%%%%%
\section{Gravitational Torque}

If the force $\vect{F}_i$ on the $i^{\rm th}$ particle is the gravitational 
force $m_i\vect{g}$, then the
total external gravitational torque about the origin is
$$\vect{N}=\sum_i\vect{r}_i\times\vect{F}_i$$
This can be written
$$\vect{N}=\sum_i\vect{r}_i\times(m_i\vect{g})
=\sum_i(m_i\vect{r}_i)\times\vect{g}$$
In terms of the centre of mass, this becomes
$$\vect{N}=\sum_i(m_i\vect{r}_i)\times\vect{g}=M\vect{r}_{\rm cm}
\times\vect{g}$$
Therefore the gravitional torque is equal to the torque acting on the total
weight of the body concentrated at the centre of mass.

%----------------------------------------------------------------------------
\begin{figure}\centering
\caption{This is a compound pendulum of mass $M$ with centre of mass at $C$. 
It pivots about the point $P$ which is a distance $l$ from the centre of 
mass.  The $z$ axis points out of the page.}
\label{rbm fig:cp}

\psset{unit=5cm}
\begin{pspicture}(-0.3,-1.3)(1.3,0.3)
% The outline of the pendulum
\psccurve[linecolor=darkgray,linewidth=2pt](-0.2,0.2)(0.1,0.2)(0.3,-0.1)(0.7,-0.4)(0.9,-0.2)%
(1.1,-0.4)(1.2,-0.8)(0.9,-1.1)(0.5,-0.9)(0.1,-0.6)(-0.1,-0.1)
% Axes
\psline{->}(0,0)(1,0)
\uput[r](1,0){$y$}
\psline{->}(0,0)(0,-1)
\uput[d](0,-1){$x$}
\qdisk(0,0){3pt}
\uput[ul](0,0){$P$}
% Centre of mass
\pcline{->}(0,0)(0.6,-0.6)
\Aput{$l$}
\uput[ur](0.6,-0.6){$C$}
\psline{->}(0.6,-0.6)(0.6,-0.9)
\uput[d](0.6,-0.9){$Mg$}
% The angle theta
\psarc{->}(0,0){0.2}{270}{315}
\uput[dr](0,-0.2){$\theta$}
\end{pspicture}
\end{figure}
%----------------------------------------------------------------------------

%============================================================================
\begin{example}[The Compound Pendulum]
The equation of motion of the compound pendulum about the pivot, 
shown in figure \ref{rbm fig:cp}, is
$$I_z\dot{\omega}=N_z$$
where $\dot{\omega}=\ddot{\theta}$, the torque is $\vect{N}=\vect{l}\times
M\vect{g}$ so that
$$N_z=-Mgl\sin\theta$$
and the moment of inertia about the pivot is
$$I_z=Ml^2+I_{z,\rm cm}$$
Therefore
$$(Ml^2+I_{z,\rm cm})\ddot{\theta}=-Mgl\sin\theta$$
which gives
$$\ddot{\theta}=-\frac{g}{l+I_{z,\rm cm}/Ml}\sin\theta$$

For small amplitude oscillations, $\left|\theta\right|\ll 1$ so that
$\sin\theta\approx\theta$.  The equation of motion is approximately
$$\ddot{\theta}=-\frac{g}{l+I_{z,\rm cm}/Ml}\theta$$
The solution is
$$\theta(t)=\theta_0\cos\omega t$$
where $\theta(0)=\theta_0$ and $\dot{\theta}(0)=0$.  The angular frequency
of oscillation is
$$\omega=\sqrt{\frac{g}{l+I_{z,\rm cm}/Ml}}$$
and the period is
$$T=\frac{2\pi}{\omega}=2\pi\sqrt{\frac{l+I_{z,\rm cm}/Ml}{g}}$$
which is independent of the initial conditions.

For large oscillations, the equation of motion is
$$\ddot{\theta}=-\omega_0^2\sin\theta$$
where
$$\omega_0^2=\frac{g}{l+I_{z,\rm cm}/Ml}$$
Using $\ddot{\theta}=\dbd{\theta}\left(\frac{1}{2}\dot{\theta}^2\right)$,
integrate to obtain
$$\frac{1}{2}\dot{\theta}^2=\omega_0^2\cos\theta+C$$
With $\theta(0)=\theta_0$ and the pendulum starting from rest so that
$\dot{\theta}(0)=0$, the solution is
$$\dot{\theta}^2=2\omega_0^2(\cos\theta-\cos\theta_0)$$
The integral of this is a bit complicated and cannot be obtained in terms of 
ordinary functions.  However an expression for the period $T$ can be
obtained.

Taking the square root of the equation for $\dot{\theta}^2$ 
$$\dot{\theta}=\pm\sqrt{2}\omega_0\sqrt{\cos\theta-\cos\theta_0}$$
During the period $0<t<\frac{T}{4}$, the pendulum swings from
$\theta=\theta_0$ to $\theta=0$ so $\dot{\theta}<0$.  Thus for these times,
the negative square root is appropriate and
$$\dot{\theta}=-\sqrt{2}\omega_0\sqrt{\cos\theta-\cos\theta_0}$$
which is a separable equation.

The solution is
$$T=\frac{\sqrt{8}}{\omega_0}\int^{\theta_0}_0\frac{d\theta}{\sqrt{\cos\theta
-\cos\theta_0}}$$
which is an elliptic integral.  Denoting the elliptic integral as
$f(\theta_0)$, the solution is
$$T=\frac{\sqrt{8}}{\omega_0}f(\theta_0)$$
which is a function of the initial conditions.
\end{example}
%============================================================================

%%%%%%%%%%%%%%%%%%%%%%%%%%%%%%%%%%%%%%%%%%%%%%%%%%%%%%%%%%%%%%%%%%%%%%%%%%%%%
\section{Rolling}

%----------------------------------------------------------------------------
\begin{figure}\centering
\caption{This diagram shows a body rolling on a plane inclined at an angle
$\theta$.  The dashed line indicates the path taken by the rolling body's
centre of mass.}
\label{rbm fig:rb}

\psset{unit=5cm}
\begin{pspicture}(-1.3,-0.1)(0.3,0.8)
\SpecialCoor
% First the ground
\psline[linecolor=darkgray,linewidth=2pt]{-}(1.1;150)(0;150)(-1;0)
% The rolling body
\rput[b]{330}(0.5;150){
	\begin{pspicture}(0,-0.2)(0,-0.2)
		\psline[linestyle=dashed]{-}(-0.4,0)(0.4,0) 
		\psline(0,0)(0.2;135)
		\psarc{->}(0,0){0.1}{0}{135}
		\uput[u](0.1;66){\rput[l]{*0}{$\phi$}}	
		\pscircle[linecolor=gray,linewidth=2pt](0,0){0.2}
		% The weight vector
		\psline{->}(0,0)(0.3;300)
		\uput[d](0.32;300){\rput[B]{30}{$Mg$}}
	\end{pspicture}
}
\rput[b]{330}(0.5;150){
	\begin{pspicture}(0,0)(0,0)
		\psline[linecolor=black]{->}(0,0)(-0.2,0) 
		\uput[d](-0.2,0){\rput[r]{30}{$F_f$}}
		\psline[linecolor=black]{->}(0,0)(0,0.1) 
		\uput[l](0,0.05){\rput[b]{30}{$R$}}
	\end{pspicture}
}
% The angle theta
\psarcn{->}(0,0){0.2}{180}{150}
\uput[l](-0.2,0.05){$\theta$}
\end{pspicture}
\end{figure}
%----------------------------------------------------------------------------

For the body rolling down the slope illustrated in figure \ref{rbm fig:rb}, 
the forces on the body lead to the following equations of motion
$$M\ddot{x}_{\rm cm}=Mg\sin\theta-F_f$$
$$M\ddot{y}_{\rm cm}=-Mg\cos\theta+R$$
The body stays in contact with the slope so that
$$y_{\rm cm}=a$$ 
and
$$\ddot{y}_{\rm cm}=0$$ 
so that the reaction is
$$R=Mg\cos\theta$$
The torque on the body about the centre of mass is
$$I_{\rm cm}\dot{\omega}=N_z=-F_fa$$
since all other forces have zero torque about the centre of mass.  The
velocity of the point of the rolling body in contact with the slope is
$$\dot{x}_{\rm cm}=-a\omega$$ 
so that
$$\omega=-\frac{\dot{x}_{\rm cm}}{a}\qquad\mif\qquad
\dot{\omega}=-\frac{\ddot{x}_{\rm cm}}{a}$$
Substituting into the torque equation gives
$$I_{\rm cm}\left(-\frac{\ddot{x}_{\rm cm}}{a}\right)=-F_fa$$
so that
$$F_f=\frac{I_{\rm cm}}{a^2}\ddot{x}_{\rm cm}$$
Substituting into the force equation shows that
$$\ddot{x}_{\rm cm}=\frac{g\sin\theta}{1+I_{\rm cm}/Ma^2}$$
which is a constant.  Thus
$$\dot{\omega}=-\frac{\ddot{x}_{\rm cm}}{a}=
\frac{g\sin\theta}{a\left(1+I_{\rm cm}/Ma^2\right)}$$
which is also a constant.

If the point of contact slips on the inclined plane, then
$$F_f=\mu R$$
where $\mu$ is the coefficient of sliding friction.  Therefore the force
equation becomes
$$M\ddot{x}_{\rm cm}=Mg\sin\theta-\mu Mg\cos\theta$$
or
$$\ddot{x}_{\rm cm}=g\left(\sin\theta-\mu\cos\theta\right)$$
which is a constant.  To calculate the angular velocity, use the torque
equation
$$I_{\rm cm}\dot{\omega}=-F_fa$$
so
$$\dot{\omega}=-\frac{a\mu Mg\cos\theta}{I_{\rm cm}}$$

%----------------------------------------------------------------------------
\begin{figure}\centering
\caption{Initially, this billiard ball is spinning with an angular velocity
of $\omega_0$ and is released with zero forward velocity.   At the time
shown in this figure, the ball is spinning with angular velocity $\omega$
and has a forward velocity of $\vect{v}_{\rm cm}$.}
\label{rbm fig:sbb}

\psset{unit=5cm}
\begin{pspicture}(-0.8,-0.6)(0.6,0.6)
% Billiard ball
\pscircle[linecolor=gray,linewidth=2pt](0,0){0.5}
% Axes
\psline{->}(-0.6,-0.55)(0.6,-0.55)
\uput[r](0.6,-0.55){$x$}
\psline{->}(-0.6,-0.55)(-0.6,0.2)
\uput[u](-0.6,0.2){$y$}
% Reaction and friction
\psline{->}(0,-0.5)(0,-0.3)
\uput[r](0,-0.3){$R$}
\psline{->}(0,-0.5)(0.3,-0.5)
\uput[r](0.3,-0.5){$F_f$}
% Gravity and velocity
\psline{->}(0,0)(0.3,0)
\uput[r](0.3,0){$\vect{v}_{\rm cm}$}
\psline{->}(0,0)(0,-0.2)
\uput[r](0,-0.2){$Mg$}
% The angle omega
\psarcn{->}(0,0){0.4}{180}{135}
\SpecialCoor
\uput[ur](0.4;135){$\omega$}
\end{pspicture}
\end{figure}
%----------------------------------------------------------------------------

%============================================================================
\begin{example}[Slipping Billiard Ball]
\problem
A billiard ball rotating with angular speed $\omega_0$ is released with
zero forward velocity such that it slips on the table (as shown in figure
\ref{rbm fig:sbb}), where the coefficient of friction is $\mu$.  When does 
it start to roll?

\solution
The reaction is $R=Mg$ and initially, the ball is slipping so that
$$F_f=\mu R$$
The force equation gives
$$M\ddot{x}_{\rm cm}=F_f=\mu Mg$$
so that
$$\ddot{x}_{\rm cm}=\mu g\qquad\dot{x}_{\rm cm}=\mu gt\qquad
x_{\rm cm}=\frac{\mu gt^2}{2}$$
which is valid up to the point when rolling starts and slipping stops as
then $F_f\neq \mu R$.

The torque equation about the centre of mass is
$$I_{\rm cm}\dot{\omega}=+\mu Mga$$
so that
$$\omega=\frac{\mu Mga}{I_{\rm cm}}t-\omega_0$$
Slipping stops when $\omega$ slows down to the point when
$$-\frac{\dot{x}_{\rm cm}}{a}=\omega$$
At this time, $t_{\rm roll}$,
$$-\frac{\mu gt_{\rm roll}}{a}=\frac{\mu Mga}{I_{\rm cm}}t_{\rm roll}
-\omega_0$$
so the time when pure rolling starts is
$$t_{\rm roll}=\frac{a\omega_0}{\mu g\left(1+Ma^2/I_{\rm cm}\right)}$$
which is a distance
$$x_{\rm roll}=\frac{a^2\omega_0^2}{2\mu g\left(1+Ma^2/I_{\rm cm}\right)^2}$$
from the point when it is first released.

After this point, we have
$$\ddot{x}_{\rm cm}=\frac{F_f}{M}$$
Now
$$I_{\rm cm}\dot{\omega}=F_fa$$
but the  ball is rotating without slipping so
$$\dot{\omega}=-\frac{\ddot{x}_{\rm cm}}{a}$$
so that
$$F_f=-\frac{I_{\rm cm}\ddot{x}_{\rm cm}}{a^2}$$
This implies that
$$\ddot{x}_{\rm cm}\left(1+\frac{I_{\rm cm}}{Ma^2}\right)=0$$
which means that the ball's velocity is constant after pure rolling has 
started.  This implies that $F_f$ does no work on a rolling body.
\end{example}
%============================================================================

%----------------------------------------------------------------------------
\begin{figure}\centering
\caption{This shows the forces on a horizontal plank of length $l$ and mass
$M$ at the instant that the right-hand end is dropped.  The dropped plank
pivots about the point $P$.}
\label{rbm fig:dl}

\psset{unit=5cm}
\begin{pspicture}(-0.4,-0.4)(1.4,0.4)
% The ladder and its support
\psline[linecolor=gray,linewidth=2pt]{-}(0,0)(1,0)
\pspolygon*[linecolor=darkgray](0,0)(-0.05,-0.1)(0.05,-0.1)
% Gravity
\psline{->}(0.5,0)(0.5,-0.2)
\uput[d](0.5,-0.2){$Mg$}
% l/2 labels
\uput[u](0.25,0){$l/2$}
\uput[u](0.75,0){$l/2$}
% Reaction and unit vectors
\psline{->}(0,0)(0,0.2)
\uput[u](0,0.2){$R$}
\psline{->}(1,0)(1,0.2)
\uput[u](1,0.2){$\tvect$}
\psline{->}(1,0)(1.2,0)
\uput[r](1.2,0){$\rvect$}
% The pivot point P
\uput[l](0,0){$P$}
\end{pspicture}
\end{figure}
%----------------------------------------------------------------------------

%============================================================================
\begin{example}
\problem
Two people hold a plank of mass $M$ and length $l$ at each end.  One person
lets their end go.  Show that the load supported by the other person drops
from $Mg/2$ to $Mg/4$ instantly.  Show that the
acceleration of the free end is $3g/2$.

\solution
A diagram of the force is shown in figure \ref{rbm fig:dl}.  The plank
rotates about $P$ according to 
$$I_P\dot{\omega}=-Mg\frac{l}{2}$$
With $R$ the reaction from the person holding the plank at $P$, the
equation of rotation about the plank's centre of mass is
$$I_{\rm cm}\dot{\omega}=-R\frac{l}{2}$$
so that
$$\frac{I_{\rm cm}}{I_P}=\frac{R}{Mg}$$
Using the parallel axis theorem to write $I_P=I_{\rm cm}+
M\left(l/2\right)^2$ gives
$$R=Mg\frac{I_{\rm cm}}{I_{\rm cm}+M\left(l/2\right)^2}$$
But the moment of inertia about the centre of mass is
$$I_{\rm cm}=\frac{1}{12}Ml^2$$ 
so the reaction is
$$R=Mg\frac{\frac{1}{12}}{\frac{1}{12}+\frac{1}{4}}=\frac{Mg}{4}$$

To work out the acceleration of the free end, use
$$\ddot{\vect{r}}=\left(\ddot{r}-r\dot{\theta}^2\right)\rvect
+\left(r\ddot{\theta}+2\dot{r}\dot{\theta}\right)\tvect$$
Here $r=l$ so $\dot{r}=\ddot{r}=0$.  Initially, $\omega=\dot{\theta}=0$ so
the acceleration is
$$\ddot{\vect{r}}=l\ddot{\theta}\tvect=l\dot{\omega}\tvect$$
Using 
$$I_{\rm cm}\dot{\omega}=-R\frac{l}{2}$$
shows that
$$\ddot{\vect{r}}=-l\frac{R\frac{l}{2}}{I_{\rm cm}}\tvect=
-\frac{Mgl^2}{8I_{\rm cm}}\tvect=-\frac{3}{2}g\tvect$$
so the initial acceleration of the free end is $3g/2$ as required.
\end{example}
%============================================================================


%----------------------------------------------------------------------------
\begin{figure}\centering
\caption{This shows the forces on a thin rod of mass $M$ and length $l$ 
falling over on a rough table.}
\label{rbm fig:fr}

\psset{unit=5cm}
\begin{pspicture}(-0.4,-0.4)(1.4,1.4)
% The rod
\psline[linecolor=gray,linewidth=2pt]{-}(0,0)(1,1)
% Gravity
\psline{->}(0.5,0.5)(0.5,0.3)
\uput[d](0.5,0.3){$Mg$}
% l/2 labels
\uput[ul](0.25,0.25){$l/2$}
\uput[ul](0.75,0.75){$l/2$}
% Reaction and friction
\psline{->}(0,0)(0,0.2)
\uput[u](0,0.2){$R$}
\psline{->}(0,0)(0.2,0)
\uput[r](0.2,0){$F$}
% The pivot point P
\uput[dl](0,0){$P$}
\end{pspicture}
\end{figure}
%----------------------------------------------------------------------------

%============================================================================
\begin{example}
\problem
A thin rod of mass $M$ and length $l$ falls over from an initial vertical
position on a rough table.  Solve the subsequent motion.

\solution
A diagram of the falling rod is given in figure \ref{rbm fig:fr}.  The rod's
equation of angular motion about $P$ is
$$I_P\ddot{\theta}=-Mg\frac{l}{2}\cos\theta$$
which can be written as
$$I_P\dbd{\theta}\left(\frac{1}{2}\dot{\theta}^2\right)=-Mg\frac{l}{2}
\cos\theta$$
Integrating and using the initial condition $\dot{\theta}=0$ at $\theta=\pi/2$
gives
$$\dot{\theta}^2=\frac{Mgl}{I_P}(1-\sin\theta)$$
Now from the parallel axes theorem, 
$$I_P=M\left(\frac{l}{2}\right)^2+\frac{1}{12}Ml^2=\frac{1}{3}Ml^2$$
so
$$\dot{\theta}^2=\left(\frac{3g}{l}\right)(1-\sin\theta)$$
Also, substituting for $I_P$ in the initial equation of angular motion
about $P$ gives
$$\ddot{\theta}=-\left(\frac{3g}{2l}\right)\cos\theta$$
The equations of motion of the centre of mass are
\begin{eqnarray*}
M\ddot{x}_{cm}&=&F\\
M\ddot{y}_{cm}&=&R-Mg
\end{eqnarray*}
If no slipping initially, then we have
\begin{eqnarray*}
x_{cm}&=&\frac{l}{2}\cos\theta\\
y_{cm}&=&\frac{l}{2}\sin\theta
\end{eqnarray*}
Therefore
$$\dot{x}_{cm}=-\frac{l}{2}\sin\theta\dot{\theta}$$
and
\begin{eqnarray*}
\ddot{x}_{cm}&=&-\frac{l}{2}\sin\theta\ddot{\theta}
-\frac{l}{2}\cos\theta\dot{\theta}^2 \\
&=&-\frac{l}{2}\left[
\sin\theta\left(-\frac{3g}{2l}\cos\theta\right)
+\cos\theta\left(\frac{3g}{l}(1-\sin\theta)\right)\right]\\
&=&\frac{3g}{4}\cos\theta(3\sin\theta-2)
\end{eqnarray*}
Similarly for $y_{cm}$,
$$\dot{y}_{cm}=\frac{l}{2}\cos\theta\dot{\theta}$$
and
\begin{eqnarray*}
\ddot{y}_{cm}&=&-\frac{l}{2}\sin\theta\dot{\theta}^2
+\frac{l}{2}\cos\theta\ddot{\theta} \\
&=&-\frac{l}{2}\left[
\sin\theta\left(\frac{3g}{l}(1-\sin\theta)\right)
-\cos\theta\left(-\frac{3g}{2l}\cos\theta\right)\right]\\
&=&-\frac{3g}{4}(1+2\sin\theta-3\sin^2\theta)
\end{eqnarray*}
Substituting into the equations of motion,
\begin{eqnarray*}
F &=& M\ddot{x}_{cm}\\
&=&\frac{3Mg}{4}\cos\theta(3\sin\theta-2)
\end{eqnarray*}
and
\begin{eqnarray*}
R&=&Mg-M\ddot{y}_{cm}\\
&=&\frac{Mg}{4}(1-6\sin\theta+9\sin^2\theta)
\end{eqnarray*}
Slipping starts at the angle $\theta^*$ where
$$F=\mu R$$
so that $\mu$ is given in terms of $\theta^*$ as
$$\mu=
\frac{3\cos\theta^*(3\sin\theta^*-2)}{1-6\sin\theta^*+9\sin^2\theta^*}$$
\end{example}
%============================================================================

%%%%%%%%%%%%%%%%%%%%%%%%%%%%%%%%%%%%%%%%%%%%%%%%%%%%%%%%%%%%%%%%%%%%%%%%%%%
%
%			Mathematics 132 Course Notes
%
%			 Department of Mathematics,
%   			  University of Melbourne
%
%		Stephen Simmons			Lee White
%
% 8 Feb-96 SS: Updated with corrections from semester 2, 1995
%
%%%%%%%%%%%%%%%%%%% Copyright (C) 1995-96 Stephen Simmons %%%%%%%%%%%%%%%%%

\chapter{Systems of Differential Equations}
\label{sde chp}


Systems of differential equations arise when a higher order differential
equation is written as a set of first order differential equations or when a
system is modelled as a set of coupled differential equations.

%============================================================================
\begin{example}
The second order differential equation 
$$\ddot{y}=f(t,y,\dot{y})$$
can be rewritten with $y_1=y$ and $y_2=\dot{y}$ as a set of coupled first
order \ODEs.
$$\dot{y}_1=y_2$$
$$\dot{y}_2=f(t,y_1,y_2)$$
\end{example}
%============================================================================

%============================================================================
\begin{example}[Volterra-Lotka Systems]
Volterra-Lotka systems are models of populations.  Let $y_1$ be the number
of foxes (the predators) and $y_2$ the number of rabbits (the prey).  Then
one model for the rabbits' and foxes' populations is
$$\dot{y}_1=-ay_1+by_1y_2$$
$$\dot{y}_2=cy_2-dy_1y_2$$
where $a$ sets the rate at which foxes die of starvation, $b$ the fox birth 
rate in the presence of a food supply, $c$ the rabbit birth rate and $d$ the
rate of rabbit deaths due to predation.  All $a$, $b$, $c$ and $d$ are
positive.

In matrix form, the system becomes
$$\begin{bmatrix}\dot{y}_1 \\\dot{y}_2\end{bmatrix}=
\begin{bmatrix}-a & by_1 \\ -dy_2 & c \end{bmatrix}
\begin{bmatrix}y_1 \\y_2\end{bmatrix}$$
\end{example}
%============================================================================

%----------------------------------------------------------------------------
\begin{figure}\centering
\caption{These coupled springs give the system of coupled differential
equations in example \protect\ref{sde ex:cs I}.  The two springs have spring
constants $k_1$ and $k_2$ respectively, and their unstretched lengths are
$l_1$ and $l_2$.}
\label{sde fig:cs}

\psset{unit=2.5cm}
\begin{pspicture}(-1,-0.4)(1,2.2)
% The line at which they are fixed
\psline[linecolor=darkgray,linewidth=2pt]{-}(-1,2)(1,2)
% The springs
\pnode(0.5,2){mC}
\cnodeput[fillstyle=solid,fillcolor=lightgray](0.5,1){mA}{$m_1$}
\cnodeput[fillstyle=solid,fillcolor=lightgray](0.5,0){mB}{$m_2$}
\nccoil[coilwidth=0.15]{-}{mC}{mA}
\nccoil[coilwidth=0.15]{-}{mA}{mB}
% The arrows
\psline{->}(0,2)(0,1)
\uput[r](0,1.5){$x_1$}
\psline{->}(-0.5,2)(-0.5,0)
\uput[r](-0.5,1){$x_2$}
\end{pspicture}
\end{figure}
%----------------------------------------------------------------------------

%============================================================================
\begin{example}[Coupled Springs I]
\label{sde ex:cs I}

For the system of coupled springs shown in figure \ref{sde fig:cs}, whose
unstretched lengths are $l_1$ and $l_2$, and whose spring constants are
$k_1$ and $k_2$, from Newton's laws
$$m_1\ddot{x_1}=m_1g-k_1(x_1-l_1)+k_2(x_2-x_1-l_2)$$
$$m_2\ddot{x_2}=m_2g-k_2(x_2-x_1-l_2)$$
In matrix form, this is
$$\begin{bmatrix}\ddot{x}_1 \\\ddot{x}_2\end{bmatrix}=
\begin{bmatrix} g+\frac{k_1l_1-k_2l_2}{m_1} 	\\
		  g+\frac{k_2l_2}{m_2}		\end{bmatrix}
-\begin{bmatrix}	\frac{k_1+k_2}{m_1} & -\frac{k_2}{m_1} 	 \\
			-\frac{k_2}{m_2} & \frac{k_2}{m_2}	 \end{bmatrix}
\begin{bmatrix}x_1 \\x_2\end{bmatrix}$$
\end{example}
%============================================================================

%============================================================================
\begin{example}[Chemical Reaction]
The following four chemical reactions occur between four chemicals, species 
1, species 2, $A$ and $B$:
\begin{enumerate}
\item $\mbox{species 1}\stackrel{k_1}{\rightarrow}\mbox{products}$
\item $\mbox{species 1}+A\stackrel{k_2}{\rightarrow}\mbox{species 2}+\mbox{products}$
\item $\mbox{species 2}\stackrel{k_3}{\rightarrow}\mbox{products}$
\item $\mbox{species 2}+B\stackrel{k_4}{\rightarrow}\mbox{species 1}+\mbox{products}$
\end{enumerate}
Then the concentrations of species 1, $c_1$ and of species 2, $c_2$, follow
$$\dot{c}_1=-k_1c_1-k_2[A]c_1+k_4[B]c_2$$
$$\dot{c}_2=-k_3c_2+k_2[A]c_1-k_4[B]c_2$$
where $[A]$ and $[B]$ are the concentrations of $A$ and $B$.

Suppose that $[A],[B]\gg c_1,c_2$ so that $[A]$ and $[B]$ hardly change
during the reaction.  Then
$$K_1=k_2[A]\qquad\mbox{and}\qquad K_2=k_4[B]$$
can be regarded as constant.  In matrix form
$$\begin{bmatrix}\dot{c}_1 \\\dot{c}_2\end{bmatrix}=
-\begin{bmatrix}     k_1+K_1 & -K_2 \\ -K_1 & k_3+K_2 	 \end{bmatrix}
\begin{bmatrix}c_1 \\c_2\end{bmatrix}$$
\end{example}
%============================================================================

%%%%%%%%%%%%%%%%%%%%%%%%%%%%%%%%%%%%%%%%%%%%%%%%%%%%%%%%%%%%%%%%%%%%%%%%%%%%%
\section{Systems of Linear First Order O.D.E.s}

The general form is
$$\dot{y}_1=a_{11}y_1+a_{12}y_2+f_1(t)$$
$$\dot{y}_2=a_{21}y_1+a_{22}y_2+f_2(t)$$
Expressed as matrices, this is
$$\begin{bmatrix}\dot{y}_1 \\\dot{y}_2\end{bmatrix}=
\begin{bmatrix}  a_{11} & a_{12} \\ a_{21} & a_{22} 	\end{bmatrix}
\begin{bmatrix}   y_1 \\ y_2				\end{bmatrix}
+\begin{bmatrix}  f_1(t) \\ f_2(t)			\end{bmatrix}$$
or
$$\vect{\dot{Y}}=\vect{A}\vect{Y}+\vect{F}$$
We will only consider systems where $\vect{A}$ is a constant matrix
independent of $t$.

%%%%%%%%%%%%%%%%%%%%%%%%%%%%%%%%%%%%%%%%%%%%%%%%%%%%%%%%%%%%%%%%%%%%%%%%%%%%%
\section{Homogeneous Systems}

The homogeneous system is
$$\vect{\dot{Y}}=\vect{A}\vect{Y}$$
where
$$\vect{A}=\begin{bmatrix}  a_{11} & a_{12} \\ a_{21} & a_{22} \end{bmatrix}
\qquad
\vect{Y}=\begin{bmatrix}   y_1 \\ y_2	\end{bmatrix}$$

%============================================================================
\begin{example}
If $a_{12}=a_{21}=0$, these equations decouple
$$\dot{y}_1=a_{11}y_1$$
$$\dot{y}_2=a_{22}y_2$$
with solution
$$\begin{bmatrix} y_1 \\ y_2 \end{bmatrix}=
\begin{bmatrix} C_1e^{a_{11}t} \\ C_2e^{a_{22}t} \end{bmatrix}$$
\end{example}
%============================================================================

The general solution is found by trying an exponential solution of the form
$$\vect{Y}(t)=\vect{K}e^{\alpha t}$$
where $\vect{K}$ and $\alpha$ are to be determined.  Now 
$$\vect{\dot{Y}}=\alpha\vect{K}e^{\alpha t}$$
so substituting into the \ODE system gives
$$\alpha\vect{K}e^{\alpha t}=\vect{A}\vect{K}e^{\alpha t}$$
Therefore we require
$$\vect{A}\vect{K}=\alpha\vect{K}$$
so $\alpha$ is an eigenvalue of $\vect{A}$ and $\vect{K}$ is the
corresponding eigenvector.

So the general method of solution is
\begin{enumerate}
\item Find the two eigenvalues of $\vect{A}$ and their corresponding
eigenvectors (which are linearly independent if $\alpha_1\neq \alpha_2$).
\item This gives two linearly independent solutions
$$\vect{Y}_1=\vect{K}_1e^{\alpha_1t}$$
$$\vect{Y}_2=\vect{K}_2e^{\alpha_2t}$$
\item The general solution is then
$$\vect{Y}=C_1\vect{K}_1e^{\alpha_1t}+C_2\vect{K}_2e^{\alpha_2t}$$
\end{enumerate}

%============================================================================
\begin{example}
To show that the same solution is obtained from a second order \ODE as from
the corresponding system, let
$$\dot{y}_1=a_{11}y_1+a_{12}y_2$$
$$\dot{y}_2=a_{21}y_1+a_{22}y_2$$
Then differentiate the first and substitute the second 
$$\ddot{y}_1=a_{11}\dot{y}_1+a_{12}\dot{y}_2
=a_{11}\dot{y}_1+a_{12}(a_{21}y_1+a_{22}y_2)$$
Then rearranging and using the first equation to eliminate $y_2$ gives
$$\ddot{y}_1=a_{11}\dot{y}_1+a_{12}a_{21}y_1+a_{22}(\dot{y}_1-a_{11}y_1)$$
This is the second order differential equation
$$\ddot{y}_1-(a_{11}+a_{22})\dot{y}_1+(a_{11}a_{22}-a_{12}a_{21})y_1=0$$
whose solution is
$$y_1=C_1e^{\alpha_1t}+C_2e^{\alpha_2t}$$
where $\alpha_1$ and $\alpha_2$ are the roots of
$$\alpha^2-(a_{11}+a_{22})\alpha+(a_{11}a_{22}-a_{12}a_{21})=0$$
\end{example}
%============================================================================

%============================================================================
\begin{example}
The system
$$\dot{y}_1=y_1+4y_2$$
$$\dot{y}_2=4y_1+y_2$$
gives
$$\vect{\dot{Y}}=\begin{bmatrix} 1 & 4 \\ 4 & 1 \end{bmatrix} \vect{Y}$$

The eigenvalues of $\vect{A}$ satisfy
$$\det \begin{bmatrix} 1-\alpha & 4 \\ 4 & 1-\alpha \end{bmatrix}=0$$
This gives the quadratic
$$(1-\alpha)^2-4^2=0$$
whose roots are $\alpha_1=5$ and $\alpha_2=-3$.

The eigenvectors are found by solving
$$(\vect{A}-\vect{I}\alpha)\vect{K}=0$$

When $\alpha_1=5$, this gives
$$\begin{bmatrix} -4 & 4 \\ 4 & -4\end{bmatrix}
\begin{bmatrix}k_1 \\ k_2 \end{bmatrix}=0$$
so that $k_1=k_2$ and the first eigenvector is
$$\vect{K}_1=\begin{bmatrix} 1 \\ 1 \end{bmatrix}$$

When $\alpha_2=-3$, this gives
$$\begin{bmatrix} 4 & 4 \\ 4 & 4\end{bmatrix}
\begin{bmatrix}k_1 \\ k_2 \end{bmatrix}=0$$
so that $k_1=-k_2$ and the second eigenvector is
$$\vect{K}_2=\begin{bmatrix} 1 \\ -1 \end{bmatrix}$$

Therefore the general solution is
$$\vect{Y}=C_1\begin{bmatrix} 1 \\ 1 \end{bmatrix}e^{5t}
+C_2\begin{bmatrix} 1 \\ -1 \end{bmatrix}e^{-3t}$$
The constants $C_1$ and $C_2$ may be determined by the initial conditions on
$y_1$ and $y_2$.
\end{example}
%============================================================================

This can be generalized to $n$ equations.  If $\vect{Y}$ has $n$ elements
and $\vect{A}$ is $n\times n$, the general solution is
$$\vect{Y}=C_1\vect{K}_1e^{\alpha_1t}+C_2\vect{K}_2e^{\alpha_2t}+\cdots
+C_n\vect{K}_ne^{\alpha_nt}$$
where
$$\vect{A}\vect{K}_i=\alpha_i\vect{K}_i$$

%%%%%%%%%%%%%%%%%%%%%%%%%%%%%%%%%%%%%%%%%%%%%%%%%%%%%%%%%%%%%%%%%%%%%%%%%%%%%
\subsection{Repeated Factors of the Characteristic Polynomial}

When two eigenvalues are equal, so that $\alpha_1=\alpha_2$, then one
solution is
$$\vect{Y}_1=\vect{K}_1e^{\alpha_1t}$$
A second solution can always be found of the form
$$\vect{Y}_2=\vect{K}_1te^{\alpha_1t}+\vect{P}e^{\alpha_1t}$$
To see this, differentiate to give
$$\vect{\dot{Y}}_2=\vect{K}_1e^{\alpha_1t}+\alpha_1\vect{K}_1te^{\alpha_1t}
+\alpha_1\vect{P}e^{\alpha_1t}$$
and substitute into $\vect{\dot{Y}}=\vect{A}\vect{Y}$, giving
$$e^{\alpha_1t}(\vect{K}_1+\alpha_1\vect{P}+\alpha_1\vect{K}_1t)
=e^{\alpha_1t}(\vect{A}\vect{K}_1t+\vect{A}\vect{P})$$
Equating coefficients of $e^{\alpha_1t}$ and $te^{\alpha_1t}$ shows that
$$\vect{A}\vect{K}_1=\alpha_1\vect{K}_1$$
which is just the eigenvalue/eigenvector equation, and
$$\vect{A}\vect{P}=\alpha_1\vect{P}+\vect{K}_1$$
which is equivalent to
$$(\vect{A}-\alpha_1\vect{I})\vect{P}=\vect{K}_1$$
This equation does not completely determine $\vect{P}$ because the
determinant of $\vect{A}-\alpha_1\vect{I}$ is zero.

%============================================================================
\begin{example}
The system
$$\vect{\dot{Y}}=\begin{bmatrix} 1 & 0 \\ 2 & 1 \end{bmatrix} \vect{Y}$$
has eigenvalues which satisfy
$$\det \begin{bmatrix} 1-\alpha & 0 \\ 2 & 1-\alpha \end{bmatrix}=0$$
which gives the quadratic
$$(1-\alpha)^2=0$$
so there is a single repeated eigenvalue $\alpha=1$.

The eigenvector for $\alpha=1$ satisfies
$$\begin{bmatrix} 0 & 0 \\ 2 & 0\end{bmatrix}
\begin{bmatrix}k_1 \\ k_2 \end{bmatrix}=0$$
so that $k_1=0$ and the eigenvector is
$$\vect{K}_1=\begin{bmatrix} 0 \\ 1 \end{bmatrix}$$
Thus the first solution is
$$\vect{Y}_1=\begin{bmatrix} 0 \\ 1 \end{bmatrix}e^t$$

The second solution has the form
$$\vect{Y}_2=\begin{bmatrix} 0 \\ 1 \end{bmatrix}te^t+\vect{P}e^t$$
where $\vect{P}$ satisfies
$$(\vect{A}-\alpha_1\vect{I})\vect{P}=\vect{K}_1$$
In this case, 
$$\begin{bmatrix} 0 & 0 \\ 2 & 0\end{bmatrix}
\begin{bmatrix} p_1 \\ p_2 \end{bmatrix}=
\begin{bmatrix} 0 \\ 1 \end{bmatrix}$$
This shows that $p_1=1/2$ but $p_2$ is not determined.  Choosing any
value of $p_2$, set $p_2=0$ so that the second solution is
$$\vect{Y}_2=\begin{bmatrix} 0 \\ 1 \end{bmatrix}te^t+
\begin{bmatrix}\frac{1}{2} \\ 0 \end{bmatrix}e^t$$

Therefore the general solution is
$$\vect{Y}=C_1\begin{bmatrix} 0 \\ 1 \end{bmatrix}e^t+C_2\left(
\begin{bmatrix} 0 \\ 1 \end{bmatrix}te^t+
\begin{bmatrix}\frac{1}{2} \\ 0 \end{bmatrix}e^t\right)$$

Note that choosing another value for $\vect{P}$ would just change the
constant $C_1$.  Suppose that $\vect{P}$ were
$$\vect{P}=\begin{bmatrix}\frac{1}{2} \\ a \end{bmatrix}=
\begin{bmatrix}\frac{1}{2} \\ 0 \end{bmatrix} + a\vect{K}_1$$
This gives the general solution
$$\vect{Y}=(C_1+aC_2)\begin{bmatrix} 0 \\ 1 \end{bmatrix}e^t+C_2\left(
\begin{bmatrix} 0 \\ 1 \end{bmatrix}te^t+
\begin{bmatrix}\frac{1}{2} \\ 0 \end{bmatrix}e^t\right)$$
which is the same as before except for the constant $C_1$.
\end{example}
%============================================================================

%%%%%%%%%%%%%%%%%%%%%%%%%%%%%%%%%%%%%%%%%%%%%%%%%%%%%%%%%%%%%%%%%%%%%%%%%%%%%
\subsection{Higher Order Degeneracy}

For $n\geq 3$, we have the possibility of $3$ degenerate eigenvalues.  We
have already found two solutions
$$\vect{Y}_1=\vect{K}_1e^{\alpha_1t}$$
$$\vect{Y}_2=\vect{K}_1te^{\alpha_1t}+\vect{P}e^{\alpha_1t}$$
where $(\vect{A}-\alpha_1\vect{I})\vect{P}=\vect{K}_1$.

A third linearly independent solution is
$$\vect{Y}_3=\vect{K}_1\frac{t^2}{2}e^{\alpha_1t}+\vect{P}te^{\alpha_1t}
+\vect{Q}e^{\alpha_1t}$$
where $(\vect{A}-\alpha_1\vect{I})\vect{Q}=\vect{P}$.

%============================================================================
\begin{example}
The system
$$\vect{\dot{Y}}=\begin{bmatrix}1&1&1\\0&1&1\\0&0&1\end{bmatrix}\vect{Y}$$
has a single eigenvalue $\alpha=1$ of multiplicity three.

The eigenvector $\vect{K}_1$ satisfies
$$\begin{bmatrix}0&1&1\\0&0&1\\0&0&0\end{bmatrix}
\begin{bmatrix}k_1\\k_2\\k_3\end{bmatrix}=\begin{bmatrix}0\\0\\0
\end{bmatrix}$$
so that
$$\vect{K}_1=\begin{bmatrix}1\\0\\0\end{bmatrix}$$

The vector $\vect{P}$ satisfies
$$\begin{bmatrix}0&1&1\\0&0&1\\0&0&0\end{bmatrix}
\begin{bmatrix}p_1\\p_2\\p_3\end{bmatrix}=\begin{bmatrix}1\\0\\0
\end{bmatrix}$$
so that choosing $p_1=0$ for simplicity gives
$$\vect{P}=\begin{bmatrix}p_1\\1\\0\end{bmatrix}
=\begin{bmatrix}0\\1\\0\end{bmatrix}$$

The vector $\vect{Q}$ satisfies
$$\begin{bmatrix}0&1&1\\0&0&1\\0&0&0\end{bmatrix}
\begin{bmatrix}q_1\\q_2\\q_3\end{bmatrix}=\begin{bmatrix}0\\1\\0
\end{bmatrix}$$
so that choosing $q_1=0$ for simplicity gives
$$\vect{Q}=\begin{bmatrix}q_1\\-1\\1\end{bmatrix}
=\begin{bmatrix}0\\-1\\1\end{bmatrix}$$

Therefore the general solution is
$$\vect{Y}=C_1\begin{bmatrix}1\\0\\0\end{bmatrix}e^t+C_2\left(
\begin{bmatrix}1\\0\\0\end{bmatrix}te^t+
\begin{bmatrix}0\\1\\0\end{bmatrix}e^t
\right)+C_3\left(
\begin{bmatrix}1\\0\\0\end{bmatrix}\frac{t^2}{2}e^t+
\begin{bmatrix}0\\1\\0\end{bmatrix}te^t+
\begin{bmatrix}0\\-1\\1\end{bmatrix}e^t
\right)$$
\end{example}
%============================================================================

%============================================================================
\begin{exercise}
Exercise 8.6 of Zill, pp. 466--469, has more examples of coupled linear
systems.
\end{exercise}
%============================================================================

%%%%%%%%%%%%%%%%%%%%%%%%%%%%%%%%%%%%%%%%%%%%%%%%%%%%%%%%%%%%%%%%%%%%%%%%%%%%%
\section{Inhomogeneous Simultaneous Systems}

An inhomogeneous simultaneous system has the form
$$\vect{\dot{Y}}=\vect{A}\vect{Y}+\vect{F}(t)$$
where $\vect{F}(t)$ is the inhomogeneous term.

Then general solution is
$$\vect{Y}=C_1\vect{Y}_1+C_2\vect{Y}_2+\cdots+C_n\vect{Y}_n
+\vect{Y}_{\rm ps}$$
where the $\vect{Y}_i$ are the $n$ linearly independent solutions of
$\vect{\dot{Y}}=\vect{A}\vect{Y}$ and $\vect{Y}_{\rm ps}$ is any particular
solution of the inhomogeneous problem.

$\vect{Y}_{\rm ps}$ is found using the method of undetermined coefficients.
If $\vect{F}(t)$ is a polynomial of degree $n$, try a polynomial of degree
$n$ for $\vect{Y}_{\rm ps}$.  If $\vect{F}(t)=\vect{C}e^{kt}$, try a
solution of the form $\vect{Y}_{\rm ps}=\vect{D}e^{kt}$.  Finally, if
$\vect{F}(t)=\vect{A}\sin\omega t+\vect{B}\cos\omega t$, try a solution of
the form $\vect{Y}_{\rm ps}=\vect{D}\sin\omega t+\vect{E}\cos\omega t$.


%============================================================================
\begin{example}
The coupled differential equations
$$\dot{y}_1=y_1+4y_2+\cos t$$
$$\dot{y}_2=4y_1+y_2+\sin t$$
gives the matrix system
$$\vect{\dot{Y}}=\begin{bmatrix}1&4\\4&1\end{bmatrix}\vect{Y}+
\begin{bmatrix}1\\0\end{bmatrix}\cos t+
\begin{bmatrix}0\\1\end{bmatrix}\sin t$$

The homogeneous problem has already been solved, so that
$$\vect{Y}_1=\begin{bmatrix}1\\1\end{bmatrix}e^{5t}$$
$$\vect{Y}_2=\begin{bmatrix}1\\-1\end{bmatrix}e^{-3t}$$
Try a particular solution of the form
$$\vect{Y}_{\rm ps}=\vect{D}\cos t+\vect{E}\sin t$$
whose derivative is
$$\vect{\dot{Y}}_{\rm ps}=-\vect{D}\sin t+\vect{E}\cos t$$
Substituting into the differential equation gives
$$-\vect{D}\sin t+\vect{E}\cos t=\vect{A}\vect{D}\cos t
+\vect{A}\vect{E}\sin t+
\begin{bmatrix}1\\0\end{bmatrix}\cos t+
\begin{bmatrix}0\\1\end{bmatrix}\sin t$$
where $\vect{A}=\begin{bmatrix}1&4\\4&1\end{bmatrix}$.  Equating
coefficients of $\cos t$ and $\sin t$ gives
$$-\vect{D}=\vect{A}\vect{E}+\begin{bmatrix}0\\1\end{bmatrix}$$
$$\vect{E}=\vect{A}\vect{D}+\begin{bmatrix}1\\0\end{bmatrix}$$
This is four equations in four unknowns
$$-d_1=e_1+4e_2$$
$$-d_2=4e_1+e_2+1$$
$$e_1=d_1+4d_2+1$$
$$e_2=4d_1+e_1$$
where $\vect{D}=\begin{bmatrix}d_1\\d_2\end{bmatrix}$ and
$\vect{E}=\begin{bmatrix}e_1\\e_2\end{bmatrix}$.  Solving for $d_1$, $d_2$,
$e_1$ and $e_2$ gives the general solution
$$\vect{Y}=C_1\begin{bmatrix}1\\1\end{bmatrix}e^{5t}+
C_2\begin{bmatrix}1\\-1\end{bmatrix}e^{-3t}+
\begin{bmatrix}d_1\\d_2\end{bmatrix}\cos t+
\begin{bmatrix}e_1\\e_2\end{bmatrix}\sin t$$
\end{example}
%============================================================================

%============================================================================
\begin{exercise}
Exercise 8.7 of Zill, pp. 472--473, has more examples of inhomogeneous 
simultaneous systems.
\end{exercise}
%============================================================================

%%%%%%%%%%%%%%%%%%%%%%%%%%%%%%%%%%%%%%%%%%%%%%%%%%%%%%%%%%%%%%%%%%%%%%%%%%%%%
\section{Coupled Oscillators}

A \name{coupled second order linear system} is one such as
$$\ddot{\vect{Y}}=-\vect{A}\vect{Y}$$
where
$$\vect{Y}=\begin{bmatrix}y_1\\y_2\\\vdots\end{bmatrix}\qquad
\ddot{\vect{Y}}=\begin{bmatrix}\ddot{y}_1\\\ddot{y}_2\\\vdots\end{bmatrix}$$
If the solution has the form
$$\vect{Y}=\vect{K}e^{i\lambda t}$$
then
$$\ddot{\vect{Y}}=-\lambda^2\vect{K}e^{i\lambda t}$$
Substituting in the original differential equation shows that
$$\vect{A}\vect{K}=\lambda^2\vect{K}$$
Thus $\lambda^2$ is an eigenvalue of $\vect{A}$ and $\vect{K}$ is its
corresponding eigenvector.  Taking the square root to recover $\lambda$
gives both a positive and a negative solution.  Therefore a $2\times 2$
second order system may have two eigenvalues and four values of $\lambda$.
The general solution is
$$\vect{Y}=\vect{K}_1\left(C_1e^{i\lambda_1t}+C_2e^{-i\lambda_1t}\right)
+\vect{K}_2\left(C_3e^{i\lambda_2t}+C_4e^{-i\lambda_2t}\right)$$
The four constants of integration are determined by the initial or boundary
conditions on $y_1$, $\dot{y_1}$, $y_2$ and $\dot{y_2}$.  Alternatively, the
solution can be expressed as
$$\vect{Y}=\vect{K}_1A\cos(\lambda_1t+\phi_1)+\vect{K}_2B\cos(\lambda_2t
+\phi_2)$$
where $A$, $B$, $\phi_1$ and $\phi_2$ are the constants to be determined.

%============================================================================
\begin{example}
To find the general solution of
$$\ddot{\vect{Y}}=-\begin{bmatrix}3&-1\\-2&2\end{bmatrix}\vect{Y}$$
the eigenvalues are found from
$$\det\begin{bmatrix}3-\alpha&-1\\-2&2-\alpha\end{bmatrix}=0$$
which is the quadratic
$$(3-\alpha)(2-\alpha)-2=0$$
This gives $\alpha=1,4$.

The eigenvector $\vect{K}_1$ corresponding to $\alpha=4$ satisfies
$$\begin{bmatrix}1&1\\2&2\end{bmatrix}
\begin{bmatrix}k_1\\k_2\end{bmatrix}=\begin{bmatrix}0\\0\end{bmatrix}$$
so that
$$\vect{K}_1=\begin{bmatrix}1\\-1\end{bmatrix}$$

The eigenvector $\vect{K}_2$ corresponding to $\alpha=1$ satisfies
$$\begin{bmatrix}-2&1\\2&-1\end{bmatrix}
\begin{bmatrix}k_1\\k_2\end{bmatrix}=\begin{bmatrix}0\\0\end{bmatrix}$$
so that
$$\vect{K}_2=\begin{bmatrix}1\\2\end{bmatrix}$$

Therefore the general solution is
$$\vect{Y}=\begin{bmatrix}1\\-1\end{bmatrix}A\cos(2t+\phi_1)
+\begin{bmatrix}1\\2\end{bmatrix}B\cos(t+\phi_2)$$
\end{example}
%============================================================================

Note that if the coupling terms in the previous example are turned off, we
would have
$$\ddot{\vect{Y}}=-\begin{bmatrix}3&0\\0&2\end{bmatrix}\vect{Y}$$
with solution
$$y_1=A_1\cos(\sqrt{3}t+\phi_1)$$
$$y_2=A_2\cos(\sqrt{2}t+\phi_2)$$
so that the uncoupled frequencies are $\sqrt{3}$ and $\sqrt{2}$.  When the
coupling is turned on, the system's frequencies are $1$ and $2$.  Note that
it is not just $y_1$ which oscillates with frequency $2$; both components
$y_1$ and $y_2$ exhibit a frequency $2$ contribution.

%----------------------------------------------------------------------------
\begin{figure}\centering
\caption{The rapid mode $\omega=\omega_{+}$ for the coupled springs has the 
two masses moving in opposite directions.}
\label{sde fig:cs II}

\psset{unit=2.5cm}
\begin{pspicture}(-2,-0.4)(2,2.2)
\rput(-1.5,0){
	\psline[linecolor=darkgray,linewidth=2pt]{-}(-0.3,2)(0.3,2)
	\pnode(0,2){mC}
	\cnodeput[fillstyle=solid,fillcolor=lightgray](0,1.2){mA}{$m_1$}
	\cnodeput[fillstyle=solid,fillcolor=lightgray](0,-0.2){mB}{$m_2$}
	\nccoil[coilwidth=0.15,coilheight=0.7]{-}{mC}{mA}
	\nccoil[coilwidth=0.15,coilheight=1.3]{-}{mA}{mB}
	\psline{->}(0.2,1.4)(0.2,1.0)
	\psline{->}(0.2,-0.4)(0.2,0.0)
}
\rput(-0.5,0){
	\psline[linecolor=darkgray,linewidth=2pt]{-}(-0.3,2)(0.3,2)
	\pnode(0,2){mC}
	\cnodeput[fillstyle=solid,fillcolor=lightgray](0,1){mA}{$m_1$}
	\cnodeput[fillstyle=solid,fillcolor=lightgray](0,0){mB}{$m_2$}
	\nccoil[coilwidth=0.15,coilheight=1]{-}{mC}{mA}
	\nccoil[coilwidth=0.15,coilheight=1]{-}{mA}{mB}
	\psline{<-}(0.2,0.8)(0.2,1.2)
	\psline{<-}(0.2,0.2)(0.2,-0.2)
}
\rput(0.5,0){
	\psline[linecolor=darkgray,linewidth=2pt]{-}(-0.3,2)(0.3,2)
	\pnode(0,2){mC}
	\cnodeput[fillstyle=solid,fillcolor=lightgray](0,1){mA}{$m_1$}
	\cnodeput[fillstyle=solid,fillcolor=lightgray](0,0){mB}{$m_2$}
	\nccoil[coilwidth=0.15,coilheight=1]{-}{mC}{mA}
	\nccoil[coilwidth=0.15,coilheight=1]{-}{mA}{mB}
	\psline{->}(0.2,0.8)(0.2,1.2)
	\psline{->}(0.2,0.2)(0.2,-0.2)
}
\rput(1.5,0){
	\psline[linecolor=darkgray,linewidth=2pt]{-}(-0.3,2)(0.3,2)
	\pnode(0,2){mC}
	\cnodeput[fillstyle=solid,fillcolor=lightgray](0,1.2){mA}{$m_1$}
	\cnodeput[fillstyle=solid,fillcolor=lightgray](0,-0.2){mB}{$m_2$}
	\nccoil[coilwidth=0.15,coilheight=0.7]{-}{mC}{mA}
	\nccoil[coilwidth=0.15,coilheight=1.3]{-}{mA}{mB}
	\psline{<-}(0.2,1.4)(0.2,1.0)
	\psline{<-}(0.2,-0.4)(0.2,0.0)
}
\end{pspicture}
\end{figure}
%----------------------------------------------------------------------------

%----------------------------------------------------------------------------
\begin{figure}\centering
\caption{The slow mode $\omega=\omega_{-}$ for the coupled springs has the 
two masses moving in the same direction.}
\label{sde fig:cs III}

\psset{unit=2.5cm}
\begin{pspicture}(-2,-0.8)(2,2.2)
\rput(-1.5,0){
	\psline[linecolor=darkgray,linewidth=2pt]{-}(-0.3,2)(0.3,2)
	\pnode(0,2){mC}
	\cnodeput[fillstyle=solid,fillcolor=lightgray](0,1.2){mA}{$m_1$}
	\cnodeput[fillstyle=solid,fillcolor=lightgray](0,0.2){mB}{$m_2$}
	\nccoil[coilwidth=0.15,coilheight=0.7]{-}{mC}{mA}
	\nccoil[coilwidth=0.15,coilheight=1]{-}{mA}{mB}
	\psline{->}(0.2,1)(0.2,1.4)
	\psline{->}(0.2,0)(0.2,0.4)
}
\rput(-0.5,0){
	\psline[linecolor=darkgray,linewidth=2pt]{-}(-0.3,2)(0.3,2)
	\pnode(0,2){mC}
	\cnodeput[fillstyle=solid,fillcolor=lightgray](0,0.8){mA}{$m_1$}
	\cnodeput[fillstyle=solid,fillcolor=lightgray](0,-0.2){mB}{$m_2$}
	\nccoil[coilwidth=0.15,coilheight=1.3]{-}{mC}{mA}
	\nccoil[coilwidth=0.15,coilheight=1]{-}{mA}{mB}
	\psline{->}(0.2,1.0)(0.2,0.6)
	\psline{->}(0.2,0.0)(0.2,-0.4)
}
\rput(0.5,0){
	\psline[linecolor=darkgray,linewidth=2pt]{-}(-0.3,2)(0.3,2)
	\pnode(0,2){mC}
	\cnodeput[fillstyle=solid,fillcolor=lightgray](0,1.2){mA}{$m_1$}
	\cnodeput[fillstyle=solid,fillcolor=lightgray](0,0.2){mB}{$m_2$}
	\nccoil[coilwidth=0.15,coilheight=0.7]{-}{mC}{mA}
	\nccoil[coilwidth=0.15,coilheight=1]{-}{mA}{mB}
	\psline{->}(0.2,1)(0.2,1.4)
	\psline{->}(0.2,0)(0.2,0.4)
}
\rput(1.5,0){
	\psline[linecolor=darkgray,linewidth=2pt]{-}(-0.3,2)(0.3,2)
	\pnode(0,2){mC}
	\cnodeput[fillstyle=solid,fillcolor=lightgray](0,0.8){mA}{$m_1$}
	\cnodeput[fillstyle=solid,fillcolor=lightgray](0,-0.2){mB}{$m_2$}
	\nccoil[coilwidth=0.15,coilheight=1.3]{-}{mC}{mA}
	\nccoil[coilwidth=0.15,coilheight=1]{-}{mA}{mB}
	\psline{->}(0.2,1.0)(0.2,0.6)
	\psline{->}(0.2,0.0)(0.2,-0.4)
}
\end{pspicture}
\end{figure}
%----------------------------------------------------------------------------

%============================================================================
\begin{example}[Coupled Springs II]

The coupled spring system of example \ref{sde ex:cs I} can be written as
$$\ddot{\vect{X}}=-\vect{A}\vect{X}+\vect{F}$$
where $\vect{A}$ is
$$\vect{A}=\begin{bmatrix}	\frac{k_1+k_2}{m_1} & -\frac{k_2}{m_1} 	 \\
  -\frac{k_2}{m_2} & \frac{k_2}{m_2}  \end{bmatrix} $$
and $\vect{F}$ is a constant
$$\vect{F}=\begin{bmatrix} g+\frac{k_1l_1-k_2l_2}{m_1} \\  
  g+\frac{k_2l_2}{m_2}\end{bmatrix}$$

The eigenvalues of $\vect{A}$ are found by solving
$$\det\begin{bmatrix} \frac{k_1+k_2}{m_1}-\alpha & -\frac{k_2}{m_1} \\
  -\frac{k_2}{m_2} & \frac{k_2}{m_2}-\alpha \end{bmatrix} =0$$
which gives the quadratic
$$\alpha^2-\left(\frac{k_1+k_2}{m_1}+\frac{k_2}{m_2}\right)\alpha+
\frac{k_1k_2}{m_1m_2}=0$$
The solutions are
$$\alpha=\frac{1}{2}\left(\frac{k_1+k_2}{m_1}+\frac{k_2}{m_2}\pm\sqrt{\left(
\frac{k_1+k_2}{m_1}+\frac{k_2}{m_2}\right)^2-4\frac{k_1k_2}{m_1m_2}}
\right)$$
Denote these two solutions $\omega_+^2$ and $\omega_-^2$.

Therefore the system of coupled springs can oscillate in two distinct
\name{modes}
$$\vect{X}_1=\vect{K}_1\cos(\omega_{+}t+\phi_1)$$
and
$$\vect{X}_2=\vect{K}_2\cos(\omega_{-}t+\phi_2)$$
and the general motion is the linear combination
$$\vect{X}=A_1\vect{K}_1\cos(\omega_{+}t+\phi_1)
+A_2\vect{K}_2\cos(\omega_{-}t+\phi_2)$$
where the values of $A_1$, $A_2$, $\phi_1$ and $\phi_2$ depend on how the
system is started off.

For simplicity, consider equal spring constants $k_1=k_2=k$ and equal masses
$m_1=m_2=m$.   Then the system is described by
$$\ddot{\vect{X}}=-\omega_0^2
\begin{bmatrix} 2&-1\\-1&1\end{bmatrix}\vect{X}$$
where $\omega_0=\sqrt{k/m}$ is the natural frequency of the single spring and
mass system.  

The eigenvalues are
$$\omega_{\pm}^2=\frac{3k}{2m}\pm\frac{1}{2}\sqrt{\left(\frac{3k}{m}\right)^2
-4\frac{k^2}{m^2}}=\frac{1}{2}(3\pm\sqrt{5})\frac{k}{m}$$

The eigenvector for $\omega_{+}$ satisfies
$$\begin{bmatrix} 2-\frac{1}{2}(3+\sqrt{5})&-1\\-1&1-\frac{1}{2}
(3+\sqrt{5})\end{bmatrix}
\begin{bmatrix} k_1\\k_2\end{bmatrix}=
\begin{bmatrix} 0\\0\end{bmatrix}$$
so that
$$\vect{K}_{+}=\begin{bmatrix} 1-\frac{1}{2}(3+\sqrt{5})\\1\end{bmatrix}$$
where $1-\frac{1}{2}(3+\sqrt{5})<0$.  In this mode,
$$x_1(t)=-\left(\frac{1}{2}(3+\sqrt{5})-1\right)x_2(t)$$
so when the second spring is extended ($x_2(t)>0$), the first spring is
compressed ($x_1(t)<0$), and vice versa.  This is shown in figure
\ref{sde fig:cs II}.


The eigenvector for $\omega_{-}$ satisfies
$$\begin{bmatrix} 2-\frac{1}{2}(3-\sqrt{5})&-1\\-1&1-\frac{1}{2}
(3-\sqrt{5})\end{bmatrix}
\begin{bmatrix} k_1\\k_2\end{bmatrix}=
\begin{bmatrix} 0\\0\end{bmatrix}$$
so that
$$\vect{K}_{-}=\begin{bmatrix} 1-\frac{1}{2}(3-\sqrt{5})\\1\end{bmatrix}$$
where $1-\frac{1}{2}(3-\sqrt{5})>0$.  In this mode,
$$x_1(t)=-\left(\frac{1}{2}(3-\sqrt{5})-1\right)x_2(t)$$
so the two springs move in the same direction.  This is shown in figure
\ref{sde fig:cs III}.
\end{example}
%============================================================================

%----------------------------------------------------------------------------
\begin{figure}\centering
\caption{The double pendulum has two segments, each of length $l$ with
weights of mass $m$ at the end.}
\label{sde fig:dp}

\psset{unit=3cm}
\begin{pspicture}(-0.1,-1.5)(2.25,1)
\SpecialCoor
% The pendulum's base
\psline[linecolor=darkgray,linewidth=2pt]{-}(-0.116,0.816)(0.484,0.816)
% The pendulum itself
\qdisk(1,0){3pt}
\qdisk(2,-0.577){3pt}
\psline[linecolor=gray,linewidth=2pt]{-}(0.184,0.816)(1,0)(2,-0.577)
% Angle at the base
\psarc{->}(0.184,0.816){0.2}{270}{315}
\uput[d](0.284,0.666){$\theta$}
\psline[linecolor=black,linestyle=dashed]{-}(0.184,0.816)(0.184,0.6) 
% Angle at the midpoint
\psarc{->}(1,0){0.2}{270}{330}
\uput[d](1.1,-0.15){$\phi$}
\psline[linecolor=black,linestyle=dashed]{-}(1,0)(1,-0.4) 
% Two weight vectors
\psline{->}(1,0)(1,-0.4)
\uput[d](1,-0.4){$mg$}
\psline{->}(2,-0.577)(2,-0.977)
\uput[d](2,-0.977){$mg$}
% Tension at the midpoint
\put(1,0){
	\psline{->}(0,0)(0.3;135)
	\uput[ur](0.3;135){$T_1$}
	\psline{->}(0,0)(0.3;-30)
	\uput[u](0.3;-30){$T_2$}
}
% Tension at the end
\put(2,-0.577){
	\psline{->}(0,0)(0.3;150)
	\uput[u](0.3;150){$T_2$}
}
% The two arrows
\pcline{->}(0.184,-0.6)(1,-0.6)
\Aput{$x_1$}
\pcline{->}(0.184,-1.2)(2,-1.2)
\Aput{$x_2$}
\end{pspicture}
\end{figure}
%----------------------------------------------------------------------------

%============================================================================
\begin{example}[Double Pendulum]
In the diagram of the double pendulum in figure \ref{sde fig:dp}, assume
that $\phi$ and $\theta$ are small so that the masses hardly move
vertically.  Therefore the vertical forces on each mass should balance,
giving
$$T_1\cos\theta=mg+T_2\cos\phi$$
$$T_2\cos\phi=mg$$
If $\phi$ and $\theta$ are small, their cosines are approximately unity, so
that
$$T_1=mg+T_2$$
$$T_2=mg$$
which shows that $T_1=2mg$.  Now 
$$x_1=l\sin\theta\approx l\theta$$
and
$$x_2=x_1+l\sin\phi\approx l(\theta+\phi)$$

In the horizontal direction on particle 1,
$$m\ddot{x}_1=-T_1\sin\theta+T_2\sin\phi$$
Thus
$$\ddot{\theta}=-\frac{2g}{l}\theta+\frac{g}{l}\phi$$

In the horizontal direction on particle 2,
$$m\ddot{x}_2=-T_2\sin\phi$$
Thus
$$\ddot{\phi}=\frac{2g}{l}\theta-\frac{2g}{l}\phi$$

These can be written as the system
$$\begin{bmatrix}\ddot{\theta}\\\ddot{\phi}\end{bmatrix}
=-\omega_0^2\begin{bmatrix}2&-1\\-2&2\end{bmatrix}
\begin{bmatrix}\theta\\\phi\end{bmatrix}$$
where $\omega_0=\sqrt{g/l}$ is the natural frequency of a single pendulum.

The general solution is 
$$\begin{bmatrix}\theta\\\phi\end{bmatrix}
=C_1\vect{K}_1\cos(\omega_1t+\phi_1)
+C_2\vect{K}_2\cos(\omega_2t+\phi_2)$$
with
$$\omega_1=\omega_0\sqrt{\alpha_1}$$
$$\omega_2=\omega_0\sqrt{\alpha_2}$$
where $\alpha_1$ and $\alpha_2$ are the eigenvalues of 
$\begin{bmatrix}2&-1\\-2&2\end{bmatrix}$ and $\vect{K}_1$ and
$\vect{K}_2$ are the corresponding eigenvectors.

The eigenvalues are found by solving
$$\det\begin{bmatrix}2-\alpha&-1\\-2&2-\alpha\end{bmatrix}=0$$
which gives the quadratic
$$(2-\alpha)^2-2=0$$
The solutions are $\alpha_1=2+\sqrt{2}$ and $\alpha_2=2-\sqrt{2}$.

The eigenvector for $\alpha_1=2+\sqrt{2}$ satisfies
$$\begin{bmatrix} -\sqrt{2}&-1\\-2&-\sqrt{2}\end{bmatrix}
\begin{bmatrix} k_1\\k_2\end{bmatrix}=
\begin{bmatrix} 0\\0\end{bmatrix}$$
so that
$$\vect{K}_1=\begin{bmatrix} 1\\-\sqrt{2}\end{bmatrix}$$

The eigenvector for $\alpha_2=2-\sqrt{2}$ satisfies
$$\begin{bmatrix} \sqrt{2}&-1\\-2&\sqrt{2}\end{bmatrix}
\begin{bmatrix} k_1\\k_2\end{bmatrix}=
\begin{bmatrix} 0\\0\end{bmatrix}$$
so that
$$\vect{K}_2=\begin{bmatrix} 1\\\sqrt{2}\end{bmatrix}$$

This gives two modes of operation of the double pendulum.  The high
frequency mode is $\omega_1=\omega_0\sqrt{2+\sqrt{2}}$ so that
$$\begin{bmatrix}\theta\\\phi\end{bmatrix}_1
=\begin{bmatrix} 1\\-\sqrt{2}\end{bmatrix}
\cos(\omega_{1}t+\phi_1)$$
Here $\phi$ and $\theta$ have opposite signs, so the two halves of the
double pendulum move in opposite directions.

The low frequency mode is $\omega_2=\omega_0\sqrt{2-\sqrt{2}}$ so that
$$\begin{bmatrix}\theta\\\phi\end{bmatrix}_2
=\begin{bmatrix} 1\\\sqrt{2}\end{bmatrix}
\cos(\omega_{2}t+\phi_1)$$
Here $\phi$ and $\theta$ have the same sign, so the two halves of the
double pendulum move in the same direction.
\end{example}
%============================================================================



%%%%%%%%%%%%%%%%%%%%%%%%%%%%%%%%%%%%%%%%%%%%%%%%%%%%%%%%%%%%%%%%%%%%%%%%%%%
%
%			Mathematics 132 Course Notes
%
%			 Department of Mathematics,
%   			  University of Melbourne
%
%		Stephen Simmons			Lee White
%
% 8 Feb-96 SS: Updated with corrections from semester 2, 1995
%
%%%%%%%%%%%%%%%%%%% Copyright (C) 1995-96 Stephen Simmons %%%%%%%%%%%%%%%%%

\chapter[Non-Linear Coupled 1st Order Equations]{Non-Linear Coupled 
First Order Equations}
\label{nlc chp}

All \ODEs of any order can be written in the autonomous form
$$\dot{\vect{Y}}=\vect{F}(\vect{Y})$$
To see why, consider the general $(n-1)$th order \ODE 
$$0=F(t, Y, Y', Y'', \ldots, Y^{(n-1)})$$
This can, in principle, be written in the alternative form
$$t=G(Y, Y', Y'', \ldots, Y^{(n-1)})$$
Now differentiate with respect to $t$, giving
$$1=\sum_{i=0}^{n-1} \frac{\del G}{\del Y^{(i)}}\,\der{Y^{(i)}}{t}$$
Since $\der{Y^{(i)}}{t}=Y^{(i+1)}$, this can be written
$$1=\sum_{i=1}^n \frac{\del G}{\del Y^{(i-1)}}\,Y^{(i)}$$
by changing the limits of $i$ in the summation.  

Then rearranging this equation gives $Y^{(n)}$ in terms of the lower-order
derivatives
$$Y^{(n)}=\Phi(Y, Y',Y'',\ldots, Y^{(n-1)})$$
This is referred to as an \name{autonomous system} because there is now no
explicit dependence on time $t$.

Finally, define the set of variables
$$y_i=Y^{(i-1)}\qquad\mbox{for $i=1,\ldots,n$}$$
so that
\begin{eqnarray*}
\dot{y}_1=&Y'&=y_2\\
\dot{y}_2=&Y''&=y_3\\
&\vdots&\\
\dot{y}_n=&Y^{(n)}&=\Phi(y_1,y_2,\ldots,y_{n-1})
\end{eqnarray*}
These relations can written more generally as
\begin{eqnarray*}
\dot{y}_1&=&f_1(y_1,y_2,\ldots,y_{n})\\
\dot{y}_2&=&f_2(y_1,y_2,\ldots,y_{n})\\
&\vdots&\\
\dot{y}_n&=&f_n(y_1,y_2,\ldots,y_{n})
\end{eqnarray*}
which can be expressed more concisely as a set of $n$ coupled first-order
differential equations of a vector argument
$$\dot{\vect{Y}}=\vect{F}(\vect{Y})$$

%%%%%%%%%%%%%%%%%%%%%%%%%%%%%%%%%%%%%%%%%%%%%%%%%%%%%%%%%%%%%%%%%%%%%%%%%%%%%
\section{Phase Space and Trajectories}

Consider the coupled system for $n=2$:
$$\begin{bmatrix} \dot{y}_1 \\ \dot{y}_2\end{bmatrix} =
\begin{bmatrix} f_1(y_1,y_2) \\ f_2(y_1,y_2)\end{bmatrix}$$
Solve this to obtain expressions for $y_1(t)$ and $y_2(t)$.  Then, for each 
time $t$, plot $(y_1,y_2)$.  This gives a \name{trajectory} of points
$(y_1(t),y_2(t))$ through the $(y_1,y_2)$ \name{phase space}.

%----------------------------------------------------------------------------
\begin{window}[0,l,{%
% General trajectory
\psset{unit=2.5cm}
\begin{pspicture}(-0.4,-0.3)(1.4,1.5)
%\psframe(-0.4,-0.3)(1.4,1.5)
% First the curve, so the axes show through
\psplot[linecolor=gray,linewidth=1.5pt,plotstyle=curve]%
{0}{1.1}{x 1.5 sub x mul 0.66 add x mul 0.1 add 3 mul}
% Now the axes and their labels
\psset{linewidth=1.2pt,linecolor=black}
\psline{->}(-0.3,0)(1.1,0)
\psline{->}(-0.1,-0.2)(-0.1,1.2)
\uput[r](1.1,0){$y_1$}
\uput[u](-0.1,1.2){$y_2$}
% Then the arrows on the trajectory
\rput{63}(0,0.3){\psline[linewidth=1.2pt]{->}(0,0)(0.3,0)\rput[tl]{*0}(0.1,-0.1){$t_0$}}
\put(0.333,0.5778){\psline[linewidth=1.2pt]{->}(0,0)(0.3,0)\rput[br](0.1,0.1){$t_1$}}
\put(0.667,0.5222){\psline[linewidth=1.2pt]{->}(0,0)(0.3,0)\rput[br](0.1,0.1){$t_2$}}
\rput{63}(1,0.8){\psline[linewidth=1.2pt]{->}(0,0)(0.3,0)\rput[br]{*0}(0.1,0.1){$t_3$}}
\end{pspicture}
},{}]
The diagram at the left shows a trajectory through phase space starting at the initial
position $(y_1(t_0),y_2(t_0))$.  The arrows indicate the direction of motion
along the trajectory at different times.
Note that there is always only one trajectory for any set of initial
conditions $(y_1(t_0),y_2(t_0))$.  This follows from the uniqueness of the
solution of the differential equations.
In this chapter, we will restrict ourselves to two-dimensional systems
($n=2$) for ease of graphical interpretation.  However the concept of phase
space and trajectories holds in higher dimensions.
\end{window}
%----------------------------------------------------------------------------

%============================================================================
\begin{example}
The coupled system for $y''+y=0$ is found by setting
$y_1=y$ and $y_2=\dot{y}$.  Then 
\begin{eqnarray*}
\dot{y}_1&=&y_2\\
\dot{y}_2&=&\ddot{y}=-y_1
\end{eqnarray*}
so that the coupled system is
\begin{eqnarray*}
\dot{y}_1&=&y_2\\
\dot{y}_2&=&-y_1
\end{eqnarray*}
The solutions have the form
\begin{eqnarray*}
y_1&=&A\sin(t+\phi)\\
y_2&=&A\cos(t+\phi)
\end{eqnarray*}
which gives
$$y_1^2+y_2^2=A^2$$
when $t$ is eliminated.  Therefore the phase portrait consists of  
concentric circles about the origin starting at the initial value $(y_1(t_0),
y_2(t_0))$.  The movement of each trajectory is clockwise because when
$y_1>0$ and $y_2=0$, $\dot{y}_2<0$.
\begin{center}
% Circular phase portraits
\phaseportrait{clsinit 0.02 0 1 -1 0 100 0.42 0.42 cls 200 0.707 0.707 cls}{}
\end{center}
\end{example}
%============================================================================

%%%%%%%%%%%%%%%%%%%%%%%%%%%%%%%%%%%%%%%%%%%%%%%%%%%%%%%%%%%%%%%%%%%%%%%%%%%%%
\subsection{Critical Points in Phase Space}

A point $(y_1, y_2, \ldots, y_n)$ where the $n$ equations
$$f_k(y_1, y_2, \ldots, y_n)=0\qquad\mbox{for $k=1,2,\ldots,n$}$$
are simultaneously satisfied is a \name{critical point} of the system.  The
trajectory which passes through a critical point is just the point itself
since the phase point has zero velocity.

A trajectory can approach a critical point arbitrarily closely as
$t\to\pm\infty$ but can never reach the critical point in finite time.

%===========================================================================
\begin{example}
\label{nlc: ex vl}

\problem
Find the critical points of the Volterra-Lotka system
\begin{eqnarray*}
\dot{y}_1&=&-ay_1+by_1y_2\\
\dot{y}_2&=&cy_2-dy_1y_2
\end{eqnarray*}

\solution
The critical points are found by setting $(\dot{y}_1,\dot{y}_2)=(0,0)$.
Now
\begin{eqnarray*}
\dot{y}_1=0&\mif& y_1=0\mbox{\ or\ }y_2=a/b\\
\dot{y}_2=0&\mif& y_2=0\mbox{\ or\ }y_1=c/d
\end{eqnarray*}
so the two critical points are $(0,0)$ and $(c/d,a/b)$.
\end{example}
%============================================================================


%%%%%%%%%%%%%%%%%%%%%%%%%%%%%%%%%%%%%%%%%%%%%%%%%%%%%%%%%%%%%%%%%%%%%%%%%%%%%
\section{Classification of Critical Points}

Critical points are named according to the trajectory's behaviour in the
neighbourhood of the critical point.  For $n=2$, there are six basic
behaviours:

\begin{description}
\item[Stable node:] Trajectories approach the critical point along an
asymptotically straight line.
\item[Stable spiral:] Trajectories approach the critical point along a
spiral.
\item[Unstable node:] Trajectories move away from the critical point along a
straight line.
\item[Unstable spiral:] Trajectories move away from the critical point 
along a spiral.
\item[Saddle Point:] Some trajectories approach; some move away.
\item[Centre:] Trajectories orbit around the critical point.
\end{description}

%%%%%%%%%%%%%%%%%%%%%%%%%%%%%%%%%%%%%%%%%%%%%%%%%%%%%%%%%%%%%%%%%%%%%%%%%%%%%
\section{Classification for Linear Systems}

For the linear system
$$\dot{\vect{Y}}=\vect{A}\vect{Y}$$
there is one critical point, at $(0,0)$.  The general solution for distinct
eigenvalues is
$$\begin{bmatrix} y_1 \\ y_2 \end{bmatrix} = 
C_1 \begin{bmatrix} \phi_1 \\ \psi_1 \end{bmatrix} e^{\alpha_1 t}
+C_2 \begin{bmatrix} \phi_2 \\ \psi_2 \end{bmatrix} e^{\alpha_2 t}$$
where $\alpha_1$ and $\alpha_2$ are the eigenvalues of $\vect{A}$ whose
corresponding eigenvectors are
$$\vect{K}_{\alpha_1}=\begin{bmatrix} \phi_1 \\ \psi_1 \end{bmatrix}
\qquad\vect{K}_{\alpha_2}=\begin{bmatrix} \phi_2 \\ \psi_2 \end{bmatrix}$$
The fundamental behaviour of the critical point depends on the values of
$\alpha_1$ and $\alpha_2$.

%%%%%%%%%%%%%%%%%%%%%%%%%%%%%%%%%%%%%%%%%%%%%%%%%%%%%%%%%%%%%%%%%%%%%%%%%%%%%
\subsection{Case 1: $\alpha_1>\alpha_2>0$}

If $C_2=0$, the solution is
$$\vect{Y}=C_1 \begin{bmatrix} \phi_1 \\ \psi_1\end{bmatrix} e^{\alpha_1 t}$$
so that
$$y_2=\frac{\psi_1}{\phi_1}y_1$$
This trajectory is a line with slope $\psi_1/\phi_1$ moving away from the
origin.

Similarly, if $C_1=0$, the trajectory is a line with slope $\psi_2/\phi_2$
moving away from the origin.  

\begin{center}
	\phaseportrait{clsinit 0.02 2 1 1 2   50 0.5 0.5 cls    
  50 -0.5 0.5 cls    50 0.5 -0.5 cls    50 -0.5 -0.5 cls}
  {\uput[u](-1,1){slope=$\frac{\psi_2}{\phi_2}$}
   \uput[u](1,1){slope=$\frac{\psi_1}{\phi_1}$}}
\end{center}

In the general case, the gradient of the trajectory is 
$$\der{y_2}{y_1}=\frac{\dot{y}_2}{\dot{y}_1}=
\frac{\alpha_1C_1\psi_1e^{\alpha_1t}+\alpha_2C_2\psi_2e^{\alpha_2t}}
{\alpha_1C_1\phi_1e^{\alpha_1t}+\alpha_2C_2\phi_2e^{\alpha_2t}}$$
As $t\to+\infty$, the gradient tends to $\psi_1/\phi_1$.  As $t\to-\infty$,
the gradient tends to $\psi_2/\phi_2$.

\begin{center}
\phaseportrait
{      clsinit 0.02 2 1 1 2
      50 0.5 0.5 cls    50 -0.5 0.5 cls    50 0.5 -0.5 cls    50 -0.5 -0.5 cls 
      50 0.9 0.1 cls	50 1 -0.3 cls	   50 0.1 0.9 cls     50 -0.3 1 cls
      50 -0.9 -0.1 cls	50 -1 0.3 cls	   50 -0.1 -0.9 cls   50 0.3 -1 cls
}
{}
\end{center}

Since the trajectories move away from the origin asymptotically along a
straight line, this is an \textbf{unstable node}.


%%%%%%%%%%%%%%%%%%%%%%%%%%%%%%%%%%%%%%%%%%%%%%%%%%%%%%%%%%%%%%%%%%%%%%%%%%%%%
\subsection{Case 2: $\alpha_1<\alpha_2<0$}

This is similar to case 1, except that the arrows point the other direction.
Since the trajectories move towards the origin, this is a \textbf{stable node}.

\begin{center}
\phaseportrait
{clsinit 0.02 -2 -1 -1 -2
      50 0.5 0.5 cls    50 -0.5 0.5 cls    50 0.5 -0.5 cls    50 -0.5 -0.5 cls 
      50 0.9 0.1 cls	50 1 -0.3 cls	   50 0.1 0.9 cls     50 -0.3 1 cls
      50 -0.9 -0.1 cls	50 -1 0.3 cls	   50 -0.1 -0.9 cls   50 0.3 -1 cls
}
{}
\end{center}

%%%%%%%%%%%%%%%%%%%%%%%%%%%%%%%%%%%%%%%%%%%%%%%%%%%%%%%%%%%%%%%%%%%%%%%%%%%%%
\subsection{Case 3: $\alpha_1>0>\alpha_2$}

As $t\to\infty$, the trajectories' gradients tend to $\psi_1/\phi_1$, and as
$t\to-\infty$, their gradients tend to $\psi_2/\phi_2$, similar to cases 1
and 2.  However, as $t\to\pm\infty$, $y_1$ and $y_2$ also tend to
$\pm\infty$.  

This gives a \textbf{saddle point}.

\begin{center}
\phaseportrait{
      clsinit 0.02 0 2 2 0
      50 0.5 0.5 cls    50 -0.5 0.5 cls    50 0.5 -0.5 cls    50 -0.5 -0.5 cls 
      80 0 0.4 cls	80 0 0.8 cls	   80 0 -0.8 cls      80 0 -0.4 cls
      80 0.4 0 cls	80 0.8 0 cls	   80 -0.8 0 cls      80 -0.4 0 cls
      80 0 1.2 cls	80 0 -1.2 cls	   80 1.2 0 cls	      80 -1.2 0 cls}{
}
\end{center}

%%%%%%%%%%%%%%%%%%%%%%%%%%%%%%%%%%%%%%%%%%%%%%%%%%%%%%%%%%%%%%%%%%%%%%%%%%%%%
\subsection{Case 4: $\alpha_1$ and $\alpha_2$ are imaginary}

When $\alpha_1=i\eta$ and $\alpha_2=-i\eta$, the solution is
$$\vect{Y}= C_1 \begin{bmatrix} \phi_1 \\ \psi_1 \end{bmatrix} e^{i\eta t}
+C_2 \begin{bmatrix} \phi_2 \\ \psi_2 \end{bmatrix} e^{-i\eta t}$$
For real solutions, we require that $C_1=C_2^*$.  Therefore, there are only
two arbitrary constants---the real and imaginary parts of $C_1$.
Then the solution is
$$\vect{Y}=
\begin{bmatrix} A_1 \\ A_2 \end{bmatrix} \cos\eta t
+\begin{bmatrix} B_1\\ B_2 \end{bmatrix} \sin\eta t$$
where
$$\begin{bmatrix} A_1 \\ A_2 \end{bmatrix} = 
2\Re \begin{bmatrix} C_1\phi_1 \\ C_1\psi_1 \end{bmatrix}$$
$$\begin{bmatrix} B_1 \\ B_2 \end{bmatrix} = 
-2\Im \begin{bmatrix} C_1\phi_1 \\ C_1\psi_1 \end{bmatrix}$$
Upon eliminating $t$ from the solution, we have
$$(B_2y_1-B_1y_2)^2 + (A_2y_1-A_1y_2)^2 = (A_1B_2-A_2B_1)^2$$
This is the equation of an ellipse.  The orientation of the ellipse's axes
are determined by $\phi_1$, $\psi_1$, $\phi_2$ and $\psi_2$.

Since the trajectories orbit the origin, this is a \textbf{centre}.

\begin{center}
\phaseportrait
{      clsinit 0.01 1 -2 5 -1
      100 0.2 0.2 cls    200 0.4 0.4 cls    300 0.6 0.6 cls}
{}
\end{center}

To obtain a rough idea of the ellipse's shape, examine $\der{y_2}{y_1}$ for
$y_1=0$ and for $y_2=0$.  Determine the sense of the rotation (clockwise or
anticlockwise) by examining $\dot{y}_2$ for $y_2=0$ and $y_1>0$.

%%%%%%%%%%%%%%%%%%%%%%%%%%%%%%%%%%%%%%%%%%%%%%%%%%%%%%%%%%%%%%%%%%%%%%%%%%%%%
\subsection{Case 5: $\alpha_1$ and $\alpha_2$ are complex conjugates}

When the eigenvalues are complex conjugates,
$$\alpha_1=\mu+i\eta\qquad\mbox{and}\qquad\alpha_2=\mu-i\eta$$
the eigenvectors are also complex conjugates
$$\begin{bmatrix} \phi_1 \\ \psi_1 \end{bmatrix} = 
\conj{\begin{bmatrix} \phi_2 \\ \psi_2 \end{bmatrix}}$$
So for a real solution, the arbitrary constants must be complex conjugates
too
$$C_1=\conj{C_2}$$
Thus the solution is
$$\vect{Y}= e^{\mu t} \left(
\begin{bmatrix} A_1 \\ A_2 \end{bmatrix} \cos\eta t
+\begin{bmatrix} B_1\\ B_2 \end{bmatrix} \sin\eta t\right)$$
This is a spiral about the origin.  If $\mu>0$, $\left\|\vect{Y}\right\|
\to\infty$ as
$t\to\infty$, so we have an \textbf{unstable spiral}. If $\mu<0$, 
$\left\|\vect{Y}\right\|\to 0$ as $t\to\infty$, so we have a 
\textbf{stable spiral}.

\begin{center}
\phaseportrait{clsinit 0.01 1 -3 3 1	  400 0.5 0.5 cls}{}
\hspace{3cm}
\phaseportrait{clsinit 0.01 1 3 -3 1      400 0.5 0.5 cls}{}
\end{center}

To obtain a rough idea of the spiral's shape, examine $\der{y_2}{y_1}$ for
$y_1=0$ and for $y_2=0$.  Determine the sense of the rotation (clockwise or
anticlockwise) by examining $\dot{y}_2$ for $y_2=0$ and $y_1>0$.

%%%%%%%%%%%%%%%%%%%%%%%%%%%%%%%%%%%%%%%%%%%%%%%%%%%%%%%%%%%%%%%%%%%%%%%%%%%%%
\subsection{Case 6: $\alpha_1$ and $\alpha_2$ are equal and non-zero}

When $\alpha_1=\alpha_2=\alpha\neq 0$, there are two subcases, depending on
whether the coefficient matrix $\vect{A}$ has one or an infinite number of
eigenvectors.

When there is only a single eigenvector, the solution is
$$\vect{Y}=
C_1 \begin{bmatrix} \phi_1 \\ \psi_1 \end{bmatrix} e^{\alpha t}
+C_2 \left(\begin{bmatrix} \phi_1 \\ \psi_1 \end{bmatrix}t 
	 + \begin{bmatrix} p_1 \\ p_2 \end{bmatrix} 
\right) e^{\alpha t}$$
where 
$$(\vect{A}-\vect{I}\alpha)\begin{bmatrix} p_1 \\ p_2 \end{bmatrix}=
\begin{bmatrix} \phi_1 \\ \psi_1 \end{bmatrix}$$
A little algebra shows that
$$\der{y_2}{y_1}\to \frac{\psi_1}{\phi_1}$$
as $t\to\pm\infty$.  Therefore, the trajectories become parallel to
$\begin{bmatrix} \phi_1 \\ \psi_1 \end{bmatrix}$ as $t\to\pm\infty$.
This is a \textbf{degenerate node}.  It is \textbf{unstable} if
$\alpha>0$ or \textbf{stable} if $\alpha<0$.  

\begin{center}
unstable degenerate nodes

\phaseportrait{      clsinit 0.02 3 -1 1 1
      100 0.5 0.5  cls  100 -0.5 -0.5 cls
      100 0.7 0.3  cls  100 -0.7 -0.3 cls
      100 0.7 0    cls  100 -0.7 0    cls
      100 0.7 -0.3 cls  100 -0.7 0.3  cls
      100 0.4 -0.7 cls  100 -0.4 0.7  cls		
      100 0.1 -1   cls  100 -0.1 1    cls}{}
\hspace{3cm}
\phaseportrait{clsinit 0.02 1 1 -1 3
      100 0.5  0.5 cls  100 -0.5 -0.5 cls
      100 0.3  0.7 cls  100 -0.3 -0.7 cls
      100 0    0.7 cls  100 0    -0.7 cls
      100 -0.3 0.7 cls  100 0.3  -0.7 cls
      100 -0.7 0.4 cls  100 0.7  -0.4 cls		
      100 -1   0.1 cls  100 1    -0.1 cls}{}

\medskip
stable degenerate nodes\par

\phaseportrait{clsinit -0.02 3 -1 1 1
      100 0.5 0.5  cls  100 -0.5 -0.5 cls
      100 0.7 0.3  cls  100 -0.7 -0.3 cls
      100 0.7 0    cls  100 -0.7 0    cls
      100 0.7 -0.3 cls  100 -0.7 0.3  cls
      100 0.4 -0.7 cls  100 -0.4 0.7  cls		
      100 0.1 -1   cls  100 -0.1 1    cls}{}
\hspace{3cm}
\phaseportrait{clsinit -0.02 1 1 -1 3
      100 0.5  0.5 cls  100 -0.5 -0.5 cls
      100 0.3  0.7 cls  100 -0.3 -0.7 cls
      100 0    0.7 cls  100 0    -0.7 cls
      100 -0.3 0.7 cls  100 0.3  -0.7 cls
      100 -0.7 0.4 cls  100 0.7  -0.4 cls		
      100 -1   0.1 cls  100 1    -0.1 cls}{}
\end{center}

When there is an infinite number of independent eigenvectors, the equations 
have become decoupled and the solution is
$$\vect{Y}=\begin{bmatrix} C_1 \\ C_2 \end{bmatrix} e^{\alpha t}$$

This gives an \textbf{unstable star point} if $\alpha>0$ or a \textbf{stable
star point} if $\alpha<0$.

\begin{center}
\hbox{\begin{minipage}{0.5\textwidth}\centering
unstable star point\par

\phaseportrait{      clsinit 0.02 1 0 0 1
      50 0.46 0.19 cls     50 0.19 0.46 cls
      50 -0.46 0.19 cls    50 -0.19 0.46 cls
      50 0.46 -0.19 cls    50 0.19 -0.46 cls
      50 -0.46 -0.19 cls   50 -0.19 -0.46 cls}{}
\end{minipage}
\begin{minipage}{0.5\textwidth}\centering
stable star point\par

\phaseportrait{      clsinit 0.02 -1 0 0 -1
      50 0.46 0.19 cls     50 0.19 0.46 cls
      50 -0.46 0.19 cls    50 -0.19 0.46 cls
      50 0.46 -0.19 cls    50 0.19 -0.46 cls
      50 -0.46 -0.19 cls   50 -0.19 -0.46 cls
}{}
\end{minipage}
}\end{center}


%%%%%%%%%%%%%%%%%%%%%%%%%%%%%%%%%%%%%%%%%%%%%%%%%%%%%%%%%%%%%%%%%%%%%%%%%%%%%
\subsection{Case 7: $\alpha_1=0$}

The case of $\alpha_1=0$ is only possible if the coefficient matrix
$\vect{A}$ has zero determinant.  Therefore there is an infinite number of
critical points which satisfy
\begin{eqnarray*}
a_{11}y_1+a_{12}y_2&=&0\\
a_{21}y_1+a_{22}y_2&=&0
\end{eqnarray*}
These two equations are equivalent because $a_{11}a_{22}-a_{12}a_{21}=0$. 
Therefore any point on the line
$$y_2=-\frac{a_{11}}{a_{12}}y_1=-\frac{a_{21}}{a_{22}}y_1$$
is a critical point.  Any point on this line is a constant solution to the 
differential equation system.  

For any point away from this line, 
$$\der{y_2}{y_1}=\frac{a_{21}y_1+a_{22}y_2}{a_{11}y_1+a_{12}y_2}
=\frac{a_{21}}{a_{11}}=\frac{a_{22}}{a_{12}}$$
Therefore the trajectory is 
$$y_2=\frac{a_{21}}{a_{11}}y_1+c$$
There are two subcases, depending on whether $\alpha_2$ is non-zero or zero.

If $\alpha_2$ is non-zero, the trace of $\vect{A}$ is non-zero and
$$a_{11}+a_{22}\neq 0$$
Therefore the slope of the trajectory, $a_{21}/a_{11}$, is not equal to
$-a_{21}/a_{22}$, the slope of the line of critical points.  Thus the
trajectories are not parallel to the line of critical points.

\begin{center}
\phaseportrait{      clsinit -0.02 1 1 1 1
      100 -0.9 0.7 cls  100 -0.7 0.9 cls 	100 0.7 -0.9 cls  100 0.9 -0.7 cls
      100 -0.3 0.9 cls  100 -0.1 0.7 cls  100 0.1 0.5 cls 100 0.3 0.3 cls
      100 0.5 0.1 cls   100 0.7 -0.1 cls  100 0.9 -0.3 cls
      100 0.3 -0.9 cls  100 0.1 -0.7 cls  100 -0.1 -0.5 cls  100 -0.3 -0.3 cls
      100 -0.5 -0.1 cls   100 -0.7 0.1 cls  100 -0.9 0.3 cls}
{\psline[linecolor=black,linestyle=dashed,linewidth=0.8pt]{-}(1,-1)(-1,1)}
\hspace{3cm}
\phaseportrait{      clsinit 0.02 1 1 1 1
      100 -0.9 0.7 cls  100 -0.7 0.9 cls 	100 0.7 -0.9 cls 100 0.9 -0.7 cls
      100 -0.3 0.9 cls  100 -0.1 0.7 cls  100 0.1 0.5 cls  100 0.3 0.3 cls
      100 0.5 0.1 cls   100 0.7 -0.1 cls  100 0.9 -0.3 cls
      100 0.3 -0.9 cls  100 0.1 -0.7 cls  100 -0.1 -0.5 cls  100 -0.3 -0.3 cls
      100 -0.5 -0.1 cls   100 -0.7 0.1 cls  100 -0.9 0.3 cls}
{\psline[linecolor=black,linestyle=dashed,linewidth=0.8pt]{-}(1,-1)(-1,1)}
\end{center}

When $\alpha_2=0$, the trace of $\vect{A}$ is 
$a_{11}+a_{22}=0$.  Thus the slope of the trajectory, $a_{21}/a_{11}$, is 
equal to $-a_{21}/a_{22}$, the slope of the line of critical points, and
the trajectories are parallel to the line of critical points.

\begin{center}
\phaseportrait
{clsinit 0.02 1 1 -1 -1
      100 -0.2 -0.2 cls  100 -0.4 -0.4 cls  100 -0.6 -0.6 cls  100 -0.8 -0.8 cls
      100 0.2 0.2 cls    100 0.4 0.4 cls  100 0.6 0.6 cls  100 0.8 0.8 cls}
{\psline[linecolor=black,linestyle=dashed,linewidth=0.8pt]{-}(1,-1)(-1,1)}
\hspace{3cm}
\phaseportrait
{      clsinit -0.02 1 1 -1 -1
      100 -0.2 -0.2 cls  100 -0.4 -0.4 cls  100 -0.6 -0.6 cls  100 -0.8 -0.8 cls
      100 0.2 0.2 cls    100 0.4 0.4 cls    100 0.6 0.6 cls    100 0.8 0.8 cls}
{\psline[linecolor=black,linestyle=dashed,linewidth=0.8pt]{-}(1,-1)(-1,1)}
\end{center}

%%%%%%%%%%%%%%%%%%%%%%%%%%%%%%%%%%%%%%%%%%%%%%%%%%%%%%%%%%%%%%%%%%%%%%%%%%%%%
\subsection{Summary of Classification of Critical Points}

Define $p$ and $q$ in terms of the coefficient matrix $\vect{A}$ as
\begin{eqnarray*}
p&=&\frac{1}{2}\,\mbox{trace}(\vect{A})\\
q&=&\left|\vect{A}\right|
\end{eqnarray*}
Then the eigenvalues of $\vect{A}$ are
$$\alpha_1, \alpha_2 = p\pm\sqrt{p^2-q}$$
The critical points can be classified in terms of $p$ and $q$ as shown in 
figure~\ref{nlc fig:cp}.

%----------------------------------------------------------------------------
\begin{figure}\centering
\caption{This illustrates the classification of critical points in terms of 
$p$ and $q$.  The abbreviations are: un - unstable node; us - unstable spiral;
c - centre; ss - stable spiral; sn - stable node; dn - degenerate node; st -
star point; dc - degenerate case; sp - saddle point.}
\label{nlc fig:cp}

\psset{unit=2.5cm}
\begin{pspicture}(-1.4,-1.4)(1.5,1.5)
%\psframe(-1.4,-1.4)(1.5,1.5)
% First the curve, so the axes show through
\psplot[linecolor=gray,linewidth=1.5pt,plotstyle=curve]{0}{1}{x sqrt}
\psplot[linecolor=gray,linewidth=1.5pt,plotstyle=curve]{0}{1}{x sqrt neg}
% Now the axes and their labels
\psset{linewidth=1.2pt,linecolor=black}
\psline{->}(-1.2,0)(1.2,0)
\psline{->}(0,-1.2)(0,1.2)
\uput[r](1.2,0){$q$}
\uput[u](0,1.2){$p$}
% The items of text
\put(0.7,0.3){us}
\put(0.7,-0.3){ss}
\put(0.7,0.02){c}
\put(0.3,0.9){un}
\put(0.3,-0.9){sn}
\put(-0.7,0.3){sp}
\put(-0.7,-0.3){sp}
\put(0,0.6){dc}
\put(0,-0.6){dc}
\put(0.81,0.85){dn or st}
\put(0.81,-0.9){dn or st}
\end{pspicture}
\end{figure}
%----------------------------------------------------------------------------

%%%%%%%%%%%%%%%%%%%%%%%%%%%%%%%%%%%%%%%%%%%%%%%%%%%%%%%%%%%%%%%%%%%%%%%%%%%%%
\subsection{Examples}

The following examples illustrate phase portraits of the system
$$\dot{\vect{Y}}=\vect{A}\vect{Y}$$
for various matrices $\vect{A}$.

%============================================================================
\begin{example}
When $$\vect{A}=\begin{bmatrix}2 & 1 \\ 1 & 2 \end{bmatrix}$$
$p=2$ and $q=3$ so
the eigenvalues are $2\pm\sqrt{4-3}$ or $1$ and $3$.  The corresponding
eigenvectors are $\vect{K}_3=\begin{bmatrix} 1 \\ 1 \end{bmatrix}$ and
$\vect{K}_1=\begin{bmatrix} 1 \\ -1 \end{bmatrix}$.  As $t\to\infty$, the
trajectories become parallel to the eigenvector with the largest eigenvalue, 
hence parallel to $\vect{K}_3$.  Since $p>0$, this is an unstable node.

\begin{center}
\phaseportrait{
      clsinit 0.02 2 1 1 2
      50 0.5 0.5 cls    50 -0.5 0.5 cls    50 0.5 -0.5 cls    50 -0.5 -0.5 cls 
      50 0.9 0.1 cls	50 1 -0.3 cls	   50 0.1 0.9 cls     50 -0.3 1 cls
      50 -0.9 -0.1 cls	50 -1 0.3 cls	   50 -0.1 -0.9 cls   50 0.3 -1 cls
}{
\uput[u](1,1){$\vect{K}_3$}
\uput[u](-1,1){$\vect{K}_1$}
}
\end{center}
\end{example}
%============================================================================

%============================================================================
\begin{example}
When $$\vect{A}=\begin{bmatrix}-2 & 1 \\ 1 & -2 \end{bmatrix}$$
$p=-2$ and $q=3$ so
the eigenvalues are $-2\pm\sqrt{4-3}$ or $-1$ and $-3$.  The corresponding
eigenvectors are $\vect{K}_{-3}=\begin{bmatrix} 1 \\ -1 \end{bmatrix}$ and
$\vect{K}_{-1}=\begin{bmatrix} 1 \\ 1 \end{bmatrix}$.  
Since $p<0$, this is a stable node.

\begin{center}
\phaseportrait{
      clsinit 0.02 -2 1 1 -2
      50 0.5 0.5 cls    50 -0.5 0.5 cls    50 0.5 -0.5 cls    50 -0.5 -0.5 cls 
      50 0.9 0.1 cls	50 1 -0.3 cls	   50 0.1 0.9 cls     50 -0.3 1 cls
      50 -0.9 -0.1 cls	50 -1 0.3 cls	   50 -0.1 -0.9 cls   50 0.3 -1 cls
}{
\uput[u](1,1){$\vect{K}_{-1}$}
\uput[u](-1,1){$\vect{K}_{-3}$}
}
\end{center}
\end{example}
%============================================================================

%============================================================================
\begin{example}
When 
$$\vect{A}=\begin{bmatrix} 1 & 2 \\ 2 & 1 \end{bmatrix}$$ 
$p=1$ and $q=-3$ so
the eigenvalues are $1\pm\sqrt{1+3}$ or $3$ and $-1$.  The corresponding
eigenvectors are $\vect{K}_{3}=\begin{bmatrix} 1 \\ 1 \end{bmatrix}$ and
$\vect{K}_{-1}=\begin{bmatrix} 1 \\ -1 \end{bmatrix}$.  
Since $q<0$, this is a saddle point.

\begin{center}
\phaseportrait{
      clsinit 0.02 1 2 2 1
      50 0.5 0.5 cls    50 -0.5 0.5 cls    50 0.5 -0.5 cls    50 -0.5 -0.5 cls 
      80 0 0.4 cls	80 -0.1 0.8 cls	   80 -0.2 1.2 cls
      80 0.1 -0.8 cls   80 0 -0.4 cls	   80 0.2 -1.2 cls	
      80 0.4 0 cls	80 0.8 -0.1 cls	   80 1.2 -0.2 cls
      80 -0.8 0.1 cls   80 -0.4 0 cls    80 -1.2 0.2 cls
}{
\uput[u](1,1){$\vect{K}_3$}
\uput[u](-1,1){$\vect{K}_{-1}$}
}
\end{center}
\end{example}
%============================================================================

%============================================================================
\begin{example}
When $$\vect{A}=\begin{bmatrix}1 & -2 \\ 5 & -1 \end{bmatrix}$$
$p=0$ and $q=9$ so
the eigenvalues are $0\pm\sqrt{0-9}$ or $\pm i3$.  The eigenvector 
corresponding to $\alpha=i3$ is $\vect{K}_{i3}=\begin{bmatrix} 1 \\
(1-i3)/2 \end{bmatrix}$.

Therefore this is a centre, and the trajectories are ellipses about the
origin.  Since $\dot{y}_2>0$ when $y_2=0$ and $y_1>0$, the trajectories are
anticlockwise.  The ellipses can be sketched by noting that
\begin{eqnarray*}
\der{y_2}{y_1}&=&5\qquad\mbox{when $y_2=0$}\\
\der{y_2}{y_1}&=&\frac{1}{2}\qquad\mbox{when $y_1=0$}
\end{eqnarray*}

\begin{center}
\phaseportrait{
      clsinit 0.01 1 -2 5 -1
      100 0.2 0.2 cls    200 0.4 0.4 cls    300 0.6 0.6 cls
}{
\psline[linecolor=black,linestyle=dashed,linewidth=0.8pt]{-}(0.5,-0.5)(0.7,0.5)
\psline[linecolor=black,linestyle=dashed,linewidth=0.8pt]{-}(-0.5,0.7)(0.5,1.2)
\uput[d](1,-0.3){slope=$5$}
\uput[ul](0,0.91){slope=$\frac{1}{2}$}
}
\end{center}
\end{example}
%============================================================================

%============================================================================
\begin{example}
When 
$$\vect{A}=\begin{bmatrix} 1 & -3 \\ 3 & 1 \end{bmatrix}$$
$p=1$ and $q=10$ so
the eigenvalues are $1\pm\sqrt{-9}$ or $1\pm i3$.  Therefore the
trajectories are unstable spirals.  To find out whether the arrow on the
spiral trajectories is pointing clockwise or anticlockwise, consider
$\dot{y}_2=3y_1$.  Thus when $y_2=0$ and $y_1>0$, the trajectory's gradient
is positive.  Hence the spiral has anticlockwise sense.  The shape of the
spiral is found by considering
\begin{eqnarray*}
\der{y_2}{y_1}&=&3\quad\mbox{when $y_2=0$}\\
\der{y_2}{y_1}&=&-\frac{1}{3}\quad\mbox{when $y_1=0$}
\end{eqnarray*}

% Example 5
\begin{center}
\phaseportrait{
      clsinit 0.01 1 -3 3 1
      400 0.5 0.5 cls 
}{
}
\end{center}
\end{example}
%============================================================================

%============================================================================
\begin{example}
When
$$\vect{A}=\begin{bmatrix} -1 & -3 \\ 3 & -1 \end{bmatrix}$$
$p=-1$ and $q=10$ so
the eigenvalues are $-1\pm\sqrt{-9}$ or $-1\pm i3$.  Therefore the
trajectories are stable spirals.  To find out whether the arrow on the
spiral trajectories is pointing clockwise or anticlockwise, consider
$\dot{y}_2=3y_1$.  Thus when $y_2=0$ and $y_1>0$, the trajectory's gradient
is positive.  Hence the spiral has anticlockwise sense.  The shape of the
spiral is found by considering
\begin{eqnarray*}
\der{y_2}{y_1}&=&-3\quad\mbox{when $y_2=0$}\\
\der{y_2}{y_1}&=&\frac{1}{3}\quad\mbox{when $y_1=0$}
\end{eqnarray*}

% Example 6
\begin{center}
\phaseportrait{
      clsinit 0.01 -1 -3 3 -1
      400 0.5 0.5 cls 
}{
}
\end{center}
\end{example}
%============================================================================

%============================================================================
\begin{example}
When 
$$\vect{A}=\begin{bmatrix} 1 & 1 \\ -1 & 3 \end{bmatrix}$$
$p=2$ and $q=4$ so
there is a repeated eigenvalue of $2$.  This is therefore an unstable
degenerate node.  The eigenvector is $\vect{K}_{2}=\begin{bmatrix} 1 \\ 1 
\end{bmatrix}$.  The form of the phase portrait can be sketched
once $\dot{y}_2=-y_1$ has been determined for $y_2=0$ and $y_1>0$.

% Example 7
\begin{center}
\phaseportrait{
      clsinit 0.02 1 1 -1 3
      100 0.5  0.5 cls  100 -0.5 -0.5 cls
      100 0.3  0.7 cls  100 -0.3 -0.7 cls
      100 0    0.7 cls  100 0    -0.7 cls
      100 -0.3 0.7 cls  100 0.3  -0.7 cls
      100 -0.7 0.4 cls  100 0.7  -0.4 cls		
      100 -1   0.1 cls  100 1    -0.1 cls	
}{
\uput[u](1,1){$\vect{K}_2$}
}
\end{center}
\end{example}
%============================================================================

%============================================================================
\begin{example}
When 
$$\vect{A}=\begin{bmatrix} -1 & 1 \\ -1 & -3 \end{bmatrix}$$
$p=-2$ and $q=4$ so
there is a repeated eigenvalue of $-2$.  This is therefore a stable
degenerate node.  The eigenvector is $\vect{K}_{-2}=\begin{bmatrix} 1 \\ -1 
\end{bmatrix}$.  The form of the phase portrait can be sketched
once $\dot{y}_2=-y_1$ has been determined for $y_2=0$ and $y_1>0$.

% Example 8
\begin{center}
\phaseportrait{
      clsinit 0.02 -1 1 -1 -3
      100 -0.5  0.5 cls  100 0.5 -0.5 cls
      100 -0.3  0.7 cls  100 0.3 -0.7 cls
      100 0    0.7 cls   100 0    -0.7 cls
      100 0.3 0.7 cls    100 -0.3  -0.7 cls
      100 0.7 0.4 cls    100 -0.7  -0.4 cls		
      100 1   0.1 cls    100 -1    -0.1 cls	
}{
\uput[u](-1,1){$\vect{K}_{-2}$}
}
\end{center}
\end{example}
%============================================================================

%============================================================================
\begin{example}
When 
$$\vect{A}=\begin{bmatrix} 1 & 1 \\ 3 & 3 \end{bmatrix}$$
$p=2$ and $q=0$ so
the eigenvalues are $2\pm\sqrt{4}$ or $0$ and $4$.  $\alpha=0$ shows that
this is a degenerate case, and $\alpha=4$ shows that the critical points are
unstable.

The line of critical points satisfies $y_1+y_2=0$ hence is given by
$y_1=-y_2$.  For points away from this line, the trajectories are straight
lines with gradients given by
$$\der{y_2}{y_1}=\frac{3y_1+3y_2}{y_1+y_2}=3$$
hence the equation of the trajectories is
$$y_2=3y_1+c$$

% Example 9
\begin{center}
\phaseportrait{
      clsinit 0.04 1 1 3 3
      50 -0.9 1.3 cls   50 0.9 -1.3 cls
      50 -0.7 1.1 cls   50 0.7 -1.1 cls 
      50 -0.5 0.9 cls   50 0.5 -0.9 cls 
      50 -0.3 0.7 cls   50 0.3 -0.7 cls 
      50 -0.1 0.5 cls   50 0.1 -0.5 cls 
      50 0.1 0.3  cls   50 -0.1 -0.3 cls 
      50 0.3 0.1  cls   50 -0.3 -0.1 cls 
      50 0.5 -0.1 cls   50 -0.5 0.1 cls 
      50 0.7 -0.3 cls   50 -0.7 0.3 cls 
      50 0.9 -0.5 cls   50 -0.9 0.5 cls 
      50 1.1 -0.7 cls   50 -1.1 0.7 cls
      50 1.3 -0.9 cls   50 -1.3 0.9 cls
}{
\psline[linecolor=black,linestyle=dashed,linewidth=0.8pt]{-}(1,-1)(-1,1)
}
\end{center}
\end{example}
%============================================================================

%%%%%%%%%%%%%%%%%%%%%%%%%%%%%%%%%%%%%%%%%%%%%%%%%%%%%%%%%%%%%%%%%%%%%%%%%%%%%
\section{Non-Linear Critical Point Analysis}

Suppose the non-linear coupled system
\begin{eqnarray*}
\dot{y}_1&=&f_1(y_1,y_2)\\
\dot{y}_2&=&f_2(y_1,y_2)
\end{eqnarray*}
has a critical point at $(y_1^*,y_2^*)$ so that
$$f_1(y_1^*,y_2^*)=f_2(y_1^*,y_2^*)=0$$
Then using Taylor's theorem, we can expand the function $f(y_1,y_2)$ about
$(y_1^*,y_2^*)$
\begin{eqnarray*}
f(y_1,y_2)&=&f(y_1^*,y_2^*)
+\left.\frac{\del f}{\del y_1}\right|_{(y_1^*,y_2^*)}(y_1-y_1^*)\\
&&{}+\left.\frac{\del f}{\del y_2}\right|_{(y_1^*,y_2^*)}(y_2-y_2^*)+\cdots
\end{eqnarray*}
where the ``$\cdots$'' indicates terms that are quadratic and higher powers
of $(y_1-y_1^*)$ and $(y_2-y_2^*)$.

Close to the critical point, the quadratic and higher order terms may be
neglected, leading to the approximate expression for the coupled system
\begin{eqnarray*}
\dot{y}_1&=&\left.\frac{\del f_1}{\del y_1}\right|_{(y_1^*,y_2^*)}(y_1-y_1^*)
+\left.\frac{\del f_1}{\del y_2}\right|_{(y_1^*,y_2^*)}(y_2-y_2^*)\\
\dot{y}_2&=&\left.\frac{\del f_2}{\del y_1}\right|_{(y_1^*,y_2^*)}(y_1-y_1^*)
+\left.\frac{\del f_2}{\del y_2}\right|_{(y_1^*,y_2^*)}(y_2-y_2^*)
\end{eqnarray*}
Define new local variables
\begin{eqnarray*}
z_1&=&y_1-y_1^*\\
z_2&=&y_2-y_2^*
\end{eqnarray*}
Then for sufficiently small $(z_1,z_2)$, we have
$$\begin{bmatrix} \dot{z}_1 \\ \dot{z}_2\end{bmatrix} =	\vect{A}
\begin{bmatrix} z_1 \\ z_2\end{bmatrix}$$
where $\vect{A}$ is the matrix
$$\begin{bmatrix} \frac{\del f_1}{\del y_1} & \frac{\del f_1}{\del y_2} \\
\frac{\del f_2}{\del y_1} & \frac{\del f_2}{\del y_2}\end{bmatrix}$$
evaluated at $(y_1^*,y_2^*)$.

Then locally in the neighbourhood of the critical point, the trajectories of
the non-linear system will be the same as those of the local linear system
$$\dot{\vect{Z}}=\vect{A}\vect{Z}$$

This gives the following procedure for classifying the critical points of a
general non-linear coupled system:
\begin{itemize}
\item Classify the critical points of
\begin{eqnarray*}
\dot{y}_1&=&f_1(y_1,y_2)\\
\dot{y}_2&=&f_2(y_1,y_2)
\end{eqnarray*}
by classifying the local linear problem
$$\dot{\vect{Z}}=\vect{A}\vect{Z}$$
for each critical point in turn.  Note that $\vect{A}$ will be different for each
critical point.
\item Construct the global phase portrait by blending the local linear phase
portraits around each critical point.
\end{itemize}

%============================================================================
\begin{example}[Volterra-Lotka System]

As previously shown in example \ref{nlc: ex vl}, the critical points of the 
Volterra-Lotka system are at $(0,0)$ and at $(c/d,a/b)$.

In the neighbourhood of $(0,0)$, $y_1$ and $y_2$ are both small so the
quadratic terms $y_1y_2$ may be neglected.  This gives the linearised system
$$\begin{bmatrix} \dot{y}_1 \\ \dot{y}_2 \end{bmatrix} = 
\begin{bmatrix} -a & 0 \\ 0 & c \end{bmatrix}
\begin{bmatrix} y_1 \\ y_2 \end{bmatrix} $$
The eigenvalues are $-a$ and $c$, hence the origin is a saddle point.  
The eigenvectors are 
$\vect{K}_{-a}=\begin{bmatrix} 1 \\ 0 \end{bmatrix}$ and
$\vect{K}_{c}=\begin{bmatrix} 0 \\ 1 \end{bmatrix}$.  
We are only interested in the system's behaviour for positive $y_1$ and $y_2$,
so the portion of the phase portrait near the origin is

\begin{center}
\phaseportrait{
      clsinit 0.02 -1 0 0 1
      50 0.5 0 cls    50 0 0.5 cls 
      80 0.4 0.4 cls	80 0.8 0.8 cls	  
}{
}
\end{center}

\noindent In the neighbourhood of $(c/d,a/b)$, shift the origin by writing
\begin{eqnarray*}
y_1&=&\frac{c}{d}+z_1\\
y_2&=&\frac{a}{b}+z_2
\end{eqnarray*}
and substitute into the Volterra-Lotka equations.  After neglecting
quadratic terms, the linearised system is
$$\begin{bmatrix} \dot{z}_1 \\ \dot{z}_2 \end{bmatrix} = 
\begin{bmatrix} 0 & cb/d \\ -ad/b & 0 \end{bmatrix}
\begin{bmatrix} z_1 \\ z_2 \end{bmatrix}$$
The eigenvalues are $\pm i\sqrt{ac}$, hence this critical point is a centre.
Looking at the signs of the coefficients shows that the direction of
rotation about the critical point is clockwise.

\begin{center}
% Circular phase portraits
\zphaseportrait{clsinit 0.02 0 1 -1 0 100 0.42 0.42 cls 200 0.707 0.707 cls}{}
\end{center}

Putting these two critical points together and smoothly combining their
trajectories gives the global phase portrait shown below.

\begin{center}
\psset{unit=2cm}
\begin{pspicture}(0,-0.3)(2.5,2.9)
% First the Volterra-Lotka curves.  These are plotted by a PostScript routine
\pscustom[linewidth=0.0278pt,linecolor=gray]{
\code{
gsave
2 2.5 div 72 mul dup scale
% Draw Volterra-Lotke system about (1,1).  The equations are
% \dot{x}=x(y-1)	\dot{y}=y(1-x)
% Volterra: dt x y  --  dt x+dx y+dy
/volterra {
	2 copy 1 sub mul	% --  dt x y x(y-1)
	4 copy pop		% --  dt x y x(y-1) dt x y
	exch 1 exch sub mul 	% --  dt x y x(y-1) dt y(1-x)
	exch dup	 	% --  dt x y x(y-1) y(1-x) dt dt
	3 -1 roll mul 		% --  dt x y x(y-1) dt dty(1-x)
	3 1 roll mul 		% --  dt x y dty(1-x) dtx(y-1)
	4 -1 roll add 		% --  dt y dty(1-x) x+dtx(y-1)
	3 1 roll add 		% --  dt x+dtx(y-1) y+dty(1-x)
} def
%
% Draw one cycle of VL phase portrait.  The starting point is assumed to be
% (x0,y0) where x0=1 and 0<y0<1.
% drawvolterra: dt y  --  dt
/drawvolterra {
	/VLstoppable false def
	newpath 1 exch		% --  dt 1 y
	2 copy moveto		% --  dt 1 y
	800 {			% --  dt 1 y
		volterra	% --  dt x y 
		2 copy lineto	% --  dt x y 
		dup 1 gt {/VLstoppable true def} if
		2 copy 1 lt exch 1 lt and VLstoppable and {exit} if
	} repeat
	pop pop			% --  dt
%	closepath
	stroke
} def
% phaseportrait
newpath
%0 0 moveto 3 0 lineto 3 3 lineto 0 3 lineto closepath stroke
%0 0 moveto 1 0 lineto 1 1 lineto 0 1 lineto closepath stroke
0 0 moveto 2.5 0 lineto 2.5 2.5 lineto 0 2.5 lineto closepath clip
0.01 0.1 drawvolterra
0.2 drawvolterra 
0.3 drawvolterra 
0.4 drawvolterra 
0.5 drawvolterra 
0.6 drawvolterra 
0.7 drawvolterra 
0.8 drawvolterra 
0.9 drawvolterra pop
grestore
}
}
% Axes
\psline{->}(0,0)(2.6,0)
\psline{->}(0,0)(0,2.6)
\uput[r](2.6,0){$y_1$}
\uput[u](0,2.6){$y_2$}
\uput[d](1,0){$\ds\frac{c}{d}$}
\uput[l](0,1){$\ds\frac{a}{b}$}
\uput[dl](0,0){$0$}
% Dashed lines
\psline[linecolor=black,linestyle=dashed]{-}(0,1)(1,1) 
\psline[linecolor=black,linestyle=dashed]{-}(1,0)(1,1) 
% Arrows on the VL curves
\psline[linewidth=2pt,linecolor=gray]{->}(1.01,0.1)(0.99,0.1)
\psline[linewidth=2pt,linecolor=gray]{->}(1.01,0.2)(0.99,0.2)
\psline[linewidth=2pt,linecolor=gray]{->}(1.01,0.3)(0.99,0.3)
\psline[linewidth=2pt,linecolor=gray]{->}(1.01,0.4)(0.99,0.4)
\psline[linewidth=2pt,linecolor=gray]{->}(1.01,0.5)(0.99,0.5)
\psline[linewidth=2pt,linecolor=gray]{->}(1.01,0.6)(0.99,0.6)
\psline[linewidth=2pt,linecolor=gray]{->}(1.01,0.7)(0.99,0.7)
\psline[linewidth=2pt,linecolor=gray]{->}(1.01,0.8)(0.99,0.8)
\psline[linewidth=2pt,linecolor=gray]{->}(1.01,0.9)(0.99,0.9)
\end{pspicture}
\end{center}
\end{example}
%============================================================================

%============================================================================
\begin{example}

For the system
\begin{eqnarray*}
\dot{y}_1&=&1-y_1^2-y_2^2\\
\dot{y}_2&=&2y_1
\end{eqnarray*}
the critical points are $(0,1)$ and $(0,-1)$.

For the critical point at $(0,1)$, let $y_1=z_1$ and $y_2=1+z_2$ and
substitute into the differential equations.  Neglecting the quadratic terms
gives the linearised system
$$\begin{bmatrix} \dot{z}_1 \\ \dot{z}_2 \end{bmatrix} = 
\begin{bmatrix} 0 & -2 \\ 2 & 0 \end{bmatrix}
\begin{bmatrix} z_1 \\ z_2 \end{bmatrix}$$
The eigenvalues are $\pm i\sqrt{ac}$, hence this critical point is a centre.
Looking at the signs of the coefficients shows that the direction of
rotation about the critical point is anticlockwise.

\begin{center}
% Circular phase portraits
\zphaseportrait{clsinit 0.02 0 -2 2 0 100 0.42 0.42 cls 200 0.707 0.707 cls}{}
\end{center}

For the critical point at $(0,-1)$, let $y_1=z_1$ and $y_2=-1+z_2$ and
substitute into the differential equations.  Neglecting the quadratic terms
gives the linearised system
$$\begin{bmatrix} \dot{z}_1 \\ \dot{z}_2 \end{bmatrix} = 
\begin{bmatrix} 0 & 2 \\ 2 & 0 \end{bmatrix}
\begin{bmatrix} z_1 \\ z_2 \end{bmatrix}$$
This gives a saddle point.

\begin{center}
\zphaseportrait{
      clsinit 0.02 0 2 2 0
      50 0.5 0.5 cls    50 -0.5 0.5 cls    50 0.5 -0.5 cls    50 -0.5 -0.5 cls 
      80 0 0.4 cls	80 0 0.8 cls	   80 0 -0.8 cls      80 0 -0.4 cls
      80 0.4 0 cls	80 0.8 0 cls	   80 -0.8 0 cls      80 -0.4 0 cls
      80 0 1.2 cls	80 0 -1.2 cls	   80 1.2 0 cls	      80 -1.2 0 cls}{
}
\end{center}

Putting these two critical points together and smoothly combining their
trajectories gives the global phase portrait shown below.

\begin{center}
\psset{unit=1.3cm}
\begin{pspicture}(-2.4,-2.4)(2.6,2.6)
%\psframe(-2.4,-2.4)(2.6,2.6)
\pscustom[linecolor=gray,linewidth=1.5pt]{
  \code{
    % First scale to make the PS coordinate system the same as PSTricks'
    1.3 2.54 div 72 mul dup dup scale CLW exch div SLW
    % Now the phase portrait, using the macros in PHASPORT.PRO
    save
    % This redefines the /clsiterate loop in the file phasport.pro 
    % to do the quadratic system x'=1-x^2-y^2, y'=2x.  
    % The four parameters a, b, c and d are ignored, for compatibility.
    % clsiterate: dt a b c d x y -- dt a b c d x+dx y+dy
    clsDict begin
    /clsiterate {
    	7 copy 2 copy 		% -- dt a b c d x y dt a b c d x y x y 
    	pairmul add 1 exch sub	% -- dt a b c d x y dt a b c d 1-x^2-y^2
	5 1 roll pop pop pop pop % -- dt a b c d x y dt 1-x^2-y^2
	4 copy pop pop pop 2 mul % -- dt a b c d x y dt 1-x^2-y^2 2x
	2 copy 2 copy pairmul	% -- dt a b c d x y dt dx dy dx^2 dy^2
	add sqrt  		% -- dt a b c d x y dt dx dy dr
	4 -1 roll   		% -- dt a b c d x y dx dy dr dt
	exch div   		% -- dt a b c d x y dx dy dd
	dup pairmul  		% -- dt a b c d x y dx*dd dy*dd
	pairadd			% -- dt a b c d x1 y1
    } def
    % The following code replaces /clsinit
    % Set the clipping path to the box from (-2.2,-2.2) to (2.2,2.2)
    newpath -2.2 -2.2 moveto 0 4.4 rlineto 4.4 0 rlineto 0 -4.4 rlineto 
    closepath clip
    % The coefficients 1 1 1 1 are dummies.  The new /clsiterate does not use them.
    0.01 1 1 1 1
    800 -0.3 1.6 cls
    800 -0.4 1.8 cls
    250 0.1 0 cls
    160 0.1 0.4 cls
    100 0.1 0.8 cls
    300 0.1 -0.4 cls
    350 0.1 -0.8 cls
    200 0.1 -1.4 cls
    200 0.1 -1.8 cls
    clsend
    end
    restore
  }
}
% Draw the axes and their labels last so they show through
\psset{linewidth=1.2pt,linecolor=black}
\psline{->}(-2.2,0)(2.2,0)
\psline{->}(0,-2.2)(0,2.2)
\uput[r](2.2,0){$y_1$}
\uput[u](0,2.2){$y_2$}
\psline[linecolor=black,linestyle=dashed,linewidth=0.8pt]{-}(-0.5,-1.5)(0.5,-0.5)
\psline[linecolor=black,linestyle=dashed,linewidth=0.8pt]{-}(-0.5,-0.5)(0.5,-1.5)
\uput[l](0,1){$1$}
\uput[l](0,-1){$-1$}
\end{pspicture}
\end{center}
\end{example}
%============================================================================

	
\end{document}

