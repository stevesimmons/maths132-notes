%%%%%%%%%%%%%%%%%%%%%%%%%%%%%%%%%%%%%%%%%%%%%%%%%%%%%%%%%%%%%%%%%%%%%%%%%%%
%
%			Mathematics 132 Course Notes
%
%			 Department of Mathematics,
%   			  University of Melbourne
%
%		Stephen Simmons			Lee White
%
% 8 Feb-96 SS: Updated with corrections from semester 2, 1995
%
%%%%%%%%%%%%%%%%%%% Copyright (C) 1995-96 Stephen Simmons %%%%%%%%%%%%%%%%%

\chapter{Vector Functions}
\label{vf chp}

A vector function $\vect{F}(t)$
$$\vect{F}(t)=F_x(t)\ivect+F_y(t)\jvect+F_z(t)\kvect$$
where $F_x(t)$, $F_y(t)$ and $F_z(t)$ are scalar functions of $t$, describes 
a trajectory in space as the parameter $t$ is varied.

Since the three unit vectors $\ivect$, $\jvect$ and $\kvect$ are constant
vectors, the first and second derivatives of $\vect{F}(t)$ are
\begin{eqnarray*}
\der{\vect{F}}{t}=\dot{\vect{F}}
&=&\lim_{\delta t\to 0}\frac{\vect{F}(t+\delta t)-\vect{F}(t)}{\delta t} \\
&=&\dot{F_x}\ivect+\dot{F_y}\jvect+\dot{F_z}\kvect
\end{eqnarray*}
and
\begin{eqnarray*}
\dder{\vect{F}}{t}=\ddot{\vect{F}}
&=&\lim_{\delta t\to 0}\frac{\dot{\vect{F}}(t+\delta t)
-\dot{\vect{F}}(t)}{\delta t} \\
&=&\ddot{F_x}\ivect+\ddot{F_y}\jvect+\ddot{F_z}\kvect
\end{eqnarray*}

In the limit as $\delta t\to 0$, $\vect{F}(t+\delta t)-\vect{F}(t)$ becomes
tangential to the trajectory curve so that $\dot{\vect{F}}(t)$ is the tangent 
at $t$.

If $t$ is time and $\vect{r}(t)$ is the position of a particle at time $t$
$$\vect{r}(t)=x(t)\ivect+y(t)\jvect+z(t)\kvect$$
where $x(t)$, $y(t)$ and $z(t)$ are the particle's Cartesian coordinates, 
the velocity of the particle is
$$\dot{\vect{r}}(t)=\dot{x}(t)\ivect+\dot{y}(t)\jvect+\dot{z}(t)\kvect$$
and the acceleration is 
$$\ddot{\vect{r}}(t)=\ddot{x}(t)\ivect+\ddot{y}(t)\jvect+\ddot{z}(t)\kvect$$


%%%%%%%%%%%%%%%%%%%%%%%%%%%%%%%%%%%%%%%%%%%%%%%%%%%%%%%%%%%%%%%%%%%%%%%%%%%%%
\section{Differentiation of Sums and Products}

Let $\vect{F}(t)$ and $\vect{G}(t)$ be vector functions and $f(t)$ be a
scalar function.  Then
$$\dbd{t}\left(\vect{F}+\vect{G}\right)=\der{\vect{F}}{t}+\der{\vect{G}}{t}$$
$$\dbd{t}\left(\vect{F}\cdot\vect{G}\right)=\der{\vect{F}}{t}\cdot\vect{G}
+\vect{F}\cdot\der{\vect{G}}{t}$$
$$\dbd{t}\left(\vect{F}\times\vect{G}\right)=\der{\vect{F}}{t}\times\vect{G}
+\vect{F}\times\der{\vect{G}}{t}$$
$$\dbd{t}\left(f\vect{F}\right)=\der{f}{t}\vect{F}+f\der{\vect{F}}{t}$$

%============================================================================
\begin{example}
A particle moving with \name{constant rectilinear motion} has Cartesian 
coordinates
\begin{eqnarray*}
x(t)&=&at+b\\
y(t)&=&ct+d\\
z(t)&=&et+f
\end{eqnarray*}
so the position of the particle can be written
$$\vect{r}(t)=(at+b)\ivect+(ct+d)\jvect+(et+f)\kvect=\vect{r}_0+\vect{v}t$$
where
\begin{eqnarray*}
\vect{r}_0&=&b\ivect+d\jvect+f\kvect \\
\vect{v}&=&a\ivect+c\jvect+e\kvect 
\end{eqnarray*}
are constant vectors.

Then the velocity is
$$\dot{\vect{r}}=\dot{x}\ivect+\dot{y}\jvect+\dot{z}\kvect
=a\ivect+c\jvect+e\kvect=\vect{v}$$
and the acceleration is
$$\ddot{\vect{r}}=\ddot{x}\ivect+\ddot{y}\jvect+\ddot{z}\kvect
=\vect{0}$$
\end{example}
%============================================================================

%============================================================================
\begin{example}
A particle moving with \name{uniform circular motion} has Cartesian 
coordinates
\begin{eqnarray*}
x(t)&=&x_0+a\cos\omega t\\
y(t)&=&y_0+a\sin\omega t\\
z(t)&=&0
\end{eqnarray*}
so the position of the particle can be written
$$\vect{r}(t)=\vect{r}_0+a\cos\omega t\ivect+a\sin\omega t\jvect$$
where
$$\vect{r}_0=x_0\ivect+y_0\jvect$$
Note that
$$\left|\vect{r}-\vect{r}_0\right|
=\left|a\cos\omega t\ivect+a\sin\omega t\jvect\right|=a$$

The velocity is
$$\dot{\vect{r}}(t)=-a\omega\sin\omega t\ivect+a\omega\cos\omega t\jvect$$
so the speed is constant
$$\left|\dot{\vect{r}}\right|=a\omega$$
and that the velocity is orthogonal to $\vect{r}-\vect{r}_0$
$$\dot{\vect{r}}\cdot\left(\vect{r}-\vect{r}_0\right)=0$$
This orthogonality can be proved by direct substitution, or alternatively,
by using 
$$\left(\vect{r}-\vect{r}_0\right)\cdot\left(\vect{r}-\vect{r}_0\right)
=\left|\vect{r}-\vect{r}_0\right|^2=a^2$$
Then differentiating with respect to time, 
$$0=\dbd{t}a^2=\dbd{t}\left|\vect{r}-\vect{r}_0\right|^2
=2\left(\dbd{t}(\vect{r}-\vect{r}_0)\right)
\cdot\left(\vect{r}-\vect{r}_0\right)$$
Since $\dot{\vect{r}}_0=0$, 
$$\dot{\vect{r}}\cdot\left(\vect{r}-\vect{r}_0\right)=0$$
as required.

The acceleration is
$$\ddot{\vect{r}}(t)=-a\omega^2\cos\omega t\ivect
-a\omega^2\sin\omega t\jvect=-\omega^2 (\vect{r}-\vect{r}_0)$$
Note that the acceleration is inwards along the radius of the circle and
that the magnitude of the acceleration is constant
$$\left|\ddot{\vect{r}}\right|=\omega^2 a$$
Also, the acceleration and velocity are at right angles
$$\dot{\vect{r}}\cdot\ddot{\vect{r}}=0$$
This can be shown by direct substitution or by differentiating 
$$\dot{\vect{r}}\cdot\dot{\vect{r}}=\left|\dot{\vect{r}}\right|^2
=a^2\omega^2$$
with respect to $t$ to give
$$\dot{\vect{r}}\cdot\ddot{\vect{r}}+\ddot{\vect{r}}\cdot\dot{\vect{r}}=0$$
so that
$$\dot{\vect{r}}\cdot\ddot{\vect{r}}=0$$
\end{example}
%============================================================================

%============================================================================
\begin{exercise}
Problems and more examples can be found in chapter 1 of Fowles.
\end{exercise}
%============================================================================

%%%%%%%%%%%%%%%%%%%%%%%%%%%%%%%%%%%%%%%%%%%%%%%%%%%%%%%%%%%%%%%%%%%%%%%%%%%%%
\section{Polar Coordinates}

%----------------------------------------------------------------------------
\begin{figure}\centering
\caption{The unit vectors for polar coordinates $(r,\theta)$ are $\rvect$ in 
the radial direction and $\tvect$ in the transverse direction.}
\label{vf fig:polar}

\psset{xunit=5cm,yunit=5cm}
\begin{pspicture}(-0.25,-0.35)(1.3,1.4)
% First the curve, so the axes show through
\psplot[linecolor=gray,linewidth=2pt,plotstyle=curve]{0.5}{1}%
{1 1 x x mul sub sqrt sub}
% Now the axes and their labels
\psline{->}(-0.25,0)(1.2,0)
\psline{->}(0,-0.3)(0,1.2)
\uput[dl](0,0){$O$}
\uput[r](1.2,0){$x$}
\uput[u](0,1.2){$y$}
\pcline[linecolor=black,linewidth=1pt]{->}(0,0)(0.707,0.293)
\Aput{$r$}

\rput[bl]{22.5}(0.707,0.293){
	\psline{->}(0,0)(0.15,0) \rput[l]{*0}(0.16,0){$\rvect$}
	\psline{->}(0,0)(0,0.15) \rput[b]{*0}(0,0.16){$\tvect$}
}
\uput[dr](0.707,0.293){$P$}
\psarc{->}(0,0){1}{0}{22.5}
\uput[r](0.2,0.05){$\theta$}
\end{pspicture}
\end{figure}
%----------------------------------------------------------------------------

Consider a point $P$ at $(x,y)$ so that $\vect{r}=x\ivect+y\jvect$.
As shown in figure \ref{vf fig:polar}, $\vect{r}$ can be written in terms of
$r$ and $\theta$, where $r$ is the particle's distance from the origin and
$\theta$ is the angle that $\vect{r}$ makes with the positive $x$ axis.
\begin{eqnarray*}
x&=&r\cos\theta\\
y&=&r\sin\theta
\end{eqnarray*}
Then the velocity can also be written in terms of $r$ and $\theta$
\begin{eqnarray*}
\dot{x}&=&\dot{r}\cos\theta-r\dot{\theta}\sin\theta\\
\dot{y}&=&\dot{r}\sin\theta+r\dot{\theta}\cos\theta
\end{eqnarray*}
so that
\begin{eqnarray*}
\dot{\vect{r}}
&=&\dot{x}\ivect+\dot{y}\jvect\\
&=&(\dot{r}\cos\theta-r\dot{\theta}\sin\theta)\ivect+
(\dot{r}\sin\theta+r\dot{\theta}\cos\theta)\jvect \\
&=&\dot{r}(\cos\theta\ivect+\sin\theta\jvect)+r\dot{\theta}(-\sin\theta\ivect
+\cos\theta\jvect)
\end{eqnarray*}
Therefore
$$\dot{\vect{r}}=\dot{r}\rvect+r\dot{\theta}\tvect$$
where $\rvect$ is the \name{radial unit vector} in the direction of
$r$ increasing
$$\rvect=\cos\theta\ivect+\sin\theta\jvect=\frac{\vect{r}}{r}$$
Note that $\left|\hat{\vect{r}}\right|=1$ because $\hat{\vect{r}}$ is a unit
vector.  $\hat{\vect{r}}$ is not a constant vector (like $\ivect$) but
changes direction as $\theta$ changes.

$\tvect$ is the \name{transverse unit vector} in the direction 
of $\theta$ increasing
$$\tvect=-\sin\theta\ivect+\cos\theta\jvect$$
which is orthogonal to $\rvect$ 
$$\tvect\cdot\rvect=0$$
Note that $\left|\hat{\vect{\theta}}\right|=1$ because $\hat{\vect{\theta}}$ 
is a unit vector.  $\hat{\vect{\theta}}$ is not a constant vector but
changes direction as $\theta$ changes.

In the expression for velocity in polar coordinates,
$$\dot{\vect{r}}=\dot{r}\rvect+r\dot{\theta}\tvect$$
$\dot{r}$ is called the \name{radial velocity} and $r\dot{\theta}$ is called
the \name{transverse velocity}.

The derivative of the radial unit vector is
$$\der{\rvect}{t}=\dbd{t}(\cos\theta\ivect+\sin\theta\jvect)
=-\dot{\theta}\sin\theta\ivect+\dot{\theta}\cos\theta\jvect$$
which can be written
$$\dot{\rvect}=\dot{\theta}\tvect$$

The derivative of the transverse unit vector is
$$\der{\tvect}{t}=\dbd{t}(-\sin\theta\ivect+\cos\theta\jvect)
=-\dot{\theta}\cos\theta\ivect-\dot{\theta}\sin\theta\jvect$$
which can be written
$$\dot{\tvect}=-\dot{\theta}\rvect$$

Taking the derivative of the polar form of the particle's velocity gives
\begin{eqnarray*}
\ddot{\vect{r}}
&=&\dbd{t}\left(\dot{r}\rvect+r\dot{\theta}\tvect\right)\\
&=&\ddot{r}\rvect+\dot{r}\dot{\rvect}+\dot{r}\dot{\theta}\tvect
   +r\ddot{\theta}\tvect+r\dot{\theta}\dot{\tvect} \\
&=&\ddot{r}\rvect+\dot{r}\dot{\theta}\tvect
+\dot{r}\dot{\theta}\tvect+r\ddot{\theta}\tvect-r\dot{\theta}^2\rvect
\end{eqnarray*}
Gathering terms shows that acceleration in polar coordinates is
$$\ddot{\vect{r}}=\left(\ddot{r}-r\dot{\theta}^2\right)\rvect
+\left(r\ddot{\theta}+2\dot{r}\dot{\theta}\right)\tvect$$
where $\ddot{r}-r\dot{\theta}^2$ is called the \name{radial acceleration}
and $r\ddot{\theta}+2\dot{r}\dot{\theta}$ is called the \name{transverse
acceleration}.

%============================================================================
\begin{example}
For a particle moving in a circular orbit around the origin, $r=a$ so
$$\dot{\vect{r}}=a\dot{\theta}\tvect$$
so that the velocity is always transverse, and
$$\ddot{\vect{r}}=-a\dot{\theta}^2\rvect+a\ddot{\theta}\tvect$$
where $\dot{\theta}$ is the \name{instantaneous angular velocity} and
$\ddot{\theta}$ is the \name{instantaneous angular acceleration}.

If we write $v=a\dot{\theta}=\left|\dot{\vect{r}}\right|$ as the speed of
the particle, then the radial acceleration is $-a\dot{\theta}^2=-v^2/a$.
\end{example}
%============================================================================


%============================================================================
\begin{example}
For a particle moving with rectilinear motion through the origin, $\theta$
is constant so that $\rvect$ is constant.  Therefore
\begin{eqnarray*}
\vect{r}&=&r\rvect \\
\dot{\vect{r}}&=&\dot{r}\rvect \\
\ddot{\vect{r}}&=&\ddot{r}\rvect 
\end{eqnarray*}
These equations for rectilinear motion in polar coordinates are only simple
if the particle's trajectory includes the origin, otherwise $\theta$ changes
along the trajectory.
\end{example}
%============================================================================

%%%%%%%%%%%%%%%%%%%%%%%%%%%%%%%%%%%%%%%%%%%%%%%%%%%%%%%%%%%%%%%%%%%%%%%%%%%%%
\section{Intrinsic Coordinates}

%%%%%%%%%%%%%%%%%%%%%%%%%%%%%%%%%%%%%%%%%%%%%%%%%%%%%%%%%%%%%%%%%%%%%%%%%%%%%
\subsection{Two-Dimensional Intrinsic Coordinates}
\label{vf sec:2D ic}

%----------------------------------------------------------------------------
\begin{figure}\centering
\caption{The unit vectors for intrinsic coordinates $(s,\psi)$ are $\Tvect$ 
in the tangential direction and $\Nvect$ in the normal direction.  $s$ is
the distance along the curve and $\psi$ the angle the tangent makes with the
$x$ axis.}
\label{vf fig:2D intrinsic}

\psset{xunit=5cm,yunit=5cm}
\begin{pspicture}(-0.25,-0.35)(1.3,1.4)
% First the curve, so the axes show through
\psplot[linecolor=gray,linewidth=2pt,plotstyle=curve]{0.25}{1}%
{1 1 x x mul sub sqrt sub}
\psplot[linecolor=gray,linewidth=2pt,plotstyle=curve,arrows=>->]{0.35}{0.55}%
{1 1 x x mul sub sqrt sub}
\rput[B](0.45,0.135){$s$}
\uput*[r](0.2,0.05){$\theta$}
% Now the axes and their labels
\psline{->}(-0.25,0)(1.2,0)
\psline{->}(0,-0.3)(0,1.2)
\uput[dl](0,0){$O$}
\uput[r](1.2,0){$x$}
\uput[u](0,1.2){$y$}
\pcline[linecolor=black,linewidth=1pt]{->}(0,0)(0.707,0.293)
\Aput{$r$}
\psline[linecolor=black,linewidth=1pt,linestyle=dashed]{-}(0.414,0)(0.707,0.293)
\psarc{->}(0.414,0){1}{0}{45}
\uput[r](0.6,0.05){$\psi$}
\rput[bl]{22.5}(0.707,0.293){
	\psline{->}(0,0)(0.15,0) \rput[l]{*0}(0.16,0){$\rvect$}
	\psline{->}(0,0)(0,0.15) \rput[b]{*0}(0,0.16){$\tvect$}
}
\rput[bl]{45}(0.707,0.293){
	\psline[linecolor=black]{->}(0,0)(0.15,0) 
	\rput[bl]{*0}(0.16,0){$\Tvect$}
	\psline[linecolor=black]{->}(0,0)(0,0.15) 
	\rput[br]{*0}(0,0.16){$\Nvect$}
}
\uput[dr](0.707,0.293){$P$}
\psarc{->}(0,0){1}{0}{22.5}
\end{pspicture}
\end{figure}
%----------------------------------------------------------------------------

The point $P$ at $(x,y)$ in Cartesian coordinates or $(r,\theta)$ is polar
coordinates can also be described by the \name{intrinsic coordinates} 
$(s,\psi)$ shown in figure \ref{vf fig:2D intrinsic}.

$s$ is arc length measured along the trajectory from some specified starting
point and $\psi$ is the angle the tangent to the trajectory makes with the
positive $x$ axis.

The associated unit vectors are $\Tvect$, the unit vector along the tangent
in the direction of increasing $s$
$$\Tvect=\cos\psi\ivect+\sin\psi\jvect$$
and $\Nvect$, the unit vector transverse to $\Tvect$ in the direction of
increasing $\psi$
$$\Nvect=-\sin\psi\ivect+\cos\psi\jvect$$

To find expressions for velocity and acceleration in intrinsic coordinates,
let $\vect{\delta r}$ be the difference in $\vect{r}$ as the particle moves a
distance $\delta s$ along the curve 
$$\vect{\delta r}=\vect{r}(s+\delta s)-\vect{r}(s)$$
As $\delta s\to 0$, $\vect{\delta r}$ becomes tangential to the curve so
that the magnitude of $\vect{\delta r}$ tends to $\delta s$
$$\frac{\left|\vect{\delta r}\right|}{\delta s}\to 1$$
and $\vect{\delta r}$ becomes a tangent to the curve
$$\frac{\vect{\delta r}}{\delta s}\to \Tvect$$
This implies that
$$\der{\vect{r}}{s}=\Tvect$$
Therefore
$$\dbd{t}\left(\vect{r}\left(s(t)\right)\right)=\der{\vect{r}}{s}\der{s}{t}
=\dot{s}\Tvect$$
so the expression for velocity in intrinsic coordinates is
$$\dot{\vect{r}}=\dot{s}\Tvect$$

To find the expression for acceleration, 
$$\ddot{\vect{r}}=\dbd{t}\left(\dot{s}\Tvect\right)
=\ddot{s}\Tvect+\dot{s}\dot{\Tvect}$$
Now $\Tvect=\cos\psi\ivect+\sin\psi\jvect$ so
$$\dot{\Tvect}=\dot{\psi}\left(-\sin\psi\ivect+\cos\psi\jvect\right)
=\dot{\psi}\Nvect$$
Therefore the expression for acceleration in intrinsic coordinates is
$$\ddot{\vect{r}}=\ddot{s}\Tvect+\dot{s}\dot{\psi}\Nvect$$

Note that from the definitions of $\Tvect$ and $\Nvect$,
$$\der{\Tvect}{s}=\der{\psi}{s}\Nvect$$
on the curve $\psi(s)$. $\der{\psi}{s}$ is the \name{curvature} of the curve
at $P(s)$ and
$$\der{\psi}{s}=\frac{1}{\rho(s)}$$
where $\rho(s)$ is the \name{radius of curvature} of the curve at the point
$P(s)$.

%%%%%%%%%%%%%%%%%%%%%%%%%%%%%%%%%%%%%%%%%%%%%%%%%%%%%%%%%%%%%%%%%%%%%%%%%%%%%
\subsection{Three-Dimensional Intrinsic Coordinates}


When the particle follows a trajectory through three-dimensional space, 
$s$ is the arc length measured along the curve from some specified point.
As for two-dimensional intrinsic coordinates,
$$\der{\vect{r}}{s}=\Tvect$$
and
$$\der{\Tvect}{s}=\kappa \Nvect=\frac{1}{\rho}\Nvect$$
where $\Nvect$ is the \name{principal normal} to the curve (transverse to
$\Tvect$ in the plane of the local curve).  Here $\kappa$ is the 
\name{curvature} of the curve at $P(s)$
$$\kappa=\left|\der{\Tvect}{s}\right|$$
and $\rho$ is the \name{radius of curvature} at $P(s)$.

Velocity in three-dimensional intrinsic coordinates is given by
$$\dot{\vect{r}}=\dbd{t}\vect{r}(s(t))=\der{\vect{r}}{s}\dot{s}$$
so that
$$\dot{\vect{r}}=\dot{s}\Tvect$$

Acceleration in three-dimensional intrinsic coordinates is 
$$\ddot{\vect{r}}=\dbd{t}(\dot{s}\Tvect)=\ddot{s}\Tvect+\dot{s}
\der{\Tvect}{t}$$
Now $\dbd{t}\Tvect=\der{\Tvect}{s}\dot{s}$ so that acceleration can be
written
$$\ddot{\vect{r}}=\ddot{s}\Tvect+\frac{\dot{s}^2}{\rho(s)}\Nvect$$





